\documentclass[a4paper,oneside,DIV=10,12pt]{scrartcl}

\usepackage{graphicx}
\usepackage{float}

\usepackage{fontspec}
\setmainfont{STIX Two Text}
%\setsansfont{Roboto}
\newfontfamily{\cyrillicfontsf}{Roboto}

\usepackage{microtype}

\usepackage{polyglossia}
\setmainlanguage{ukrainian}

\usepackage{amsmath}
\usepackage{unicode-math}
\setmathfont{STIX Two Math}
%\usepackage[retainorgcmds]{IEEEtrantools}

%\usepackage{booktabs}

\usepackage{enumitem}

\usepackage{siunitx}
\sisetup{output-decimal-marker = {,},
exponent-product = {\cdot}}

\begin{document}
	\section{Що таке поляризація світла?}
		\emph{Поляризація світла} --- це процес упорядкування площини коливань вектора напруженості електричного поля і пов'язаного з ним вектора напруженості магнітного поля у просторі.
			
	\section{Чим відрізняється плоскополяризоване світло від неполяризованого?}
		\emph{Поляризованим} називається світло, в якому вектор напруженості електричного поля виконує коливання в одній площині. У \emph{неполяризованому} світлі площина коливань вектора для кожного пучка орієнтована випадково, тому площини коливань векторів напруженості електричного поля орієнтовані у будь-яких перпендикулярних до вектора швидкості напрямках.
			
	\section{Які типи поляризованого світла ви знаєте?}
		Електромагнітні хвилі з певним хвильовим вектором $\mathbf{k}$ в системі координат, вісь $z$ якої збігається із напрямом поширення, загалом, можна записати так:
		\[
			\mathbf{E} = E_{x} \mathbf{i} + E_{y} \mathbf{j} = E_{0x} \mathbf{i} \cos(kz - \omega t - \varphi_{x}) + E_{0y} \mathbf{j} \cos(kz - \omega t - \varphi_{y}),
		\]
		де $\mathbf{i}$ та $\mathbf{j}$  — це орти у напрямку осей $x$ та $y$, $\omega$ — частота, $E_{0x}$ і $E_{0y}$ — дві незалежні амплітуди, $\varphi_{x}$ і $\varphi_{y}$— дві незалежні фази.
		
		\subsection{Пласка (лінійна) поляризація}
			Якщо фази $\varphi_x$ і $\varphi_y$ збігаються, то для хвилі в будь-який момент часу виконується співвідношення:
			\[
				\frac {E_{x}}{E_{y}}= \frac {E_{0x}}{E_{0y}} = {\text{const}}.
			\]
			
			Тобто в цьому випадку $E_{x}$ і $E_{y}$ зв'язані лінійним співвідношенням. Така поляризація електромагнітної хвилі називається лінійною поляризацією.
			
			До цього випадку відносяться також хвилі, для яких $E_{0x}=0$ або E 0 y = 0 $E_{0y}=0$. Будь-яку лінійно-поляризовану хвилю можна звести до одного з цих двох випадків, вибравши відповідним чином напрям осей $x$ та $y$.
			
		\subsection{Циклічна (колова) поляризація}
			Циклічна або колова поляризація виникає тоді, коли $E_{0x}=E_{0y}$, а фази різняться на чверть періоду:
			\[
				\varphi _{y}-\varphi _{x}=\pm {\frac {\pi }{2}}.
			\]
			
			У такому разі електромагнітна хвиля записується:
			\[
				\mathbf {E} =E_{x}\mathbf {i} +E_{y}\mathbf {j} =E_{0x}\mathbf {i} \cos(kz-\omega t-\varphi _{x})\pm E_{0y}\mathbf {j} \sin(kz-\omega t-\varphi _{x}).
			\]
			
			Для такої хвилі справджується рівність:
			\[
				\displaystyle E_{x}^{2}+E_{y}^{2}=E_{0x}^{2}+E_{0y}^{2}={\text{const}},
			\]
			яка є рівнянням кола щодо змінних $E_{x}$ та $E_{y}$.
			
			Залежно від знаку зсуву фази вектор напруженості електричного поля в будь-якій точці простору для циклічно-поляризованої хвилі обертається за чи проти годинникової стрілки, виконуючи повний оберт за період.
			
			Будь-яку лінійно поляризовану хвилю можна подати у вигляді суперпозиції двох циклічно поляризованих хвиль (з обертанням «за» та «проти» годинникової стрілки).
			
		\subsection{Еліптична поляризація}
			У загальному випадку між змінними $E_{x}$ та $E_{y}$ існує співвідношення, яке задається рівнянням:
			\[
				\left({\frac {E_{x}}{E_{0x}}}\right)^{2}+\left({\frac {E_{y}}{E_{0y}}}\right)^{2}-2{\frac {E_{x}}{E_{0x}}}{\frac {E_{y}}{E_{0y}}}\cos(\varphi _{y}-\varphi _{x})=\sin ^{2}(\varphi _{y}-\varphi _{x}).
			\]
			
			Це рівняння еліпса, тож така поляризація називається еліптичною.
			
		\subsection{Неполяризоване}
		\subsection{Часткова поляризація}
		
	\section{Які способи одержання поляризованого світла}
		Поляризоване світло можна дістати під час відбивання і заломлення природного світла на межі поділу двох діелектриків.
	\section{Закон Брюстера}
		\[
			\operatorname{tg} \alpha_B = \frac{n_2}{n_1}.
		\]
\end{document}