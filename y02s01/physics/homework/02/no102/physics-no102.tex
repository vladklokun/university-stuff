\documentclass[a4paper,oneside,DIV=9,12pt]{scrartcl}

\usepackage{fontspec}
\setmainfont{STIX Two Text}

\usepackage{amsmath}
\usepackage{unicode-math}
\setmathfont{STIX Two Math}

\usepackage{microtype}

\usepackage{polyglossia}
\setmainlanguage{ukrainian}

\usepackage{siunitx}
\sisetup{output-decimal-marker = {,},
		inter-unit-product = \ensuremath{{}\cdot{}},
		exponent-product = \cdot}

\newcommand\given{\noindent\textbf{Дано:}}
\newcommand\find{\noindent\textbf{Знайти:}}
\newcommand\solution{\noindent\textbf{Розв'язання:}}
\newcommand\answer[1]{\noindent\textbf{Відповідь:} #1}

\begin{document}
	
	\given\par
		$\nu = \SI{4e6}{\hertz}$;
		
		$\mu = 1$;
		
		$\Delta \lambda = \SI{37,5}{\hertz}$.
		
	\find\par
		$n$ --- ?.
		
	\solution\par
		Показник заломлення середовища визначається за формулою:
		\[
			n = \frac{c}{v}.
		\]
		
		Знайдемо $v$:
		\[
			\Delta \lambda = \frac{v}{\nu} \quad \implies \quad v = \Delta \lambda \nu.
		\]
		
		Знаходимо $n$:
		\[
			n = \frac{c}{\Delta \lambda \nu} = \frac{\num{3e8}}{\num{37,5} \cdot \num{4e6}} = \frac{\num{3e8}}{\num{1,5e8}} = \num{2}.
		\]
	\answer{2}.
	
\end{document}