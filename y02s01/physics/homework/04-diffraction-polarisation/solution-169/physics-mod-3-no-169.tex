\documentclass[a4paper,oneside,DIV=10,12pt]{scrartcl}

\usepackage{graphicx}
\usepackage{float}

\usepackage{fontspec}
\setmainfont{STIX Two Text}
%\setsansfont{Roboto}
\newfontfamily{\cyrillicfontsf}{Roboto}

\usepackage{microtype}

\usepackage{polyglossia}
\setmainlanguage{ukrainian}

\usepackage{amsmath}
\usepackage{unicode-math}
\setmathfont{STIX Two Math}

\usepackage{booktabs}

\usepackage{siunitx}
\sisetup{output-decimal-marker = {,},
exponent-product = {\cdot}}

\begin{document}
	\section*{№169}
		Кут між площинами пропускання поляроїдів дорівнює \SI{50}{\degree}. Природне світло, проходячи через таку систему, послаблюється у 8 разів. Нехтуючи втратою світла при відбиванні, визначити коефіцієнт поглинання світла в поляроїдах.
		
		\subsection*{Дано}
		\noindent${I_a} / {I_0} = {1} / {8}$; \\
		$\varphi = \SI{50}{\degree}$.
			
		\subsection*{Знайти}
			$k$ — ?
			
		\subsection*{Розв'язання}
			Запишемо закон Малюса:
			\begin{equation}
			\label{eq:malus}
				I_a = (1 - k) \cos^2 \varphi I_p.
			\end{equation}
			Виразимо $I_p$ через $I_0$:
			\[
				I_p = \frac{(1 - k)}{2} I_0.
			\]
			Підставивши в~\eqref{eq:malus} отримаємо:
			\[
				I_a = \frac{(1 - k) (1 - k) \cos^2 \varphi}{2} I_0.
			\]
			Поділимо обидві частини на $I_0$:
			\[
				\frac{I_a}{I_0} = \frac{(1 - k)^2 \cos^2 \varphi}{2}.
			\]
			З умови маємо:
			\[
				\frac{(1 - k)^2 \cos^2 \varphi}{2} = \frac{1}{8}.
			\]
			Тобто:
			\[
				(1 - k)^2 = \frac{
					2
				}{
					8 \cos^2 \varphi
				}
			\]
			Звідси:
			\[
				k = 1 - \sqrt{\frac{2}{8 \cdot \cos^2(\SI{50}{\degree})}} = 0{,}22.
			\]
\end{document}