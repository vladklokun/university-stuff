\documentclass[a4paper,oneside,12pt,DIV=9]{scrartcl}

\usepackage{fontspec}
\setmainfont{PT Serif}
\setsansfont{PT Sans}
\setmonofont{PT Mono}

\usepackage{microtype}

%Localization
\usepackage{polyglossia}
\setdefaultlanguage{ukrainian}
\PolyglossiaSetup{ukrainian}{indentfirst=true}
\setotherlanguage[variant=uk]{english}

\begin{document}

	\textbf{Exercise 8.} Translate into English.
	\begin{enumerate}
		\item Терміном «конструювання програмного забезпечення» називають детальне створення робочого, змістовного програмного забезпечення шляхом поєднання кодування, перевірки, тестування компонентів системи, тестування взаємодії компонентів системи та налагодження.
		
		\begin{english}
			The term “software construction” refers to the detailed creation of working meaningful software through a combination of coding, verification, unit testing, integration testing and debugging.
		\end{english}
		
		\item Конструювання програмного забезпечення пов'язане з усіма іншими видами роботи з розробки програмного забезпечення, найбільше з його проектуванням та тестуванням.
		
		\begin{english}
			The software construction process is linked to all the other software engineering activities, most strongly to software design and software testing.
		\end{english}
		
		\item Сам процес конструювання програмного забезпечення включає багато проектування та тестування.
		
		\begin{english}
			The software construction process itself involves significant software design and test activity. 
		\end{english}
		
		\item Також в процесі конструювання програмного забезпечення використовуються вихідні дані проектування, які забезпечують один із вхідних параметрів для тестування.
		
		\begin{english}
			It also uses the output of design and provides one of the inputs to testing.
		\end{english}
		\item Чіткі межі між розробкою, конструюванням та тестуванням залежать від процесів життєвого циклу програмного забезпечення, що використовуються в проекті.
		
		\begin{english}
			Detailed boundaries between design, construction, and testing will vary depending upon the software life cycle processes that are used in the project.
		\end{english}
		
		\item Одним з ключових процесів конструювання є поєднання окремо створених підпрограм, класів, компонентів та підсистем.
		
		\begin{english}
			One of the key activities during construction is the integration of separately constructed routines, classes, components, and subsystems.
		\end{english}
		
		\item Крім цього окремі системи програмного забезпечення можуть потребувати поєднання з іншими програмними чи апаратними системами.
		
		\begin{english}
			In addition, a particular software system may need to be integrated with other software or hardware systems.
		\end{english}
		
		\item Спочатку необхідно перевірити, чи закладені основи для успішного проведення конструювання.
		
		\begin{english}
			First you have to verify that the groundwork has been laid so that construction can proceed successfully.
		\end{english}
		
		\item Потім потрібно визначити спосіб тестування коду, написати класи та підпрограми, створити й назвати змінні та іменовані константи.
		
		\begin{english}
			Then you have to determine how your code will be tested, write classes and routines, create and name variables and named constants.
		\end{english}
		
		\item Після цього потрібно вибрати керівні конструкції та створити блоки операторів.
		
		\begin{english}
			After this you have to select control structures and organize blocks of statements.
		\end{english}
		
		\item Всі члени команди повинні перевірити низькорівневі програмні структури та код один одного.
		
		\begin{english}
			All the team members have to review each others' low-level designs and code.
		\end{english}
		
		\item Код має бути налагоджений, «відшліфований» та відрегульований для його пришвидшення та використання меншої кількості ресурсів.
		
		\begin{english}
			The code has to be debugged, polished and tuned in order to make it faster and use fewer resources.
		\end{english}
		
		\item Під час конструювання програмного забезпечення зазвичай створюються елементи конфігурації, що потребують керування в програмному проекті.
		
		\begin{english}
			Software construction typically produces configuration items that need to be managed in a software project.
		\end{english}
		
		\item Такими елементами конфігурації є вихідні файли, зміст та набори тестових даних.
		
		\begin{english}
			These elements are source files, content, test cases and so on.
		\end{english}
		
		\item Отже, конструювання програмного забезпечення також тісно пов'\-я\-за\-не з керуванням конфігурацією програмних засобів.
		
		\begin{english}
			Thus, the software construction is also closely linked to the software configuration management.
		\end{english}
	\end{enumerate}
\end{document}