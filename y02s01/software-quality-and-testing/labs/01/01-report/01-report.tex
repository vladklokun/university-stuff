\documentclass[a4paper,oneside,12pt,DIV=12,titlepage,toc]{scrartcl}

%Typography
\usepackage{fontspec}
\setromanfont{PT Serif}
\setsansfont{PT Sans}
\setmonofont{PT Mono}

\usepackage{microtype}

%Localization
\usepackage{polyglossia}
\setdefaultlanguage{ukrainian}
\PolyglossiaSetup{ukrainian}{indentfirst=true}

%Math typesetting
\usepackage{amsmath}
\usepackage{unicode-math}
\setmathfont{STIX Two Math}

%Table typesetting
\usepackage{booktabs}
\usepackage{longtable}
\usepackage{array}
\newcolumntype{x}[1]{>{\raggedright\arraybackslash\hspace{0pt}}p{#1}}

%Including graphics
\usepackage{graphicx}

%Itemize separator
\def\labelitemi{—}
\renewcommand{\arraystretch}{1.4}

\begin{document}
	\begin{titlepage}
		\begin{center}
			Міністерство освіти і науки України\\
			Національний авіаційний університет\\
			Навчально-науковий Інститут\\
			Комп'ютерних інформаційних технологій\\
			Кафедра комп'ютеризованих систем управління
			
			\vspace{\fill}
				Лабораторна робота №2\\
				з дисципліни «Якість програмного забезпечення та тестування»\\
				на тему: «Тестування специфікації вимог»\\
				
			\vspace{\fill}
			
			\begin{flushright}
				Виконав:\\
				студент ННІКІТ СП-225\\
				Клокун Владислав\\
				Перевірила:\\
				Апенько Н. В.
			\end{flushright}
			Київ 2017
		\end{center}
	\end{titlepage}
	
	\section{Недоліки наданої специфікації}
		\subsection{Загальні}
			У наданій специфікації цілком проігнорована структура, рекомендована стандартом IEEE 830: жоден з розділів, передбачених стандартом, не представлений у документі в належному вигляді.
			Натомість документ складається з розділів «Користувацький функціонал», «Опис панелей», «Опис вкладок».
			Загалом, наданий документ не відповідає таким критеріям SRS:
			\begin{enumerate}
				\item Коректність --- наданий документ не містить порівнянь з будь-якою якісною специфікацією, документацією проекту і є неузгодженим.
				\item Однозначність --- надана специфікація містить пункти, що визначені неоднозначно. 
				\item Несуперечливість --- документ містить вимоги, які містять конфлікти.
				\item Упорядкування за значимістю і/або стійкістю --- жодна з вимог, поставлених у документі, не має ідентифікатора, що показує її значимість або стійкість.
				\item Модифікованість --- структура та стиль наданого документа не передбачає зміст, алфавітний покажчик та явно виражених перехресних посилань. Не всі вимоги, передбачені у документі, виражені окремо (табл. \ref{tab:sect01comp}). 
			\end{enumerate}
		\subsection{Розділ «Користувацька функціональність»}
			Розділ описує функції, що мають бути доступні для користувача при роботі з системою. Розглянемо відповідність поставлених задач стандарту.
				
			%\begin{table}[h]
			%	\centering
				\begin{longtable}{x{0.25\textwidth}x{0.2\textwidth}x{0.45\textwidth}}
					\toprule
					Вимога & Незадоволені критерії & Пояснення \\
					\midrule
					\endhead
					\bottomrule
					\caption{Відповідність формулювань поставлених вимог стандарту IEEE 830}\label{tab:sect01comp}
					\endfoot
					%\caption[]{(продовження)}
					Управління адресами & Однозначність, повнота & Не визначений перелік дій, що розуміється під поняттям «управління адресами» (додавання, видалення, редагування тощо).\\
					Групування адрес та почтова розсилка & Однозначність, модифікованість & Дві вимоги згруповані в одну. Не визначені критерії для групування адрес (фізичне розташування, алфавітний порядок тощо). Не визначені деталі почтової розсилки.\\
					Підтримка 20 мов & Однозначність, повнота & Відсутній перелік мов, підтримка яких вимагається.\\
					Інтеграція з Gmail, Hotmail, Yahoo & Однозначність, повнота & Не визначені деталі інтеграції.\\
					Google map інтеграція & Коректність, однозначність, повнота & Невірно названий сервіс. Не визначені деталі інтеграції.\\
					%\bottomrule
					
				\end{longtable}
				%\caption{Відповідність формулювань поставлених вимог стандарту IEEE 830}\label{tab:sect01comp}
			%\end{table}
		\subsection{Розділ «Опис панелей»}
			\subsubsection{Верхня панель}
				Текстова специфікація вказує, що верхня панель має панель налаштування мови та панель інтеграції з сервісами Gmail, Hotmail та Yahoo. Панель налаштування мови за замовчуванням показує 4 мови: англійську, німецьку, російську та польську. Однак, на рисунку \ref{fig:langpanel}, показані 5 мов: арабська, китайська, іспанська, французька та російська. Отже, порушена умова несуперечливості вимог. 
				
				\begin{figure}[h]
					\centering
					\includegraphics{01.png}
					\caption{Макет панелі мов}\label{fig:langpanel}
				\end{figure}
				
			\subsubsection{Панель меню}
				Текстова специфікація вказує, що панель меню містить такі вкладки: «home page», «add new», «groups», «next birthdays». Однак, зображені на макеті (рис. \ref{fig:menupanel}) дані не співпадають.
				
				\begin{figure}[h]
					\centering
					\includegraphics{02.png}
					\caption{Макет панелі меню}\label{fig:menupanel}
				\end{figure}
				
			\subsubsection{Нижня панель}
				Розбіжності текстової специфікації з графічним макетом не виявлено.
			
		\subsection{Розділ «Опис вкладок»}
			\subsubsection{Домашня сторінка}
				Недоліків не виявлено.
			
			\subsubsection{Додавання нової сторінки}
				Текстова специфікація вказує обмеження довжини полів «First Name / Last Name» (дане поле відсутнє на макеті вкладки), «Telephone», «E-mail». Не вказані обмеження на формат введених даних для кожного з полів.
				
			\subsubsection{Сторінка редагування запису}
				Відсутні графічні позначення та вказівки з обробки випадку, коли користувач намагається повністю видалити значення обов'язкових полів.
			
			\subsubsection{Сторінка перегляду запису}
				Специфікація передбачає підтримку інтеграції з сервісами Gmail, Hotmail та Yahoo. Однак, графічне зображення механізма автентифікації не наведене. 
				
			\subsubsection{Сторінка управління групами}
				Специфікація передбачає можливість користувача видалити групу. Однак, не уточнений процес видалення групи: звідки саме видаляється група (з користувацького інтерфейсу, внутрішніх структур даних), що саме відбувається під час видалення групи.
				
			\subsubsection{Сторінка створення нової групи}
				Не вказані обмеження на довжину та формат даних у полях.
				
			\subsubsection{Сторінка редагування групи}
				Не вказані обмеження на довжину та формат даних у полях. 
				
			\subsubsection{Наступні дні народження}
				Недоліків не виявлено.
				
		\section{Вказівки для вдосконалення специфікації}
			Для виправлення наданої специфікації необхідно підійти до її розробки з початку і передбачити такі кроки:
			\begin{enumerate}
				\item Привести структуру документа до вигляду, рекомендованого стандартом IEEE 830: визначити мету, аудиторію та масштаб проекта; розробити загальний опис продукта; розділити систему на функціональні блоки; описати вимоги до інтерфейсів тощо.
				\item Ввести придатну метрику або порівняння для забезпечення процесів верифікації та перевірки коректності вимог. Використовувати ідентифікатори значимості або стійкості вимог.
				\item Перевіряти відповідність як вимог, так і документа, що їх описує, відповідним локальним та міжнародним стандартам (IEEE, ISO, ГОСТ, ДСТУ тощо).
				\item Дати точне визначення термінам, що використовуються при розробці та підтримці цільової системи.
				\item Виражати кожну з вимог окремо. Наприклад, «групування адрес» та «почтова розсилка» --- дві різні, незалежні вимоги.
				\item Виправити змістові помилки, переконатись у відповідності текстових вимог графічним макетам. Забезпечити наявність графічних макетів в усіх випадках, де вони потрібні (повідомлення про помилку, вікна авторизації тощо).
				\item Підвищити точність формулювання вимог. Наприклад, надавати повний перелік мов, якими повинна бути локалізована цільова система.
				\item Передбачити правильну обробку даних, що вводяться користувачем, та точну специфікацію змісту передбачених полів. Наприклад: «поле „Телефон“ має формат \texttt{+38~OPC~NNN~NN~NN}, де \texttt{OPC} --- код українського оператора, складається з трьох цифр; \texttt{N} --- цифра; кодування ASCII-сумісне». Описати вимоги до засобів для тестування, наприклад, набори тестових даних, регулярні вирази тощо.
			\end{enumerate}
			
			Виконання вищезазначених вимог дозволить спростити не лише роботу відповідальних людей, а й сам процес аналізу, розробки, введення та підтримки цільової системи.
			
		\section{Висновок}
			Було проведено тестування наданої специфікації вимог, яке показало повну невідповідність її структури та змісту навіть базовим принципам стандарту IEEE 380 «Recommended Practice for Software Requirements Specifications». 
			
			%Також, при формуванні функціональних вимог, необхідно виправити змістові помилки. Приділяти значно більше уваги до обробки даних, введених користувачем. Наприклад, навести перелік мов, якими повинна бути локалізована система.
			%Для цього необхідно:
			%\begin{enumerate}
			%	\item Визначити мету, терміни, аудиторію та масштаб проекту.
			%	\item Описати бачення продукта, його функціональність, класи та характеристики користувачів.
			%	\item Визначити та описати середовище роботи продукта. 
			%	\item Встановити обмеження та правила за допомогою наявних стандартів, або розробити нові.
			%	\item Написати документацію для користувачів.
			%	\item Описати залежності програми.
			%	\item Розподілити систему на функціональні блоки та описати їх.
			%\end{enumerate}
\end{document}