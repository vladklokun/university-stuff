\documentclass[a4paper,oneside,DIV=12,12pt]{scrartcl}

\usepackage{graphicx}

\usepackage{fontspec}
\setmainfont{STIX Two Text}
\setsansfont{PT Sans}
\setmonofont{PT Mono}

\usepackage{unicode-math}
\setmathfont{STIX Two Math}

\usepackage{microtype}

\usepackage{polyglossia}
\setmainlanguage{ukrainian}

%%%
\usepackage{booktabs}
\usepackage{longtable}
%%%

%%%
\usepackage{siunitx}
\sisetup{output-decimal-marker = {,}}
%%%

\renewcommand{\arraystretch}{1.2}

\begin{document}
	\begin{titlepage}
    \begin{center}
	Міністерство освіти і науки України\\
	Національний авіаційний університет\\
	Навчально-науковий інститут комп'ютерних інформаційних технологій\\
	Кафедра комп'ютеризованих систем управління

	\vspace{\fill}

	Лабораторна робота №6\\
	з дисципліни «Якість програмного забезпечення та тестування»\\
	на тему «Програмний код і його метрики»

	\vspace{\fill}
	
	\begin{flushright}
		Виконав:\\
		студент ННІКІТ СП-225\\
		Клокун Владислав\\
		Перевірила:\\
		Апенько Н. В.
	\end{flushright}

	Київ 2017

    \end{center}
    \end{titlepage}
	
	\section{Хід роботи}
		\subsection{Кількісні метрики коду}
			Кількісні метрики розробленої програми наведені у табл.~\ref{tab:quantitative-metrics}.
			
			\begin{longtable}[c]{lr}
				\toprule
					Метрика & Значення\\
				\midrule
				\endhead
				\bottomrule
				\caption{Кількісні метрики розробленої програми}
				\endfoot
				\label{tab:quantitative-metrics}
				
					Кількість рядків                 & $188$\\
					Кількість порожніх рядків        & $47$\\
					Кількість коментарів             & $36$\\
					Відсоток коментарів              & $26$\\
					Середнє число рядків для функцій & $7$\\
			\end{longtable}
		
		\subsection{Метрики Халстеда}
			Були обчислені метрики за Халстедом~(табл.~\ref{tab:software-metrics-halstead}). Нехай $\eta_1$~— кількість унікальних операторів, $\eta_2$~— кількість унікальних операндів, $N_1$~— загальна кількість операторів та $N_2$~— загальна кількість операндів, тоді за виміряними значеннями обчислимо інші метрики за формулами. Словник програми~$\eta$:
			\[
				\eta = \eta_1 + \eta_2 = 21 + 18 = 39.
			\]
			
			Довжина програми~$N$:
			\[
				N = N_1 + N_2 = 76 + 93 = 169.
			\]
			
			Обчислена довжина програми~$\hat{N}$:
			\[
				\hat{N} = \eta_1 \log_2 \eta_1 + \eta_2 \log_2 \eta_2
						= 21 \log_2 21 + 18 \log_2 18
						= \num{167,30}.
			\]
			
			Об'єм програми~$V$:
			\[
				V = N \cdot \log_2 \eta
				  = 169 \cdot \log_2 39
				  = \num{893,20}.
			\]
			
			Складність програми~$D$:
			\[
				D = \frac{\eta_1}{2} \cdot \frac{N_2}{\eta_2}
				  = \frac{21}{2} \cdot \frac{93}{18}
				  = \frac{217}{4}
				  = \num{54,25}.
			\]
			
			Зусилля~$E$:
			\[
				E = D \cdot V
				  = \num{54,25} \cdot \num{893,2}
				  = \num{48456,10}.
			\]
			
			\begin{longtable}[c]{llr}
					\toprule
						Метрика & Позначення & Значення\\
					\midrule
					\endhead
					\bottomrule
					\caption{Метрики розробленої програми за Халстедом}
					\endfoot
					\label{tab:software-metrics-halstead}
					
						Кількість унікальних операторів & $\eta_1$  & $21{,}00$\\
						Кількість унікальних операндів  & $\eta_2$  & $18{,}00$\\
						Загальна кількість операторів   & $N_1$     & $76{,}00$\\
						Загальна кількість операндів    & $N_2$     & $93{,}00$\\
						Словник програми                & $\eta$    & $39{,}00$\\
						Довжина програми                & $N$       & $169{,}00$\\
						Обчислена довжина програми      & $\hat{N}$ & $167{,}30$ \\
						Об'єм                           & $V$       & $893{,}20$ \\
						Складність                      & $D$       & $54{,}25$ \\
						Зусилля                         & $E$       & $48456{,}10$ \\
			\end{longtable}
		
\end{document}