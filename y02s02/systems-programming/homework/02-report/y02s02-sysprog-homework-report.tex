\documentclass[a4paper,oneside,BCOR=1cm,DIV=12,12pt,headings=normal]{scrartcl}

%%% Support for color
\usepackage{xcolor}
\definecolor{lightblue}{HTML}{03A9F4}
\definecolor{red}{HTML}{F44336}
%%%

%%% Links and hyperreferences
\usepackage{hyperref}
\hypersetup{
	colorlinks      = false,
	linkbordercolor = red,
	urlbordercolor  = lightblue,
	pdfborderstyle  = {/S/U/W 1.5},
}
%%%

%%% Graphics inclusion
\usepackage{graphicx}
%%%

%%% Font selection
\usepackage{fontspec}

\setromanfont{STIX Two Text}[
]

\setsansfont{Source Sans Pro}[
]

\setmonofont{Source Code Pro}[
	% Scale = 1.05,
]

% Used for suppressing polyglossia errors regarding missing cyrillic script
% \newfontfamily\cyrillicfonttt[
	% Script = Cyrillic,
	% Scale  = MatchUppercase,
% ]{Inconsolata}

\usepackage{unicode-math}
\setmathfont{STIX Two Math}

%%%

%%% Font settings for different KOMA Script elements
\setkomafont{pagenumber}{\rmfamily}
\setkomafont{disposition}{\rmfamily\bfseries}
%%%

%%% Typographic enhancements
\usepackage{microtype}
%%%

%%% Language-specific settings
\usepackage{polyglossia}
\setmainlanguage{ukrainian}
%%%

%%% List settings
\usepackage{enumitem}
\setlist[enumerate]{
	leftmargin = *,
}
%%%

%%% Captions
\usepackage{caption}
%%%

%%% Code listings
\usepackage{minted}

%%%

%%% Framing code listings
\usepackage{tcolorbox}
\tcbuselibrary{breakable}
\tcbuselibrary{minted}
\tcbuselibrary{skins}

\newtcbinputlisting[auto counter, list inside, number within = section]{\inputasm}[4][]{%
	minted language = nasm,
	minted style    = bw,
	minted options  = {
		linenos,
		tabsize = 4,
		breakbytokenanywhere,
		fontsize = \small,
	},
	%
	% empty,
	sharp corners,
	colframe         = black,
	colback          = black!0,
	leftrule         = 0em,
	rightrule        = 0em,
	toprule          = 0pt,
	bottomrule       = 0pt,
	titlerule        = 0.5pt,
	colbacktitle     = black!0,
	coltitle         = black,
	toptitle         = 0.3em,
	bottomtitle      = 0.1em,
	borderline north = {1pt}{0pt}{black},
	borderline south = {1pt}{0pt}{black},
	before skip      = \intextsep,
	after  skip      = \intextsep,
	title            = {Лістинг \thetcbcounter: #3},
	list entry       = {\protect\numberline{\thetcbcounter}#3},
	left = 0em,
	right = 0em,
	%
	listing file={#2},
	listing only,
	breakable,
	%
	label = {#4},
	%
	#1
}

% Customize minted line numbers
\renewcommand{\theFancyVerbLine}{\ttfamily\scriptsize\arabic{FancyVerbLine}}

%%%

%%% Count figures within sections
\usepackage{chngcntr}
\counterwithin{figure}{section}
%%%

%%% Custom commands
% Print filenames
\newcommand{\filename}[1]{\texttt{#1}}

% Print command names
\newcommand{\progname}[1]{\texttt{#1}}

% Print function names
\newcommand{\funcname}[1]{\texttt{#1}}

% Typeset all caps
\newcommand{\allcaps}[1]{{\addfontfeatures{LetterSpace = 3}#1}}

%%%

\setlength{\emergencystretch}{1em}


\begin{document}
	\begin{titlepage}
	\centering
		Міністерство освіти і науки України\\
		Національний авіаційний університет\\
		Навчально-науковий інститут комп'ютерних інформаційних технологій\\
		Кафедра комп'ютеризованих систем управління

		\vspace*{\fill}

		Домашня робота\\
		з дисципліни «Системне програмування»\\
		Варіант №24

		\vspace*{\fill}
		
		\begin{flushright}
			Виконав:\\
			студент \allcaps{ННІКІТ} \allcaps{СП}-225\\
			Клокун В.\,Д.\\
			Перевірив:\\
			Артамонов Є.\,Б.
		\end{flushright}

		Київ 2018
    \end{titlepage}

	\section{Завдання}
		Робота складається з 2~завдань, які повинні знаходитись в одному виконуваному файлі, створеного за допомогою мови асемблера. Вибір завдань виконується за допомогою діалогового меню. Для виконання домашньої роботи згідно з номером варіанта були поставлені такі завдання:
		\begin{enumerate}
			\item Дані числа~$a$, $b$ $(0 < a < b)$ і набір з 10~елементів. Знайти мінімальний і~максимальний з елементів, що містяться в інтервалі~$(a, b)$. Якщо необхідні елементи відсутні, то вивести~$-1$.
			\item Відображення вмісту директорії~(аналог \progname{DIR}). Можливість виводу структури каталогів у вигляді дерева.
		\end{enumerate}

	\section{Хід роботи}
		Початковий код необхідної програми був розроблений для платформи macOS з~64-бітною розрядністю на діалекті мови асемблера \allcaps{NASM} з використанням засобів інфраструктури~\allcaps{LLVM}.

		Початковий код необхідного виконуваного файлу був розділений на 4 файли: \filename{arrload.asm}, \filename{main.asm}, \filename{p01findminmax.asm} та~\filename{p02dirtree.asm}.
		Файл \filename{arrload.asm}~(ліст.~\ref{lst:01-arrload}) містить початковий код функції, яка завантажує, обробляє та виводить початковий набір даних із заданого файлу в масив.

		\inputasm{../01-solution/arrload.asm}{Завантаження початкових даних для виконання завдання~1}{lst:01-arrload}

		Файл \filename{main.asm}~(ліст.~\ref{lst:02-main}) містить початковий код основної функції програми, яка відповідає за взаємодію з користувачем, виклик функцій для виконання завдань, підготовку та вивід даних, а також завершення роботи програми.
		\inputasm{../01-solution/main.asm}{Основна частина програми}{lst:02-main}

		Файл \filename{p01findminmax.asm}~(ліст.~\ref{lst:03-p01findminmax}) містить початковий код, що описує роботу функцій \funcname{findmin()} та~\funcname{findmax()}. Ці функції відповідають за пошук у~початковому наборі даних мінімального та максимального елементів, які містяться у заданому інтервалі.
		\inputasm{../01-solution/p01findminmax.asm}{Функції для пошуку мінімального та максимального елементів, що містяться в інтервалі~$(a, b)$}{lst:03-p01findminmax}

		Файл \filename{p02dirtree.asm}~(ліст.~\ref{lst:04-p02dirtree}) містить початковий код функцій \funcname{dir()} та~\funcname{tree()}, які відповідають за вивід вмісту директорії та структури каталогу у вигляді дерева.
		\inputasm{../01-solution/p02dirtree.asm}{Функції для виводу змісту директорії та структури каталогів у вигляді дерева}{lst:04-p02dirtree}

	\section{Висновок}
		Під час виконання домашньої роботи ми закріпили розуміння написання простих програм на мові асемблера та застосування простих і складних алгоритмів, а також використання спеціальних команд управління.

\end{document}

