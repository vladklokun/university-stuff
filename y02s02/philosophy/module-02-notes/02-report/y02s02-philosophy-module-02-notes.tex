\documentclass[a5paper,oneside,DIV=12,12pt,headings=small]{scrartcl}

%%% Length calculations
\usepackage{calc}
%%%

%%% Support for color
\usepackage{xcolor}
\definecolor{lightblue}{HTML}{03A9F4}
\definecolor{red}{HTML}{F44336}
%%%

%%% Graphics inclusion
\usepackage{graphicx}
%%%

%%% Font selection
\usepackage{fontspec}

\setromanfont{PT Serif}[
]

\setsansfont{PT Sans}[
]

\setmonofont{PT Mono}[
]

\usepackage{amsmath,unicode-math}
\setmathfont{STIX Two Math}

%%%

%%% Font settings for different KOMA Script elements
\setkomafont{pagenumber}{\rmfamily}
\setkomafont{disposition}{\rmfamily\bfseries}
%%%

%%% Typographic enhancements
\usepackage{microtype}
%%%

%%% Language-specific settings
\usepackage{polyglossia}
\setmainlanguage{ukrainian}
%%%

%%% List settings
\usepackage{enumitem}
\setlist[enumerate]{
	leftmargin = *,
	label*={\arabic*.},
}
\renewcommand{\labelitemi}{—}
%%%

%%% Captions
\usepackage{caption}
\usepackage{subcaption}

\DeclareCaptionLabelFormat{closing}{#2)}
\captionsetup[subtable]{labelformat = closing}
\captionsetup[subfigure]{labelformat = closing, position = auto}
%%%

%%% Tables
\usepackage{booktabs}
\usepackage{longtable}

\usepackage{multirow}

\usepackage{array}
\newcolumntype{v}[1]{>{\raggedright\arraybackslash\hspace{0pt}}p{#1}}
\newcolumntype{b}[1]{>{\centering\arraybackslash\hspace{0pt}}p{#1}}
\newcolumntype{n}[1]{>{\raggedleft\arraybackslash\hspace{0pt}}p{#1}}
%%%

%%% Links and hyperreferences
\usepackage{hyperref}
\hypersetup{
	colorlinks      = false,
	linkbordercolor = red,
	urlbordercolor  = lightblue,
	pdfborderstyle  = {/S/U/W 1.5},
}
%%%

%%%
\usepackage{ragged2e}
%%%

%%% All caps
\newcommand{\allcaps}[1]{{\addfontfeatures{LetterSpace = 3}#1}}
%%%

\setlength\emergencystretch{1em}


\begin{document}
	\tableofcontents
	
	% \setlength{\RaggedRightParindent}{1em}
	% \RaggedRight
	\setcounter{section}{30} % Start section numbering from 31 to follow order from reference material
	% 31. Проблема конечності людського існування. Сенс життя людини.
	\section{Проблема конечності людського існування. Сенс життя людини}
		Важливим аспектом філософського осмислення людини є врахування її природного руху замкненим колом: народження~— життя~— смерть. З~давніх давен людина намагалась якось осягнути цей вічний круговорот життя. У~чому смисл природного процесу народження, розвитку, зрілості, старіння і~смерті як людини, так і~будь-якого іншого організму? Це питання виникає як намагання виправдати свою присутність на Землі, свою долю й~призначення. Знайшовши таке виправдання, людина може змиритися з~думкою про скінченність індивідуального буття. Таємниця людського існування полягає не в~тому, щоб тільки жити (існувати), а~й~у~тому, як і~для чого (чи для кого) жити. Отже, в~чому полягає сенс життя?

		Сенс життя~— це поняття, яке відбиває постійне прагнення людини співвідносити свої вчинки із системою суспільних цінностей, з~вищим благом, щоб у~такий спосіб діставати можливість виправдовувати себе у~своїх власних очах, в~очах інших людей чи перед якимось авторитетом, Богом. Інакше кажучи, це пояснення собі й~іншим, для чого ти живеш.

		Сенс життя кожної людини унікальний і~неповторний, як і~її життя. Людина завжди вільна у~виборі сенсу і~в його реалізації. Але свободу не можна ототожнювати зі свавіллям. її слід сприймати з~точки зору відповідальності. Людина відповідає за вірно знайдений і~реалізований сенс свого життя, життєвих ситуацій, що в~них вона потрапляє. Людина повинна йти за своїм покликанням, у~якому життя набуває сенсу. Відчути і~знайти своє покликання їй допомагає самопізнання, відповідальність за реалізацію свого призначення, що на Землі допомагає узгодити універсальні життєві цінності з~конкретними життєвими ситуаціями.

		З точки зору змісту вищого блага вирізняють такі типи обгрунтування життя: гедонізм, аскетизм, евдемонізм, корпоративізм, прагматизм, перфекціоналізм, гуманізм.

		Схематично і~досить умовно можна окреслити такі варіанти вирішення проблеми сенсу життя в~історії людської культури:

		\begin{enumerate}
			\item Сенс життя споконвічно існує в~глибинах самого життя. Для цього варіанта характерне релігійне тлумачення життя. Єдине, що робить осмисленим життя і~має для людини абсолютний сенс, є не що інше, як активна співучасть у~Боголюдському житті. Не перероблення світу на основах добра, а~вирощування в~собі субстанціального добра, зусилля жити з~Христом і~у Христі. Бог створив людину за своїм образом і~подобою. І~ми своїм життям повинні проявити його, бо емпіричне життя світу, як писав Семен Франк, безглузде, як безладно вирвані з~книги сторінки.
			\item Сенс життя перебуває за межами життя. Його можна назвати «життям заради інших людей». Для людини життя стає осмисленим, коли вона служить інтересам родини, нації, суспільства, коли вона живе заради щастя прийдешніх поколінь. їй небайдуже, що вона залишить після себе. Недаремно прожити життя~— це і~продовжитися у~своїх нащадках, і~передати їм результати своєї матеріальної і~духовної діяльності. Але на цьому шляху існує небезпека опинитися в~ситуації, коли все твоє неповторне життя перетворюється на засіб для створення якоїсь ідеї чи ідеалу (це може бути ідея комунізму, «світлого майбутнього» тощо). Якщо така позиція не пов'\-я\-за\-на з~духовною еволюцією людської особистості, людина стає на шлях фанатизму (історія знає безліч варіантів і~класового, і~національного, і~релігійного фанатизму).
			\item Сенс життя створюється самим суб'\-єктом. Цей варіант можна розуміти як «життя заради життя». Його фундатором був давньогрецький філософ Епікур. Жити потрібно так, вважав філософ, щоб насолоджуватися життям, отримувати задоволення від життєвих благ і~не думати про смерть. Цінність епікурейської позиції полягає в~тому, що вона застерігає нас від ситуації, за якої пошук сенсу життя відсуває на другий план саме життя. Життя саме по собі є цінністю, рідкісним дарунком, і~людині до нього слід ставитися із вдячністю і~любов'ю. Адже їй дана можливість переживати неповторність власного існування в~усіх його проявах~— від радощів, злетів і~перемог до падіння, відчаю і~страждань. Разом з~тим епікурейське ставлення до життя, якщо воно позбавлене відповідальності за цей дарунок, утверджує в~людині егоїстичну позицію «життя заради себе» і~веде до втрати відчуття його повноцінності.
		\end{enumerate}

% 32. Постановка проблеми свідомості в~історії філософії.
	\section{Постановка проблеми свідомості в~історії філософії.}
		Проблема свідомості~— одна з~найважливіших і~загадкових. Вона як філософська категорія має складну і~суперечливу історію, характеризується багатозначністю підходів і~тлумачень. Це свідчить, водночас, про пильну увагу філософів до проблеми свідомості, актуальність якої пояснюється тим, що:
		\begin{itemize}
			\item без з'я\-су\-ван\-ня природи людської свідомості не можна визначити місце і~роль людини в~світі, особливості її взаємовідносин з~навколишньою дійсністю;
			\item питання про сутність свідомості, про її зв'я\-зок з~буттям є одним із найважливіших світоглядних і~методологічних аспектів кожного філософського напрямку;
			\item всі проблеми сучасної суспільної практики органічно пов'\-я\-за\-ні з~дослідженнями свідомості. Це стосується гострих і~актуальних проблем суспільного розвитку, взаємодії людини і~техніки, відношення науково-технічного прогресу та~природи, проблем виховання, спілкування людей тощо.
		\end{itemize}

		Одним із важливих філософських питань завжди було і~залишається питання про зв'я\-зки між свідомістю й~буттям. Матеріалістична позиція, виходячи з~примату буття над свідомістю, не відкидає того, що людська діяльність завжди передбачає свідомість, що вона «пронизана» свідомістю. Буття виступає як більш широка система, всередині якої свідомість є специфічною умовою, засобом, «механізмом» вписування людини в~цю цілісну систему буття.
		
		Отже, вторинність людської свідомості стосовно людського буття виступає як вторинність елементу стосовно системи, вторинність умови й~передумови відносно цілісної структури діяльності.

		Свідомість~— це особлива форма відображення, регуляції та~управління ставленням людей до навколишньої дійсності, до самих себе та~своїх способів спілкування, які виникають і~розвиваються на основі практично-перетворювальної діяльності.

		Матеріалістична філософія, виходячи з~принципу матеріальної єдності світу, органічної включеності людини в~цілісність живої і~неживої природи, розглядає свідомість як властивість високоорганізованої матерії~— мозку. Тому необхідно простежити генетичні витоки свідомості саме в~тих формах організації матерії, які передують людині в~процесі її еволюції.

		Важливішою передумовою такого дослідження є аналіз відображення форми матеріальної взаємодії, на основі якої виникають психіка та~свідомість.

		Відображення~— це здатність матеріальних явищ, предметів, систем відтворювати у~своїх властивостях особливості інших явищ, предметів, систем в~процесі взаємодії з~ними.

		Підхід у~дослідженні відображення має бути послідовно генетичним, історичним. Слід розглядати розвиток конкретних форм та~видів відображення, їх ускладнення, вдосконалення в~процесі розвитку форм руху матерії.

		Характер відображення залежить від:
		\begin{enumerate}
			\item Природи впливу.
			\item Особливостей, якісної специфіки тіла, що відображає. Тобто поява більш складних матеріальних об'\-єк\-тів зумовлює появу нових, більш складних форм відображення.
		\end{enumerate}

		Так, найбільш простим матеріальним об'\-єк\-там відповідає фізична форма відображення. З~появою білкових тіл виникає біологічна форма відображення~— чуттєвість.

		Виникнення живої речовини супроводжувалося появою ще більш складної біологічної форми відображення~— подразливості. Це властивість найпростішої живої речовини відповідати на вплив зовнішнього світу (поворот голівки соняшника за сонцем протягом дня).

		Більш складною властивістю живої речовини є відчуття, що виникає на основі ускладнення подразливості. Відчуття~— це певний внутрішній стан живої речовини, який полягає в~мобілізації можливостей організму, його ресурсів для здійснення реальних дій, необхідних для задоволення потреб організму.

		Форми відображення в~живій природі розвиваються в~напрямку зростання ролі цього внутрішнього стану мобілізації, налаштування організму на роз\-в'я\-зан\-ня життєвих задач.

		Виникнення відчуттів пов'\-я\-за\-но з~формуванням особливої матеріальної структури, що відповідає за відображення,~— нервової тканини, яка поступово розвивається у~складні нервові системи.

		Спочатку примітивні нервові клітини під впливом зовнішнього середовища спеціалізуються, відбувається розподіл функцій між окремими групами нервових клітин. Потім виникає центральна нервова система, тобто дії організму регулюються з~одного центру~— головного мозку.

		Відчуття є елементарною формою психічного. Більш складною формою є сприйняття та~уявлення. Сприйняття~— це синтез відчуттів, отриманих від різних органів чуття. Уявлення~— це здатність зберігати образ предмета в~мозку не лише тоді, коли предмет безпосередньо впливає на органи чуття, а~й~тоді, коли цього впливу немає.

		Крім відчуттів, які дають безпосереднє знання про світ, людині властива вища форма прояву свідомості~— понятійне мислення. Лише людині властиві вищі психічні функції~— мислення, пам'\-ять, воля, емоції.

		Свідомості відповідає специфічно людський спосіб буття в~світі, взаємодія зі світом. Цим способом є практика, тобто практично-перетворювальне ставлення до дійсності, за допомогою якого людина створює своє «неорганічне тіло», «другу природу» і~взагалі творить культуру. Формування культури на основі практики спричиняє виникнення свідомості. В останній з~необхідністю фіксуються навички, способи, норми практичної діяльності. Оскільки ці навички, способи та~норми мають суспільну природу, тобто виникають, реалізуються та~відтворюються в~сумісній, колективній діяльності, то і~форми відображення, в~яких вони закріплюються, завжди мають соціальний характер.

		Навички, способи, норми практичної діяльності завжди передбачають певне спілкування людей, їх кооперацію. Звідси~— людська свідомість має суспільну природу.

% 33. Роль праці, спілкування й~мовлення у~формуванні та~розвитку свідомості. 
	\section{Роль праці, спілкування й~мовлення у~формуванні та~розвитку свідомості}
		Здатність людини творити суб'\-єк\-тив\-ний образ об'\-єк\-тив\-но\-го світу, тобто бути свідомою істотою, з~необхідністю спонукає її до пізнання і~перетворення світу — до діяльності. Особливим проявом діяльності є праця. Саме праця, з~погляду еволюційної теорії Ч.\,Дарвіна, є однією з~найважливіших передумов виникнення свідомості у~людини. Трудова діяльність людини мала два вирішальні наслідки:
		\begin{enumerate}
			\item Організм предків людини почав пристосовуватися не просто до умов середовища, а~до трудової діяльності. Специфічні особливості фізичної організації людської істоти~— пряме ходіння, диференціація функцій передніх і~задніх кінцівок, розвиток руки (згадаймо відомий вислів Гегеля~— «знаряддя знарядь»), головного мозку виробились у~процесі тривалого пристосування організму до виконання трудових операцій.
			\item Праця, яка з~самого початку була спільною діяльністю, стимулювала виникнення і~розвиток видів спілкування, членороздільної мови як засобу спілкування, нагромадження і~передавання трудового, соціального досвіду.
		\end{enumerate}

		Людська праця відрізняється від діяльності навіть найрозвинутіших тварин тим, що являє собою активний вплив людини на природу, а~не просте пристосування до неї, характерне для тварин. Вона передбачає систематичне використання і, найголовніше, виробництво знарядь виробництва. Праця є цілеспрямованою свідомою діяльністю людини, вона з~самого початку має суспільний характер і~немислима поза суспільством.

		Отже, за еволюційною теорією, хоча вона, зараз уточнюється й~школи ставиться під сумнів, думка і~руки, розум і~знаряддя праці почали творити нову істоту, яка отримала назву розумної, тобто людини. Цього факту не заперечують ні космічні гіпотези походження людини, ні релігійна (про що вже була мова), бо всюди є акт цілеспрямованої діяльності по створенню благ, необхідних для задоволення людських потреб.

		Оскільки ж праця органічно пов'\-я\-за\-на із застосуванням мускульної сили й~сили розуму, то відповідно до цього виділяються два основні види праці — фізична і~розумова.

		Цей поділ праці виник ще за доби рабовласництва і~зберігся донині.

		Унікальність свідомості як особливої здатності людини творити образи оточуючого її світу виявляється не тільки у~характері її функціонування і~розвитку, а~й~у~способі виникнення самої свідомості. Однією із основ виникнення феномена свідомості є спілкування і~мова.

		Говорячи про спілкування, ми насамперед, підкреслюємо, що цей феномен притаманний лише людині, воно виникає як особливий стан відношення, вза\-є\-мо\-зв'яз\-ку людей. У~найширшому розумінні слова, спілкування — це процес вза\-є\-мо\-зв'яз\-ку, взаємовідношення, взаємодії між людьми, людськими спільнотами, у~якому відбувається обмін діяльністю, інформацією, досвідом, здібностями, уміннями та~навичками, а~також результатами діяльності, тобто процес передачі, опрацювання і~використання інформації в~широкому розумінні. Спілкування~— одна з~необхідних і~загальних умов формування і~розвитку особистості і~суспільства. Адже основою розвитку свідомості нині, як і~раніше, залишається живий контакт, який містить у~собі все багатство людських стосунків і~дає змогу засвоїти не тільки думки інших людей, а~й~почуття, і~не лише побачити, а~й~відчути чужу суб'\-єктивність, ЇЇ наявність, отримати не тільки знання, а~й~моральні навички.

		Особливим способом, засобом спілкування є мова, особливістю якої є те, що вона водночас і~універсальний засіб спілкування, і~найдосконаліша форма спілкування (адже у~процесі безпосереднього спілкування величезну роль відіграють міміка, жести, посмішка тощо). Мова~— це такий засіб спілкування, особливим та~основним елементом якого є слово, речення. Але найголовніша функція мови, причому двоєдина,~— вона є знаряддям мислення і~способом спілкування. Загальновідомо, що слово~— велика сила, здатна і~«вбити» людину, і~«піднести» її. Мова~— це вся система культури, це спеціалізована, інформаційно-знакова діяльність; це безпосередня діяльність думки, мислення, свідомості. Одиниця ж мови~— слово~— є єдністю значення і~звучання, адже ми розрізняємо усну, писемну та~внутрішню мову. Мова відображає стан розвитку інтелекту людини, її душу. Мова~— це та~сила, за допомогою якої людська спільність не тільки здатна саморозвиватись, а~й~плекає, творить свою власну культуру, організовує життя у~ціннісних межах його існування. Слово є дороговказом не тільки до оптимізму, воно може сіяти недовіру, песимізм, вести людину до деградації її як особи, так і~як істоти суспільної. Тому людина і~людство з~необхідністю мають пам'\-я\-та\-ти, що мова~— це оберіг, творець людини, тому її необхідно плекати.

		Отже, варто зупинитися окремо на тезі, що мова є засобом розуміння світу. Тобто спосіб життя людей зумовив формування мови, котра в~свою чергу зумовлює формування певного типу свідомості (те, що ми називаємо ментальністю), яка знову ж таки відображає світ відповідним набором понять.

% 34. Взаємозв'язок діяльності, мислення та~мови.
	\section{Взаємозв'язок діяльності, мислення та~мови}
		Мислення~— узагальнене й~абстрактне відображення мозком людини явищ дійсності в~поняттях, судженнях й~умовиводах. Мисленню властиві такі процеси, як абстракція, узагальнення, аналіз, синтез, постановка певних завдань і~знаходження шляхів їх роз\-в'я\-зан\-ня, висунення гіпотез тощо.

		Щодо мови і~мислення в~науці існували два протилежні й~неправильні погляди~— ототожнення мови й~мислення (Д.\,Шлейєрмахер, Й.\,Г.\,Гаман) і~відривання мови від мислення (Ф.\,Е.\,Бенеке). Представники першої точки зору вважали, що мова~— це всього лише форма мислення. А~оскільки відомо, що кожне явище має форму і~зміст, то мова й~мислення разом становлять один об'\-єкт. Представники протилежного погляду стверджували, що мова й~мислення між собою абсолютно не пов'\-я\-за\-ні, мислення не залежить від мови, воно здійснюється в~інших формах.

		Насправді мова й~мислення тісно пов'\-я\-за\-ні між собою, але цей зв'я\-зок не є простим, прямолінійним, тому єдність мови та~мислення не є їх тотожністю. З~одного боку, немає слова, словосполучення, речення, які б не виражали думки. Однак мова~— це не мислення, а~лише одне з~найголовніших знарядь, інструментів мислення. З~іншого боку, існують й~інші форми мислення, які здійснюються невербально (несловесно).

		Загалом існує три типи мислення:
		\begin{enumerate}
			\item Чуттєво-образне (наочно-образне).
			\item Технічне (практично-дійове).
			\item Поняттєве (словесно-логічне).
		\end{enumerate}

		Чуттєво-образне мислення~— мислення конкретними образами, картинами (в мозку прокручується своєрідний фільм). Воно притаманне не тільки людині, а~й~вищим тваринам~— собакам, кішкам, мавпам тощо. Уявіть собі таку картину. Увечері господиня залишила незакритою сметану на кухні. Вранці приходить на кухню приготувати сніданок і~бачить, як кішка доїдає сметану. Розлютившись, господиня хапає кішку і~вдаряє її. Наступного ранку, побачивши господиню на кухні, кішка стрибає зі столу й~ховається за буфетом, хоч на цей раз вона нічого поганого не вчинила. Перед нею виник учорашній образ розлюченої господині.

		Чуттєво-образне мислення властиве всім людям, а~особливо представникам творчих професій: письменникам, художникам, артистам, режисерам, балетмейстерам тощо. Існування цього типу мислення переконливо заперечує поширену донедавна думку, що мислення протікає тільки в~словесній формі. Коли художник-мариніст І.\,Айвазовський писав картину «Дев'\-я\-тий вал», він не обмірковував її за допомогою слів, а~переніс на полотно образ, який визрів у~його романтичній уяві: неосяжна велич й~буйна могутність морської стихії, вогненні присмерки, грайливе на хвилях місячне світло, відвага людей, які мужньо борються з~розбурханою стихією.

		Технічне (практично-дійове) мислення~— здійснюється без участі мови. Воно, як і~наочно-образне мислення, властиве вищим тваринам і~людині. Чи не першим на цей тип мислення звернув увагу німецький філософ Г.\,Гегель, який, зокрема, вказав на те, що безпосередня трудова діяльність, скажімо праця каменяра, обо\-в'яз\-ко\-во вимагає мислення. Трудові дії людини, як елементарні дії вищих тварин, осмислені. Так, наприклад, якщо високо підвісити банан, то мавпа, щоб його дістати, ставить ящик, бере в~передні кінцівки палку, вилазить на ящик і~збиває банан.

		Практично-дійове мислення притаманне всім людям, але найбільшою мірою спеціалістам технічних професій. Інколи інженеру легше створити нову машину, ніж захистити свій проект (важко підбирати потрібні слова та~вирази). А~від тих, хто працює на ком\-п'ю\-те\-рах, можна почути, що вони мислять машинною (ком\-п'ю\-тер\-ною) мовою.

		Поняттєве мислення~— здійснюється за допомогою мови. Абстрактні поняття про любов і~ненависть, життя і~смерть, мову й~мислення, науку й~культуру, теорію відносності А.\,Ейнштейна чи гіпотезу вроджених граматичних структур Н.\,Хомського осмислити і~передати без участі мови неможливо. Однак слід зауважити, що людина і~в цьому випадку не завжди мислить вслух, тобто вимовляючи слова. Частіше люди мислять за допомогою внутрішнього мовлення («про себе»), яке відрізняється від зовнішнього тим, що є згорнутим, зредукованим.

		У людей усі типи мислення переплітаються, але превалює поняттєве, тобто основним знаряддям мислення є мова.

		Про те, що мова і~мислення не тотожні, засвідчують й~інші факти. Так, зокрема, мислення не має властивостей матерії, воно є ідеальним, тоді як мова має ідеальний (семантика) і~матеріальний (звукова оболонка слів, матеріально виражені граматичні форми тощо) аспекти. Будова мови і~будова мислення не збігаються. Мова і~мислення оперують різними одиницями (фонема, морфема, слово, речення~— поняття, судження, умовивід). Щоправда, багатовіковий процес оформлення й~вираження думок за допомогою мови зумовив розвиток низки граматичних категорій, які частково збігаються з~деякими категоріями мислення (підмет~— суб'\-єкт, присудок~— предикат, додаток~— об'\-єкт, означення~— атрибут).

		Нарешті, ще одним вагомим доказом того, що мова і~мислення нетотожні явища, є їх неодночасне виникнення. Історично мислення виникло раніше, воно передує мові. Саме тому й~функції мови щодо мислення змінювалися. Спершу мова лише включалася в~процеси мислення, доповнювала практично-дійове і~наочно-образне мислення. З~часом вплив мови на мислення зростав і~мова стала основним знаряддям мислення.

% 35. Структура свідомості. Свідомість і~несвідоме. Самосвідомість.
	\section{Структура свідомості. Свідомість і~несвідоме. Самосвідомість}
		Свідомість має надзвичайно складну структуру. Фахівці не мають відносно неї одностайної думки. Це пов'\-я\-за\-но зі складністю такого явища, як свідомість, яка взагалі вирізняється складністю, важкодоступністю наукового вивчення. Багато аспектів, властивостей свідомості ми ще не знаємо, спостерігається дискусійність, навіть протилежність поглядів відносно механізмів, властивостей, функцій, структури свідомості.

		Можна виділити такі рівні свідомості та~їх елементи:
		\begin{enumerate}
			\item Базовим і~найбільш давнім рівнем свідомості є чут\-тє\-во-афек\-тив\-ний пласт, до якого належать:
			\begin{enumerate}
				\item Відчуття~— відображення в~мозкові окремих властивостей предметів та~явищ об'\-єк\-тив\-но\-го світу, що безпосередньо діють на наші органи чуттів.
				\item Сприйняття~— образ предмета в~цілому, який не зводиться до суми властивостей та~сторін.
				\item Уявлення~— конкретні образи таких предметів чи явищ, які в~певний момент не викликають у~нас відчуттів, але які раніше діяли на органи чуттів.
				\item Різного роду афекти, тобто сильні мимовільні реакції людини на зовнішні подразники (гнів, лють, жах, відчай, раптова велика радість).
			\end{enumerate}

			\item Ціннісно-вольовий рівень, до якого належать:
			\begin{enumerate}
				\item Воля~— здатність людини ставити перед собою мету і~мобілізовувати себе для її досягнення.
				\item Емоції~— ціннісно-забарвлені реакції людини на зовнішній вплив. Сюди можна віднести мотиви, інтереси, потреби особи в~єдності зі здатностями у~досягненні мети.
			\end{enumerate}

			\item Абстрактно-логічне мислення. Це найважливіший пласт свідомості, який виступає в~таких формах:
			\begin{enumerate}
				\item Поняття~— відображення в~мисленні загальних, найбільш суттєвих ознак предметів, явищ об'\-єк\-тив\-ної дійсності, їх внутрішніх, вирішальних зв'\-яз\-ків і~законів.
				\item Судження~— форма думки, в~якій відображається наявність чи відсутність у~предметів і~явищ яких-небудь ознак і~зв'\-яз\-ків.
				\item Умовивід~— форма мислення, коли з~одного чи кількох суджень виводиться нове судження, в~якому міститься нове знання про предмети та~явища.
				\item Різні логічні операції.
			\end{enumerate}

			\item Необхідним компонентом свідомості можна вважати самосвідомість і~рефлексію:
			\begin{enumerate}
				\item Самосвідомість~— це виділення себе, ставлення до себе, оцінювання своїх можливостей, які є необхідною складовою будь-якої свідомості.
				\item Рефлексія~— це така форма свідомості, коли ті чи інші явища свідомості стають предметом спеціальної аналітичної діяльності суб'\-єкта.
			\end{enumerate}
			\end{enumerate}

		Частина психічних процесів і~явищ усвідомлюється людиною, але існує велика кількість психічних процесів і~явищ, перебіг і~вияв яких не відображається у~свідомості людини. Ці процеси належать до групи неусвідомлюваних процесів, або несвідомого.
		
		Неусвідомлювані процеси почали активно вивчатися на початку XX століття. Перші дослідження цієї проблеми показали, що проблема несвідомого досить широка, а~свідомість по відношенню до несвідомого~— це лише вершина айсберга.

		Усі неусвідомлювані психічні процеси прийнято поділяти на три класи:
		\begin{enumerate}
			\item Неусвідомлювані механізми свідомих дій.
			\item Неусвідомлювані спонукання свідомих дій.
			\item «Надсвідомі» процеси (Ю.\,Б.\,Гіппенрейтер).
		\end{enumerate}

		Свідомість має бути програмою, що управляє людською діяльністю, а~також внутрішнім життям людини. Такі умови забезпечуються завдяки певним характерним рисам, властивим свідомості та~функціям, які вона виконує.

		Однією з~важливих рис свідомості є її універсальність. Це означає, що у~свідомості можуть відображатися будь-які властивості предметів, що так чи інакше залучаються до діяльності. Це відбувається тому, що праця і~спілкування «змушують» предмети подати себе багатогранно в~думках людини. Відомий приклад: орел бачить набагато далі, ніж бачить людина, але людське око помічає в~речах значно більше, ніж око орла.

		Свідомості властива об'\-єк\-тив\-ність. Тобто свідомість відображає предмети такими, якими вони є в~дійсності. Тварина бачить у~предметі лише об'\-єкт потреби або небезпеки. Людина бачить речі незалежно від тієї чи іншої потреби.

		Для свідомості характерний нерозривний зв'я\-зок із мовою. Мова виконує важливі функції:
		\begin{enumerate}
			\item Збереження знань (акумулятивна функція). 
			\item Зв'я\-зок між людьми, передача досвіду (комунікативна функція). 
			\item Засіб вираження думки, знань (експресивна функція).
		\end{enumerate}

		Свідомість містить чітко виражене цілеспрямоване відображення дійсності. Їй властиве цілепокладання. Перед тим, як щось зробити, людина створює ідеальний проект майбутнього результату і~розробляє план дій. Матеріальне виробництво продукує речі, предмети. Духовне~— їх проекти.

		Людина активно ставиться до дійсності. Вона оцінює ситуацію, фіксує своє ставлення до дійсності, виділяє себе як суб'\-єкта такого ставлення. Активне ставлення до дійсності~— характерна риса свідомості як специфічної форми відображення.

		Активність як не\-від'\-єм\-на риса свідомості тісно пов'\-я\-за\-на з~такою властивістю свідомості, як творчість. Адже універсальне й~об'\-єк\-тив\-не відображення дійсності передбачає не просто активне ставлення до неї, а~творчо-активне, тобто перетворювальне, а~не руйнівне ставлення. Людина прагне створювати нове. А~для цього потрібні нові ідеї, конструктивне зображення того, чого реально ще немає, але може бути створено відповідно пізнаним об'\-єк\-тив\-ним законам цієї реальності.

		Вже зазначалося, що людина активно ставиться до дійсності. Активність передбачає оцінювання не лише ситуації навколишньої дійсності, а~й~аналіз носія свідомості, тобто людини, виділення суб'\-єктом самого себе як носія певної активної позиції відносно світу. Все це проявляється в~самосвідомості. Отже, самосвідомість~— це виділення себе, ставлення до себе оцінювання своїх можливостей, які є необхідною складовою будь-якої свідомості.

		Формування самосвідомості має певні ступені та~форми. Перший ступінь~— самопочуття. Самопочуття~— це елементарне усвідомлення свого тіла та~його гармонійне поєднання зі світом оточуючих речей та~людей. Щоб правильно орієнтуватися в~світі речей, необхідно насамперед усвідомлювати, виділяти ті зміни, які відбуваються з~тілом людини на відміну від того, що відбувається у~зовнішньому світі. Якби цього і~не відбувалося, то людина не змогла б розрізнити процеси, що відбуваються в~самій дійсності від суб'\-єк\-тив\-них процесів. Наприклад, людина не змогла б зрозуміти, чи предмет наближається чи віддаляється від неї.

		Усвідомлення себе як такого, що належить до тієї чи іншої спільності людей, тієї чи іншої культури і~соціальної групи~— є більш високим рівнем самосвідомості.

		Виникнення свідомості «Я» як зовсім особливого утворення, схожого на «Я» інших людей і~разом із тим у~чомусь унікального, неповторного~— це найвищий рівень розвитку самосвідомості. Завдяки йому людина може здійснювати вільні дії і~нести за них відповідальність, що в~свою чергу вимагає самоконтролю та~оцінювання своїх дій.

		У поняття самосвідомості входить, як уже говорилося, також самооцінка, самоконтроль. Самосвідомість передбачає співставлення себе з~певним ідеалом «Я», що формується і~вибирається самою людиною. Людина порівнює себе з~цим ідеалом, самооцінює і, як наслідок, виникає відчуття задоволення чи незадоволення собою.

		Самооцінка і~самоконтроль можливі лише за наявності такого «дзеркала», як колектив інших людей. У~цьому «дзеркалі» людина бачить саму себе, і~з його допомогою вона починає ставитися до себе, як до людини, тобто виробляє форми самосвідомості. Самосвідомість формується в~процесі колективної практичної діяльності і~міжлюдських взаємовідносин, а~не в~результаті внутрішніх потреб ізольованої свідомості.

		Об'\-єк\-том вивчення людини може бути сама свідомість. У~цьому випадку ми говоримо про рефлексію. Рефлексія~— це така форма самосвідомості, коли ті чи інші явища свідомості стають предметом спеціальної аналітичної діяльності суб'\-єкта. Рефлексія не обмежується лише усвідомленням, аналізом того, що є в~людині, а~й~одночасно переробляє саму людину, спричиняє перехід за межі того рівня розвитку особистості, якого було досягнуто.

		Людина аналізує себе в~світі певного ідеалу особистості, порівнює себе з~ним, прагне досягти цього ідеалу. Вона немовби прагне «обґрунтувати» себе, закріпити системи своїх власних орієнтирів. Але свій образ, який формує людина, не завжди відповідає (адекватний) реальній людині та~її свідомості. Адже людина здатна помилятися. Тому те, наскільки правильно людина «розуміє» себе, адекватно «подає» себе, можуть визначити оточуючі її люди.

% 36. Розуміння пізнання як діалектичного й~культурно-історичного процесу взаємодії людини і~світу. об'\-єкт і~суб'\-єкт пізнання.
	\section{Розуміння пізнання як діалектичного й~куль\-тур\-но-іс\-то\-рич\-но\-го процесу взаємодії людини і~світу. Об'\-єкт і~суб'\-єкт пізнання}
		Пізнання — це особливий вид духовної діяльності, основною метою якої є встановлення об'\-єк\-тив\-них (істинних) знань про світ, суспільство, людину. Процес пізнання, будучи процесом активного творчого відтворення дійсності у~свідомості людини в~результаті її діяльного предметно-практичного відношення до світу, можливий лише при взаємодії людини з~явищами дійсності. Цей процес у~гносеології осмислюється через категорії «суб'\-єкт пізнання» та~«об'\-єкт пізнання»~— протилежностями, через взаємодію яких він реалізується. У~цьому випадку суб'\-єкт є носієм свідомості і~знання, а~об'\-єкт є тим, на що спрямована пізнавальна діяльність суб'\-єкта.

		Суб'\-єкт пізнання~— це реальна людина, суспільна істота, наділена свідомістю у~вигляді мислення, чуття, розуму та~волі, яка засвоїла історично вироблені людством форми та~методи пізнавальної діяльності, розвинула свої пізнавальні здібності та~оволоділа історично конкретними здатностями до цілеспрямованої пізнавальної діяльності.

		Об'\-єкт пізнання~— це те, на що спрямовується пізнавальна діяльність суб'\-єкта на основі практики. Об'\-єк\-том пізнання може бути вся дійсність, але лише в~тій мірі, в~якій вона увійшла у~сферу діяльності суб'\-єкта.

		Взаємодія суб'\-єкта і~об'\-єк\-та фіксує єдність матерії і~свідомості, буття і~мислення, природи і~духу. І~об'\-єкт, і~суб'\-єкт формуються в~процесі практичної діяльності і~неподільні одна від одного в~своєму виникненні та~функціонуванні. В цьому сенсі без об'\-єк\-та не може бути суб'\-єкта і~навпаки. Тобто без суспільно-історичної практики немає ні суб'\-єкта, ні об'\-єк\-та. Звичайно, об'\-єкт не конструюється суб'\-єктом, його свідомістю, а~існує об'\-єк\-тив\-но, але об'\-єк\-тив\-на реальність не може стати об'\-єк\-том пізнання без активної діяльності суб'\-єк\-та. Процес пізнання можливий лише при наявності взаємодії суб'\-єк\-та та~об'\-єк\-та, в~якій суб'\-єкт є носієм діяльності, а~об'\-єкт~— предметом, на який вона спрямована.

	% 37. Проблема істини у~філософії.
	\section{Проблема істини у~філософії}
		Проблема істини завжди була серцевиною теорії пізнання, до якої спрямована вся гносеологічна проблематика, тому всі філософські напрями і~школи намагались сформулювати своє розуміння природи і~сутності істини.

		Класичне визначення істини (кореспондентний підхід) дав Арістотель. Він визначив істину як відповідність наших знань дійсності. Проте це визначення було настільки широким і~абстрактним, що його дотримувались усі філософи: матеріалісти та~ідеалісти, діалектики та~метафізики. Однак, погляди філософів різняться в~питанні про характер відображуваної реальності та~про механізм відповідності.

		Крім того, безсумнівною є обмеженість такого підходу, оскільки очевидність не завжди є показником істини. Так, відкриття кривизни Всесвіту не узгоджується з~відчуттям тривимірності простору і~одновимірністю часу, емпірична очевидність причинно-наслідкового зв'язку не відповідає вимогам квантової фізики.

		Когерентну концепцію істини пов'язують з~А.\,Пу\-ан\-ка\-ре. Виходячи з~системності наукового знання, він запропонував тлумачити істину як процес і~результат домовленості, якого досягає наукове співтовариство з~приводу певної проблеми. В основі дискусії завжди лежать емпіричні дані, наукові концепції, теорії, факти та~гіпотези. Нерідко аргументами виступають правові чи соціокультурні аспекти. Такий підхід виявився найбільш продуктивним для вирішення соціальних, політичних та~економічних проблем, а~також є цілком застосовним у~точних науках. 

		Специфіка сучасного розуміння істини полягає, по-перше, в~тому, що дійсність, відображена в~істині, трактується як об'\-єк\-тив\-на реальність, яка існує незалежно від свідомості і~сутність якої виявляється через явище. По-друге, пізнання та~його результат~— істина~— нерозривно пов'язані з~предметно-чуттєвою діяльністю людини, з~практикою. Достовірне знання сутності та~її проявів відтворюється в~практиці. \emph{Істина}~— це адекватне відображення об'\-єк\-та суб'\-єк\-том, яке відтворює об'\-єкт таким, яким він існує незалежно від свідомості суб'\-єк\-та пізнання. 

	% 38. Практика як критерій істини.
	\section{Практика як критерій істини}
		Найбільш надійним критерієм істини вважають практику. \emph{Практика}~— це суспільно-історична, матеріальна, чуттєво-предметна діяльність, спрямована на перетворення природи, суспільства і~самої людини. Вона є джерелом пізнання, його рушійною силою і~кінцевою метою.

		Практика як критерій істини полягає у~відображенні, відтворенні та~перенесенні теоретичних знань на об'\-єк\-тив\-ний світ. Вона дозволяє отримати певні результати, які можуть дозволити зробити висновки щодо істинності теоретичних знань, що перевіряються.

		Практика має діалектичний характер: вона є єдністю абсолютного і~відносного. Якщо розглядати її як історичний процес, то вона є абсолютним критерієм істини. Абсолютність практики як критерію істини полягає в~тому. що тільки вона є єдиним засобом, здатним виявити об'\-єк\-тив\-но історичний зміст наших знань. Але розвиток практики обмежений рівнем розвитку суспільства на його певному історичному етапі. Якщо розглядати практику як окремий практичну дію, то вона є відносним критерієм істини, оскільки практика розвивається разом з~суспільством, і~тому знання, які вона реалізує, є неповними і~неточними, вони потребують подальшого дослідження.
		
		Практика виконує функцію критерію істини як цілісний процес у~діалектичній єдності перетворюючого (продуктивного) і~використовуючого (репродуктивного) рівнів. Однак, пріорітетне значення все ж таки має перетворюючий рівень, бо саме там знаходять реалізацію вищі досягнення творчої діяльності людини, засновані на тому чи іншому рівні розвитку пізнання. Проте, навіть в~єдності двох своїх основних рівнів, практика все ж не може бути абсолютним критерієм істини, оскільки вона завжди є практикою певного історичного етапу розвитку. Повним і~абсолютним критерієм вона могла б бути лише як цілісний історичний процес, у~якому здійснюється постійне розширення сфери олюднення світу через призму минулого, сучасного і~майбутнього. Але практика, яка сама є історичним процесом, завжди функціонує як діалектична єдність абсолютного та~відносного. І~хоч відомого, що практика є одночасно і~абсолютним, і~відносним критерієм істини, значення її в~пізнанні визначається тим, що це основний і~всезагальний критерій істини, єдиний спосіб виявлення об'\-єк\-тив\-но\-го змісту наших знань про дійсність.

	% 39. Діалектична єдність абсолютної й~відносної істини. Істина й~омана. Істина та~правда.
	\section{Діалектична єдність абсолютної й~відносної істини. Істина й~омана. Істина та~правда}
		Відносна істина~— це таке знання, яке в~принципі правильно, але не повністю відображає дійсність, не дає її всебічного вичерпного образу. Відносна істина включає і~такі моменти, які в~процесі подальшого розвитку пізнання і~практики будуть змінюватись, поглиблюватись та~уточнюватись, замінюючись новими. Відносність знань не означає, що в~них відсутній об'\-єк\-тив\-ний зміст. У~тій мірі, у~якій картина світу визначається не волею і~бажанням суб'\-єкта, а~реальним становищем речей, вона є об'\-єк\-тив\-ною істиною.

		Абсолютна істина~— це такий зміст людських знань, який тотожний своєму предмету і~який не буде спростований подальшим розвитком пізнання та~практики.

		Абсолютна і~відносна істини не існують окремо одна від одної. Існує лише одна істина~— об'\-єк\-тив\-на за змістом, яка є діалектичною єдністю абсолютного та~відносного, тобто є абсолютною істиною відносно певних меж. Абсолютне та~відносне~— це два необхідних моменти об'\-єк\-тив\-ної істини. Відносна істина обов'язково містить у~собі момент абсолютної, зумовлює і~передбачає її. Абсолютна істина, в~свою чергу, дається людині лише через відносні істини, які перебувають у~нескінченному процесі постійного розвитку. Відносні істини~— це сходинки, певні етапи на шляху досягнення абсолютної істини. На кожному етапі пізнання ми маємо справу лише з~відносно істинним, обмеженим знанням. Але сама здатність людини долати цю обмеженість, одержувати досконаліші знання свідчить про принципову можливість, рухаючись до об'\-єк\-тив\-ної істини, досягати одночасно і~абсолютної істини.

		Омана~— це такий зміст людського знання, в~якому дійсність відтворюється неадекватно і~який зумовлений історичним рівнем розвитку суб'\-єкта та~його місцем у~суспільстві. Це ненавмисне спотворення дійсності в~уявленнях суб'\-єкта. Вона має певні закономірні підстави для свого існування, будучи необхідним моментом і~результатом пізнання та~практики.

		Існування омани обумовлене законами розвитку як пізнання, так і~його основи~— практики. В процесі пізнання суб'\-єкт змушений застосовувати ті знання та~засоби пізнання, які були вироблені в~інших історичних умовах попередніми поколіннями і~поширювати їх на нові об'\-єк\-ти і~умови. В результаті розвиток науки може породжувати оману, проте в~процесі освоєння явищ, виробляючи нові поняття та~уточнюючи старі, омана у~процесі пізнання стає ще одним кроком на шляху до істини.

		Правда~— це суб'\-єк\-тив\-не знання про певну частину дійсності. На відміну від об'\-єк\-тив\-ної істини, яка в~своїй сутності є всеосяжною, повною, незалежною від суб'\-єкта, правда є лише частковим, відносним знанням, що завжди піддається впливу суб'\-єкта, його моральних переконань, відчуттів та~обмежується доступною йому інформацією.

	% 40. Єдність чуттєвого й~раціонального в~пізнанні. Основні форми чуттєвого та~раціонального пізнання.
	\section{Єдність чуттєвого й~раціонального в~пізнанні. Основні форми чуттєвого та~раціонального пізнання.}
		Раціональне і~чуттєве~— це діалектично взаємозв'язані сторони єдиного пізнавального процесу, які лише в~єдності можуть давати адекватну картину дійсності. Кожен момент чуттєвого відображення в~пізнанні опосередкований мисленням. У~свою чергу, раціонально пізнання одержує свій зміст з~чуттєвих даних, які забезпечують постійний зв'язок мислення з~конкретними предметами та~явищами дійсності. І~в мисленні людина не може відокремитись від наочності, оскільки без цього неможлива проекція наших знань на дійсність та~результативне здійснення предметно-практичної діяльності.

		Єдність чуттєвого та~раціонального у~пізнання обумовлена суспільно-історичною практикою, але безпосередньо вона втілюється і~виявляється в~діяльності такої пізнавальної здібності людини як творча уява, яка в~своїх образах відображає зміст уявлень раціонально оформлених пізнавальних ситуацій. 

		До основних форм чуттєвого пізнання відносять відчуття, сприйняття та~уявлення. \emph{Відчуття}~— це відображення окремих властивостей предметів та~явищ внаслідок їх безпосереднього впливу на органи чуття людини. Відчуття~— це ті канали, які зв'язують суб'\-єкт із зовнішнім світом. Але, будучи результатом безпосереднього впливу лише окремих властивостей та~сторін об'\-єк\-тів, відчуття хоч і~є джерелом пізнання, дає не цілісну характеристику дійсності, а~лише однобічну її картину.

		\emph{Сприйняття}~— чуттєве відображення предметів та~явищ дійсності в~сукупності притаманних їм властивостей при безпосередній дії їх на органи чуття. Це цілісний, багатоаспектний чуттєвий обрах дійсності, який виникає на основі відчуттів, але не є лише їх сумою. Це якісно нова форма чуттєвого відображення дійсності, яка виконує дві взаємозв'язані функції: пізнавальну та~регулятивну. Пізнавальна функція розкриває властивості та~структуру об'\-єк\-тів, а~регулятивна спрямовує практичну діяльність суб'\-єкта згідно з~цими властивостями об'\-єк\-тів. Сприймання має активний характер. Воно відображає в~єдності з~всебічними характеристиками об'\-єк\-та також і~все багатогранне життя суб'\-єкта: його світоглядні установки, минулий досвід, інтереси, прагнення, надії.

		\emph{Уявлення}~— це чуттєвий образ, форма чуттєвого відображення, яка відтворює властивості дійсності за відбитими в~пам'\-я\-ті слідами предметів, що раніше сприймались суб'\-єк\-том. Це чуттєвий образ предмета, який уже не діє на органи чуття людини, узагальнений образ дійсності. Уявлення поділяються на~образи пам'\-я\-ті та~образи уяви. За допомогою образів уяви твориться картина майбутнього.

		До основних форм раціонального пізнання відносять поняття, судження та~умовиводи. \emph{Поняття}~— це форма раціонального пізнання, в~якій відображається сутність об'\-єк\-та і~дається його всебічне пояснення. Поняття як знання сутності, знання про загальне і~закономірне формується на основі практики, оскільки сами в~процесі практики суб'\-єкт може визначити суттєві і~несуттєві сторони дійсності. В поняттях предмети та~явища відображаються в~їхніх діалектичних взаємозв'язках та~розвитку, тому самі вони мають бути рухливими, гнучкими, діалектичними. Своїм логічним змістом поняття відтворює таку діалектичну закономірність пізнання, як зв'язок одиничного, особливого і~загального, хоча у~понятті вони і~не розчленовані. Їх розчленування і~виявлення співвідношення розкривається в~судженні.

		\emph{Судження}~— це елементарна найпростіша форма вираження змісту поняття; така логічна форма мислення, в~якій стверджується або заперечується дещо відносно об'\-єк\-ту пізнання. В судженнях виражається зв'я\-зок між поняттями, розкривається їх зміст, дається визначення. Будучи формою вираження змісту понять, окреме судження не може певною мірою розкрити цей зміст.

		\emph{Умовивід}~— це такий логічний процес, у~ході якого із кількох суджень на основі закономірних, суттєвих і~необхідних зв'яз\-ків виводиться нове судження, яке своїм змістом має нове знання про дійсність. Перехід до нового знання в~умовиводі здійснюється не шляхом звернення до даних чуттєвого досвіду, а~опосередковано, на основі логіки розвитку самого знання, його власного змісту. За характером одержання нового знання умовиводи поділяються на такі види:
		\begin{enumerate}
			\item Індуктивні~— рух думки від суджень менш загального характеру до більш загального.
			\item Дедуктивні~— від більш загального до менш загального.
			\item За аналогією~— на підставі подібності чи відмінності деяких точно виявлених властивостей ряду об'\-єк\-тів доходять висновку про подібність чи відмінність інших властивостей об'\-єк\-тів.
		\end{enumerate}
		
% 41. Буденне та~наукове пізнання.
	\section{Буденне та~наукове пізнання}
		Буденне пізнання основане на пізнанні буденного, або повсякденного життя. Головною ланкою у~цьому типі пізнання виступає буденна свідомість. У~буденному світі~— в~світі повсякденності в~найрізноманітніших сферах діяльності триває процес самореалізації особистості. Людина, живучи повсякденним життям, постійно в~різноманітних ситуаціях і~обставинах шукає і~знаходить уже сталі норми, засоби, способи поведінки, найкоротший шлях, що веде до такої мети. Причому мета найрізноманітніша, майже до мети визволитись, позбутись бажань і~потреб. Повсякденності властиві риси: стандартність, рецептурність, рекомендаційність буденної свідомості.

		Буденна свідомість характеризується специфічними рисами, безпосередньо знайомими кожному, що різко відрізняються від тих, що наука подає як раціональні еталони.

		Не вірно вважати, що повсякденна свідомість людей складається тільки  з~завмерлих стереотипів, непорушних схем, рецептів, обмежуючи поведінку  людей жорсткими рамками. І~та складова буденної свідомості~— здоровий  сенс, дає можливість людям правильно оцінювати зміни суспільстві, ситуацію, що складається, пристосовуватися, вести відповідно з~обставинами, дотримуватися свого людського інтересу в~усіх перепитіях життя, повсякденності. Саме це логіка соціальних норм, логіка досягнення мети.
		
		Наукове пізнання містить у~собі два основних взаємопов'язаних, але якісно різних рівня~— емпіричний і~теоретичний.
		
		На емпіричному рівні переважає чуттєве пізнання; раціональне пізнання і~його форми тут присутні, але мають підпорядковане значення. Тому на емпіричному рівні досліджуємий об'\-єкт відображається переважно з~боку своїх зовнішніх зв'язків і~проявів. Характерними ознаками емпіричного пізнання є: збирання фактів, їх первинне узагальнення, опис спостерігаємих та~експериментальних даних, їх систематизація, класифікація.
		
		Теоретичний рівень характеризується перевагою раціонального пізнання. Теоретичне пізнання відображає явище і~процеси з~боку їх універсальних внутрішніх зв'язків і~закономірностей. Що досягаються за допомогою раціональної обробки емпіричних даних, яка дозволяє сформулювати наукові проблеми, гіпотези, теорії. Найважливіше завдання теоретичного пізнання~— досягнення істини.
		
		Емпіричний та~теоретичній рівні наукового пізнання вза\-є\-мо\-пов'\-я\-за\-ні, тобто емпіричне може переходити в~теоретичне і~навпаки. Неприпустимо абсолютизувати значення якогось одного з~них.

% 42. Форми і~методи наукового пізнання.
	\section{Форми і~методи наукового пізнання}
		В науковому пізнанні істинним повинен бути не тільки його результат, але й~спосіб отримання цього результату~— метод. 

		Очевидно, що наукове пізнання не обмежується констатацією фактів. Тому важливе значення мають методи обробки і~систематизації фактів (ці методи називають також загальнологічними методами пізнання):
		\begin{enumerate}
			\item Аналіз і~синтез.
			\item Індукція, дедукція.
			\item Аналогія.
			\item Класифікація та~інші.
		\end{enumerate}
		
		Аналіз~— логічний прийом, який полягає в~уявному розчленуванні предмета дослідження на складові елементи (властивості, відношення, параметри тощо) з~метою їх детального вивчення. Виконання аналізу супроводжується здійсненням синтезу, коли знання про частини предмета чи явища залучають до їх комплексного, цілісного сприйняття.
		
		Індукція~— це форма мислення, в~якій висновок про загальне робиться на основі знання про часткове. Індукція ґрунтується на існуванні причинно-наслідкової залежності між частковим і~загальним. Тому індуктивний висновок завжди імовірний.
		
		Дедукція~— це форма мислення, яка передбачає неухильне дотримання законів логіки при переході між думками в~процесів міркування. Інколи дедукцію визначають як шлях думки від загального до часткового. Висновок, отриманий дедуктивним шляхом, завжди достовірний. В науковому пізнанні індукція і~дедукція взаємозв'язані. Індукція розширює наявне знання, дозволяє висувати гіпотези, припущення, версії, тоді як дедукція направлена на систематизацію існуючих знань, створення теорій та~їх обґрунтування.
		
		Аналогія~— це вид умовиводу, застосування якого дозволяє зробити висновок про властивості досліджуваного об'\-єкта на основі його схожості з~іншими предметами чи явищами. Наприклад, модель атома Резерфорда—Бора була побудована аналогічно до планетарної системи.
		
		Класифікація~— це розподіл предметів якого-небудь виду на взаємозв'язані класи відповідно до певного критерію чи ознаки. Здійснення класифікації виявляє глибинні, неочевидні на перший погляд зв'язки між об'\-єктами, дозволяє формулювати узагальнені висновки щодо предмета дослідження.
		
		Результатом залучення методів обробки і~систематизації фактів є формулювання гіпотез та~емпіричних законів (припущень, версій тощо)
		Зрозуміло, що емпіричні факти не проникають глибоко в~сутність речей, явищ і~процесів, а~дозволяють сформулювати поверхове уявлення про їх структуру, виявити деякі причинно-наслідкові залежності, висунути первинні гіпотези. Іншими словами, емпіричний рівень пізнання дозволяє досліднику сформулювати лише імовірнісне знання про об'\-єкт. Більш глибоке освоєння дійсності можливе лише при залучені методів теоретичного пізнання, які вимагають абстрагування від несуттєвих властивостей об'\-єктів.
		
		Через це першим етапом теоретичного рівня пізнання є побудова ідеалізованого об'\-єкта, який у~подальшому дослідженні заміняє реально існуючу предметну дійсність. Наприклад, такі поняття як матеріальна точка, абсолютно чорне тіло, ідеальний газ, маса та~інші є результатами ідеалізації. У~такий спосіб теоретичне пізнання втрачає з~дійсністю безпосередній зв'язок. З~одного боку, це дозволяє здійснити цілісне дослідження предмета, а~з іншого~— ставить питання про критерії істини. Усвідомлюючи проблемність цієї ситуації, В.\,І.\,Ленін писав, що правильне мислення не відходить, а~наближається до істини. Тому абстракції, утворені на основі наукових фактів, не лише не спотворюють істинної природи явищ, а~навпаки~— глибше, точніше, повніше її відображають, дозволяють прогнозувати, бачити в~об'\-єкті те, що не дано безпосередньому емпіричному спогляданню.
		
		Отже, для теоретичного пізнання важливим є створення ідеалізованого об'\-єкта та~побудова теоретичного знання.
		
		Методи створення ідеалізованого об'\-єкта:
		\begin{enumerate}
			\item Абстрагування.
			\item Формалізація.
			\item Ідеалізація.
			\item Математичне моделювання.
		\end{enumerate}
		
		Результатом застосування цих методів є формулювання принципів, ідей, аксіом, постулатів тощо, які визначають напрямки теоретизування та~обумовлюють методику емпіричних досліджень.
		
		Методи побудови теоретичного знання:
		\begin{enumerate}
			\item Дедуктивні (аксіоматичний, гіпотеко-дедуктивний);
			\item Історичні (конкретно-історичний, аб\-стракт\-но-іс\-то\-рич\-ний);
			\item Системні.
		\end{enumerate}
		
		Наукове знання, яке будується аксіоматичним методом є дедуктивною системою, в~якій весь зміст теорії може бути виведений з~її початкових положень~— аксіом (твердження, істинність яких не піддається сумніву). Гіпотеко-дедуктивний метод передбачає існування сукупності гіпотез і~множини емпіричних фактів, між якими встановлюється складна система взаємодії та~взаємозалежності.
		 
		Історичний підхід, на відміну від попереднього, орієнтується на вивченні особливостей виникнення, формування та~розвитку об'\-єкта дослідження. Широко застосовується в~мовознавстві, геології, астрономії, психології та~інших науках, які вивчають складні, розтягнені в~часі процеси.
		
		В основі групи системних методів лежить поняття системи~— упорядкованої, структурованої сукупності елементів. Системний підхід ґрунтується на таких принципах:
		\begin{enumerate}
			\item Системний об'\-єкт~— це сукупність елементів, пов'\-я\-за\-них між собою скінченною множиною струк\-тур\-но-функ\-ці\-о\-наль\-них залежностей.
			\item Функціонування системи залежить і~може бути пояснено лише з~урахуванням її структурної організації.
			\item Структурна організація системи може бути інтерпретована на інших об'\-єк\-тах-моделях.
		\end{enumerate}
		
		Результатом застосування цих методів є оформлення знання у~вигляді наукових проблем, гіпотез, теорій, концепцій.
		
		Проблема~— це питання чи комплекс питань, які об'\-єктивно виникають у~процесі наукового пізнання. Вирішення проблеми має допомогти розв'язати актуальні наукові проблеми.
		
		Гіпотеза~— це наукове припущення, істинність якого не встановлена. Гіпотези є джерелом розвитку наукового знання, оскільки вони дають можливість передбачати деякі явища та~їх властивості.
		
		Теорія~— це вища форма організації наукового знання. Вона дає цілісне уявлення про закономірності та~суттєві зв'язки предметів і~явищ. Теорія є найбільш розвиненою формою наукового знання, оскільки вона дає адекватне відображення об'\-єктивного світу, направляє наукові пошуки, передує практиці. Достовірність теорій лише в~незначних випадках може бути перевірена на практиці. Тому їх істинність визначають через методи верифікації (встановлення відповідності з~досвідними даними та~науковими фактами), фальсифікації (виявлення суперечливих моментів теорії, перевірка її на нових даних), редукції (зведення до простих, вже відомих фактів, аксіом, положень), логічного і~математичного доведення та~інші.
		
		Розглянуті вище методи і~форми наукового пізнання взаємозв'язані та~доповнюють одне одного. В комплексі вони дозволяють максимально наблизитися до пізнання істини, яка є ідеалом і~метою пізнавального процесу.
		 
% 43. Предмет соціальної філософії. 
	\section{Предмет соціальної філософії}
		Питання «Що таке суспільство? В чому полягає його сутність?»~— не нові. Протягом століть, з~моменту зародження наукового пізнання, людство робило спроби відповісти на ці питання, зрозуміти специфіку суспільних явищ, характер зв'язку між людиною й~суспільством. Це знайшло своє відображення в~різних концепціях. Осмислюючи світ, люди намагалися зрозуміти зміст і~мету свого існування, відповісти на питання, що таке суспільство, чому відбуваються його зміни, яка роль людини, її свідомості в~цьому процесі.

		Розкрити тезу, що в~розумінні суспільства, його сутнісних засад існують різноманітні теорії. Чи не першими в~цьому були античні мислителі (Платон, Арістотель, Епікур та~ін.) зі своїми теоретичними уявленнями про соціум як систему співбуття людей. Фундаментальною тезою грецької ідеї держави була гармонія (пропорційність), справедливість життя усіх її громадян. Саме гармонія і~справедливість, на думку грецьких мислителів, були єднальною силою суспільства, яке ототожнювалось ними з~державою. Для стоїків суспільство є зразком світової, космічної єдності, яка проникає й~охоплює будь-яку множинність. Не лише людську спільноту, а~й~Землю, увесь світ (Всесвіт) вони розглядали як своєрідне «суспільство»~— «державу богів і~людей». В античній філософії побутували й~інші точки зору. Так, Епікур і~його прибічники вважали, що люди, відчуваючи потребу одне в~одному, об'єдналися в~суспільство. Вони розподілили між собою окремі суспільні обов'язки, особливо ті, що стосувалися безпеки, і~затвердили правила стосунків між собою.

		Показати, що в~епоху Середньовіччя в~поглядах на суспільство переважало його тлумачення як Граду Земного на противагу Граду Небесному, божественому.

		Підкреслити, що мислителі Нового часу Т.\,Гоббс, Дж.\,Локк і~Ж.—Ж.\,Руссо знову висунули ідеї добровільної угоди між людьми як вихідного принципу влаштування громадського життя. Подібне уявлення про суспільство переважало у~XVІІІ ст. На противагу цьому Г.\,Гегель (1770–1831) стверджував, що держава~— це божественна воля у~тому розумінні, що вона є дух, присутній на землі, який розгортається, щоби бути формою організації світу. Вона є абсолютною раціональністю, божеством, вічним і~необхідним буттям абсолютної ідеї, створюючи громадянське суспільство для досягнення своїх цілей. Соціологи Джон—Стюарт~Мілль (1806–1873) і~Герберт Спенсер~(1820–1903), продовжували розвивати «універсальну» теорію суспільства. Інакше інтерпретував суспільство Е.\,Дюркгейм (1858–1917), розглядаючи його як єдність різноманітних ідей, вірувань~— релігійних, моральних, естетичних, правових, політичних,~— що реалізуються завдяки посередництву індивідів.

		Принципово іншою є точка зору К.\,Маркса (1818–1883), який, розкриваючи джерела та~основу виникнення і~розвитку суспільства, писав: «В суспільному виробництві свого життя люди вступають в~певні, необхідні, від їх волі незалежні відносини~— виробничі відносини, які відповідають певному ступеню розвитку їх матеріальних продуктивних сил… На певному ступені свого розвитку матеріальні продуктивні сили суспільства приходять у~суперечність з~існуючими виробничими відносинами, або~— що є тільки юридичним виразом останніх~— з~відносинами власності, всередині яких вони досі розвивалися. З~форм розвитку продуктивних сил ці відносини перетворюються в~їх окови». З~його точки зору, діалектичний взаємозв'язок продуктивних сил і~виробничих відносин відображається в~законі відповідності виробничих відносин характеру й~рівню розвитку продуктивних сил. Обумовленість удосконалення виробничих відносин характером і~рівнем розвитку продуктивних сил, і~в той же час активний вплив виробничих відносин на продуктивні сили є об'\-єктивною закономірністю. Порушення відповідності між продуктивними силами і~виробничими відносинами, відставання останніх перешкоджає розвиткові продуктивних сил, внаслідок чого відбувається зниження темпів росту продуктивності праці, впровадження досягнень науково-технічного прогресу у~виробництво. Продуктивні сили становлять матеріальну основу практичної діяльності, але розвиток цієї матеріальної основи забезпечується чи стримується системою виробничих відносин.

		На підставі аналізу різних підходів до пояснення сутності суспільства можна стверджувати, що суспільство є певною єдністю, яка виявляє себе в~об'єднаності, спільності, упорядкованості життя й~функціонує як єдине конкретне ціле. Людське життя в~усіх його сферах, починаючи від сім'ї і~закінчуючи найвищими духовними цінностями~— мораллю, мистецтвом, релігією, філософією,~— має форму суспільного життя, співбуття людей, їх груп.

		Показати, що сучасні дослідження сутності суспільства ґрунтуються на двох головних поняттях~— суспільного буття та~суспільної свідомості. Суспільне буття~— сукупність всіх форм життєдіяльності людей, їх груп в~суспільстві. Суспільна свідомість~— система ідеального відображення, освоєння та~перетворення суспільного буття людей. Суспільна свідомість, духовна діяльність, які є відображенням суспільного буття, виступають регулюючим чинником усіх суспільних відносин.

		Обґрунтувати, що люди характеризуються не лише індивідуальними ознаками і~якостями, а~й~мають загальнозначуще~— соціальне, зумовлене обставинами їх суспільного існування: умовами матеріального виробництва, соціальним устроєм, політичною організацією, рівнем суспільної свідомості й~культури. Звідси постає питання суспільної природи соціального. Соціальним є все те, що характеризує співіснування людей і~що є відмінним від їх природної, біологічної основи. Виникає воно як системна характеристика безпосередньої чи опосередкованої взаємодії людей. Без такої взаємодії соціальне неможливе. Соціальне є сукупністю суспільних відносин індивідів, їх певних груп у~процесі спільної діяльності, які виявляються у~їх ставленні одне до одного, до свого місця і~ролі в~суспільстві, до явищ і~процесів соціального життя.
		 
% 44. Специфіка буття суспільства як спів-буття людей. 
	\section{Специфіка буття суспільства як спів-буття людей}
		На шляху до трансцендентного людина зустрічається з~чимось спорідненим з~її власним буттям. Це~— буття іншої людини. Й коли це дійсно «зустріч», а~не зовнішнє зіткнення, тоді відбувається взаємний обмін смислами, взаємне збагачення й~взаємне піднесення. Отже, дійсне людське буття здійснюється як співбуття.
		
		Ідея співбуття як дійсно достотного стану людського буття викристалізувалася якраз у~межах екзистенційної філософської традиції. Вихідним тут завжди було наголошення на винятковій специфічності людського існування як особливого типу реальності, на своєрідності й~неповторності кожної окремої людини. Відповідно предметом філософської думки ставали передусім окремішність людини, її відособленість від інших людей і~від світу взагалі. Але справді продуктивними для подальшого культурного розвитку виявились ідеї про необхідність для людини виходити за межі свого індивідуального існування, вступати в~смислову взаємодію, в~комунікацію з~іншими людьми.
		
		Реальне сумісне буття людей є таким, що між ними далеко не завжди відбувається безпосередній обмін смислами. Багато частіше і~старіше вони пов'язані між собою через обмін. Людина виявляється суцільно оточеною речами і~знаками~— продуктами діяльності інших людей. Її буття виявляється суцільно опосередкованим, але саме тому~— суцільним співбуттям з~іншими людьми.
		 
% 45. Соціальний простір і~соціальний час як буття людини в~культурі. 
	\section{Соціальний простір і~соціальний час як буття людини в~культурі}
		Просторово-часові характеристики об'\-єктів на різних рівнях організації матерії мають свої особливості. Найбільш вагомими вони є на соціальному рівні. Що ж таке соціальний простір і~соціальний час?

		Соціальний простір~— це простір, освоєний людством у~процесі його існування. Це є частина фізичного простору, яка складає спосіб життєдіяльності суспільства, тобто олюднений простір. Він вписаний у~простір біосфери. Соціальний простір є простором людського буття, який має перш за все соціально-культурний і~духовно-практичний виміри. Мерло—Понті писав, що простір~— це не місце, в~якому розташовані речі, а~він характеризує лише можливу послідовність їх розташування. А~С.\,Б.\,Кримський підкреслював, що якщо людина як тілесний індивід займає певне місце у~фізичному просторі, то у~своєму соціальному бутті людина ніби «розкидана» у~світі, у~справах і~подіях життя та~духу, у~тимчасовості свого унікального існування. Людина у~своєму бутті втягнута у~багатовімірні просторові структури, сенс яких визначається тим, якою мірою вони відповідають буттєвій ситуації людини в~даний конкретних момент часу; якою мірою вони стали «обжитими» людиною, «своїми» для неї. Наприклад, для людини має значення рівень комфортності її житла, робочого місця, місця відпочинку, автомобіля, перебування в~оточенні інших людей тощо.

		На відміну від фізичного простору, соціальний простір твориться самими людьми. Освоюючи фізичний простір, вони перетворюють частину його відповідно до своїх потреб та~інтересів: створюють штучні умови свого життя, будують населені пункти, освоюють надра землі, ближній космос, обробляють землю, вирощують все нові види рослин і~тварин тощо. Усе це є підпросторами соціального простору, які складають поле людської культури. Кожна людина одночасно перебуває у~просторі життя, соціуму, культури, інших людей, свого внутрішнього світу. Людина не перебуває в~просторі, а~організовує, структурує простір свого буття в~культурі. Вона створює свій особистий простір, який вписаний у~простір соціуму. Але разом з~тим, відбувається автономізація, «приватизація» власного соціального простору. Соціальний простір має культурно-історичні межі, але й~тенденцію до розширення: суспільство освоює все нові ділянки фізичного простору, окультурює їх, усуспільнює.

		Соціальний час~— це тривалість існування людства, його історія. Він теж має культурний вимір. Він указує на міру мінливості суспільного життя, матеріальних та~духовних процесів, які відбуваються в~соціумі. Його можна зобразити у~вигляді променя, який має початок (перехід від тваринного до людського стану) і~не має кінця. На відміну від фізичного часу, який протікає рівномірно й~вимірюється хвилинами, годинами, добами, роками тощо, соціальний час протікає нерівномірно й~вимірюється історичними епохами. Протягом деяких історичних періодів відбувається значна кількість суспільних змін, а~в~інші~— незначна. Отже, соціальний час у~деяких історичних епохах прискорює ходу («біжить», «летить»), а~в~деяких~— сповільнює («повзе», «тягнеться»).

		Унікальність людського буття полягає в~тому, що воно, за влучним виразом С.\,Б.\,Кримського, протікає як синтез основних форм часу: екзистенціально-особистісного, історичного й~космічного. Для соціального часу особливе значення має майбутнє. Воно не визначається цілком і~повністю минулим і~теперішнім, а~є завжди втаємниченим і~невизначеним, оскільки існуючі в~теперішньому часі тенденції суспільного розвитку можуть привести до непередбачуваних подій, ситуацій, станів. Окрім того, кожна особистість по-своєму переживає, оцінює соціальний час, а~отже, по-своєму діє в~тих чи інших суспільних умовах, що може вплинути (більшою чи меншою мірою) й~на хід історії.

		Особистісний час інтегрує в~собі як об'\-єктивні часові відношення, так і~суб'\-єктивне їх відображення у~процесі переживання та~оцінки тривалості й~послідовності історичних подій. Визначальним для кожної людини є володіння своїм часом. На думку Сенеки, частину часу у~нас крадуть, відбирають інші люди, але значну частину ми даремно втрачаємо самі. А~тому, якщо бажати бути господарем власного часу завтра, треба навчитися тримати в~руках день сьогоднішній (Кримський). Соціальний час, на відміну від фізичного, передає екзистенціальне переживання людиною свого буття. Справжнє людське буття насичене подіями. Не випадково Хайдеггер ставив у~взаємну залежність людське буття і~час, а~саме буття визначав через термін «присутність». Отже, через категорії «соціальний простір» та~«соціальний час» люди виражають своє чуттєво-емоційне ставлення до світу, уявлення про початок і~кінець свого буття, про життя та~смерть, про тимчасовість і~вічність, безконечність та~конечність буття тощо. З~розвитком людства змінюється й~зміст цих категорій.
		 
% 46. Суспільне буття та~суспільна свідомість.
	\section{Суспільне буття та~суспільна свідомість}
		Суспільне буття~— це реальний процес життя, система соціальних зв'язків і~відносин, що зумовлюють зміст, характер, спрямованість життєдіяльності людини, виникнення, функціонування, розвиток свідомості як особливого типу, способу її орієнтації в~навколишній діяльності. Тут на перший план виступає зумовленість виникнення свідомості суспільним буттям. Виникнувши свідомість перетворюється з~результату на передумови людської життєдіяльності.
		
		Суспільна свідомість~— це виражене в~логічній, понятійній формі і~закріплене в~мові усвідомлення людиною сутності світу свого буття і~своєї сутності, свого місця у~світі, ставлення до нього і~до себе, можливостей його пізнання і~перетворення. Свідомість як усвідомлення буття є, з~однієї сторони, відображенням об'\-єктивною за своїм змістом, а~з другої сторони~— містить у~собі моменти суб'\-єктивного ставлення людини до дійсності. Усвідомлення, отже, виступає як виявлення активного відношення людини до дійсності. Це відношення реалізується насамперед через практичне освоєння людиною навколишнього світу і~є відображенням форм людської діяльності і~створюваного цією діяльністю предметного світу олюдненої природи. Усвідомлення~— це єдність знань і~переживань, єдність об'\-єктивного змісту свідомості і~його оцінки. Характер і~спрямованість ставлення людини до умов свого буття зумовлюється насамперед характером і~спрямованістю потреб і~інтересів. Вони визначають характер і~спрямованість ставлення людини до світу, і~саме той кут зору, під яким людина сприймає його. Отже, суспільна свідомість виникає як результат залежності життєдіяльності людини від соціальних умов життя, як реалізація потреби в~новому типі орієнтації. Тобто свідомість~— це породження, продукт суспільного буття.
		 
		Суспільне буття~— матеріальна, предметно-чуттєва сторона життя суспільства, соціально-практична діяльність людей і~її продукти-світ культури як теоретично і~практично освоєної природи.
		
		Суспільна свідомість~— ідеальна, духовна сторона життя суспільства, сукупність поглядів, ідей, вчень, теорій, якими люди керуються у~своїй практичній діяльності (наукові знання, політичні, правові ідеї, моральні принципи і~норми, художні образи, релігійні переконання, філософські погляди).
		
		В проблемі взаємозв'язку природного і~соціального (суспільного) буття марксизм виходить з~первинності природного, а~в~проблемі взаємозв'язку суспільного буття і~суспільної свідомості з~первинності суспільного буття, розглядаючи суспільну свідомість як творче відображення суспільного буття.

% 47.Структурованість соціальної системи. Основні елементи соціальної структури суспільства та~їх історичний характер. 
	\section{Структурованість соціальної системи. Основні елементи соціальної структури суспільства та~їх історичний характер}
		Соціальна структура суспільства — це сукупність взаємозв'язаних і~взаємодіючих між собою соціальних груп, спільностей та~інститутів, пов'язаних між собою відносно сталими відносинами. Отже, соціальна структура суспільства являє собою будову цієї соціальної системи, визначає характер взаємозв'язків і~взаємовідносин між її складовими частинами. Сутність соціальної структури суспільства найбільш повно виражається у~її загальних рисах, до яких можна віднести:
		\begin{enumerate}
			\item Багатоманітність соціальних елементів, що утворюють соціальну структуру суспільства (соціальний інститут, соціальна група, соціальна спільність тощо).
			\item Різний ступінь впливу кожного складового елемента соціальної структури суспільства на соціальні процеси і~явища, відмінність їхніх соціальних ролей.
			\item Наявність відносно стабільних зв'язків між складовими елементами соціальної структури суспільства, взаємозалежність останніх. Це означає, що жоден елемент соціальної структури не може існувати в~суспільстві автономно. В будь-якому випадку він поєднаний соціальним зв'язком з~іншими структурними підрозділами суспільства. У~цьому випадку є цікавою історія про Робінзона Крузо, який навіть опинившись на безлюдному острові, перебував у~тісному зв'язку з~суспільством (користувався речами, що виготовили інші люди, займався тими ж видами занять, що і~в Англії — облаштовував власну оселю, вирощував сільськогосподарські культури, молився Господові тощо).
			\item Взаємопроникність елементів, що забезпечує цілісність соціальної структури, тобто одні й~ті ж соціальні суб'\-єкти можуть бути частинами різних складових одиниць суспільства. Наприклад, одна і~та ж особистість може бути включена до різних соціальних груп та~спільностей.
			\item Багатофункціональність і~стабільність — кожен елемент соціальної структури суспільства виконує свої специфічні функції, які є відмінними від ролей інших соціальних елементів, що й~передбачає значну кількість соціальних функцій суспільства. У~зв'язку з~вищесказаним можна зробити висновок, що основними складовими суспільства є соціальні спільності, оскільки їх вплив на соціальні процеси є незрівнянно більший ніж участь окремої особистості. Що стосується соціальних організацій та~соціальних інститутів, то вони формуються у~результаті діяльності і~взаємодії соціальних спільностей та~груп, є похідними від них*1. Важливим елементом соціальної структури суспільства є також соціальні групи.
		\end{enumerate}
		
		Отже, основними підструктурами (елементами) суспільства є:
		\begin{enumerate}
			\item Соціально-етнічна структура;
			\item Соціально-демографічна структура;
			\item Соціально-професійна структура;
			\item Соціально-класова структура;
			\item Соціально-територіальна структура.
		\end{enumerate}

% 48.Матеріальні фактори функціонування суспільства. Єдність продуктивних сил та~виробничих відносин. 
	\section{Матеріальні фактори функціонування суспільства. Єдність продуктивних сил та~виробничих відносин}
		Значна частина філософів висувають на перший план матеріальні фактори, що спричиняє більш плідне і~послідовного пояснення надзвичайно розмаїтої і~складної суспільної життєдіяльності людини. Наприклад, новий підхід до розкриття природи суспільства, характеру суспільних відносин, рушійних сил їхнього розвитку зробив англійський економіст і~філософ А.\,Сміт (1723–1790). Він вважав, що праця, прагнення людини до благополуччя виступають головним джерелом суспільного багатства, основою життя суспільства. На думку А.\,Сміта, розподіл праці є основою суспільних зв'язків людини. Внаслідок розподілу праці відбувається обмін плодами праці кожної людини. Отже, кожен працює на себе і~в той же час змушений працювати на інших, а~працюючи на інших, одночасно працює на себе. За А.\,Смітом, у~трудовому суспільстві, щоб воно успішно функціонувало і~розвивалося, необхідно дотримуватися трьох основних умов: панування приватної власності, невтручання держави в~економіку та~відсутність перешкод для всебічного розвитку особистої ініціативи.

		Послідовники матеріалістичних ідей, поглядів, спираючись на досягнення передової суспільної думки, суттєво переглянули попередні уявлення про суспільство, його основу та~рушійні сили. Вони висунули принцип матеріалістичного розуміння історії. Відповідно до цього принципу стверджується, що першоосновою розвитку є трудова, виробнича Діяльність людей, яка спрямована на задоволення потреб особи і~насамперед матеріальних потреб суспільної людини.

		\subsection{Єдність продуктивних сил та~виробничих відносин}
			Вони завжди перебувають у~тісному взаємозв'язку і~діалектичній єдності. Слід зазначити, що виробничі відносини похідні, вторинні щодо продуктивних сил. Нові продуктивні сили, хоч і~стихійно, але зароджуються в~надрах старого устрою. Люди, які вдосконалюють знаряддя виробництва, не задумуються над соціальними наслідками цього процесу, їх цікавить насамперед задоволення матеріальних, а~потім і~духовних потреб. Люди, по суті, не вільні у~виборі своїх продуктивних сил, що становлять основу всієї їхньої історії. Продуктивні сили~— це результат практичної енергії попередніх поколінь людей.

			З іншого боку, люди, які розвивають свої продуктивні сили, розвивають одночасно і~відносини один з~одним. Характер цих відносин з~необхідністю змінюється разом з~перетворенням та~зростанням продуктивних сил.

			Продуктивні сили вимагають лише певної, а~не довільної форми виробничих відносин, яка найбільш повно відповідала б наявним продуктивним силам.

			Отже, виробничі відносини не можуть встановлюватися довільно. Це — об'\-єктивні, матеріальні відносини, які не залежать від волі та~бажання людей, а~залежать від рівня та~характеру розвитку продуктивних сил. Процес стихійного розвитку продуктивних сил відбувається доти, доки нові продуктивні сили не вступили в~конфлікт із старими виробничими відносинами. Відтак починається руйнування старих виробничих відносин і~заміна їх новими.

% 49.Історична періодизація суспільного розвитку: формаційний та~цивілізаційний підходи.
	\section{Історична періодизація суспільного розвитку: формаційний та~цивілізаційний підходи}
		Формаційний підхід був розроблений К.\,Марксом і~його послідовниками. Суть його полягає в~тому, що продуктивні сили суспільства у~сукупності з~виробничими відносинами становлять певний спосіб виробництва, а~спосіб виробництва у~поєднанні з~політичною надбудовою суспільства~— соціально-економічну формацію. Основоположним економічним ядром кожного способу виробництва, а~відповідно і~формації, є панівна форма власності, оскільки саме вона визначає спосіб поєднання працівника із засобами виробництва.

		Формаційний підхід передбачає, що розвиток людського суспільства відбувається як послідовна зміна одного способу виробництва іншим:
		\begin{enumerate}
			\item Первіснообщинний;
			\item Рабовласницький;
			\item Феодальний;
			\item Капіталістичний;
			\item Комуністичний.
		\end{enumerate}

		Формаційний підхід виходить із того, що вирішальна роль у~суспільному розвитку належить процесу виробництва, відносинам власності, а~його головною рушійною силою є протиріччя між продуктивними силами й~виробничими відносинами і~загострення класової боротьби в~суспільстві.

		Однак у~сучасних умовах формаційний підхід при визнанні певних його положень вважають методологічно недостатнім і~справедливо піддається критичному аналізу.
		
		По-перше, п'ятиланкова періодизація розвитку суспільства не має всеохоплюючого значення. Вона більш-менш прийнятна в~основному для країн Західної Європи, але не відображає повною мірою своєрідності розвитку азійського способу виробництва, еволюції цивілізацій Китаю, Індії, а~також не висвітлює особливостей історичного розвитку Росії, України.

		По-друге, формаційний підхід не розкриває багатоваріантності життя, збіднює історію людського суспільства, зводячи її в~основному до одного фактора~— розвитку матеріального виробництва, практично не враховує ролі соціокультурного та~інших неекономічних факторів у~розвитку суспільства (національних, релігійних, етнічних, ментальних тощо).

		По-третє, уявляючи історію розвитку людства як процес «революційного» руйнування старого способу виробництва і~заміни його новим, формаційний підхід, таким чином, припускає певну перервність (дискретність) природно-історичного процесу.

		По-четверте, формаційний підхід надмірно абсолютизує класову конфронтаційність між власниками і~не власниками, між роботодавцями і~найманими працівниками.

		На потребу глибшого наукового пізнання закономірностей розвитку суспільства світова суспільна наука розробила і~широко використовує цивілізаційний підхід щодо пізнання історії розвитку людства.
		
		Цивілізація~— історично конкретний стан суспільства, який характеризується досягнутим рівнем продуктивних сил, особливою формою виробництва і~відповідною духовною культурою людей.

		Цивілізаційний підхід по-іншому визначає закономірні ступені розвитку економічних систем. В основу цивілізаційного підходу покладено такі принципи:
		\begin{enumerate}
			\item Багатовимірності аналізу економічних систем.
			\item Природної еволюційної поступовості історичного процесу. 
			\item Відмови від класових, конфронтаційних оцінок змісту і~цілей системи.
			\item Пізнання системи в~єдності її економічних і~соціокультурних елементів.
			\item Посилення ролі людського фактора у~суспільному розвитку.
			\item Визнання світової історії як єдиного планетарного цілого.
		\end{enumerate}

		Як бачимо, цивілізаційний підхід не страждає економічним детермінізмом, оскільки передбачає рівномірність впливу та~інших чинників на розвиток людського суспільства. Він зорієнтований не на особливості способу виробництва, а~перш за все на цілісність людської цивілізації, домінуюче значення загальнолюдських цінностей, інтегрованість кожного суспільства у~світову спільноту.

		Найважливішою рисою цивілізації є її гуманістична спрямованість. Людина виступає не лише головним суб'\-єктом виробництва та~цивілізації в~цілому, але і~їх безпосередньою метою, цільовою функцією.

		Цивілізація~— категорія історична, її рівень і~сутність визначаються тими багатогранними со\-ці\-аль\-но-еко\-но\-міч\-ни\-ми умовами, в~яких відбувається життєдіяльність того чи іншого народу.

		Американський економіст У.\,Ростоу висунув теорію «стадій росту», в~якій виділив п'ять стадій економічного розвитку:
		\begin{enumerate}
			\item Традиційне суспільство.
			\item Перехідне суспільство.
			\item Стадія зрушення.
			\item Стадія зрілості.
			\item Стадія високого рівня масового споживання.
		\end{enumerate}

% 50.Історична періодизація суспільного розвитку: осьовий і~хвильовий підходи. 
	\section{Історична періодизація суспільного розвитку: осьовий і~хвильовий підходи}
		Сучасний американський соціолог Олвін Тоффлер підкреслив безпосередній зв'язок між змінами техніки й~способу життя. Техніка, як він вважає, обумовлює тип суспільства й~тип культури. Причому вплив техніки має хвилеподібний характер. Перша, аграрна, «хвиля» тривала тисячоріччя. У~схемі Д.\,Белла, у~якій зіставляються закрите й~відкрите, традиційне й~сучасне суспільство, їй відповідає образ традиційного суспільства. Головний зміст другої «хвилі»~— заводське виробництво. Нинішня, третя, «хвиля» асоціюється з~«інформаційним суспільством». Вона викликана повсюдним поширенням комп'ютерів, турбореактивної авіації, гнучких технологій. В «інформаційному суспільстві» складаються нові види родини, стилі роботи, життя, нові форми політики, економіки й~свідомості. Олвін Тоффлер стверджує: бурхливий технічний прогрес, небачені досі досягнення науки, розвиток систем глобального зв'язку, і~водночас~— численні природні катастрофи, які не в~останню чергу зумовлені непродуманою людською діяльністю, що лиш констатують загальну екологічну кризу та~ставлять людство на межу самознищення~— це свідчення подальших змін у~розвитку суспільства, свідчення про прихід третьої хвилі цивілізації~— інформаційної. Інформаційна революція створює новий поділ цивілізацій на «швидкі» та~«повільні» економіки. Інформаційне суспільство~— соціологічна і~футурологічна концепція, що вважає головним чинником суспільного розвитку виробництво і~використання науково-технічної й~іншої інформації. Концепція інформаційного суспільства є різновидом теорії постіндустріального суспільства, основу якої заклали 3.\,Бжезинский, Д.\,Белл, О.\,Тоффлер.

		Автор концепції «осьового часу» К.\,Ясперс стверджував, що вісь світової історії слід віднести до часу близько 500 років до~н.е., до того духовного процесу, який йшов між 800 і~200~pp. до~н.е. Тоді відбувся найрізкіший поворот в~історії. З'явилася людина такого типу, який зберігся і~до цього дня. Цей час називається осьовим. В цей час відбувається багато надзвичайного~— в~Китаї жили Конфуцій і~Лао-цзи, виникли всі напрями китайської філософії, мислили Мо-цзи, Чжуан-цзи, Ле-цзи і~інші. В Індії виникли Упанішади, жив Будда; у~філософії~— в~Індії, як і~в Китаї,~— були розглянуті всі можливості філософського збагнення дійсності, аж до скептицизму, до матеріалізму, софістики і~нігілізму; в~Персії Заратустра учив про світ, де йде боротьба добра із злом; в~Палестині виступали пророки~— Ілія, Ісайя, Ієремія і~Второїсайя; в~Греції~— це час Гомера, філософів Парменіда, Геракліта, Платона, трагіків, Фукидіда і~Архімеда. Все те, що пов'язано з~цими іменами, виникло майже одночасно протягом небагатьох сторіч в~Китаї, Індії і~на Заході.

		Спрямованість культурно-історичного процесу завжди має альтернативні можливості, і~яка з~них стане дійсністю, до яких наслідків вона приведе, залежить від багатьох випадковостей і~дій конкретних людей, які завжди мають власні цілі, хоча й~можуть діяти неусвідомлено або навіть несвідомо. Вибір напрямку руху суспільної системи залежить від багатьох чинників і~ними зумовлених обставин, враховує свободу людини як її сутнісну визначеність, істотну рису людського буття, свободу як вияв здатності до самоорганізації, певного механізму саморегуляції культурно-історичного руху.
		
% 51. Поняття політики та~політичної системи суспільства. Основні елементи політичної системи. 
	\section{Поняття політики та~політичної системи суспільства. Основні елементи політичної системи}
		Політика та~політична система суспільства у~своїх багатоманітних проявах є предметом вивчення різних галузей наукового знання. 
		Наприклад, політологія вивчає суспільство та~його політичне життя як об'\-єктивне утворення, систему з~притаманними їй закономірностями.
		
		Філософія намагається, перш за все, з'ясувати питання генезису політики як специфічного виду діяльності людини.
		
		Якщо звернутися до терміну «політика», то він має походження від грецького слова «polis», що означає буквально «місто» або «місто-держава». У~творчості античних філософів є різні трактування цього поняття. Платон розглядає політику як «мистецтво управління людьми». Аристотель у~трактаті «Політика» стверджує, що політика~— це діяльність, спрямована на досягнення загального блага, щасливого життя людей в~межах держави. Людину ж він визначає як «суспільну», «політичну істоту». У~Давній Греції політикою називалася діяльність, пов'язана з~організацією господарського і~громадського життя «полісу».
		
		Відомий філософ ХVІ століття Н.\,Макіавелі в~праці «Государ» практично ототожнює поняття «політика» і~«влада». Відповідно його твердження, людиною рухає насамперед жага влади, вона прагне спочатку її придбати, а~потім усі її політичні дії націлені на збереження цієї влади.
		
		Марксистська теорія запропонувала ідею виникнення політики на історичному етапі розвитку суспільства. З~появою економічної й~соціальної нерівності між людьми структура суспільного життя ускладнюється, вона охоплює відносини між класами, націями і~державами. Основним суб'\-єктом політики виступає панівний клас суспільства, саме він визначає зміст і~спрямованість політики держави.
		
		Приступаючи до вивчення питання про сутність політики та~складової політичної системи суспільства треба звернути увагу на те, що у~змістовому відношенні політична система включає в~себе різні елементи, об'єднані між собою однією, спільною основою~— сферою політичного життя. Основними її структурними елементами є:
		\begin{enumerate}
			\item Політичні відносини.
			\item Політична влада.
			\item Політична діяльність.
			\item Політична організація суспільства.
			\item Політична свідомість.
			\item Політична культура, тощо.
		\end{enumerate}

		Треба також звернути увагу на той факт, що чим вище рівень розвитку суспільства, тим складнішою за своєю структурою і~змістом є його політична система.
		
		Політична система, її зміст, характер розвитку і~спрямованість завжди детерміновані рядом факторів. Найважливішими серед них є політичний інтерес, політична цілеспрямованість, об'\-єкт і~суб'\-єкт політики. Політика становить собою найсуттєвішу іпостась соціальної організації людей.
		
		Що являє собою політика? Перш за все, політика має пряме відношення до управління, якого потребує будь-яка система. Узгодження взаємопов'язаних функцій елементів, підпорядкованість самозбереженню, реалізація відповідних тенденцій розвитку потребує організації системи на певній управлінській основі, що передбачає підкорення, виконання команд і~тому подібних управлінських операцій, які долають хаотичність що періодично виникає в~системі.
		
		В системних теоріях (синергетиці, кібернетиці) функції управління докладно досліджені, вони мають пряме відношення до функціонування соціальних систем.  Управління в~них набуває статусу політичної дії. Для збереження соціуму необхідна корекція процесів управління, яка і~існує у~формі політики.
		
		В основі явища політики лежать владні стосунки між людьми. В сучасних визначеннях політики це всіляко підкреслюється. Принаймні її не розглядають виключно як функцію похідну від економіки. При будь-якому суспільному устрої політична проблематика не зійде на ніщо, оскільки залишається питання: хто панує, як комплектуються органи управління, як здійснюється влада, якою мірою досягнуто згоду або яка міра незгоди між тими, хто управляє і~хто їм підкоряється. Політика така ж важлива як економіка.
		
		Отже, політика~— це сфера діяльності, яка пов'язана з~відношеннями між соціальними групами, в~центрі яких стоїть проблема завоювання, утримання і~використання державної влади.
		
		Саме влада, як форма стосунків між людьми, є основою політики. Влада є невід'ємним атрибутом людської, а~тим паче соціальної нерівності людей. Вона існує як влада над собою, так і~як влада над іншими.
		
		Щоб усвідомити сутність політичної системи суспільства, треба звернути увагу на той факт, що політична система суспільства~— явище історичне.
		
		Суспільні відносини між різними соціальними групами і, відповідно, політика, яка відображає корінні інтереси цих груп, випливають з~їх місця в~економічному житті суспільства. Політика є надбудовою над економічним базисом. У~ній найбільш повно й~глибоко відображаються корінні економічні інтереси різних соціальних груп. Тому вона є концентрованим вираженням економіки, її узагальненням і~завершенням. Політика здійснює великий вплив на економіку і~всі інші сфери суспільного життя. Яскравим доказом цього є соціально-політичні процеси, які відбуваються нині на етапі становлення української державності.
		
		Політика як суспільне явище виконує ряд важливих функцій. До них необхідно віднести:
		\begin{enumerate}
			\item Вираження політично значущих інтересів усіх соціальних суб'\-єктів;
			\item Управління соціально-політичними процесами в~суспільстві;
			\item Визначення пріоритетів розвитку суспільства і~забезпечення у~ході їх реалізації гармонії інтересів соціальних груп та~окремих індивідів;
			\item Узгодження інтересів різних соціальних груп населення і~відвернення конфліктів, збереження цілісності й~стабільності функціонування соціальної системи.
		\end{enumerate}
		
		Таким чином, політика~— багатовимірне соціальне явище. В процесі суспільного розвитку змінюються суб'\-єкти політичних відносин, функції політики, форми політичного устрою й~правління, характер політичних організацій, відносин, ідеологій.
		
		Разом з~політичними відносинами формується і~політична система суспільства. Це система дер\-жав\-но-пра\-во\-вих, політичних і~громадських інститутів, установ і~організацій, за допомогою яких регулюються політичні відносини між державами, народами, націями та~іншими спільностями людей. Її елементами здійснюються завоювання, утвердження і~функціонування політичної влади в~залежності від політичної культури суспільства. Основним політичним інститутом виступає держава.

% 52. Держава як основний політичний інститут.
\section{Держава як основний політичний інститут}
	Проблема природи і~походження держави перебуває в~центрі уваги і~сьогоденної філософської думки не зважаючи на величезний матеріал, що міститься у~творах Ксенофонта, Протагора, Демокрита, Платона, Аристотеля, Цицерона, Бодена, Гоббса, Локка, Монтеск'є, Фіхте, Гегеля, Маркса, Спенсера, Вебера, Ніцше, Фрейда та~інших.
	
	У період античності найбільш вагомий вклад до тлумачення проблеми держави внесли Платон і~Аристотель. Платонівська теорія, викладена у~«Державі» і~»Законах», була нормативною теорією «ідеальної держави», побудованої на принципах добра і~справедливості. Його ідеальна держава засновується на знанні про доброчесність. Ідеальною формою правління за Платоном є монархія. Філософ вважав, що держава стане досконалою лише тоді, коли до влади прийдуть носії найбільшого знання~— філософи. Аристотель, навпаки, вважав верховенство закону ознакою досконалої держави. На його думку конституційне правління, а~не монархічне є ідеальною формою у~державі. Філософ твердив, що закон є необхідною умовою цивілізованого життя.
	
	Вагомий внесок щодо розробки теорії про державу як публічно-правову спільноту зробив Цицерон. Держава за Цицероном~— це «справа народу». Тобто держава постає не лише як вираження загального інтересу всіх громадян, що було характерним для давньогрецької концепції, а~й~як узгоджене правове спілкування цих громадян, як певне правове утворення.
	
	Суттєвий вклад до розробки проблеми держави як основного політичного інституту зробив відомий німецький філософ І.\,Кант. Згідно його теорії, держава~— це об'єднання значної кількості людей, підпорядкованих дії правового закону. Перспективи, темпи розвитку держави залежать від того, наскільки її державний устрій узгоджений із правовим принципом.
	
	У філософії Гегеля держава постає конституційною монархією з~поділом влад на законодавчу, урядову і~владу монарха. Його ідеалом була органічна єдність влад. Саме в~пануванні цілого, у~залежності й~підпорядкуванні різних влад державній єдності, на думку Гегеля, і~полягає сутність внутрішнього суверенітету держави. Гегель захищав суверенітет державно-правового цілого, піддаючи критиці беззаконня.
	
	Звернення до філософії Нового часу дозволяє відновити державу у~правах засобу інтеграції суспільства, його вищої управлінської ланки і, нарешті, організації для забезпечення безпеки членів суспільства. При повсюдному прагненні індивіда до влади виникає потреба в~обмеженні сваволі окремих осіб, що може виявитись згубною для інших людей і~людської спільноти взагалі. В зв'язку з~цим державі властиві функції придушення природних потягів індивідів. М.\,Штирнер з~цього приводу зауважував: «Власна воля і~держава~— дві сили, ворожі одна одній не на життя, а~на смерть, недоречно думати про встановлення між ними «вічного миру».
	
	Але на те й~існує політика, щоб регулювати, послаблювати, змінювати спрямування цього репресивного тиску. Саме вона є «вид діяльності, шляхом якої люди перерішують свої долі і~змінюють статус в~суспільстві».
	
	Засобами її здійснення забезпечується зворотний вплив індивіда на державну владу.
	
	Основними ознаками держави є:
	\begin{enumerate}
		\item Територіальний принцип закріплення населення замість кровно родинного, який був панівним до виникнення держави; наявність кордонів, які фіксують територію держави, де діє її юрисдикція~— правові повноваження, що поширюються на всіх людей на цій території;
		\item Наявність особливої системи органів і~установ, які здійснюють функції державної влади~— публічної, яка стоїть над населенням. До цих органів зокрема належить спеціальний апарат примусу: армія, поліція, служба безпеки, суд, прокуратура тощо;
		\item Система податків з~населення, необхідних для відкриття джерела надходження коштів, потрібних для утримання державного апарату, здійснення влади в~різних сферах суспільного життя.
	\end{enumerate}

	Держава як орган політичної влади й~управління пройшла тривалий шлях свого розвитку. Відомі такі її історичні типи, як рабовласницький, феодальний, буржуазний, соціалістичний.
	
	Крім держави до політичної системи суспільства відносяться також політичні партії, суспільно-політичні рухи, профспілки, громадські організації тощо.
	
	В сучасному суспільстві політичним партіям належить вирішальна роль в~організації соціальних сил для політичної боротьби, у~здійсненні державної влади.
	
	Необхідно звернути увагу на те, що важливим елементом політичної системи суспільства є також політична свідомість і~політична культура. Політична свідомість~— це відображення людиною чи певними спільностями людей в~ідеальних образах політичного життя, політичних відносин як різновидності суспільних відносин в~цілому.
	
	Кожний суб'\-єкт політики має свої інтереси і~їх контролює. Зіткнення різних інтересів відбувається на перехрестях боротьби за політичну владу. Саме боротьба за владу та~її устрій і~є центральною проблемою політичного мислення і~політичної свідомості.
	
	При усвідомленні питання про державу слід мати на увазі, що держави відрізняються за формою правління, устрою і~політичного режиму.
	
	Форма правління~— це система вищих органів державної влади, спосіб їх утворення, порядок здійснення ними державної влади. Історично склалися дві основні форми правління:
	\begin{enumerate}
		\item Монархія, де вища державна влада здійснюється однією особою і~передається у~спадок. У~межах монархічної форми правління виділяються:
		\begin{enumerate}
			\item Абсолютна монархія~— всевладдя глави держави.
			\item Конституційна~— коли глава держави наділений переважно виконавчою владою, або фактично тільки представницькими функціями.
		\end{enumerate}
		\item Республіка~— така форма, при якій вища державна влада здійснюється виборним колегіальним органом, що обирається населенням на певний строк.
	\end{enumerate}
	
	Пряме відношення до можливостей впливу індивіда на державну політику має електорат, виборці державних органів, права і~обов'язки яких докладно вивчаються відповідними розділами соціології.
	
	Підсумовуючи питання, необхідно, на нашу думку, підкреслити, що найпрогресивнішим спрямуванням еволюції держави щодо особистості є створення держави, основною характеристикою якої є орієнтація на особистість, захист її прав, усунення невиправданих обмежень її буття.
	
	Шлях до реалізації можливостей існування такої форми державності тернистий. Адже вона виникає на підставі відповідного рівня розвитку саме індивідів як елемента суспільності. На типі державного устрою позначаються не лише культурні досягнення суспільства, але і~ступень культури особистості, а~також типологічні особливості індивідів і~їх «соціальний характер», про який докладно говорить відомий представник «гуманістичного психоаналізу» Е.\,Фромм. Радикальний гуманізм Фромма полягає у~вимозі докорінно змінити внутрішню природу людини, що виявляється у~переході від домінуючої установки на «володіння» до наступної установки на «буття».

% 53. Громадянське суспільство й~держава.
	\section{Громадянське суспільство й~держава}
		Ідея громадського суспільства закладена ще філософами античності. Зокрема вона розроблялась у~творчості Цицерона. Сучасні уявлення про громадське суспільство ґрунтуються на положенні Гегеля про розрізнення понять «держава» і~«суспільство».
		
		Громадське суспільство~— це суспільство в~якому існує і~постійно розширюється сфера вільного волевиявлення, яке сприяє розкриттю внутрішнього потенціалу людей і~досягається через систему інституцій і~відносин, покликаних забезпечити умови для самореалізації окремих індивідів та~їхніх об'єднань.
		
		Основою громадського суспільства є вільні громадяни та~їхні добровільні об'єднання, існування яких регулює не політична влада, а~самоуправління, вільне волевиявлення громадян і~правовий закон.
		
		Громадське суспільство має складну структуру: це комплекс соціальних груп, приватних осіб, їх асоціацій та~інститутів (школа, сім'я, церква, добровільні об'єднання за професійними, віковими, творчими та~іншими інтересами, клуби, спілки, товариства, політичні партії, тощо), взаємодія яких регулюється правом.
		
		Громадське суспільство забезпечує простір для самореалізації індивіда поза державними структурами. Тому таке суспільство часто розглядають як сферу суспільного буття, не охоплену безпосередньо діяльністю держави. Однак вони не ізольовані, а~взаємно доповнюють одне одного. Громадське суспільство сприяє виникненню й~конденсації громадських ініціатив, які підтримують, коригують діяльність державного організму. В процесі становлення і~розвитку громадського суспільства відбувається заміна традиційних форм регулювання життєдіяльності людей, утверджуються демократичні, правові норми в~усіх сферах суспільного буття. Це дає підставу стверджувати, що громадське суспільство і~правова держава є одним цілим і~виражають міру демократизації політичного життя та~політичної системи як її правового механізму.
		
		Отже, суспільство, що прагне до правової держави і~ефективної політичної влади, в~першу чергу повинно не обмежувати можливості високих устремлінь особистостей, а~навпаки, стимулювати їх шляхом забезпечення відповідних матеріальних і~духовних умов.

% 54. Проблема взаємозв'язку свободи і~необхідності в~житті людини та~суспільства.
	\section{Проблема взаємозв'язку свободи і~необхідності в~житті людини та~суспільства}
		Суспільство, як і~природа, розвивається закономірно. Закони суспільного життя, подібно до законів природи, існують і~діють, незалежно від свідомості й~волі людини. Люди не можуть створювати або знищувати їх. Вони можуть лише відкривати, пізнавати їх і~використовувати в~процесі суспільної практики.
		
		Разом з~тим, закономірності суспільного життя не тотожні закономірностям природи. В природі закони діють стихійно, у~суспільстві~— прокладають собі дорогу через свідому діяльність людини, яка ставить перед собою певну мету з~тим, щоб її реалізувати. Це зовсім не означає, що закони втрачають свою об'\-єктивність. Вони діють так само нездоланно, неминуче, як і~в природі. Це~— історична необхідність. Але закони суспільства, їх дія проявляються як тенденція, що зумовлена взаємодією об'\-єк\-тив\-них умов і~суб'\-єктивного фактора.
		
		Під об'\-єктивним фактором розуміють такі умови, які не залежать від волі й~свідомості людей і~визначають напрями й~межі їх діяльності. Це передусім природні умови регіону, досягнутий рівень розвитку продуктивних сил, історично назрілі потреби суспільного розвитку та~ін.
		
		Суб'єктивним фактором є діяльність народних мас, держави, класів, політичних партій, громадських рухів, окремих осіб~— їх свідомість, воля, рівень розуміння об'\-єк\-тив\-них потреб розвитку суспільства тощо.
		
		Люди у~своїй діяльності змушені зважати на об'\-єктивні умови. Тільки їх врахування в~конкретних історичних обставинах дає змогу вирішувати проблеми суспільного життя й~розвитку. Разом з~тим, наявність об'\-єк\-тив\-них умов недостатня для перемоги нового, перетворення можливості в~дійсність. Рушійною силою історичного процесу завжди виступають прогресивні, революційні елементи суб'\-єктивного фактора. Своєю активно-творчою діяльністю вони дають простір для дії об'\-єк\-тив\-них законів розвитку. Проте поряд з~ними є й~такі елементи, які перешкоджають соціальному прогресу~— фашизм, расизм, неонацизм та~ін.
		
		Дію об'\-єк\-тив\-них умов та~суб'\-єк\-тив\-них факторів у~суспільному розвитку слід розглядати нероздільно, в~їх діалектичній єдності. Абсолютизація ролі свідомості, волі людей чи соціальних інституцій, ігнорування об'\-єк\-тив\-них умов і~законів веде до волюнтаризму та~авантюризму. Фетишизація об'\-єк\-тив\-них законів і~умов при ігноруванні ролі суб'\-єктивного фактора породжує фаталізм, схиляння перед стихійністю.
		
		Співвідношення історичної закономірності та~свідомої діяльності людей треба розглядати в~діалектичному взаємозв'язку, який розкривається в~категоріях необхідності й~свободи.
		
		Історична необхідність~— це те, що закономірно випливає з~дії об'\-єк\-тив\-них законів розвитку суспільства. Вона впливає на дії, вчинки людей, які, в~подальшому, зворотньо впливають (як позитивно, так і~негативно) на цю необхідність, відкриваючи шляхи для розширення своєї свободи.
		
		Свобода є продуктом історичного розвитку людства. Дії об'\-єк\-тив\-них законів людина уникнути не може. Свобода не означає протиставлення суб'\-єкта цим законам або «звільнення» від них. Такий шлях до свободи є ілюзорним. Реальна свобода досягається шляхом пізнання і~використання необхідності. Свобода людини, як підкреслював Ф.\,Енгельс, полягає не в~уявній незалежності від законів природи Й суспільства, а~в~пізнанні їх та~вмінні використовувати у~своїй діяльності.
		
		Оволодіння природною необхідністю реалізується в~розвитку продуктивних сил. їх прогрес можна інтерпретувати як поступовий процес звільнення людства від підкорення стихійним силам природи, тобто розширення свободи суспільства по відношенню до природи. Відношення до природи завжди опосередковане певною формою суспільних відносин, законами суспільного розвитку. Свобода суспільства визначається мірою оволодіння цими законами, їх перебування під свідомим контролем, подолання панування соціальної стихії. В міру розвитку суспільства, його продуктивних силі виробничих відносин свобода розширює свої межі, панування людини над природними і~суспільними процесами посилюється.
		
		У живій та~неживій природі відсутні приклади абсолютної свободи, свободи від усього, від будь-якої залежності. Ще складнішим і~багатограннішим є соціальний зміст поняття свободи. Аналізуючи його, необхідно підходити завжди конкретно-історично. Людина як частина природи і~суспільства всіма своїми діями вплітається в~різноманітні відносини з~природою та~суспільством, державою і~нацією, класом і~партією, трудовим колективом та~сім'єю. Тому при визначенні свободи особи природно виникають питання: свобода від чого і~від кого, а~також свобода для чого і~для кого? Свобода в~чому?
		
		Абсолютна свобода неможлива не тільки тому, що людина включена у~всезагальні зв'язки з~природою і~суспільством, а~ще й~тому, що завжди обмеженими є її власні можливості. Природні й~соціальні умови завжди обмежували і~будуть обмежувати свободу особи. Але можливості останньої постійно розширюються. Людина не може бути повністю незалежною від зовнішніх природних і~суспільних умов. Це означає, що вона завжди має лише відносну свободу. При цьому, зрозуміло, рівень її свободи залежить від міри свободи суспільства, в~якому вона живе.
		
		Для розуміння свободи важливе значення має пізнання необхідності. Однією із особливостей взаємозв'язку свободи і~необхідності є те, що необхідність виступає основним елементом свободи, її об'\-єктивним змістом. Характеристика свободи як пізнаної необхідності складає сутність її гносеологічного аспекту. Однак свободу не можна зводити тільки до необхідності. Досягнення свободи пов'язане не лише з~пізнанням дійсності, а~й~з активною практичною діяльністю людини. Історична необхідність, на відміну від необхідності в~природі, реалізується в~діяльності людей. Необхідність обмежує свободу діяльності і~разом з~тим утверджує її.
		
		Існування необхідності в~реальному світі створює умови для свідомої діяльності особи, для вибору. Людина в~процесі та~в~результаті своєї діяльності пізнає необхідність не тільки природи, але й~людського суспільства. В результаті пізнання необхідності в~матеріальному світі особа спрямовує свою діяльність у~відповідність з~нею. В процесі практичної діяльності наше пізнання необхідності може коригуватися, доповнюватися, уточнюватися тощо. За рахунок цього відбувається розширення нашої свободи, яка необхідна нам для розвитку. Чим повніше ми пізнаємо природні й~суспільні явища, власну природу, потреби, ідеали і~таке ін., тим ефективніше використовуємо свої сили, тим ширшої свободи набуваємо.
		
		На шляху розвитку людства простежуються більш-менш чіткі історичні етапи зростання свободи. Як уже говорилось, для їх позначення К.\,Маркс ввів поняття «суспільно-економічна формація», показавши, що зміна однієї з~них на іншу~— більш досконалу є послідовним і~закономірним історичним процесом зростання свободи людства, де в~історичній перспективі майбутнє людства~— реальний гуманізм (тобто комунізм)~— завершиться утвердженням дійсної свободи людства.
		
		При характеристиці історичного процесу набуття свободи людством поряд з~категорією «формація» широко використовуються такі категорії, як «епоха» і~«цивілізація». Якщо «формація» фіксує і~характеризує певний історичний тип окремого суспільства, то «епоха»~— інший відрізок всесвітньої історії, контролюючи при цьому провідну (для того часу) тенденцію суспільного розвитку. Ця тенденція може обмежуватися однією формацією або навіть її певною частиною, а~може охоплювати міх формаційні періоди розвитку людського суспільства (епоха первісного суспільства, епоха феодалізму, епоха первісного нагромадження капіталу, епоха монополістичного капіталізму, епоха переходу від капіталізму до соціалізму тощо).  Стосовно культурного феномену розрізняють, наприклад, епоху Відродження, епоху Просвітництва і~таке ін.
		
		Цивілізаційний підхід характеризує не просто рівень розвитку суспільства, а~привертає увагу й~до ступеня розвитку його матеріальної та~духовної культури; це особливий стан організації суспільного життя, який характеризується з~різних сторін, як зростанням матеріального добробуту (благополуччя), так і~невпинним удосконаленням моралі та~розширенням меж людської свободи (істини й~краси).

% 55. Поняття духовного життя суспільства. Специфіка духовних відносин і~духовної культури.
	\section{Поняття духовного життя суспільства. Специфіка духовних відносин і~духовної культури}
		Духовне життя суспільства надзвичайно складне. Воно не обмежується різноманітними проявами свідомості людей, їх помислів та~почуттів, хоча свідомість є стрижнем, ядром духовного життя людини і~суспільства.
		
		До основних елементів духовного життя суспільства відносяться духовні потреби людей, спрямовані на створення певних духовних цінностей, духовне виробництво в~цілому. До елементів духовного життя відноситься також споживання духовних цінностей; сюди ж відноситься і~міжособистісне духовне спілкування.
		
		Основу духовного життя суспільства складає духовна діяльність, яку можна розглядати як діяльність свідомості, в~процесі якої виникають ті чи інші думки й~почуття людей, їх образи та~уявлення про природні та~соціальні явища. Результатом цієї діяльності стають певні погляди людей на світ, наукові ідеї і~теорії, мораль, мистецтво, релігія тощо. Вони втілюються в~моральних принципах та~нормах поведінки, творах народного та~професійного мистецтва, в~релігійних обрядах, ритуалах і~т.п.
		
		Особливим видом духовної діяльності є розповсюдження духовних цінностей з~метою їх засвоєння людьми, що має вирішальне значення для підвищення їх освіченості, культури. Основною рушійною силою духовної діяльності є духовні потреби~— внутрішні спонукання людини до духовної творчості, до створення духовних цінностей та~їх споживання, до духовного спілкування.
		
		Виробництво і~споживання духовних цінностей опосередковується духовними відносинами. Вони існують як відносини людини безпосередньо до тих чи інших духовних цінностей, тобто~— схвалює вона їх чи відкидає, а~також відносини між людьми щодо цих цінностей: до їх виробництва, розповсюдження, споживання, збереження.
		
		Духовні відносини опосередковують будь-яку духовну діяльність. Тому виділяються такі види духовних відносин, як пізнавальні, моральні, естетичні, релігійні та~інші. Отже, вони є перш за все відносинами розуму і~почуттів людини до тих чи інших духовних цінностей і, відповідно,~— до всієї дійсності. Ті духовні відносини, які існують у~суспільстві, проявляються у~повсякденному міжособистісному спілкуванні людей у~будь-яких сферах їх життєдіяльності~— у~сімейній, виробничій, міжнаціональній тощо; вони створюють інтелектуальний та~емоційно-психологічний фон міжособистісних відносин і~багато в~чому визначають його зміст.
		
		Духовна культура~— це досягнення, що виражають рівень та~глибину духовно-теоретичного освоєння дійсності, це сукупність досягнутого суспільством у~сфері морального та~художнього життя, в~науці і~філософії, політиці та~праві тощо. Такі явища духовного життя, як мова, логіка мислення, правила поведінки, рівень та~глибина почуттів також характеризують духовну культуру суспільства.
		
		Духовна культура як елемент духовного життя, суспільних, духовних відносин включає в~себе певну систему цінностей, знань, переконань, світоглядних орієнтацій, норм, традицій в~органічній єдності з~соціальною гуманістично значимою діяльністю людей щодо освоєння, творення буття. Духовна культура створюється діяльністю соціальних суб'\-єктів і~спрямована на перетворення суспільного буття, розвиток сутнісних сил людини, зокрема її духовності, їх всебічну самореалізацію; це не тільки свідомість, а~й~соціальна активність,  перетворююча діяльність особистості, яка вимірюється обсягом створюваних нею духовних, соціогуманістичних цінностей. Така культура свідчить про здатність кожної особистості до сприйняття передового, прогресивного в~суспільному бутті, так само як і~до його поширення, творення у~відповідності з~творчими силами та~здібностями кожної індивідуальності; про готовність особистості до самовіддачі, саморозвитку своєї духовності як свого особистого, так і~всього суспільства.
		
% 56. Культура як символічний світ людського буття. Співвідношення культури й~цивілізації.
	\section{Культура як символічний світ людського буття. Співвідношення культури й~цивілізації}

		Існує більше ніж  півтисячі визначень феномену культури.  Першим  термін «культура» запровадив у~філософію Цицерон у~значенні «мистецтво вдосконалення своєї душі, розуму», протиставивши його терміну «натура», тобто «природа». Зараз у~філософії існують різні підходи до розуміння культури. Релігійно-ідеалістична філософія акцентує увагу на духовному аспекті культури, визначає її як вираження прагнення людини добудувати своє природне та~соціальне буття до  рівня духовності.
		
		Матеріалістична філософія розглядає  як домінантний діяльнісний аспект культури, її орієнтованість на людину, що виступає творцем і~споживачем культурних цінностей. Спільним у~всіх підходів до  розуміння культури є підкреслення того, що світ культури створений людиною, природа входить у~нього як перетворена людиною. Г.\,Гегель  особливо підкреслив, що «культура~— це створена людиною „друга природа“». 
		
		Можна сказати, що в~культурі конституюється світ, який складає безпосередню дійсність буття людини. Тільки через входження в~нього людина стає людиною~— творцем культури, суспільства, себе на основі засвоєння надбань, уже створених людством. Людина є суб'\-єктом і~об'\-єктом культури. Культура орієнтує людину в~світі, вміщує  її в~світ суспільства, людства; регулює відносини в~суспільстві; визначає  систему цінностей. У~широкому значенні під культурою розуміються всі основні сфери людського буття~— виробництво матеріальних благ;  духовна культура в~її розмаїтті та~складності,  включаючи  міфологію, релігію, науку, мистецтво; різні форми стосунків у~суспільстві~— від особистісних до політичних.
		
		Але культура не тотожна суспільству. Культура перебуває у~діалектичній взаємодії із суспільством. Вона функціонує за власними законами і~в той же час підлягає дії суспільних законів. На стан культури впливають спосіб виробництва, політична система суспільства, становище індивіда в~суспільстві, тощо.
		
		Якщо розглядати культуру як певну людську діяльність, то можна виділити в~ній матеріальну і~духовну культуру, що взаємодіють у~процесі реалізації людини як суспільної істоти і~неповторної індивідуальності. Якщо ж розглядати культуру з~точки зору її носія, то можна виділити світову культуру, яка складається з~етнічних та~національних культур. 
		
		Отже, культура  виступає як специфічний спосіб існування людини в~світі, є засобом взаємозв'язку з~природою. 
		
		Усі сфери суспільного життя~— це сфери культури, а~культура є символічним світом, створеним людиною. Йдеться не про одного індивіда, а~про спів-буття людей, які в~процесі спілкування, яке є системою всіх  суспільних відносин, вибудовують світ свого буття. Це вже не природний, а~позаприродний світ, в~основі якого лежить формування певних символів.
		
		Символи виступають як умовні знаки для позначення загальних ідей, кількість значень у~нього~— необмежена. А~отже, культура як світ символів теж є необмеженою. Матеріальна культура також має символічне забарвлення. Культура має історичний характер, який виявляється в~створенні нових символів, і~в зміні значень уже наявних. Одні й~ті ж символи стають носіями нових значень. Так відбувається розширення семантичного поля. Окрім того з'являються нові символи. Особливе значення у~розвитку людської цивілізації відіграє мова, яка теж є системою символів і~знаків, у~яких закодовані результати людського пізнання та~діяльності.

% 57. Суспільна свідомість та~її структура: буденна й~наукова свідомість, суспільна психологія та~ідеологія.
	\section{Суспільна свідомість та~її структура: буденна й~наукова свідомість, суспільна психологія та~ідеологія}

		Свідомість~— це найбільш загальна категорія філософії, яка співвідноситься з~такою ж всезагальною категорією як матерія.  Суспільний характер свідомості як феномена людської життєдіяльності виявляється у~розмаїтті таких її форм, як філософія, релігія, наука, мистецтво, мораль, правосвідомість, соціальна психологія та~ін.
		
		Свідомість має складну структуру. Вона включає не лише усвідомлювані компоненти, але й~несвідомість, а~також самосвідомість.
		
		Носіями суспільної свідомості є суб'\-єкти суспільнох діяльності. Суспільна свідомість~— це усвідомлене суспільне буття. У~суспільній свідомості виділяють такі елементи, як: суспільна психологія та~ ідеологія; форми суспільної свідомості (політична, правова, моральна, естетична, релігійна, а~також філософія і~наука.) Люди судять про рівень духовної культури суспільства чи соціальної групи  через прояви суспільної свідомості. В ній виділяються наступні рівні: побутовий та~теоретичний.
		
		Побутовий рівень~— це нижчий рівень суспільної свідомості, який включає сукупність знань, уявлень суспільних почуттів, настроїв, емоцій, звичаїв, традицій, чуток, ілюзій і~т.~д., які стихійно виникають у~процесі повсякденного життя тієї  чи іншої соціальної групи. У~цих проявах суспільної свідомості виявляються, схоплюються зовнішні сторони суспільного життя.
		
		Теоретичний рівень~— це теоретично узагальнене й~обґрунтоване знання про суспільне життя. Воно включає в~себе суспільні ідеї, теорії, концепції, гіпотези, програми, тощо, які розкривають сутність тих чи інших суспільних явищ, тобто виявляють внутрішні закономірності розвитку суспільства.
		Суспільна психологія є частиною побутового рівня суспільної свідомості. Вона ґрунтується на психічному складі відповідної соціальної групи і~включає в~себе: соціальні почуття, настрої, звичаї, традиції і~т.~д., які формуються у~тих або інших класів, націй та~інших соціальних груп під впливом  матеріальних умов їх життя. 
		
		Суспільна психологія~— це стан масової свідомості, це погляд на суспільне життя мільйонів людей, а~ідеологія вносить у~масову свідомість теоретично обґрунтовані ідеї, концепції, програми, озброює людей теоретичними знаннями для усвідомлення стану справ у~суспільстві і~можливостей його подальшого розвитку.
		
		Ідеологія піднімається над почуттями, емоціями, віруваннями, заблудженнями, які народжуються у~повсякденній практиці, формує ідеали суспільства, цілі та~засоби їх досягнення.


% 58. Форми суспільної свідомості. 
\section{Форми суспільної свідомості}
		Форми суспільної свідомості~— це структурні елементи, що розрізняються за предметом, який вони відображають; за способом і~характером цього відображення та~за його результатами. Виділяють  такі форми: політичну свідомість, правову, моральну, естетичну, релігійну, науку та~філософію. Всі вони відображають різні сторони суспільного буття, а~отже, застосовують  різні методи і~засоби пізнання соціальної дійсності.
		
		Наприклад, політична свідомість відображає відносини між класами, націями, державами, народами з~приводу державної влади. Для цього використовуються засоби масової інформації, партії та~інші політичні організації. Але треба розрізняти політику як діяльність і~політичну свідомість як сукупність ідей, поглядів, програм і~ідеалів тощо.
		
		Правова свідомість відображає відносини між групами людей, між людиною і~суспільством, людиною й~державою з~приводу законного та~незаконного. Результатами виступають: юридичні норми, кодекси, закони, правові концепції, теорії, тощо.
		
		Моральна свідомість відображає відносини між групами, всередині груп, між людиною та~суспільством з~приводу справедливості, гідності, честі, обов'язку через моральні ідеали, норми, принципи, правила і~т.~д.
		
		Релігійна свідомість~— це уявлення про надприродне, в~яке людина вірить, від якого відчуває себе залежною. Вона виражається у~віруваннях, релігійних  вченнях, образах, символах, релігійній філософії. Провідне поняття цієї форми суспільної свідомості~— священне.
		
		Естетична свідомість відображає відносини між людьми з~приводу прекрасного і~потворного, красивого і~некрасивого, формує естетичні смаки, почуття, емоції, естетичні образи і~т.~д., які  виражаються у~творах мистецтва, літератури.
		
		Наука~— відображає внутрішні суттєві зв'язки у~суспільних явищах та~між ними; виробляє наукові поняття, концепції, прогнози суспільного розвитку.
		
		Філософія є узагальненим вченням про суспільне життя, його закони. Вона виробляє відповідні філософські категорії, принципи, теорії, концепції.
		
		Усі форми суспільної свідомості тісно зв'язані між собою, не існують відокремлено одна від одної.  Розділити їх можна лише умовно, як і~взагалі суспільне життя. Вони доповнюють одна одну, даючи в~єдності уявлення про суспільне життя і~перспективи суспільного розвитку.
		
	% 59. Особистість як міра соціальності в~людині. Особистість та~особа. Роль індивідуальних якостей у~процесі формування особистості. 
	\section{Особистість як міра соціальності в~людині. Особистість та~особа. Роль індивідуальних якостей у~процесі формування особистості}
		Проблема особистості є одною з~найскладніших у~філософії. Особистість~— «виступає як  динамічна, відносно стійка цілісна система інтелектуальних, соціально-культурних і~морально-вольових якостей людини, які виражені в~індивідуальних особливостях її свідомості і~діяльності.»~— визначає  І.\,Т.\,Фролов. Якщо провідним у~визначенні індивідуальності є її  неповторний вроджений талант, то у~особистості~— воля, самостійність. Людина виконує в~суспільстві визначені ним ролі, виступає як істота соціальна. Як особистість проявляє себе  у~ставленні до виконання своєї ролі або ролей в~суспільстві. Особистість втілює цінності своєї епохи, культури. Сприймаючи цінності  як власні, обираючи їх за основу свого світогляду і~діяльності, обирає роль в~суспільстві, чи виконує визначену так, щоб максимально реалізувати свою систему цінностей.  Особистість передбачає  самостійність діяльності на основі вільно обраних принципів і~відповідальності. Автономність~— одна з~провідних якостей особистості. Вона сама обирає цілі, обов'язки в~залежності від своїх переконань, інтересів, а~не від зовнішніх впливів. Свободу особистості забезпечує її  вольова готовність  здійснювати вибір і~брати на себе відповідальність за нього. В особистості домінує свідомо- вольовий початок, який забезпечує не тільки розуміння мети і~шляхів її досягнення, але і~силу для їх здійснення, незважаючи на всі  зовнішні перешкоди. Особистість діє на основі  усвідомлення сенсу життя. 
		
		Ю.\,Хабермас підкреслює, що людина, щоб стати особистістю, має перейти до відкритого морального керівництва життям. Вона повинна «…відважитись усвідомити свою індивідуальність і~свою свободу. Разом з~емансипацією з~стану оречовлення, через який людина відчуває почуття провини, вона набуває дистанції по відношенню до себе. Індивід виводить себе з~анонімно розсіяного, фрагментарного життя і~надає власному життю послідовність і~прозору ясність. В соціальному вимірі така особистість здатна відповідати за власну поведінку і~встановлювати зв'язки з~іншими особистостями. В часовому ж вимірі турбота про себе формує свідомість історичності екзистенції, яка здійснюється в~межах, одночасно визначених майбутнім і~минулим. Власне таким чином особистість, що усвідомила сама себе, розполагає собою як певним, поставленим перед нею завданням, і~при цьому цілком може бути, що вона сама обрала його для себе.» 
		
		Процес формування особистості~— це процес духовної еволюції. Людина в~процесі життя, завдяки матеріальній і~духовній діяльності, стає єдністю універсально-людського і~неповторно-індивідуального. В розвиток «Я» включається цілий світ. Як підкреслювало багато філософів, особливо Г.\,Сковорода, людина стає «мікрокосмом». Розвиток людини як особистості можливий лише через діалог, творчу комунікацію з~природою, іншими людьми, суспільством, культурою людства. Це і~є сутність процесу духовної еволюції, саморозвитку людини до рівня особистості в~процесі співтворчості з~світом. Конкретні шляхи духовною еволюції людини, розвитку особистості в~величезній мірі залежать від сенсу життя.

	\section{Діалектика об'\-єктивного й~суб'\-єктивного в~історичному процесі, його рушійні сили}
		Дотримуючись принципу єдності мислення і~буття, введеного античним мислителем Парменідом, К.\,Маркс та~Ф.\,Енгельс розробили теорію матеріалістичної діалектики. На відміну від Г.\,Гегеля, який за початок і~джерело розвитку взяв ідеальне начало, К.\,Маркс і~Ф.\,Енгельс говорять про розвиток як характеристику, що внутрішньо притаманна природі та~суспільству. Людське мислення здатне відтворювати цей розвиток через формування й~наповнення змістом відповідних категорій і~законів. Тому філософія марксизму розрізняє об'\-єктивну і~суб'\-єктивну діалектику.
		
		Об'єктивна діалектика виявляє закони розвитку об'\-єктивної реальності, незалежної від волі і~свідомості людини. Суб'єктивна діалектика~— це відображення об'\-єктивної діалектики у~свідомості людей. Цей вид діалектики, на думку К.\,Маркса та~Ф.\,Енгельса, є об'\-єктивним за змістом і~суб'\-єктивним за формою. Іншими словами, закони об'\-єктивної і~суб'\-єктивної діалектики хоча й~відрізняються за формою, все ж є тотожними за змістом. Усвідомлення об'\-єктивної діалектики йде у~зворотному напрямку від дійсного розвитку. Осмислення та~викриття законів об'\-єктивної діалектики і~є діалектикою суб'\-єктивною.
		
		Це відкриття дозволило К.\,Марксу застосувати закони і~категорії діалектики до аналізу суспільства, а~Ф.\,Енгельс показав, що діалектичним є розвиток природи і~проілюстрував це на конкретних прикладах у~працях «Діалектика природи» та~«Анти-Дюрінг». Цю думку підтверджує й~розвиток природознавства в~ХХІ ст. Сучасна наука синергетика (міждисциплінарна наука про складні, відкриті структури) доводить, що саме діалектичний метод дозволяє пізнати світ як складну самоорганізовану систему.
		
		Більшість напрямків некласичної західної філософії трансформували діалектичні ідеї Г.\,Гегеля, К.\,Маркса та~Ф.\,Енгельса відповідно до своїх світоглядних принципів. У~наш час існують: розуміння діалектики в~неотомізмі (діалектична теологія К.\,Барта, П.\,Тілліха), яка протиставляє релігію і~віру; діалектика людського існування (екзистенціалізм Ж.—П.\,Сартра, К.\,Ясперса), яка тлумачить існування протилежностей як ознаку свободи; негативна діалектика (Т.\,Адорно, Г.\,Маркузе, Ю.\,Хабермас, М.\,Хоркхаймер та~ін.), яка прагне усунути опозиції класичної діалектики (необхідність~— випадковість, можливість~— дійсність тощо), подолати «одномірність» людини (Г.\,Маркузе), надавши мисленню ознак метафоричності, образності, асоціативності. Негативна діалектика основним своїм завданням вбачає не подолання суперечностей, а~їх пошук, «послідовне логічне усвідомлення нетотожності», проблемності світу. Загалом сучасна діалектика~— це пошук людиною цілісності, її прагнення осягнути безкінечне, вічне, істинне. Завдяки діалектиці людина долає обмеженості формальної логіки, прагнучи узгодити розчленованість власного світу з~уявленнями про Абсолют.
		
		Розглянуті вище історичні форми діалектики дозволяють стверджувати, що на сьогодні діалектика існує в~трьох формах: ідеалістична діалектика (розроблена на основі об'\-єктивного ідеалізму Г.\,Гегелем), марксистська діалектика (розроблена на матеріалістичних засадах К.\,Марксом і~Ф.\,Енгельсом) та~негативна діалектика (розроблена Т.\,Адорно і~М.\,Хоркхаймером для аналізу суперечностей розвитку сучасного суспільства).

	% 61. Співвідношення еволюційного та~революційного моментів у~розвитку людської цивілізації.
	\section{Співвідношення еволюційного та~революційного моментів у~розвитку людської цивілізації}
		Суспільний процес відбувається як неперервно, так і~перервно, через еволюційні і~революційні зміни, періоди війни та~миру. Еволюція і~революція~— це два основні типи суспільного розвитку, які взаємопов'язані та~взаємозумовлені. 
		
		Еволюція~— це такі зміни в~рамках су\-спіль\-но-еко\-но\-міч\-ної формації, які не руйнують її базису та~надбудови, зберігають її основну якість. Такі зміни відбуваються поступово, протягом тривалого часу.
		
		Революція~— це докорінні, якісні зміни в~стані системи, які переривають еволюційний період її розвитку і~переводять систему на якісно новий ступінь розвитку. В теорії діалектики поняття «еволюція» та~«революція» розкривають зміст закону взаємного переходу кількісних і~якісних змін за допомогою стрибка.  Теорія соціальної революції розглядається філософією марксизму. Згідно з~нею соціальна революція~— це якісний стрибок, що усуває стару суспільну формацію і~замінює її новою, більш прогресивною, через корінний переворот у~всіх соціально-економічних структурах. У~цей період відбуваються якісні зміни у~всіх сферах суспільного життя (економічній, соціальній, політичній та~духовній). 
		
		Необхідно розрізняти революції і~реформи. Реформа~— перетворення, нововведення, перевлаштування певної сторони суспільного життя (порядків, інститутів, організацій, закладів), без докорінної зміни засад існуючої соціально-політичної структури і~наявного соціального ладу. 
		
		Реформа є способом внесення тих чи інших змін у~систему суспільних відносин, які не торкаються зміни самого способу виробництва. Вони здійснюються з~метою зняти напругу соціальних суперечностей. Будучи спрямованими на певні прогресивні зміни в~суспільстві, соціальні реформи покликані водночас зберегти існування певного суспільного ладу, тих чи інших соціально-політичних, економічних та~ідеологічних структур, інститутів, установ тощо. Але, оскільки реформи не торкаються самого характеру виробництва, який приводить до загострення суспільних суперечностей, вони мають тимчасовий ефект, який відволікає суспільство від причини таких суперечностей. У~процесі розвитку суспільства відносна протилежність соціальних реформ та~соціальної революції не усувається. Якщо реформа є способом кількісних змін у~системі суспільних відносин з~метою їх збереження, то революція є якісною, фундаментальною зміною системи економічних, соціальних, політичних та~ідеологічних інститутів даного суспільства. Якщо реформа є виявом поступового стрибка у~історичному розвитку, то революція є виявом прискореного стрибка в~цьому розвитку.
		
		Розрізняють соціальні, науково-технічні революції. Соціальна революція відбувається тільки при наявності певних об'\-єк\-тив\-них умов. Головна з~них~— дія закону відповідності виробничих відносин рівню і~характеру продуктивних сил. Коли продуктивні сили, які розвиваються, вступають у~суперечність з~існуючими виробничими відносинами, тоді виникає економічна основа соціальної революції. Соціальна революція~— закономірність у~розвитку суспільства, так як продуктивні сили у~своєму розвитку неминуче переростають існуючі виробничі відносини і~останні мають бути приведені у~відповідність до продуктивних сил. Соціальна революція може містити в~собі декілька політичних революцій. Типи соціальних революцій залежать від того, які системи суспільних відносин (який спосіб матеріального виробництва) ними породжуються. Класичними типами соціальних революцій витупають буржуазні і~соціалістичні революції. ХХ століття породило нові типи соціальних революцій: народно-демократичні революції в~Європі та~Азії, національно-визвольні революції в~країнах Азії, Африки та~Латинської Америки.
		
		Для того щоб відбулася політична революція, необхідно, щоб склалася революційна ситуація. Її ознаки характеризує В.\,І.\,Ленін: мають створитися такі умови, за яких пануючі класи не можуть зберігати в~незмінному вигляді своє панування; криза «верхів», криза політики пануючих класів; загострюється вище звичайного бідування пригноблених, трудящих мас; низи не можуть жити по-старому; зростає активність народних мас.
		
		Революційна ситуація~— необхідна, але не достатня умова. Для того, щоб революційна ситуація переросла в~революцію, необхідні суб'\-єктивні фактори~— енергія революційного класу, організованість народних мас. Історія знає приклади, коли революційна ситуація не переростала в~революцію (Росія 1859–1861~рр.).
		
		Співвідношення еволюції та~революції у~розвитку людства~— це проблема співвідношення еволюційного поступового процесу та~різких революційних змін. Внаслідок таких змін в~історії здійснюється перехід до нових рівнів і~форм суспільного життя, щось закінчується, відмирає, але з'являється дещо принципово нове. 
		
		Науково-технічні революції розділяють між собою найважливіші епохи розвитку людства, у~які здійснюються значні відкриття. Серед цих епох виділяються:
		\begin{enumerate}
			\item Архаїчна епоха, коли домінувало полювання, рибальство і~збиральництво, тобто відбувалося безпосереднє присвоєння готових продуктів, дарів природи (споживаючий тип господарства).
			\item Аграрна епоха, яка наступила внаслідок неолітичної революції в~VII—III тисячоліттях до н.~е., коли на зміну полюванню і~збиральництву прийшло землеробство і~скотарство, що в~декілька раз збільшило чисельність населення, призвело до появи надлишкового продукту, появи приватної власності, нерівності та~держави.
			\item Індустріальна епоха, яка виникла в~XVII—XVIII~ст. нашої ери внаслідок промислової революції, коли були винайдені різні механізми, які збільшили продуктивність праці, що супроводжувалося новим зростанням чисельності населення та~початком масової урбанізації. 
			\item Постіндустріальне суспільство, коли у~виробництві вирішальну роль стали відігравати здібності і~можливості людського інтелекту. Нове багатократне збільшення продуктивності праці створює достаток матеріальних благ і~породжує нові засоби задоволення фізичних та~духовних потреб людини, яких раніше не існувало.
		\end{enumerate}

	% 62. Проблема сенсу історії у~філософії: діалектико-матеріалистичний, релігійний та~ідеалістичний підходи.
	\section{Проблема сенсу історії у~філософії: діалектико-матеріалистичний, релігійний та~ідеалістичний підходи}
		У наш час майже безсумнівним постає твердження, що історія є результатом діяльності людей. Але при тому досить поширеним є уявлення, що за поверхнею історичних дій лежать деякі невидимі рушійні сили, приховані чинники, які виконують якісь вищі задуми або програми. Людину, в~даному випадку, розглядають лише як виконавця цих задумів або програм. Досить часто спостерігається тлумачення історії в~аспекті підпорядкованості дій людини законам долі, фатуму, божественному промислу, світовому розуму, абсолютній ідеї, волі і~т.~д. Саме вони, а~не людина, часто визнаються справжнім суб'\-єктом історії. Проте поза життям та~діями реальних людей ми не можемо вести мову про людську історію. Звідси випливає, що в~історії реально відбувається тільки те, що проходить через людські дії або вступає з~ними в~певний зв'язок. 
		
		Хоч життя суспільства складається з~дій людей, але це не значить, що все, що відбувається у~суспільстві, повністю контролюється свідомістю людей і~підпорядковане їх діям. Існують об'\-єктивні умови, які обмежують можливості здійснення людських бажань та~намірів~— як для окремих особистостей, так і~соціальних спільнот. Процес суспільного розвитку здійснюється закономірно і~свідома діяльність людей є необхідним елементом цього закономірного процесу. Люди самі творять свою історію, але творять її відповідно до можливостей, які визначаються об'\-єктивними, не залежними від волі і~свідомості людей, законами. 
		
		Така ситуація часто приводила до того, що мислителі або визнавали існування незалежних від людини законів (в релігійній філософії вони виступають як доля, рок, фатум), вважаючи при цьому, що діяльність людей не здатна впливати на хід історичного розвитку (фаталізм), або стверджували існування вільної від будь-яких законів діяльності людей (волюнтаризм).
		
		У суспільстві немає однозначної детермінації поведінки людей. У~людини є певна свобода дії, і~вибір того чи іншого вчинку обумовлений її суб'\-єктивними намірами. Кожне покоління, вступаючи у~життя, включається у~певні об'\-єктивні умови, які склалися на час його появи на світ (певний стан виробництва, предмети побуту, звичаї та~традиції, соціальні інститути). Ці обставини загалом визначають спосіб життя і~мислення тих соціальних груп, серед яких живуть люди. «Люди самі творять свою історію,~— писав К.\,Маркс,~– але вони її творять не так, як їм заманеться, при обставинах, які не самі вибрали, а~які безпосередньо наявні, дані їм і~які перейшли від минулого». Нове покоління своєю діяльністю може змінити дійсність, але воно може це зробити, якщо в~цій дійсності наявні об'\-єктивні умови і~засоби для цієї діяльності. Так, перехід від одного суспільного ладу до іншого стає об'\-єктивно можливим внаслідок виникнення нових продуктивних сил, які вступають у~протиріччя із старими виробничими відносинами. Необхідність такого переходу рано чи пізно усвідомлюють передові сили суспільства, які вступають у~протиріччя із інтересами тих консервативних сил, які намагаються зберегти старий лад.
		
		Хто ж рухає історію: видатні особистості, еліта чи народні маси? Історія зберегла імена історичних діячів, з~якими пов'язані найважливіші події минулого. Звідси часто робиться висновок, що саме видатні особи є авторами і~головними дійовими особами історії. Світовий Дух у~Гегеля реалізує себе через діяльність окремих індивідів. Для реалізації своєї мети (за висловом Гегеля~— «хитрощів розуму») Дух маніпулює народами як матеріалом. І~тільки всесвітньо-історичні особи, герої якої-небудь епохи, яких варто визнати проникливими людьми, вважає Гегель, розуміють те, що є потрібним і~своєчасним. Саме видатні люди являються тими, які краще всього розуміли суть справи і~від яких потім інші люди засвоювали їх розуміння та~схвалювали чи, принаймні, змирялися з~ним. Але Гегель і~тих індивідів, які змінюють історію, розглядає лише як провідників об'\-єктивного духу, через яких реалізується мета Абсолютного Духу, а~не їх цілі. Отже, він вважає, що розвиток історії підпорядкований об'\-єктивним законам і~не залежить від прагнень, волі та~свідомості людей, хоча й~реалізується через окремих особистостей.
		
		Ф.\,Ніцше писав, що народ~— це матеріал без форми, з~якого творять, простий камінь, який потребує різьб'яра. Звідси робився висновок про необхідність вождя, надлюдини, з~сильною волею до влади. Ніцше вважав, що суспільство і~особистість ворожі один до одного, тому неминучими є  конфлікти між суспільними та~особистими інтересами, так як ціле намагається підпорядкувати собі частини, а~особистість прагне зберегти свою індивідуальність. Але боротьбу із суспільством можуть винести тільки сильні особистості. Тому ця концепція визнає вирішальну роль видатних особистостей в~історичному процесі. 
		
		Думки про те, що саме видатні особистості рухають історію, дотримувався Х.\,Ортега-і-Гасет. Маси не здатні до творчості, вважав він, оскільки вони у~своєму розвитку ніколи не зможуть піднятися до рівня видатної особистості. Він висловлював думку, що маси, народ~— це руйнівна сила, що приходить на зміну еліті. «Повстанням мас» Х.\,Ортега-і-Гассет називає захоплення масами суспільної влади, кризу європейських народів та~культур. Це повстання виявилося, по-перше, в~тому, що маси досягнули такого рівня життя та~добробуту, який раніше був можливий лише для еліти. По-друге, маси вийшли з~підпорядкування, не підкоряються меншості, не рахуються з~нею, а~витісняють та~заміняють її. Натовп, що виник на авансцені суспільства, став значимим. Маси придушують меншість. Сутність цього явища, вважає Х.\,Ортега-і-Гассет, полягає у~зрівнялівці~— зрівнюються багатства, культури, слабка та~сильна статі, зрівнюються континенти. Суспільство завжди було єдністю меншості та~маси. Меншість поставала як сукупність людей, які наділені особливими якостями, а~маса~— не виділена нічим «пересічна людина», «спільний тип». Під поняттям «маса» Ортега-і-Гассет розуміє не соціальну приналежність, а~«той людський склад чи спосіб життя, який сьогодні переважає та~панує в~усіх верствах суспільства, зверху донизу, і~тому знаменує собою наш час».
		
		Близькою до цієї концепції є концепція еліти. Так, Дж.\,Бернхейм у~праці «Революція управлінців» стверджував, що в~результаті науково-технічної революції влада перейде до науково-технічної інтелігенції, управлінців підприємствами, концернами, банками.
		
		Марксизм вважає, що рушійною силою історії є народні маси. Народ розглядається як головний творець і~суб'\-єкт історії. Народ творить історію тим, що своєю працею створює цінності культури. Марксизм не заперечує роль видатних особистостей в~історії. Певні історичні умови, особливо в~перехідні епохи, створюють можливості для появи видатних особистостей. Вони з'являються у~відповідь на історичну потребу в~них. Ці особистості завжди є представниками певних соціальних спільностей~— верств, станів, народу, держави тощо. Тому вони, з~одного боку, є виразниками актуальних запитів соціальних груп, а~з другого~— виявленням непересічних здібностей та~можливостей людини.
		
		Особливої актуальності проблема активності мас набула у~ХХ~ст. Перший дослідник цієї проблеми французький соціолог, соціальний психолог і~філософ Г.\,Лебон у~праці «Психологія натовпів» зазначав, що «головною характерною рисою нашої епохи є саме заміна свідомої діяльності індивідів несвідомою діяльністю натовпу». Інтерес до феномену «маси» обумовлений тим, що хоч у~ХХ~ст. значно зростає революційність мас, проте ці маси були вже іншого ґатунку, ніж ті, які спричинили соціальні зрушення в~період Великої французької революції. У~ХХ~ст. зростання міст та~розвиток промислового виробництва вимагало значної кількості нових працівників, які вербувалися із села та~з представників маргінальних груп. Крім того, формується система управління цими масами, яка передбачає використання спеціально спрямованих методів контролю, нагляду й~маніпулювання. Виникають масові ідеології, розраховані на легкість сприйняття і~швидку зворотну реакцію.
		
		Г.\,Лебон розрізняє поняття «маси» та~«натовп». «Маса», за Лебном, є людська сукупність, що має психічну спільність, але це не є скупченням людей в~одному місці. Термін «натовп» виражає найвипадковіший зв'язок людей, поєднаних у~даному просторі суто тимчасовим минущим інтересом. Якщо маса~— це стійке явище, яке не обмежується часом та~простором, то натовп~— явище тимчасове, нестійке, хоч і~є безпосереднім породженням маси, її локальним виявом. 
		
		Основною ознакою натовпу є те, що люди в~ньому не мають жодної внутрішньої організації. Характерною психологічною особливістю поведінки особи в~натовпі є те, що знижується її інтелектуальний контроль, індивіди зливаються в~єдині розум та~почуття, внаслідок чого вольова, інстинктивна, пристрасна сторона людської природи виявляється з~більшою свободою. Кожен прагне бути схожим на ближнього, з~ким він перебуває поруч, хоч би до якого соціального класу, походження і~культури він належав. Натовп може піти на грабіж, убивство, лінчування, тобто робити те, чого жодна людина не дозволила б собі. Г.\,Лебон відводить натовпу украй негативну роль в~історії: «Цивілізації створювалися й~дотепер управлялися й~зберігалися нечисленною аристократією, а~зовсім не масами. Сила натовпу завжди спрямована на руйнування. Їхнє панування завжди є якимось безладним періодом». З~другого боку, маси іноді здаються героїчнішими, справедливішими, ніж кожен зокрема. Вони мають ентузіазм та~великодушність, їхня безкорисливість буває безмежною, коли їх захоплюють ідеалом або зачіпають їхню віру.
		
		Філософія історії ставить питання не тільки про напрямок руху, рушійні сили історії, а~й~питання про її сенс, тобто можливості втілення суспільного ідеалу, про засоби, через які цей ідеал можна утвердити. Чи здійснює історія якийсь певний сенс? Чи має історія кінець? Питання про сенс історичного буття людини постає ще в~міфології, яка намагається дати відповідь на нього. Міфи допомагають осмислити історію етносу, встановити походження роду, обґрунтувати традиції. Представник релігійної філософії Святий Августин створив узагальнену модель історії, яка на основі Біблії дає відповідь на питання про сенс історії: історія має сенс як повчання Богом людства. Сама ж історія є процесом поширення віри та~зростання Граду божого.
		
		Філософія ставить питання про сенс історії залежно від системи цінностей, що позначають позитивне або негативне у~житті людської цивілізації. Для просвітників сенс історії уявлявся як рух до царства розуму, що ґрунтується на науковому прогресі (Тюрбо, Кондорсе). Представник німецької класичної філософії І.\,Кант сенс історії бачив у~досягненні вічного миру, єдності людей та~утвердженні універсальних моральних нори. Для Гегеля витоком та~кінцевою метою історичного процесу є «абсолютна ідея», а~сенсом історії~— її самопізнання. К.\,Маркс сенс історії бачить у~подоланні відчуження людини на ґрунті перетворення матеріальних умов суспільного буття та~наближення до «царства свободи».
		
		Представники російського космізму (світоглядної позиції, відповідно до якої людина, її минуле, теперішнє та~майбутнє розглядаються крізь призму співвідношення з~космосом) В.\,С.\,Соловйов, М.\,Ф.\,Федоров, М.\,О.\,Бер\-дя\-єв сенс історії бачили у~духовному вдосконаленні людства. В.\,С.\,Соловйов виношував надію на об'єднання людства в~майбутньому, і~перш за все~— в~Боголюдстві. Про суспільство, де панувало б добро без зла, мріяв М.\,Ф.\,Федоров. А~таким, на його думку, може бути тільки суспільство, в~якому за допомогою науки буде подолана смерть, повернуті до життя всі померлі покоління. Цим і~буде досягнута вища справедливість, а, відповідно, й~вище добро, здійсниться ідея єдності людства в~сукупності всіх його минулих і~майбутніх поколінь.
		
		М.\,О.\,Бердяєв указував, що єдине людське суспільство має бути облаштоване перш за все за духовим принципом, як духовна спільнота екзистенційних суб'\-єктів, об'єднаних однією історією. У~такому суспільстві, яке стане «колективним тілом свободи», матеріальне перевтілюється в~духовне. Людство для нього ~— не сукупність безликих індивідів, а~єдність незалежних особистостей, кожна з~яких є відповідальною за себе, за інших, а~в~кінцевому підсумку~— й~за долю світу.
		
		К.\,Ясперс сенс історії бачив у~подоланні відчуження через справжню комунікацію, що ґрунтується на філософській вірі. Філософська віра відрізняється від релігійної віри, оскільки існує в~єдності із знанням, а~релігійна віра ґрунтується на одкровенні. Одкровення відділяє віруючих від тих, хто не поділяє релігійних поглядів. Тим самим релігійна віра перешкоджає взаєморозумінню, створюючи у~віруючих претензію на винятковість, що призводить до фанатизму та~нетерпимості. Отже, філософи по-різному бачать сенс історії, адже сенс історії~— це певний орієнтир для певних соціальних груп, які мотивують вольовий вибір дальшого шляху.
		
		Філософія історії, спираючись на визначену нею світоглядну позицію, на синтез історичного знання, бачить своїм призначенням зрозуміти історію як ціле, що має сенс. Сенс історії не можна розглядати як якусь певну формулу. Він інтерпретується людьми залежно від проблем сьогодення, від осмислення навколишнього світу й~пошуку власного сенсу життя. Сенс залежить від того, яке значення надає суб'\-єкт певному об'\-єкту, які переживання виникають у~суб'\-єкта і~яка поведінка передбачається. Людина постійно прагне зробити усвідомленим своє перебування в~світі та~в~історії, зрозуміти спрямованість історії. 
		
		Уся історія суспільства є цілісним процесом, який складається з~тісного переплетіння протиборства різних людей, націй, народів. І~в цьому розумінні будь-яка діяльність людей є рушійною силою. Рушійною силою кожної людини, соціальної групи і~людства загалом є інтереси. Зміст інтересу визначається умовами життя людей та~їхніх спільнот, місцем у~системі суспільних відносин. Інтерес є реальною причиною соціальних подій, звершень, що стоять за безпосередніми мотивами, помислами, ідеями індивідів, соціальних груп чи суспільства. Серед численних інтересів особливе місце належить матеріальним, особливо інтересам власності на засоби виробництва, адже історію людської цивілізації можна характеризувати за певними формами власності. Відносини власності є найхарактернішою ознакою, що поєднує людей у~різні соціальні спільноти чи роз'\-єд\-нує їх. 
		
		Отже, рушійні сили суспільного розвитку~— це діяльність людей і~соціальних груп, в~основі якої лежать певні інтереси і~яка здійснюється через державні і~недержавні органи, колективи, первинні соціальні осередки тощо. 

	
\end{document}

