\documentclass[a5paper,oneside,DIV=12,12pt,headings=small]{scrartcl}

%%% Length calculations
\usepackage{calc}
%%%

%%% Support for color
\usepackage{xcolor}
\definecolor{lightblue}{HTML}{03A9F4}
\definecolor{red}{HTML}{F44336}
%%%

%%% Graphics inclusion
\usepackage{graphicx}
%%%

%%% Font selection
\usepackage{fontspec}

\setromanfont{PT Serif}[
]

\setsansfont{PT Sans}[
]

\setmonofont{PT Mono}[
]

\usepackage{amsmath,unicode-math}
\setmathfont{STIX Two Math}

%%%

%%% Font settings for different KOMA Script elements
\setkomafont{pagenumber}{\rmfamily}
\setkomafont{disposition}{\rmfamily\bfseries}
%%%

%%% Typographic enhancements
\usepackage{microtype}
%%%

%%% Language-specific settings
\usepackage{polyglossia}
\setmainlanguage{ukrainian}
%%%

%%% List settings
\usepackage{enumitem}
\setlist[enumerate]{
	leftmargin = *,
}
%%%

%%% Captions
\usepackage{caption}
\usepackage{subcaption}

\DeclareCaptionLabelFormat{closing}{#2)}
\captionsetup[subtable]{labelformat = closing}
\captionsetup[subfigure]{labelformat = closing, position = auto}
%%%

%%% Tables
\usepackage{booktabs}
\usepackage{longtable}

\usepackage{multirow}

\usepackage{array}
\newcolumntype{v}[1]{>{\raggedright\arraybackslash\hspace{0pt}}p{#1}}
\newcolumntype{b}[1]{>{\centering\arraybackslash\hspace{0pt}}p{#1}}
\newcolumntype{n}[1]{>{\raggedleft\arraybackslash\hspace{0pt}}p{#1}}
%%%

%%% Links and hyperreferences
\usepackage{hyperref}
\hypersetup{
	colorlinks      = false,
	linkbordercolor = red,
	urlbordercolor  = lightblue,
	pdfborderstyle  = {/S/U/W 1.5},
}
%%%

%%% All caps
\newcommand{\allcaps}[1]{{\addfontfeatures{LetterSpace = 3}#1}}
%%%

\setlength\emergencystretch{1em}


\begin{document}
	\section{Поняття світогляду та~його суспільно-історичний характер.}
		Під світоглядом слід розуміти результат духовного осягнення людиною, людством світу. Об'єкт світогляду~— світ як~цілісність. А предмет світогляду~— відношення «людина~— світ» Тобто центром уваги світогляду є питання про співвідношення активної, цілеспрямованої, розумної частини світу (людини) із світом~— як~об'єктивно існуючою цілісністю (протилежністю людині). Співвідношення «ми» (люди) і~«він» (світ) можна розглядати як~сутність світогляду.
		
		А виходячи із природи людини, світогляд є система узагальнених почуттів, інтуїтивних уявлень і~теоретичних поглядів на~навколишній світ і~на місце людини в~ньому, на~відношення людини до світу, до самої себе і~до Інших людей. Водночас світогляд~— це система не завжди усвідомлених основних життєвих настанов людини, певної соціальної групи і~суспільства, політичних, моральних, естетичних, релігійних, наукових, філософських принципів пізнання і~оцінок.
		
		Наявність світогляду є виявом системності духовності людини, суспільства і~водночас показником зрілості не лише особистості, а~й соціальних груп, тих чи інших політичних сил.
		Світогляд~— це різнорівневе духовне утворення, в~якому органічно поєднуються і~житейське бачення, з його раціональними та~ірраціональними елементами, розсудок і~передсуди, наукові, художні і~політичні погляди.
		
		Розвиток світогляду відбувається шляхом зміни форм практичного освоєння людиною світу і~її теоретичного самоусвідомлення.
		
		Тому світогляд~— найвища форма самоусвідомлення людини, логічно впорядкована система світовідчуття, світосприйняття та~світорозуміння. В світоглядній культурі світові наявного буття протиставляється світ мети, світ ідеалів, цілей і~сподівань, а~також уявлення, почуття, ідеї, знання, вірування і~переконання, опосередковані досвідом особистості і~які є основою формування її життєвої позиції.
		
		Як правило, виділяють такі історичні типи світогляду: міфологічний, релігійний, науковий і~філософський. Більшість дослідників світоглядної культури вважає, що~подана типізація не є найбільш досконалою («істиною в~останній інстанції»). Проте всі дослідники цього унікального явища стверджують високу ефективність цієї типізації, особливо для розкриття природи і~сутності світоглядного вибору людини.
		
	\section{Історичні типи світогляду: міфологічний, релігійний, філософський}
		Історичними формами світогляду прийнято вважати такі: міфологія, релігія, філософія. Оскільки філософії як~світоглядові присвячено окремі питання, зупинимося спочатку на~короткій характеристиці міфології і~релігії, з'ясуємо основні риси цих явищ.
		
		Найбільш ранньою формою світогляду є міфологія. Вона виникала в~первісному суспільстві. У міфах людина насамперед прагнула відповісти на~так звані космічні питання.
		Основні риси міфологічного світогляду:
		\begin{enumerate}
			\item Уявлення про кровно-родинні зв'язки природних сил та~явищ.
			\item Персоніфікація, уособлення природних сил та~способів людської діяльності.
			\item Міфологічне мислення оперує образами, а~не поняттями;
			\item Міфи сприймалися як~реальність, що~не потребує доказовості, обґрунтування та~перевірки.
			\item Людина розумілася як~іграшка в~руках природних чи надприродних сил, її життєвий шлях визначався долею, фатумом.
		\end{enumerate}
		
		Як і~міфологія, релігія вдається до фантазії і~почуттів. Але на~відміну від міфу релігія розрізняє земне і~неземне, надприродне, вона розводить їх на~два протилежних полюси. Релігія формувалася на~основі постійного вторгнення в~життя людей «чужих» їм природних і~соціальних процесів. Ці таємничі, непомірні сили усвідомлювалися безсилими перед ними людьми як~«вищі сили». «Вищі сили» виступали як~уособлення добра і~зла, як~демонічні і~божественні начала. Звідси~— поєднання страху і~поваги у~людей, прагнення знайти захист і~порятунок у~зверненні до божественних сил.
		Основні риси релігійного світогляду:
		\begin{enumerate}
			\item Поділ світу на~поцейбічний («земний») і~потойбічний («небесний»).
			\item Віра в~існування надприродних сил та~відведення їм головної ролі у~світобудові та~житті людей.
			\item Наявність культу~— системи усталених ритуалів, догматів.
			\item Поклоніння Богові як~вищій істоті.
		\end{enumerate}
		
		Філософський світогляд за сучасних умов розглядається як~один із впливових і~дійових типів. Він, як~і~релігія, розвинувся з первинної міфології, успадкувавши її світоглядні функції. Що ж до сучасного філософського світогляду, то необхідно звернути увагу на~такі його особливості:
		\begin{enumerate}
			\item Філософському світогляду властива не чут\-тє\-во-об\-раз\-на, як~у~попередніх світоглядах, форма освоєння дійсності, а~абс\-тракт\-но-по\-ня\-тій\-на.
			\item Філософський світогляд~— теоретична форма світогляду, яка~виникла історично, і~перша форма систематизованого теоретичного мислення взагалі.
			\item Відмінність філософського світогляду від міфологічного та~релігійного полягає в~тому, що~релігія і~міфологія збігаються з відповідним світоглядом, тоді як~філософія становить ядро наукового світосприйняття.
			\item На відміну від релігії і~міфології, філософія в~осмисленні світу систематично спирається на~наукові знання;
			\item Філософія прагне поставити і~розв'язати граничні, абсолютні проблеми людського буття.
			\item Філософія досліджує пізнавальне, ціннісне, со\-ці\-аль\-но-по\-лі\-тич\-не, моральне та~естетичне ставлення людини до світу; виробляє певні критерії і~принципи суспільної та~індивідуальної діяльності, спираючись не на~авторитет, а~на знання необхідності.
		\end{enumerate}
		
	\section{Специфіка філософського світогляду. Основне питання філософії}
		Філософія як~світогляд є системою найбільш загальних поглядів на~світ, природу, суспільство, людину, пізнання. Філософія як~світогляд теоретично обґрунтовує свої положення і~висновки, основні принципи соціально-політичної, наукової, моральної, естетичної діяльності людини, тобто освоює світ як~духовно (теоретично), так і~практично.
		
		Для філософії особливими проблемами є загальні проблеми існування світу, як~природи, його єдності, походження, тенденцій розвитку тощо. Суб'єктом же пізнання, осмислення цих проблем є людина як~творча, діяльна істота. Тому в~предмет будь-якої філософської системи, будь-якого її напрямку необхідно включається, як~основне,~— відношення «людина~— світ».
		Основне питання філософії має дві сторони:
		\begin{enumerate}
			\item Про природу і~сутність світу, що~лежить в~основі світу. Матерія чи дух, ідея, свідомість.
			\item Про пізнавальний світ.
		\end{enumerate}
		
		Два протилежні висновки щодо розкриття природи світу знайшли своє вираження у~вирішенні питання про те, що~ж є первинним~— матерія чи свідомість. Це перша сторона основного питання філософії. Різні відповіді у~вирішенні цього питання обумовили виникнення двох основних напрямів філософії~— матеріалізму та~ідеалізму.
		
		Матеріалізм виходить з того, що~світ за своєю природою матеріальний, вічний, несотворимий, безкінечний у~часі і~просторі. Матерія~— первинна. Свідомість є продуктом, властивістю високоорганізованої матерії~— мозку, вона~— вторинна. Матеріальний світ згідно з матеріалізмом існує сам по собі, незалежно ні від людини, ні від будь-яких надприродних сил. Людина~— частка природи, її свідомість породжена природою, є її специфічною властивістю. Матеріалізм у~різні історичні епохи набував різних форм та~видів: наївний та~зрілий (науковий матеріалізм), стихійний та~філософськи осмислений, метафізичний та~діалектичний.
		
		Ідеалізм виходить із визнання первинності духу, свідомості, мислення та~вторинності природи, матерії. Ідеалізм, як~і~матеріалізм, набував конкретних форм, наповнювався конкретним змістом на~різних етапах історії.
		Крім цих основних способів вирішення основного питання філософії є ще дуалізм, який визнає матерію і~свідомість, дух і~природу, мислення і~буття як~дві самостійні основи.
		До сфери основного питання філософії належить і~питання про здатність і~можливість людини пізнати навколишній світ, про межі пізнання, про його природу та~істинність. Матеріалізм стверджує, що~людина здатна пізнати світ, що~наші знання відповідають матеріальним об'єктам, тримають в~собі їх образи, можуть бути і~є істинними знаннями.
		
		Об'єктивний ідеалізм також ствердно відповідає на~питання про пізнання світу, але вирішує його на~ідеалістичній основі, виходячи з абсолютної тотожності розуму і~дійсності.
		
		Філософський напрямок, що~заперечує пізнаванність світу, називається агностицизмом. Це, як~правило, суб'\-єк\-тив\-ний ідеалізм. Згідно з агностицизмом людина не може мати вірогідних (достовірних) знань, оскільки реальний світ для неї – всього лише світ її почуттів, які внаслідок обмеженості та~індивідуальних особливостей людських органів чуття, спотворюють дійсність, не можуть дати вірогідного знання про неї.
		
		Найважливішими у~філософії є питання про загальну структуру світу та~стан, в~якому він перебуває. Це питання знаходить своє вирішення у~двох основних концепціях – діалектичній та~метафізичній.
	
	\section{Філософія в~системі культури. Функції філософії}
		Всі явища культурного життя виростають на~ґрунті світогляду. Ще більше значення, чим просто світогляд, займає в~культурі філософія, як~система теоретичних і~усвідомлених (продуманих, проаналізованих) елементів світогляду. Філософія є теоретичним фундаментом культури й у~той же самий час~— її, культури, найвищим усвідомленим вираженням. Не випадково Гегель, а~за ним і~Карл Маркс називали філософію «квіткою культури нації».
		
		До основних функцій філософії слід зарахувати світоглядну, пізнавальну (гносеологічну), методологічну, практично-діяльну (праксеологічну).
		
		Світоглядна функція філософії полягає в~тому, що~вона, озброюючи людей знаннями про світ та~про людину, про її місце у~світі, про можливості його пізнання і~перетворення, здійснює вплив на~формування життєвих установ, на~усвідомлення людиною цілей та~сенсу життя.
		
		Пізнавальна (гносеологічна) функція полягає в~тому, що~вона, орієнтуючи пізнавальні прагнення людини на~пізнання природи і~сутності світу, природи та~сутності самої людини, загальної структури світу, зв'язків і~законів його розвитку, з одного боку, озброює людей знанням про світ, людину, про зв'язки і~закони, а~з іншого~— здійснює вплив на~кожну форму суспільної свідомості, детермінуючи необхідність для кожної з них (в своїй сфері) усвідомлювати дійсність крізь призму відношення «людина~— світ».
		
		Методологічна функція. Виділення її як~однієї з основних зумовлено тим, що~філософія займає особливе місце у~процесі усвідомлення буття у~структурі суспільної свідомості. Кожна з форм суспільної свідомості, виступаючи як~усвідомлення залежності життєдіяльності людини від певної сфери дійсності, є відображенням саме цієї сторони людського буття. Специфіка філософії полягає в~тому, що~вона в~найузагальненішій формі вивчає ставлення людини до світу і~до самої себе. Тому основні положення філософії мають важливе методологічне значення для кожної з форм суспільної свідомості в~процесі усвідомлення свого специфічного предмету.
		
		Практично-діяльна (праксеологічна) функція філософії полягає в~тому, що~вона стає знаряддям активного, перетворювального впливу на~оточуючий світ і~на саму людину. Філософія відіграє важливу роль у~визначенні цілей життєдіяльності, досягнення яких є найважливішою умовою забезпечення існування, функціонування і~розвитку людини.
		
	\section{Загальна характеристика  основних етапів  розвитку античної філософії}
		Перший етап охоплює VII~— V століття до н.~е. і~називається досократівським (натурфілософським). Філософи, які жили до Сократа, так і~називаються~— досократики. До них належать мудреці із Мілета (так звана Мілетська школа: Фалес, Анаксимандр, Анаксимен), Геракліт із Ефеса, Елейська школа (Парменід, Зенон), Піфагор і~піфагорійці, атомісти (Левкіпп і~Демокріт). У центрі ранньої~— досократівської~— грецької натурфілософії стояли проблеми фізики та~Космосу.
	
		Другий етап (середина V~ст. до н.~е.~— кінець IV~ст. до н.~е.)~— класичний або «висока класика». Софісти й Сократ, які вперше спробували визначити суть людини, здійснили антропологічний поворот у~філософії. Філософська спадщина Платона й Арістотеля, що~характеризується відкриттям надчуттєвого і~органічним формулюванням основних класичних проблем, найповніше узагальнює і~відображає досягнення класичної епохи грецької античності.
	
		Третій етап у~розвитку античної філософії (кінець IV~— II століть до н.~е.) зазвичай називають елліністичним. На відміну від попереднього, пов'язаного з виникненням значних, глибоких за змістом і~універсальних за тематикою філософських систем, формуються різноманітні еклектичні конкуруючі філософські школи: перипатетики, академічна філософія (Платонівська Академія), стоїчна й епікурейська школи, скептицизм. Усі школи об'єднує одна особливість: перехід від коментування вчень Платона й Арістотеля до формування проблем етики, моралізаторською відвертістю в~епоху присмерку й занепаду елліністичної культури. Тоді популярна творчість Теофраста, Карнеада, Епікура, Піррона та~ін.
	
		Четвертий етап у~розвитку античної філософії (І ст. до н.~е.~— VI ст. н.~е.)~— римський період розвитку~— період, коли вирішальну роль в~античності став відігравати Рим, під вплив якого потрапляє Греція. Римська філософія формується під впливом грецької, особливо елліністичної. У римській філософії виділяються три напрямки: стоїцизм (Сенека, Епіктет, Марк Аврелій), скептицизм (Секст Емпірик), епікуреїзм (Тит Лукрецій Кар). У III~— V століттях н.~е. в~римській філософії виникає і~розвивається неоплатонізм, визначний представник якого є філософ Плотін. Неоплатонізм значно вплинув не тільки на~ранню християнську філософію, але й на~всю середньовічну релігійну філософію.
		
	\section{Проблема людини в~античній філософії. Софісти, Сократ}
		У софістів людина~— це розумна істота, яка творчо діє і~в сфері культури, і~в сфері пізнання. Тому істини~— продукт людської творчості, так само як~і~культура. Отже, пізнання не відображає об'\-єк\-тив\-ний світ, а~відтворює суб'єктивний світ людини.

		Стародавньо-грецький філософ Протагор відмічав: «Людина~— міра всіх речей, існуючих, що~вони існують, і~не існуючих, що~вони не існують». Усі істини відносні та~мають значення лише для людини, яка виступає джерелом морально-правових норм. У просвітництві людей софісти вбачали зміст свого життя і~на відміну від попередників вважали, що~доброчесності можна навчитися. Під доброчесністю софісти розуміли не тільки моральні якості, але й усю сукупність людських здібностей.
		
		Софісти заперечували привілеї народження і~в такому контексті розвивали ідею рівності. Якщо раніше слово людина використовувалось лише стосовно грека, то софісти твердили, що~всі люди родичі та~співгромадяни одного царства не за звичаями та~мораллю, а~за природою.
		
		Душа для Сократа~— щось демонічне, сам Ерос, невгасиме завзяття, спрямованість йти вгору. Сократ закликає пізнати самого себе.
		Сократ рішуче повертає філософські дослідження від вивчення Космосу, природи до людини як~духовної істоти. "Пізнай самого себе»~— такою є головна теза сократівського філософського вчення. І таке знання можна здобути в~практичній зустрічі умів. Сократ принципово відмовляється від записування своїх думок, вважаючи дійсною сферою знання, мудрості живу бесіду з опонентами, живу полеміку. Саме він ввів поняття «діалектика» (вміння вести бесіду, сперечатися).
		Філософія Сократа~— своєрідна межа в~історії античної філософії. У всіх досократівських мислителів («досократики») світ виступає у~вигляді цілісності, яка підпорядковує собі людину~— «одну» з частинок Космосу. Сократ же вирізняє людину, визначаючи предметом філософії відношення «людина~— світ».
		
	\section{Філософські системи Платона та~Арістотеля}
		В плані погляду на~основу буття Платон і~Аристотель є антагоністами, представляючи різні світогляди~— ідеалізм і~матеріалізм.

		Платон першоосновою буття вважає свідомість, яка не є породженням матерії, але сама є причиною матеріальних форм. Світ речей, які сприймаються чуттями (тобто матеріальний світ) не є в~кінцевому підсумку реальним. Матеріальні речі існують лише завдяки тому, що~існують їх безтілесні прообрази, які Платон називає ідеями.
		
		Філософія Платона характеризується також своєрідним протиставленням душі і~тіла. Тіло~— смертне, а~душа безсмертна.
		Індивідуальна душа складається з двох частин: розумної і~нерозумної (тобто чуттєвої). За допомогою першої частини людина здатна мислити, а~друга сприяє почуттям: завдяки їй людина закохується, відчуває голод і~спрагу, буває охоплена іншими почуттями.

		Аристотель началом буття вважає матерію і~форму. Він вважає що~суть буття будь-якої речі~— її форма. Все, що~існує в~природі, складається з матерії і~форми. Матерія вічна, при цьому вона не поступається формі. Матерія є чиста можливість, потенціал речі, а~форм~— реалізація цього потенціалу. Форма робить матерію дійсністю, тобто втіленням у~конкретну річ.
		
		Не заперечуючи існування душі і~Бога, Аристотель відводить їм другорядну роль. Він не вважає що~матеріальні форми створені Богом. Але все ж таки Бог вніс в~матеріальний світ першооснову форм. Душа~— це здійснення можливості життя природного тіла. Аристотель розрізняє три види душі. Два з них належать до фізичної психології, оскільки не можуть існувати без матерії. Третя душа~— метафізична. Ця метафізична душа незалежна від тіла і~лише вона має здатність уявляти і~мислити. Але в~кінцевому підсумку душа існує лише заради існування фізичного тіла.
		
	\section{Основні філософські школи пізньої античної Філософії. Скептицизм, епікуреїзм, стоїцизм}	
		\begin{figure}[!htbp]
		\centering
			\includegraphics[width = \linewidth]{./assets/08.png}
		\end{figure}
		Філософія завершального циклу античної філософії була чітко орієнтована на~захист окремого індивіда в~умовах поступового руйнування класичного античного полісу (Олександр Македонський завоював не лише більшість полісів, а~й колосальні території за межами Греції, створивши грандіозну імперію). Пізня антична філософія поставала індивідуалістичного, суб'єктивно забарвленою. Через це тут не стільки продукували нові ідеї, скільки використовували вже наявні (епігонство), які часто сполучали між собою без достатньої внутрішньої єдності (еклектика). Зупинимося на~ідеях найавторитетніших шкіл цього етапу.
		
		Епікурейство заснував Епікур (342~— 271 рр. до Р. X.) і~продовжив римлянин Тит Лукрецій Кар (95~— 55 рр. до Р. X.). Епікур ставив собі завдання захистити людину від можливих страхів життя, для чого доводив невмирущість матерії, а, значить, і~певне безсмертя людини (посилаючись на~атомізм), відсутність фатуму та~необхідності в~космосі (припускаючи, що~атом володіє здатністю довільно відхилятись від траєкторії свого польоту), можливість різних пояснень тих самих явиш через відсутність тісного зв'язку думки з фактами та~відчуттям. Найбільше, чого може досягти людина в~житті,~— це звільнити себе від страхів та~неприємних відчуттів, отримувати від життя насолоди, серед яких найбільша—уміння запобігати стражданням та~зберігати душевну рівновагу, незворушність і~безпристрасність (давньогрецькою мовою~— досягнення стану «атараксії»).
		
		Скептицизм (від давньогрецького «сумнів»), що~його заснував Піррон (360~— 270 рр. до Р. X.), звертав увагу на~те, що~всі філософи запевняли в~істинності своїх теорій, але висували різні ідеї; звідси випливав висновок про неможливість створення істинної філософії. В основі всіх суджень скептицизму лежали три знамениті запитання з трьома відповідями:
		\begin{enumerate}
			\item Якими є усі речі?~— Не більше такими, ніж будь-якими іншими.
			\item Що можна сказати про такі речі?~— Краще не казати нічого, утримуючись від суджень.
			\item Що робити людині, яка перебуває у~стосунках з такими речами?~— Зберігати самовладність (автаркію).
		\end{enumerate}
		
		Більше поширеним у~цю епоху був стоїцизм, що~його заснував Зенон-стоїк. Стоїцизм також закликав людину до життєвої мудрості та~самовладності, але з позиції зовсім іншого розуміння і~буття, і~людини. Скептики вважали, що~весь світ пронизаний єдиним потоком вогняної пневми (дихання), що~несе всьому закон і~долю. Дія долі неминуча й невблаганна. Тому людині не варто впадати у~відчай, адже змінити долю неможливо. («Того, хто бажає, доля приваблює, того, хто не бажає, штовхає»). Стоїки пояснювали свою думку притчею про карету та~собаку: собака прив'язаний до карети, що~їде; як~би не смикався пес, рух визначає карета, тому втрачають сенс смикання, опір руху, відчайдушні крики. Через це і~людині у~її відношенні до долі лишається одне: визначити внутрішнє ставлення до того, чого змінити не можна. Гідне для людини ставлення до будь-чого~— спокій, незворушність, зберігання внутрішньої автономії. Стоїки сприяли систематизації філософського знання; вони поділили філософію на~фізику, логіку та~етику, залучивши до логіки і~розуміння пізнання.
		
	\section{Загальна характеристика релігійної філософії Середньовіччя}
		Середньовічна європейська філософія~— надзвичайно важливий змістовний і~тривалий етап в~історії філософії, пов'язаний передусім з християнством. Головна відмінність середньовічного мислення полягає в~тому, що~рух філософської думки був сповнений проблемами релігії. «Філософія~— служниця богослов'я»,~— такою була поширена думка освічених кіл середньовічної Європи. Не можна забувати й те, що~більшість учених були представниками духовенства, а~монастирі~— осередками культури та~науки.

		Специфіку типу філософствування середньовіччя можна визначити в~таких моментах:
		\begin{enumerate}
			\item Їй був властивий біблійський традиціоналізм і~ретроспективність. Біблія вважалася «Книгою книг», богонатхненним твором, словом Бога, Заповітом і~об'єктом віри. Тут містилася ідея єдиного, унікального Бога, який знаходився у~позамежному світі. Така тенденція виключала багатобожжя в~будь-якому варіанті та~стверджувала ідею про єдину сутність світу.
			\item Оскільки Біблія розумілася як~повне зведення законів буття та~велінь Бога, особливого значення набула екзегетика- мистецтво правильного тлумачення та~роз'яснення положень Заповіту.
			\item Філософії середньовіччя була властива тенденція до повчання. Це сприяло загальній настанові на~цінність навчання та~виховання стосовно просування до спасіння, до Бога.
			\item Філософія середньовіччя цуралась античного скептицизму та~агностицизму, хоча прямо чи опосередковано переймала вчення античних філософів.
			\item Світ не уявлявся таким, що~може бути осягнутим, побудованим на~раціональних засадах, історичним, тобто таким, що~має початок від створення світу та~кінець у~вигляді «страшного Суду».
			\item Фізична природа світу, історія в~окремих проявах, ряд моральних вимог осягались розумом людини, а~релігійні питання~— Одкровенням.
			\item Для середньовічного світогляду характерний теїзм (грец. Teos~— бог)~— світогляд, в~основі якого лежить розуміння Бога, котрий не лише створив світ, але й втручається в~усі його події.
		\end{enumerate}
		
	\section{Натурфілософські та~гуманістичні ідеї філософії епохи Відродження}
		Епоха Відродження (або Ренесансу)~— це період європейської історії з середини~XIV до~кінця XVI~століття. Цю епоху часто називають перехідною~— від середньовіччя до Нового часу, від феодалізму до капіталізму. В Європі це був час зародження мануфактурної промисловості, розвитку міського життя, торгівлі, приватного підприємництва, великих географічних відкриттів. У сфері духовного життя поруч із церковною культурою з'являється світська.

		Основні принципи філософії Відродження:
		\begin{enumerate}
			\item Антропоцентризм~— визнання людини вихідною точкою філософського осмислення дійсності; естетичний антропоцентризм епохи Відродження протиставляється теоцентризму середньовіччя та~космоцентризму античності.
			\item Натурфілософія.
			\item Сплеск мистецтва (естетичне розуміння дійсності).
			\item Гуманізм~— напрям суспільної думки, що~визнає людину вищою цінністю, спрямований на~ствердження особистості у~діяльності, творчості та~свободі;
			\item Титанiзм~— гуманістичне уявлення про необмеженість можливостей людини, що~проявилося у~прагненні до всебічного самовираження особистості у~різних сферах діяльності: мистецтві, науці, філософії тощо.
			\item Антиклерикалізм~— заперечення повної влади церкви у~всіх сферах людського життя.
			\item Антисхоластичність~— заперечення схоластики як~методу філософствування та~пізнання як~такого, що~не дає інформації про дійсний світ.
			\item Пантеїзм (від грец.~— бог скрізь)~— ототожнення Бога та~природи (це принципово інше у~порівнянні з традиційним осмисленням поняття Бога).
		\end{enumerate}
		
		Періоди філософії Відродження:
		\begin{enumerate}
			\item Гуманiстичний (XIV~— до сер. XV ст.). Відбувається звернення до людини як~богорівної творчої істоти, становлення гуманізму та~антропоцентризму. Представники: Данте, Петрарка, Валла.
			\item Неоплатонiчний (середина XV~— перша третина XVI ст.). Поряд з питаннями природи людини здійснюється постановка онтологiчних проблем та~вирішення їх у~дусі оригінально інтерпретованого вчення античності. Розвиваються магія та~алхімія як~способи пізнання природи та~її перетворення. Представники: Кузанський, Фiчiно, Бьоме.
			\item Натурфiлософський (друга половина XVI~— початок XVII ст.). Відхід від розуміння людини як~центру світу і~визнання її частиною природи. Розвиток пізнання природи як~механізму. Представники: Леонардо да Вiнчi, Еразм Роттердамський, Копернiк.
		\end{enumerate}
		
	\section{Загальна характеристика філософії Нового часу. Емпіризм та~раціоналізм}
		У XVI–XVII~століттях філософія інтенсивно розвивалася поза межами університетів, звільнившись від тісних рамок схоластики і~зв'язку з теологією. Томас Гоббс, на~якого найбільший вплив мали Платон і~Аристотель, сам здійснив грандіозний вплив на~подальшу політичну філософію.
	
		У теорії пізнання для філософії Нового часу характерне протистояння раціоналізму та~емпіризму. Раціоналізм, найвизначнішими прихильниками якого були Бенедикт Спіноза, Рене Декарт, Готфрід Лейбніц, притримується думки про те, що~основою для пізнання є розум, логічне мислення. Емпіризм, представлений у~ці часи Френсісом Беконом, Джоржем Берклі, Томасом Гоббсом, Джоном Локком, Девідом Г'юмом, вважає, що~джерелом будь-якого знання є передусім досвід.
		
		Один із найяскравіших представників епохи, Френсіс Бекон став родоначальником емпіризму, вважав, що~вища мета науки полягає у~забезпеченні панування людини над природою. На противагу йому Рене Декарт, основоположник раціоналізму, прагнув розробити універсальний метод для всіх наук. Характерна риса вчення Декарта~— дуалізм субстанцій. Бенедикт Спіноза дотримувався пантеїстичних поглядів і~протиставив дуалізму Декарта матеріалістичний монізм, за що~був відлучений від іудаїзму. Джон Локк розвинув сенсуалістичну теорію пізнання. Ідеї суб'єктивного ідеалізму у~різних варіаціях пропонували Джордж Берклі і~Девід Г'юм. Об'\-єк\-тив\-но-іде\-ал\-і\-стич\-не вчення розробив Готфрид Лейбніц, який висловив ряд глибоких діалектичних ідей.
		
		Раціоналізм (від лат. «ratio»~— «розум»)~— цілісна гносеологічна концепція, згідно з якою істинним началом буття, пізнання та~поведінки людей є принципи розуму. У філософію термін «розум» перейшов з теології, де ним позначали такий напрям, прихильники якого наполягали на~очищенні релігії від всього, що~не може знайти розумового пояснення, піддавали догмати віри логічному аналізу. Філософський раціоналізм своїм корінням сягає античності: до вчення Сократа про ототожнення істини і~моралі, до вчення Платона про ідеї як~істинні субстанції дійсності, до вчення Аристотеля про космічний розум як~всезагальну умову буття і~мислення.
		
		Першою та~історично найбільш розповсюдженою формою емпіризму став сенсуалізм (від лат. «sensus»~— «чуття», «відчуття»)~— філософська настанова, згідно з якою єдиним джерелом та~основою знань виступає чуттєвий досвід. Центральною проблемою сенсуалізму є проблема встановлення онтологічного статусу даних чуттєвого досвіду. Ця проблема у~своїй радикальній редакції вперше була сформульована в~античній філософії. Вже у~Протагора і~Секста Емпірика сенсорні дані дають змогу судити не про зовнішні об'єкти як~про свої причини, а~про стан суб'єкта як~його вияв. У середньовічній схоластиці проблематика сенсуалізму експліцитно формулюється у~контексті відомої суперечки про універсали: номіналізм є принциповою основою сенсуалізму (Ж. Бурідан), а~реалізм~— основою раціоналізму (А. Кентерберійський, І. Еріуґена). У контексті за\-галь\-но-на\-ту\-ра\-ліс\-тич\-ної орієнтації ренесансної культури домінувала сенсуалістична тенденція (досвідний індуктивізм Телезіо, Кампанелли, Парацельса, аналітизм Галілео).
		
	\section{«Докритичний» та~«критичний» періоди у~творчості І.\,Канта}
		\subsection{Докритичний (догматичний) період}	
			Визначальним для цього періоду є робота над проблемами природознавства та~математики, викладання при\-род\-ни\-чо-на\-у\-ко\-вих дисциплін. Астрономія, математика, фізика, антропологія, фізична географія, мінералогія~— такий неповний перелік галузей знання, які його цікавили. У своїх природо наукових дослідженнях Кант перебував під впливом ньютонівської концепції всесвітнього тяжіння (розробив пояснення явищ припливів та~відпливів на~основі визначення присутності відштовхуючих сил, дії сил на~відстані) та~еволюційної концепції Ж. Бюффона, його ідеї закономірної зміни природних тіл та~явищ у~часі. Кант виступає як~натураліст-спостерігач, обґрунтовує необхідність того, щоб в~природознавстві все було пояснено природним чином. У роботах цього періоду було поставлене питання про розвиток у~природі. Зокрема, в~роботі «Всезагальна природна історія та~теорія неба» (1775) була розвинена космогонічна гіпотеза, в~якій на~основі законів механіки пояснювалось, яким чином виникла сонячна система, які етапи вона пройшла. У працях «докритичного» періоду Кант перебуває під впливом раціоналістичної філософії Лейбніца. Він дотримується погляду, згідно з яким зв'язок між причинами і~наслідками подій не відрізняється від логічного зв'язку між підставою та~наслідком. Однак поступово Кант відмовляється від такої позиції під впливом філософії Юма. Кант починає визнавати, що~зв'язок між причиною і~дією має емпіричний характер (факт буття), а~не характер логічного висновку. Разом з цим Кант залишається на~позиціях раціоналізму і~наголошує, що~наука (математика, природознавство), не може мати своїм джерелом досвід, який завжди є обмеженим, а~тому не може бути підставою для універсальних узагальнень. Водночас джерелом для таких знань не може бути і~розум сам по собі. 
		
		\subsection{Критичний період}
			Другий етап~— так називається «критичний період». В роботах цього періоду послідовно викладалися: «критична теорія пізнання», етика, естетика s вчення про доцільності природи. Основна увага філософа зосередилося на~критичному аналізі пізнавальних здібностей людини, на~розробці відповідної теорії пізнання. Під впливом скептицизму та~емпіризму Юма, Кант ввів у~філософію поняття про негативні величини, висміяв захоплення сучасників містикою і~«духовідення». У цей період він надає великого значення використанню в~філософії досвідченого знання. Кант розширює рамки критичної філософії: принципи, знайдені в~теорії пізнання, застосовує в~соціальних сферах. Знайдена істина піддається багаторазовій перевірці, витримавши яку отримує додаткове обґрунтування, а~не витримавши, заміняється новою, що~піддається, в~свою чергу, перевірці, уточненню і~вдосконаленню. Перейшовши на~позиції критичної філософії, Кант не забував про природознавство. Він продовжував читати курси фізичної географії та~теоретичної фізики. Зберігав інтерес до астрономії і~«небесної механіки» і~написав дві статті на~цю тему: «Про вулкани і~місяць» і~«Дещо про вплив Місяця на~погоду». За два роки до того, як~заговорили про берлінський конкурс, він випустив роботу «Метафізичні початки природознавства». 
		
		\subsection{Теорія пізнання. Апріоризм}
			Знання завжди виявляють себе у~формі судження, в~якій думкою фіксується відношення чи зв'язок між поняттями~— суб'єктом і~предикатом судження. У математиці, філософії, природознавстві судження займають визначальне місце. Тому Кант ставить три питання:
			\begin{enumerate}
				\item Як можливі такі судження в~математиці.
				\item Як~вони можливі в~теоретичному природознавстві.
				\item Як вони можливі в~метафізиці.
			\end{enumerate}
			Вирішення цих питань Кант пов'язує з дослідженнями трьох головних пізнавальних властивостей. Чуттєвість~— здатність до почуттів, розум~— здатність до умовиводів, які доходять до ідеї, розсуд~— здатність до понять і~суджень. Кожна з пізнавальних властивостей розглядається Кантом крізь проблему апріорності. Проведене дослідження уможливило зробити йому висновок, що~існують апріорні форми чуттєвості (простір і~час), апріорні форми розсуду (категорії). Спроба знайти апріорні форми розуму призвела до висновку про існування антиномій розуму. Кант розділяє математику на~дві форми: чуттєве споглядання простору (геометрія) та~чуттєве споглядання змін у~часі (арифметика). Простір~— апріорна форма чуттєвого споглядання. Час~— апріорна форма чуттєвого споглядання змін, яка надає послідовностям подій загальності та~необхідності. 
		
		\subsection{Трансцендентальна логіка}
			Питання про можливість апріорних синтетичних суджень у~філософії («метафізиці») Кант пов'язує з дослідженням властивостей розуму. Він розглядає розум як~здатність створювати умовиводи, що~призводить до виникнення ідей. За Кантом, «ідеї»~— поняття про незаперечне. Розум створює три головні ідеї:
			\begin{enumerate}
				\item Ідею про душу як~цілісність всіх психічних явищ.
				\item Ідею про світ як~цілісність нескінченного ряду причиново зумовлених явищ (причин і~наслідків).
				\item Ідею про Бога як~причину усіх причиново створених явищ.
			\end{enumerate}
			Однак спроба розуму дати вичерпну відповідь про те, що~є світ як~ціле, призводить до суперечності. А саме: можна довести, що~світ не має початку в~часі, не має обмежень в~просторі. І можна довести, що~світ почав існувати лише з якогось моменту часу, що~світ обмежений просторово. Кожна зі схем доведення буде логічно істинною. Такі суперечності, як~виявив Кант, виникають у~розумі з необхідністю. А це свідчить, що~розум сам по собі має суперечливий характер. Кант пропонує розділити логіку на~два типи~— загальну логіку і~трансцендентальну логіку. Загальна логіка досліджує форми поняття, судження, умовиводу, повністю абстрагуючись від питання про зміст, який мислиться за допомогою тієї чи іншої форми думки. Трансцендентальна логіка звертає увагу на~те, що~надає знанню апріорного характеру, створює можливість виникнення загальних і~необхідних істин. Таким чином, фактично вчення Канта про пізнання є розгорненою трансцендентальною логікою.
		
		\subsection{Етика}
			Людина діє з необхідністю, оскільки вона з своїми думками, чуттями розміщується серед інших явищ природи і~в цьому відношенні підкоряється необхідності світу явищ. Разом з цим людина є моральною істотою. Як моральна істота, вона належить до світу духовного. І в~цій якості людина~— вільна. У разі, коли дії збігаються з моральним законом, здійснюються людиною на~підставі власної схильності, їх не можна назвати моральними. Дія людини буде моральною тільки в~тому випадку, коли індивід здійснює її з поваги до морального закону. Суперечність між моральною поведінкою людини і~результатом цієї поведінки в~емпіричному житті не знімається нашою моральною свідомістю. Не знаходячи справедливого стану речей у~світі явищ, моральна свідомість діє у~світі уявному. Існування таких понять, як~«свобода», «безсмертя», «Бог», пояснюється вірою в~уявний світ, даний тільки думкам, але неосяжний для думок (трансцендентний світ). Трансцендентність уявного світу буде існувати завжди, бо людина не здатна своїм розумом вийти за межі антонімічності загальних і~необхідних понять.
			
	\section{Філософська система Г. Гегеля і~його діалектичний метод}
		Вчення Гегеля є вищим досягненням німецької класичної філософії. Воно характеризується виключною широтою та~глибиною змісту, важливістю та~багатогранністю поставлених проблем. Вихідним пунктом філософської концепції Гегеля є тотожність буття та~мислення. Ця тотожність, гадає Гегель, є відносною, як~і~їх взаємопротилежність, і~в ній виникає роздвоєння на~протилежності, проте поки що~тільки в~думці на~суб'єкт думки та~на думку як~змістовний об'єкт. Мислення, з точки зору Гегеля, є не лише суб'єктною людською діяльністю, а~й незалежною від людини об'єктивною сутністю, першоосновою всього сущого. Мислення, стверджує Гегель, відчужує своє буття у~формі матерії, природи, яка є «інобуттям» цього об'єктивно існуючого мислення, або абсолютної ідеї. Гегель розглядає мислення (абсолютну ідею) не як~нерухому, незмінну першосутність, а~як процес неперервного розвитку пізнання, як~процес сходження від нижчого до вищого. Абсолютна ідея є активною і~діяльною, вона мислить і~пізнає себе, проходячи в~цьому розвитку три етапи:
		\begin{enumerate}
			\item До виникнення природи і~людини, коли абсолютна ідея перебуває поза часом і~простором у~стихії «чистого мислення» і~виступає системою логічних понять та~категорій, як~система логіки.
			\item Це духовне начало з самого себе породжує природу, яку Гегель називає «інобуттям» абсолютної ідеї.
			\item Третій етап розвитку абсолютної ідеї~— це абсолютний дух. На цьому етапі абсолютна ідея залишає створену природу і~повертається до самої себе, але вже на~основі людського мислення (самопізнання ідеї).
		\end{enumerate}
		Ці три етапи сформувались у~Гегеля в~самостійні складові частини його філософської системи: логіку, філософію природи та~філософію духу. Логіка є найважливішою частиною гегелівської системи, оскільки тотожність буття та~мислення означає, що~закони мислення, які й досліджує логіка, є дійсними законами буття. Логіка, на~думку Гегеля, є вченням про сутність усіх речей. Гегель виділяє три сходинки діалектичного методу, які діють у~розвитку як~загальнолюдського, так і~індивідуального мислення і~проявляються у~кожному конкретному пізнавальному аналізі, утворюючи при цьому потрійну єдність. Першою сходинкою «логічного» діалектичного методу, на~думку Гегеля, є розсудок. Розсудковий~— це підлеглий, але необхідний бік діалектичного мислення. Другою сходинкою діалектичного методу у~Гегеля є негативний розум як~проміжний етап між розсудком та~розумом. Третя сходинка~— спекулятивний розум, якого не спроможне досягти мислення, спрямоване на~кінечні природні речі. Саме в~цьому останньому вияві діалектика як~метод, на~думку Гегеля, досягає найвищої зрілості. Позиція Гегеля щодо розуму та~розсудку значно відрізняється від кантівської. У нього розум є не нижчим, а~вищим виявом мислення. Він вважає, що~розум та~розсудок мають складати єдину висхідну послідовність, де вони співвідносяться як~провідний та~підлеглий чинники, що~взаємообумовлюються та~проникають один в~одного. Завдання логіки, на~думку Гегеля, полягає в~аналізі наукового методу мислення. Він намагався показати, що~походження багатоманітного з єдиного начала може бути предметом раціонального пізнання, знаряддям якого є логічне мислення, а~основною формою~— поняття. Тому логіка, збігається з наукою про речі, що~осягаються думкою. 
	
		У філософії Гегеля слід чітко розрізняти діалектичний метод та~систему, що~суперечать одне одному, перебувають у~суперечності, яка виявляється в~таких моментах:
		\begin{enumerate}
			\item Метод виходить з визнання всезагальності розвитку. Система ж заперечує всезагальність розвитку, оскільки природа розвивається лише в~просторі, а~не в~часі. Система вимагає обмеження розвитку.
			\item Метод заснований на~визнанні всезагальності суперечності. Система ж вимагає вирішення усіх суперечностей та~встановлення несуперечливого стану.
			\item Метод вимагає відповідності руху думки стану, характерному для реальних процесів. Система ж передбачає конструювання зв'язків з голови.
			\item Метод вимагає постійного перетворення дійсності, а~система~— незмінності існуючого стану речей.
		\end{enumerate}
		
	\section{Антропологічна філософія Л. Фейєрбаха}
		Л. Фейєрбах був першим філософом, який піддав критиці філософську систему об'єктивного ідеалізму Гегеля. У Фейєрбаха був гегельянський період філософського розвитку, але в~лоні гегелівського «абсолютного ідеалізму» визрів і~його антитезис~— антропологічний матеріалізм. Згідно з Фейєрбахом, єдиними об'єктивними реальними речами є природа і~людина. Він закликає перейти від роздумів про потойбічні суті, як~це роблять ідеалісти, до вивчення природи і~людини. Основою філософії, її початковим пунктом має бути людина, а~не абсолютна ідея. Тому Фейєрбах сам назвав свою філософію «антропологією». Фейєрбах робить спробу, виходячи з антропологічного матеріалізму, розглянути різні форми суспільної свідомості і~передусім релігію. Не Бог створив людину, а~людина Бога. Божественна суть, стверджує Фейєрбах, це не що~інше, як~людська суть, звільнена від індивідуальних меж, об'єктивувалася, а~потім~— що~обожнює, шановна в~якості потойбічної суті, тобто Бога. Фейєрбах виступав не лише проти ідеалізму, але і~проти вульгарного матеріалізму Фохта, Молешотта, які зводили психічні явища до матеріальних фізико-хімічних і~фізіологічних процесів. Він постійно підкреслював, що~істина не є ні матеріалізм (маючи на~увазі вульгарний матеріалізм), ні ідеалізм, а~тільки антропологія.
		
	\section{Основні ідеї марксистської філософії. Проблема відчуження людини}
		Теоретичним джерелом філософії марксизму є німецька класична філософія, зокрема, філософія Гегеля і~Фейєрбаха. В результаті досить вправного поєднання гегелівської діалектики і~фейєрбахівського матеріалізму було створене нове філософське вчення, яке отримало назву діалектичного матеріалізму. Якісно відрізняючись від німецької класики за духом і~змістом, марксистська філософія успадкувала від неї раціоналізм як~спосіб пояснення і~осягнення дійсності. Віра в~силу розуму, наукове знання, соціальний прогрес представлені тут максимально повно, що~певною мірою визначило її оптимістичний характер і~забезпечило широку підтримку цього вчення з боку багатьох послідовників. Від багатьох інших філософських вчень філософія марксизму відрізняється чітким визначенням предмету дослідження: це найбільш загальні закономірності розвитку природи, суспільства та~людського мислення. При цьому вона використовує знання конкретних наук. Характерним для марксистської філософії є постановка і~чітке формулювання так званого основного питання філософії. Найбільш зрозуміло суть його розкрита в~праці Енгельса «Людвіг Фейєрбах і~кінець класичної німецької філософії» (1886 р.). На його думку, кожна філософська концепція незалежно від часу її створення та~конкретних філософських проблем, які в~ній аналізуються, в~кінцевому підсумку ставить одне і~теж питання, яке він називає основним філософським питанням. Суть його в~відношенні двох видів реальності: матеріальної і~ідеальної. Це питання має дві сторони:
		\begin{enumerate}
			\item Що є первинне, а~що~– вторинне, що~існувало раніше – матерія чи свідомість.
			\item Чи спроможна людська свідомість пізнати навколишній світ, отримати об'єктивно істинне знання про нього.
		\end{enumerate}
			
		Проблема людини у~марксизмі тісно пов'язана з теоретичним осмисленням такого суспільного феномена як~відчуження. Під останнім розуміється складне явище, змістом якого є перетворення самого процесу людської діяльності і~її результатів (творів, соціальних інститутів і~організацій, грошей, духовних цінностей та~ін.) в~силу, що~панує над людиною, тисне на~неї, диктує певні вимоги, силу, протилежну її бажанням та~прагненням.
		
		Проблема відчуження була вже досить глибоко проаналізована в~німецькій класичній філософії Регелем і~Фейєрбахом. Маркс високо оцінив роботу своїх попередників по осмисленню проблеми відчуження людини, але вважав, що~їхні концепції проблеми відчуження мають принципові недоліки. У Гегеля знищення відчуження людини зображається як~чисто духовний акт, здійснюваний філософом як~уособлення універсального розуму. Фейєрбах бачив корінь зла в~релігійному відчуженні, Маркса таке трактування і~вирішення проблеми відчуження не задовольняли. Фундаментальним, базовим для всякого відчуження людини, за Марксом, є економічне відчуження, або відчуження праці. Маркс виділяє такі форми відчудження як: відчудження людини від природи (природа постає як~знаряддя праці, чим більше людина створює речей, тим далі вона від природи), відчудження людини від своєї родової сутності (людина за своєю сутністю є істота універсальна, але в~капіталістичному суспільстві, де існує глибокий розподіл праці і~вузька спеціалізація людина може розвивати свої сутнісні сили тільки однобічно), відчудження від праці (праця постає не як~свободна, а~як примусова діяльність, викликана нуждою та~матеріальною необхідністю, а~продукти праці постають як~ворожі для людини сили, які закабаляють її, людина відчуває себе свободною тільки здійснюючи функції, які є спільними у~неї з твариною) та~відчудження людини від людини (це і~індивідуалізм буржуазного суспільства, й байдужість людей один до одного, коли відносини між людьми постають як~відносини між виробниками товарів, зростання конкурентної боротьби).
		
	\section{Основні етапи розвитку філософської думки в~Україні}
		\subsection{Часи існування Київської Русі (XI–XIV~ст.)}
			На зміну язичницькому світосприйманню приходить християнізована картина світу. Християнський світогляд стає панівним і~визначає формування філософського мислення.
			
			Саме християнська філософія першою збудила інтерес до філософської думки на~Русі. Характерною рисою філософської думки Київської Русі було «софійне» розуміння філософського знання, де «софія тлумачиться як~особистісно-плюралістичне знання. В давньоруській філософській думці світоглядна орієнтація спрямовується в~духовний світ людини. Людина мислилась не просто часткою космосу, природи, а~усвідомлювала себе її господарем, її «вінцем».
			
		\subsection{Доба Козаччини (XVI–XVIIІ ст.)}
			У розвитку філософської думки починається новий, ранньомодерний етап, характерною особливістю якого була відкритість переважної частини тогочасної філософії для впливу західного Ренесансу, Реформації і~контреформації. Філософська проблематика стосувалась насамперед бачення острожцями проблем Бога і~світу, Бога і~людини.
			
			Відповідно до характерної для цієї давньоукраїнської культури мислення філософію розуміли як~мудрість, а~засобом її осягнення вважали релігійний, містичний досвід.
			
		\subsection{XIX~— перша третина ХХ ст.}
			Цей період позначений впливом культури романтизму, взаємодією романтичної та~просвітницької філософії. Упродовж всього ХІХ століття існувало два основних філософських напрямків: «неофіційний», чи публіцистичний. У філософії цього напрямку  ішов напружений пошук самобутності української думки. Але цей напрямок відрізнявся несистемністю. Другий напрямок «офіційний», чи професійно-академічний. Який предметом філософського осмислення робив найактуальніші проблеми духовного життя суспільства та~здобутки природничих наук.
			
		\subsection{ХХ–ХХІ століття}
			Цей період визначався умовами існування України в~межах багатонаціональної країни СРСР, а~також   складним і~неоднозначним за соціально-економічними, культурними наслідками життям емігрантів, відірваних від своєї Батьківщини.
			
			Вся українська філософія, як~і~філософія інших радянських республік того часу, була «розчинена» в~єдиній пануючій тоді марк\-сист\-сько-ле\-нін\-ській філософії з єдиними вимогами, єдиними завданнями, цілями і~єдиним світоглядом. Склався догматичний розгляд традиційних філософських проблем, а~філософія перетворилася на~служницю політики і~офіційної ідеології.
			
	\section{Сутність позитивізму, його основні історичні форми}
		Незвичайність новітніх наукових відкриттів наприкінці ХІХ століття гостро поставила питання про  природу наукових понять, співвідношення чуттєвого і~раціонального моментів пізнання, емпіричного і~теоретичного знання, про істину та~її критерії, закономірності розвитку науки, наукової революції тощо. 
		
		З кінця ХІХ століття впливовим став позитивістський напрям з його установкою на~точне знання. Водночас у~ньому були чітко виражені й суб'\-єк\-тив\-но-іде\-а\-ліс\-тич\-на та~агностична тенденції. У своєму історичному розвитку позитивізм пройшов кілька етапів.
		
		Історичні форми:
		\begin{enumerate}
			\item 30-40 рр. ХІХ ст. У світоглядному плані позитивізм був негативною реакцією на~натурфілософські вчення; у~плані  соціальному він виражав типову для  буржуазного суспільства установку на~ використання даних науки, практичне оволодіння природою, коли загальносвітоглядні питання  взагалі перестають бути предметом інтересу дослідника. Згідно з позитивістськими поглядами, наука повинна бути позитивним знанням, яке спирається на~данні досвіду. Справа науки – констатувати, описувати, класифікувати факти досвіду, встановлювати зв'язки між ними, послідовність явищ, тобто виявляти закони. Щодо філософії, то вона, на~думку позитивізму, повинна займатися узагальненням висновків конкретних наук, систематизацією наукового знання. Філософії в~системі Конта відводиться роль науки про науки, енциклопедичної суми знань.
			\item Махізм~— «досвід», який, за Махом, є сукупність вихідних чуттєвих даних, «елементів», нібито нейтральних по відношенню до фізичного і~психічного. Філософ стверджував, що~поняття фізики, якими б абстрактними вони не були, завжди можна простежити до чуттєвих елементів, з яких вони побудовані.   Філософію Е.Мах зводив фактично до методології наукового пізнання. Саме пізнання  він розглядав як~процес прогресивної адаптації людини до середовища. Він вважав, що~основою наукового знання є не факти, а~відчуття.  
			\item Неопозитивізм (або логічний позитивізм) виник у~20-х роках ХХ~століття і~розвивався як~течія, що~претендувала на~аналіз філософсько-методологічних проблем, висунутих у~ході науково-технічної революції. Неопозитивізм зводить філософію до аналізу мови науки і~намагається вилучити з науки поняття («метафізичні залишки») які, на~його думку, не ґрунтуються на~фактах. Головним предметом  дослідження неопозитивізму стає наукова мова.
			\item Під впливом ідей Карла Поппера в~70-роки ХХ~ст. склалась течія постпозитивізму. Характерними для  постпозитивізму є проблеми фальсифікації: правдоподібності наукових теорій; раціональності; розуміння; соціології знання.
		\end{enumerate}

	\section{Екзистенціалізм як~течія в~сучасній західній філософії: релігійний та~атеїстичний напрямки}
		Екзистенціалізм стає однією з провідних течій філософії Західної Європи саме на~початку ХХ~ст.~— епохи втрати традиційних релігійних цінностей, моральних орієнтирів, епохи самотності людини в~потоці Буття. На відміну від філософської антропології, екзистенціалізм запропонував іншу парадигму людини.  
		
		Фундаментальною рисою екзистенціалізму стало усвідомлення людини як~унікальної, неповторної істоти. Ця парадигма дістала назву  філософії існування людини. Буття кожної людини розглядається екзистенціалістами як~абсолютне. Центральним поняттям цієї філософської течи стає поняття екзистенції~— як~специфічно людського  буття, буття на~межі, буття в~граничних станах~— відчаю, жаху.
		
		Треба мати на~увазі ще одну сутнісну рису екзистенціалізму~— це осмислення людини за межами її раціоналізму в~само-переживанні і~співпереживанні, які відкривають двері до істинних таємниць людського Я.
		
		Релігійний екзистенціалізм є результатом розвитку християнської культури. Релігійний екзистенціалізм виходить з ідеї творення людини Богом. Проте Бог створює людину не завершеною, а~відкритою до діалогу і~розвитку. Бог не обмежує людину рамками якоїсь готової сутності, вона повинна стати собою через своє існування. На відміну від атеїстичного екзистенціалізму, релігійний екзистенціалізм переконаний, що~сутність людини виходить за межі земного існування~— вона знаходиться як~у~вічному, божественному існуванні, так і~у існуванні людського Я.
		
		Атеїстичний екзистенціалізм ґрунтується на~твердженні, що~людська сутність є розгортання людського існування в~цьому світі. Трансцендентна природа людської сутності заперечується. Атеїстичний екзистенціалізм залишає людину наодинці з собою, без ідеї Бога. Філософія атеїстичного екзистенціалізму~— це філософія абсолютно самотньої людини. Це філософія людини, що~звалила на~себе знання про ілюзорність всіх обіцянок релігії і~несе його, подібно Сізіфу, на~вершину свого життя.
		
	\section{Основні ідеї неотомізму}
		З кінця ХІХ і~у ХХ столітті (в епоху значних зрушень у~суспільному житті та~со\-ці\-аль\-но-пси\-хо\-ло\-гіч\-них катаклізмів) значно підвищився інтерес до релігії як~способу розв'язання людських проблем. Ре\-лі\-гій\-но-фі\-ло\-соф\-ські доктрини, які, здавалося б, відійшли в~минуле, здобули «нове дихання», почали переосмислюватися до потреб нашого часу і~менталітету сучасної науки.
		
		Найбільш  типовим у~цьому відношенні  напрямом вважається неотомізм~— сучасна версія вчення Фоми Аквінського (ХІІІ ст.), який у~свій час  поставив за мету поєднати релігію з наукою, примирити віру і~розум. Томізм почав широко розповсюджуватися насамперед у~католицьких країнах~— як~«вічна філософія». У своєму оновленому вигляді він став називатися «неотомізмом».  В  основі цього напряму лежить принцип зверхності теології, яка на~думку прибічників неотомізму, втілює єдність пізнавального та~практичного відношення до світу.
		
		Вони не заперечують наукові знання про природу і~суспільство, їх реальність, але стверджують їх залежність  від Бога. Людський розум пізнає ідеї, закладені в~світ Богом.      
		
		Таким чином, релігійна філософія розробляє обґрунтування теїзму, існування Бога, його природи і~відношення до світу і~людини.
		
		До найбільш відомих представників неотомістського напряму відносяться Е.\,Жиль\-сон (1884–1978) та~Ж.\,Марітен (1882–1973)~— у~Франції; К.\,Ранер (1904–1984)~— у~Німеччині. Вони відмовляються від «раціональних» доведень і~переносять акцент на~ек\-зис\-тен\-ці\-аль\-но-ан\-тро\-по\-ло\-гіч\-ні мотиви віри в~Бога~— в~тому розумінні, що~ця віра, ідея Бога як~абсолютного буття, розумного й морального першопочатку світу виражає фундаментальну потребу й установку людини, дає їй правильну орієнтацію для розв'язання своїх життєвих проблем, забезпечує гуманістичний вимір сучасного на\-у\-ко\-во-тех\-ніч\-но\-го і~соціального прогресу.  
		
		Зазначимо, що~специфічною рисою неотомізму, як~і~релігійної філософії в~цілому, є залучення до своєї доктрини проблем суспільства, науки, буття людини.
		
	\section{Історичні форми діалектики як~вчення про універсальні зв'язки, зміни та~розвиток. Діалектика й метафізика}
		Різні філософи та~філософські напрямки по-різному вирішують проблему відношення людини до світу, але спільним залишається те, що~вони вирішують її специфічно філософськими засобами. Тобто всі філософи займаються не частковим, а~цілим, прагнучи кожне конкретне явище підвести під всезагальні поняття філософії і~користуються для цього не конкретними методами, а~специфічним філософським інструментарієм~— відповідним способом мислення~— діалектичним чи метафізичним.
		
		«Діалектика» у~перекладі з давньогрецької мови означає суперечку, суперечність. Діалектика як~діалог стала першим способом теоретизування про світ, людину, суспільство. Вона дозволила зрозуміти, що~категорії~— це найзагальніші поняття, якими оперує людина. Простір, матерія, рух, форма та~інші категорії є не просто словами, а~формами мислення.
		
		Отже, в~Античній філософії сформувалися два підходи до розуміння діалектики. Перший тлумачив її як~мистецтво суперечки, форму діалогу, який направлений на~пошук істини, узгодження суперечливих точок зору, їх узагальнення та~подолання. Другий підхід характеризував діалектику як~метод філософствування, направлений на~пізнання загального, істинного, об'єктивного.
		
		В епоху Середньовіччя осмислення діалектики відбувалося в~контексті дискусії щодо природи універсалій), зміст якої полягав у~вирішенні питання: чи існують реальні прообрази загальних понять? Фактично ця суперечка, яка поділила філософів на~реалістів (вважали, що~універсалії існують реально і~передують виникненню одиничних речей) та~номіналістів (стверджували, що~в~світі існують лише одиничні речі, які позначаються відповідними поняттями), мала на~меті розв'язати проблему адекватного відображення дійсності в~мисленні людини.
		
		Наприкінці XVIII~— на~поч. ХІХ ст.ст. уявлення про діалектику змінюється. Її протиставляють метафізичному, догматичному стилю мислення, який був характерним для методологічних та~наукових пошуків.
		
		Представники німецької класичної філософії (І. Кант, І. Фіхте, Ф. Шеллінг, Г. Гегель) протиставили діалектику й метафізику, назвали їх різноспрямованими, хоча й взаємозалежними способами мислення.
		
		Дотримуючись цього ж принципу, К. Маркс та~Ф. Енгельс розробили теорію матеріалістичної діалектики. На відміну від Г. Гегеля, який за початок і~джерело розвитку взяв ідеальне начало, К. Маркс і~Ф. Енгельс говорять про розвиток як~характеристику, що~внутрішньо притаманна природі та~суспільству. Людське мислення здатне відтворювати цей розвиток через формування й наповнення змістом відповідних категорій і~законів. Тому філософія марксизму розрізняє об'єктивну і~суб'єктивну діалектику. Об'єктивна діалектика виявляє закони розвитку об'єктивної реальності, незалежної від волі і~свідомості людини. Суб'єктивна діалектика~— це відображення об'єктивної діалектики у~свідомості людей.
		
		Метафізика у~перекладі з давньогрецької мови означає буквально після фізики. А фізикою (фізіс) називали науки про природу. Для метафізичного способу мислення речі та~їх мислене відображення є завжди незмінними, нерухомими, тотожними собі. Метафізика не схоплює всезагальних зв'язків між речами та~явищами, не розглядає їх взаємної зумовленості. Метафізичне мислення розуміє рух лише як~просте переміщення речей у~просторі, що~відбувається по колу або по прямій лінії.
		
		Схематично протилежність діалектики й метафізики як~способів осягнення світу показана у~табл.~\ref{tab:dialectics-metaphysics}.
		\begin{table}[!htbp]
		\centering
		\caption{Протилежність діалектики й метафізики}
		\label{tab:dialectics-metaphysics}
			\begin{tabular}{v{7.05em}v{9em}v{9em}}
				\toprule
					& Діалектика & Метафізика \\
				\midrule
					Речі & Змінні, перебувають у~зв'язку та~розвитку & Незмінні, ізольовані одна від~одної \\
					Джерело руху & Внутрішня суперечність & Поштовх ззовні \\
					Механізм руху & Стрибок & Поступальність \\
					Спрямованість руху & По~спіралі, що~піднімається вгору & По~колу або по~прямій лінії \\
					Світ у~цілому & Процес & Стан \\
					Поняття & Змінні, взаємозв'язані, тотожно-протилежні & Незмінні, ізольовані одне від~одного, несуперечливі \\
				\bottomrule
			\end{tabular}
		\end{table}

	\section{Сутність закону єдності й боротьби протилежностей. Роль суперечностей у~процесі розвитку}
		Єдність і~боротьба протилежностей~— один з основних законів діалектики. Він характеризує джерело саморуху й розвитку явищ природи і~соціально-історичної реальності. Закон єдності й боротьби протилежностей в~діалектиці займає центральне місце. Це – сутність, «ядро» діалектики.

		Кожний предмет~— це єдність протилежних сторін, властивостей, тенденцій. У кожному предметі, явищі є позитивні й негативні сторони, те, що~росте, розвивається, і~те, що~відживає. Якщо на~перший погляд здається, що~в~процесі розвитку предметів чи явищ відсутні суперечності, то завдання дослідника полягає в~тому, щоб віднайти їх. Лише шляхом розкриття внутрішніх суперечностей можна пізнати предмети, їх сутність, закони їх розвитку.
		
		Що означає «єдність протилежностей»? По-перше, взаємообумовленість протилежностей, тобто існування однієї протилежності передбачає необхідну наявність іншої протилежності. Наприклад, лівий~— правий, добро~— зло, притягання~— відштовхування, низ~— верх тощо.
		
		По-друге, перехід однієї протилежності в~іншу шляхом заперечення одна одної. Протилежності перебувають в~боротьбі одна з одною. їх боротьба~— це природний закономірний наслідок того, що~протилежності всередині предмета чи явища одночасно взаємно обумовлюють і~заперечують одна одну.
		
		Єдність протилежностей має відносний характер. Це пояснюється тим, що~сталість, незмінність предмета чи явища тимчасова, що~предмет має свій початок і~кінець. А боротьба протилежностей має абсолютний характер, тому що~рух (розвиток) не припиняється ні на~хвилину в~результаті боротьби протилежностей.
		
		Боротьба протилежностей~— це складний процес виникнення, розвитку та~вирішення суперечностей. Зміст закону єдності й боротьби протилежностей виражається через взаємодію категорій тотожності, відмінності, протилежності, суперечності.
		
		А відносини між протилежностями називаються суперечностями. Суперечності~— це система відносин, в~межах якої протилежності породжують одна одну, взаємопроникають і~переходять одна в~одну, породжуючи щось нове. Наприклад, мінливість і~спадковість, які є факторами біологічної еволюції, взаємопроникають і~переходять одна в~одну, породжуючи кожного разу новий вид живої матерії.
		
		Суперечності бувають різних видів: внутрішні, зовнішні, основні, неосновні, антагоністичні, неантагоністичні.
		
		Внутрішні суперечності~— це взаємодія протилежностей у~системі внутрішніх відносин предмета (наприклад, соціальні відносини в~суспільстві).
		
		Зовнішні суперечності~— це взаємодія протилежностей, що~належать різним предметам (наприклад, будь-яка система і~навколишнє середовище).
		
		Основні суперечності~— взаємодія протилежностей, які створюють джерело саморозвитку предмета в~певний період. Вони можуть проявлятися як~складний комплекс глобальних проблем (наприклад, екологічна, енергетична, продовольча тощо).
		
		Неосновні суперечності впливають на~основні, але не визначають їх форму (наприклад, споживча вартість~— вартість в~ринкових відносинах).
		
		Антагоністичні суперечності~— це взаємодія протилежностей, що~мають максимально діаметральні тенденції своєї еволюції. Вирішення таких суперечностей нерідко здійснюється революційним шляхом, тобто переходом в~нову якість.
		
		Прикладом неантагоністичних суперечностей можуть бути відносини між однотипними соціально-класовими спільнотами.
		
		Кожна конкретна суперечність виникає і~проходить певний шлях. Це не означає, що~предмет може бути без суперечностей. Мова йде про ту чи іншу конкретну суперечність. Предмет же може мати інші суперечності. Абсолютно тотожним самому собі предмет не може бути. На певній стадії розвитку окремі сторони досягають такого ступеня суперечностей, що~вже не можуть існувати в~єдності. Наступає момент вирішення суперечностей. Це вирішення відбувається шляхом боротьби. Суперечності не примиряються, а~лише долаються.
		
		Подолання суперечностей означає усунення старого і~виникнення нового. Дуже важливо для практичної діяльності знати, «зловити» момент вирішення суперечностей.
		
	\section{Взаємоперехід кількості та~якості в~процесу розвитку. Сутність категорії «міра»}
		Щоб з'ясувати суть закону взаємного переходу кількісних змін у~якісні, його прояви й діяння, необхідно розкрити зміст таких категорій, як~якість, кількість, властивість, міра, стрибок. Якість~— це тотожна буттю визначеність. Якщо річ втрачає визначеність, то вона втрачає і~свою якість. Однак таке визначення ще не дає повного уявлення про якість речі. Розрізняють якість як~безпосередню визначеність, що~сприймається органами чуття, і~якість як~сукупність суттєвих властивостей речі, що~сприймається опосередковано через мислення, абстрагування. Якість і~відчуття~— це одне й те ж, вважав Л. Фейербах.

		Властивість як~категорія визначає одну із сторін речі. Якість речі визначається виключно через її властивості. Між властивістю і~якістю існує діалектичний взаємозв'язок. Поняття якості у~буденному і~філософському розумінні не збігаються. Отже, є така якість, яка сприймається відчуттям (мова може йти про відчуття несуттєвих властивостей предмета), і~якість як~філософська категорія, котра означає сукупність суттєвих властивостей предмета, із втратою яких предмет неодмінно втрачає свою визначеність. Суттєві властивості речі не сприймаються на~рівні відчуттів, бо є результатом теоретичного узагальнення. Гегель стверджував, що~якість~— це «сутнісна визначеність».
		
		Кількість~— філософська категорія, що~відображає такі параметри речі, явища чи процесу, як~число, величина, обсяг, вага, розміри, темп руху, температура тощо. За висловлюванням Гегеля, кількість~— це «визначеність у~межах даної якості». Спочатку кількісні зміни не зачіпають якості предмета і~тому на~це не завжди звертають увагу, зауважує Гегель. Якісні зміни, що~відбуваються в~об'єктивному світі, здійснюються лише на~основі кількісних змін. Іншого шляху до появи нового просто не існує.
		
		У гегелівській філософії взаємоперехід кількісних змін у~якісні виступає як~закон мислення, логіки. Однак суть цього закону і~полягає в~тому, що~він має відношення не лише до мислення, логіки, а~й до самої дійсності, її розвитку. Кожний перехід кількісних змін у~якісні означає одночасно і~перехід якісних змін у~нові кількісні зміни. «Зміна буття,~— писав Гегель,~— суть не лише перехід однієї величини в~іншу, а~й перехід якісного в~кількісне і~навпаки».
		
		Єдність, взаємозв'язок і~взаємозалежність якості і~кількості виявляються в~понятті міра. Будь-який предмет, явище, процес мають свою міру, тобто якісно-кількісну визначеність. Міра~— це межа кількісних змін, в~рамках якої предмет залишається тим, чим він є, не змінюючи своєї якості як~сукупності корінних його властивостей. Порушення міри предмета веде до переходу в~інше. Стара якість зникає, а~нова виникає. Разом з тим виникає і~нова міра. Так відбувається розвиток всього сущого.
		
	\section{Поняття й сутність діалектичного заперечення. Категорія «зняття»}
		Закон заперечення заперечення~— один із основних законів діалектики, який відображає поступальність, спадкоємність у~процесі розвитку предметів і~явищ об'єктивної дійсності. Вперше у~філософії розгорнуту характеристику закону заперечення заперечення дав Гегель, але він тлумачив його з ідеалістичних позицій.

		В основі закону лежить діалектичне заперечення як~об'\-єк\-тив\-ний і~суттєвий момент процесу розвитку. Діалектичне заперечення означає не просте знищення чи механічне відкидання старої якості, а~її подолання, зняття. Воно включає момент внутрішнього зв'язку зі старим, утримання та~збереження позитивного, що~міститься у~старій якості, і~тим самим становить умову подальшого розвитку, можливість нового заперечення. Діалектичне заперечення~— це насамперед зумовлена суперечливістю предмета внутрішня неминучість його якісного перетворення. Як уже зазначалося, все існуюче має свої внутрішні суперечності, які наростають, загострюються і~зрештою досягають такого стану, коли подальший розвиток предмета стає неможливим без їхнього розв'язання. Процес розвитку відношення протилежностей у~рамках певної суперечності має такі етапи:
		\begin{enumerate}
			\item Вихідний стан об'єкта.
			\item Розгортання протилежностей, роздвоєння єдиного, перетворення об'єкта~— набуття нового якісного стану (перше заперечення), поява нових протилежностей.
			\item Розгортання нових протилежностей, роздвоєння єдиного (як нової якості), перетворення об'єкта~— набуття нової якості (друге заперечення), і~т.д.
		\end{enumerate}
		В цьому процесі кожний з етапів виступає запереченням попереднього, а~весь процес розвитку~— запереченням заперечення.
		
		Сутність закону заперечення заперечення полягає у~відображенні напрямку і~форми процесу розвитку в~цілому. Повторимо, він відображає спадкоємність як~характерну рису процесу розвитку; бо в~кожному новому ступені розвитку зберігається те позитивне, що~було на~попередній стадії розвитку. Водночас кожний новий ступінь розвитку~— це не просте, механічне поєднання позитивного змісту попередніх стадій розвитку, а~перехід у~вищу фазу розвитку, ствердження якісно нового, більш багатого змісту. Умовно процес розвитку можна зобразити у~вигляді спіралі. Чому? Оскільки має місце повторюваність старого на~вищій основі, то виявляється, що~розвиток іде не по прямій, а~начебто по спіралі, наближаючись з кожним циклом до старого, оскільки є повторюваність, і~віддаляючись від нього, оскільки це нове.
		
		Важливо підкреслити, що~розвиток~— це складний і~суперечливий процес. У ньому є і~висхідна, і~низхідна лінії, і~прогрес, і~регрес. Нелогічно виключати регрес з розвитку. Бо регрес~— це теж форма розвитку, хоч і~за низхідною лінією. Закон заперечення заперечення, як~і~інші закони діалектики, є законом будь-якого розвитку, будь-якого руху взагалі і~не лише за прогресивною висхідною лінією, а~й за низхідною, регресивною.
		
	\section{Розкрити взаємозв'язок категорій діалектики, вибрати один із запропонованих варіантів: «форма» і~«зміст»~— співвідношення зовнішньої та~внутрішньої форми в~процесі розвитку}

		Кожен предмет, явище, процес становлять певну цілісність, що~складається з безлічі структурних елементів, вза\-є\-мо\-зв'я\-за\-них між собою, має відмітні ознаки, що~характеризують предмет як~окреме утворення і~відрізняють даний предмет від інших. Тобто усі вони мають свій зміст і~форму. Зміст~— це система зв'язків і~відносин між елементами, яка поєднує їх у~цілісність, із притаманними їй властивостями, особливостями, якісною визначеністю. Сукупність складових елементів, спосіб їх розташування і~зв'язку визначає структуру предмета. Наприклад, соціальна структура суспільства містить у~собі всіх соціальних суб'єктів із усім різноманіттям і~суперечливістю їх відносин (особи, групи, класи, нації тощо). Форма~— це спосіб організації змісту, його вираження й існування. Розрізняють внутрішню і~зовнішню форму. Наприклад, у~книзі внутрішньою формою є спосіб побудови, викладу матеріалу (розповідь, повість, роман). Зовнішню її форму складає розмір, обсяг, особливості обкладинки.
		
		Зміст і~форма взаємозв'язані і~взаємно обумовлюють один одного. їх відмінність не можна абсолютизувати. Те, що~в~одній системі зв'язків і~відносин виступає як~зміст, в~іншій~— як~форма. Наприклад, економічні відносини виступають як~соціальна форма, у~якій і~за допомогою якої здійснюється виробництво. У той же час економічні відносини характеризуються притаманним їм змістом, який виступає як~відношення між людьми в~процесі виробництва, зумовлені їх взаємним відношенням до засобів виробництва і~результатів праці. Ці відносини самі проявляються в~конкретно-історичних формах власності, у~способі з'єднання робочої сили зі знаряддями праці, у~формах розподілу, обміну, споживання.
		
		Зміст завжди зв'язаний певною формою, а~форма наповнена певним змістом. Зміст є визначальним у~його співвідношенні з формою. Розвиток і~зміна змісту зумовлює зміну форми. Життєві сили, закладені в~зерні, розвиваючись, розривають, скидають стару форму~— форму зерна~— і~здобувають форму рослини, з його кореневою системою, стеблом, листами, квітами, плодами… У той же час форма може активно впливати на~зміст, визначаючи його специфіку. Так, зміст суспільного життя багато в~чому залежить від форми державного управління, що~може або сприяти прогресу, або призвести суспільство в~стан кризи. Тоді виникає необхідність заміни старої форми новою. Важливо також зазначити, що~один і~той же зміст може приймати різні форми. Наприклад, утвердження капіталізму як~суспільного ладу проявилося в~специфічних формах в~Англії, Німеччині, Франції, CШA…
		
		Зміст, структура, форма~— це тільки різні сторони предметів, явищ, процесів, у~них знаходить свій вираз складність предметів, явищ і~взаємин між ними. Вони один без одного не існують. Наприклад, суспільна свідомість за своїм змістом є відображена в~голові людини і~закріплена пам'яттю система знань про навколишній світ і~про саму людину. Вона є об'єктивною за своїм походженням. Цей зміст виявляється й існує у~формах суспільної свідомості: наукова свідомість, політична, правова, моральна, естетична, релігійна, філософська, ідеологія. Взаємозв'язок форм суспільної свідомості складає його структуру.
		
	\section{Буття як~філософська категорія. Форми буття та~їх діалектична єдність}
		
		Питання про те, як~все існує, яке його буття, розглядаються в~онтології. Онтологія~— це вчення про суще, про першооснови буття: система найзагальніших понять буття, за допомогою яких здійснюється осягнення дійсності. Термін «онтологія» запровадив у~XVII ст. німецький філософ Р.Гоклініус. Під онтологією розуміється окрема галузь філософського знання, яка досліджує сутність буття світу, основи всього сущого: матерію, рух, розвиток, простір, час, необхідність, причинність та~інше.
		
		Розглядаючи проблему буття, філософія виходить із того, що~світ існує. Філософія фіксує не просто існування світу, а~більш складний зв'язок всезагального характеру: предмети та~явища світу. Вони разом з усіма їхніми властивостями, особливостями існують і~тим самим об'єднуються з усім тим, що~є, існує у~світі. За допомогою категорії «буття» здійснюється інтеграція основних ідей, які виділяються в~процесі осягнення світу «як цілого»: світ є, існує як~безмежна та~неминуча цілісність; природне і~духовне, індивіди і~суспільство існують у~різних формах; їх різне за формою існування~— передумова єдності світу; об'єктивна логіка існування та~розвитку світу породжує сукупну реальну дійсність, яка наперед задана свідомості.
		
		Всезагальні зв'язки буття проявляються через зв'язки між одиничними і~загальними відношеннями предметів та~явищ світу. Цілісний світ~— це всезагальна єдність, яка включає в~себе різноманітну конкретність і~цілісність речей, процесів, станів, організмів, структур, систем, людських індивідів та~інше. За існуючою традицією їх можна назвати сущими, а~світ в~цілому~— сущим. Кожне суще~— унікальне, неповторне в~його внутрішніх і~зовнішніх умовах існування. Визначеність сущого характеризує місце і~час його індивідуального буття. Умови цього буття ніколи не відтворюються знову і~не залишаються незмінними.
		
		Визнання унікальності кожного сущого особливо важливе для вчення про людину, воно націлене на~визнання в~кожній людині неповторної істоти. Разом із цим пізнання та~практика потребують того, щоб будь-яке одиничне явище знаходило своє місце в~системі зв'язків, об'єднувалось у~групи, узагальнювалось у~всеосяжну цілісність. Визначаючи подібність умов, способів існування одиничних сущих, філософія об'єднує їх у~різноманітні групи, яким притаманна загальність форм буття. Серед основних форм буття розрізняються:
		\begin{enumerate}
			\item Буття речей (тіл), процесів, які у~свою чергу поділяються на~буття речей, процесів, стан природи, буття природи як~цілого; буття речей і~процесів, вироблених людиною.
			\item Буття людини, яке поділяється на~буття людини у~світі речей і~специфічне людське буття.
			\item Буття духовного (ідеального), яке існує як~індивідуальне духовне і~об'єктивне (позаіндивідуальне) духовне.
			\item Буття соціального, яке ділиться на~індивідуальне (буття окремої людини в~суспільстві та~в~історичному процесі) і~суспільне буття.
		\end{enumerate}
		
		Виділяючи головні сфери буття (природу, суспільство, свідомість), слід враховувати, що~розмаїття явищ, подій, процесів, які входять у~ці сфери, об'єднані певною загальною основою. Особливе місце в~онтології посідає буття духовного та~різноманітних форм його прояву. Дух, душа, духовне, духовність, свідомість, ідеальне~— поняття, вживані в~різних значеннях і~смислах в~міфології, релігії, філософії.
		
	\section{Категорія «матерія» у~філософії. Сучасна наука про основні форми й структурність матерії}
		Термін «матерія» у~перекладі з~латинської означає «речовина». В матеріалістичній філософії під матерією розуміють субстанцію~— те, що~лежить в~основі всіх речей, явищ і~процесів. Категорія «матерія» дозволяє зрозуміти єдність різноманітних природних та~штучно створених предметів і~систем, а~також встановити відношення й зв'яз\-ки між ними. Нею не вичерпуються лише ті предмети і~явища,  які безпосередньо доступні органам чуття людини. Вона описує й ті можливі речі, процеси, системи, світи що~стануть доступними для пізнання у~майбутньому, коли будуть удосконалені засоби спостереження.
	
		Термін «матерія» був уведений Арістотелем. Проаналізувавши погляди своїх попередників~— натурфілософів, він прийшов до висновку, що~досократики за першооснову світу брали саме матерію: Фалес~— воду, Анаксімен~— повітря, Геракліт~— вогонь, Демокріт~— атоми тощо. Він писав, що~«більшість перших філософів вважали початком усього одні лише матеріальні початки, а~саме те, з~чого складаються всі речі, з~чого як~першого вони виникають і~на що~як останнє вони, гинучи, перетворюються». У власній філософії він використав термін «матерія» для позначення складової частини будь-якої речі як~її можливість (потенція). На його думку, дійсністю (або деякою річчю) матерія ставала лише завдяки активній формі.
		
		В епоху Відродження Джордано Бруно розглядав Всесвіт як~такий, у~якому всі існуючі речі мають дві субстанції: формальну та~матеріальну. У нього матерія є єдиною і~пізнається лише за допомогою розуму. На відміну від Арістотеля, Бруно вважав, що~матерія є одночасно потенційною (можливою) і~актуальною (дійсною). Будучи абсолютною, вічною, єдиною, матерія у~нього отримує першість перед формами, які постійно змінюють одна одну в~матерії. Матерію, що~містить у~собі всі форми, він назвав природою, яка виступає прообразом і~верховною силою Всесвіту. Він звільнив поняття об'\-єк\-тив\-ної першооснови світу від обов'\-яз\-ко\-во\-го (для попередніх поглядів на~матерію) зв'яз\-ку з~конкретним субстратом~— загальним матеріальним носієм властивостей речей. За Джордано Бруно будь-які конкретні його види (вода, вогонь, атоми тощо) є речами, але не початком.
		
		У Новий час зміст категорії «матерія» змінюється. Рене Декарт під матерією розуміє протяжну субстанцію, сутність якої він зводив до наявності трьох вимірів: довжини, ширини та~висоти. А всі властивості матерії, які сприймаються органами чуття (вагу, колір, твердість тощо), він вважав випадковими. У нього матерія є пасивною субстанцією, яка може ділитися безкінечно, заповнює весь простір і~всюди залишаться тотожною собі. Це був раціоналістичний погляд на~матерію. Дж. Локк запропонував протилежний погляд на~сутність матерії. У нього як~емпірика матерія є умовним поняттям, яке можна одержати шляхом абстрагування від конкретних і~змінних властивостей окремих речей. П. Гольдбах ототожнив матерію з~природою, що~є єдиним цілим, поза яким ніщо не може існувати. Вона є нескінченною у~просторі та~часі, протяжна, подільна, має непроникливість, здатна набувати будь-яких форм, які сама ж і~породжує.
		
		Наукове розуміння матерії в~Новий час різко відрізнялось від філософського. Оскільки природознавці експериментально досліджували конкретні види речовини в~її трьох агрегатних станах, вони зводили поняття матерії саме до речовини. При цьому виділяли деякі її властивості як~абсолютні, незмінні. Зокрема, вважали, що~речовина складається з~неподільних атомів, які мають незмінну масу. Відкриття у~фізиці на~рубежі XIX і~XX століть довели необґрунтованість таких тверджень. 
		
		У середині XIX століття формується марксистське вчення про матерію, в~якому розмежовуються філософське та~при\-род\-ни\-чо-на\-у\-ко\-ве її розуміння. Енгельс писав, що~матерія~— це дещо об'\-єк\-тив\-но існуюче, тілесне, наділене певними загальними властивостями, такими як~притягування й відштовхування, дискретність та~неперервність і~т.д. Він писав: «Такі слова як~„матерія“ і~„рух“ є не більше, ніж скорочення, в~яких ми охоплюємо відповідно до їх загальних властивостей множину різних речей, які сприймаються чуттями. Тому матерію і~рух можна пізнати лише шляхом вивчення окремих речовин і~окремих форм руху; і~поскільки ми пізнаємо останні, постільки ми пізнаємо також і~матерію, і~рух як~такі». Отже, в~марксистській філософії матерія є лише створеною в~теорії розумовою абстракцією, яка відображає об'\-єк\-тив\-но існуючий світ.
		
		Розвиваючи марксистське вчення про матерію, В.\,І.\,Ленін дав їй таке визначення: «Матерія є філософською категорією для позначення об'\-єк\-тив\-ної реальності, яка дана людині у~відчуттях її, яка копіюється, фотографується, відображається нашими відчуттями, існуючи незалежно від них». У наведеному визначенні матерії зазначається лише одна абсолютна властивість матерії~— бути об'\-єк\-тив\-ною реальністю, тобто існувати незалежно від волі й свідомості людини. Дане визначення має методологічне значення для розвитку природознавства, оскільки вказує на~невичерпність матерії, змінність її форм, які підлягають конкретно-науковому дослідженню.
		
		Сучасна наука вивчає конкретні форми та~рівні організації матерії, поглиблюючи наші знання про навколишній світ. Вона виробила уявлення про матерію як~складну самоорганізовану систему, яка перебуває у~постійних змінах, елементи якої зв'я\-за\-ні маж собою. Зокрема, природничі науки розглядають такі рівні організації матерії:
		\begin{enumerate}
			\item Нежива природа. Вона має складну будову. Вона складається з~елементарних частинок: атомів, молекул, макротіл, планет, галактик, систем галактик.
			\item Біологічний рівень організації матерії. До нього входять системи доклітинного рівня: нуклеїнові кислоти і~білки, клітини, багатоклітинні організми (рослини і~тварини); надорганізмові структури (популяції, види, біоценози тощо). Біологічний рівень організації утворює біосферу.
			\item Соціальний рівень організації матерії. Він є особливим типом матеріально системи, що~називається людським суспільством. У ньому формуються та~функціонують такі структурн.~е.ементи як~сім'я, історичні форми спільності людей (рід, плем'я, община, народність, нація), класи, страти та~інші соціальні групи, що~утворюються за різними ознаками. Суспільство має свою історію, яка включає минуле, теперішнє й майбутнє у~їх нерозривній єдності.
		\end{enumerate}
		
		Названі рівні організації матерії взаємодіють між собою, складаючи у~своїй єдності Метагалактику (Всесвіт). Різні рівні і~форми організації матерії досліджуються різними природничими та~суспільними науками. Проте оскільки світ єдиний, то й названі науки утворюють цілісну пізнавальну систему, яка виявляє не тільки особливості кожної, окремо взятої форми існування матерії, але й відношення та~зв'яз\-ки між ними. Тому особливу роль у~системі сучасної науки відіграють так звані міждисциплінарні науки: синергетика, теорія систем, інформатика, біогеохімія, кібернетика та~інші.


	\section{Рух як~спосіб існування матерії та~його основні форми. Рух і~розвиток}
		Рух~— це будь-які зміни і~переміщення тіл. Вже Геракліт не лише визнавав всезагальний характер руху, але й виявив його суперечливість: все існує і~в той же час не існує; все тече й постійно змінюється; все перебуває у~постійному процесі виникнення та~зникнення. Причиною змін він називав взаємодію речей.
		
		Справді, сучасна наука виявила складну будову матерії і~нерозривний зв'язок між її структурними елементами. Цей вза\-є\-мо\-зв'язок, взаємодія різних рівнів організації матерії, а~також взаємодія елементарних частинок в~окремих тілах і~спричинюють якісні та~кількісні зміни, що~відбуваються у~матеріальному світі. Отже, рух матеріальних тіл викликають їх внутрішні і~зовнішні взаємодії, поза якими існування цих тіл неможливе. Тобто рух є способом існування матерії.
		
		Кожен рівень організації матерії має свої форми руху. Але так само як~всі рівні організації матерії та~їх елементи перебувають у~взаємодії, усі форми руху нерозривно пов'\-яз\-а\-ні між собою. У середині XIX ст. Ф. Енгельсом була запропонована така класифікація форм руху матерії:
		\begin{enumerate}
			\item Механічний рух~— переміщення макротіл у~просторі.
			\item Фізичний рух~— рух молекул.
			\item Хімічний рух~— взаємодія атомів.
			\item Біологічний рух~— життя~— рух білкових тіл.
			\item Соціальний рух~— розвиток суспільства і~людини.
		\end{enumerate}
		
		За своїм змістом поняття «рух» більш загальне, ніж поняття «розвиток». Розвиток~— це такий рух (взаємодія), під час якого відбувається не просто зміна вже існуючих властивостей, стану якоїсь системи, а~виникають нові властивості, нова якість (сутність чогось), яких раніше не було.
		
		Традиційно виділяють два типи розвитку:
		\begin{enumerate}
			\item Розвиток у~межах однієї форми руху матерії (наприклад, поява нових ознак у~тварин, рослин і~т. д.).
			\item Розвиток, за якого відбувається перехід від однієї форми руху матерії, від одного рівня її структурної організації до іншого, вищого (наприклад, виникнення органічної природи з~неорганічної). Цей тип розвитку називають прогресивним.
		\end{enumerate}
		
	\section{Простір і~час як~форми існування матерії}
		У діалектично-матеріалістичній філософії категорії «простір» і~«час» постають як~форми існування матерії, її не\-від'\-єм\-ні характеристики, атрибути. Простір виражає взаємне розташування предметів один відносно одного, їх конфігурацію; час позначає тривалість існування матеріальних тіл, явищ, процесів.
		
		Матеріальні об'\-єк\-ти перебувають у~постійному русі, вони змінюють своє положення відносно інших об'\-єк\-тів, набувають іншої конфігурації, мають певну тривалість свого існування. Окремі об'\-єк\-ти мають свої часові межі: вони виникають у~певний момент часу, а~через деякий його проміжок перетворюються на~інші об'\-єк\-ти. Природничі науки підтверджують незнищенність матерії, руху, простору й часу та~їх нерозривність між собою. Зокрема, теорія відносності доводить, що~в~реальному фізичному світі просторові та~часові інтервали змінюються залежно від системи відліку. Якщо тіло рухається не в~умовах земного тяжіння, а~його швидкість наближається до швидкості світла, то це призводить до значних змін просторових і~часових інтервалів: просторові інтервали скорочуються, а~часові~— подовжуються.

	\section{Поняття і~сутність антропосоціогенезу. Філософія й наука про походження людини.}
		Антропосоціогенез~— це єдиний процес становлення людини і~суспільства.
		
		Філософи зосереджують увагу на~рушійних силах антропогенезу і~підходять до вирішення цієї проблеми або з~позицій ідеалізму, або з~позицій матеріалізму.
		
		З матеріалістичних позицій проблема антропосоціогенезу глибоко розроблена Ф. Енгельсом. Він висунув положення про визначальну роль праці в~процесу антропогенезу. Головна ознака присутності людини в~світі~— знаряддя праці. Людина є перш за все творець знарядь праці. Вона перейшла від пристосування до природи до пристосування природи до себе. Праця потребувала взаємодії для свого розвитку. Тобто виготовлення знарядь праці робило перехід від біологічного до надбіологічного об'\-єд\-нан\-ня, від стада до суспільства, необхідним, єдино можливим шляхом розвитку. Могутнім фактором став розвиток мови як~засобу спілкування, об'\-єд\-нан\-ня зусиль в~праці, згуртування спільноти. В ході антропосоціогенезу відбувся перехід існування від біологічно визначеного до морального. Це створило умови для розвитку праці і, в~свою чергу, було породжено потребами праці як~способом буття людей. Праця була як~рушієм антропосоціогенезу, так і~його результатом. Вона розвивала інтелект, здібності людей, сприяла формуванню суспільства, що, в~свою чергу, ускладнювало працю і~збільшувало різноманіття її видів.
		
		Антрополог і~філософ Тейяр де Шарден розглядає антропосоціогенез як~складову космогенезу в~цілому, рушійною силою якого є взаємодія механічної та~психічної енергій. Накопичення психічного, примноження станів його вияву і~є фактором появи людини. Людина усвідомлює свою свідомість, і~це робить її перебування у~світі новою формою буття, підносить процес еволюції до нового рівня. Погляди Тейяра де Шардена, Чижевського і~Вернадського лягли в~основу концепції космічного походження людини: поява людини, виникнення соціальної форми матерії~— це не випадковість, а~загальна тенденція космічної еволюції; людина є природньо-космічною істотою, яка містить в~собі нескінченність космосу.
		
		Серед наукових концепцій походження людини найбільш поширеною і~впливовою є еволюційна, у~джерел якої стоять Ламарк і~Дарвін. Під впливом мінливості, спадковості, природного відбору здійснюється еволюція в~органічному світі, поява нових видів. Де Фріс додає і~такий фактор як~великі одиничні мутації. Вищим щаблем еволюційного розвитку є людина. Еволюційна концепція~— не тільки основа антропології, а~й раціональна передумова філософського вчення про людину, тому що~поєднує і~відокремлює світ природи і~світ людської культури, історії.

	\section{Постановка проблеми людини в~історії філософії}
		Філософи Стародавньої Греції розглядали людину і~світ як~єдине ціле. Людина~— мікрокосм, Космос в~мініатюрі. Софісти, Сократ започаткували антропоцентризм, поставивши людину, її буття у~центр філософії, оголосивши людину мірою всіх речей. Аристотель розглядає людину як~соціальну, політичну істоту. Соціальність, розумність, мова є тими основними характеристиками, що~виокремлюють людину з-поміж живих істот. Античні філософи орієнтують людину на~пізнання світу, себе, суспільства.
		
		В Середньовіччі антропоцентризм поступився місцем теоцентризму. Людина~— творіння Боже, протирічива єдність душі і~тіла. Земне життя є часом випробувань, вдосконалення, і~від індивіда залежить перспектива вічності~— рай чи пекло.
		
		Філософи і~митці доби Відродження повертають у~філософію і~суспільну свідомість антропоцентризм. Людина не просто творіння Боже, а~особлива істота, що~отримала від Бога розум і~дар творчості. Вона займає виключне місце в~світі, вільна в~виборі своєї долі, життєвого шляху. Людина може піднятись до небесних висот або впасти до скотського стану. Людина вибирає і~несе земну і~потойбічну відповідальність за здійснений вибір. Гуманістам доби Відродження притаманна віра в~безмежні можливості людини, її само реалізацію в~повноті сил і~здібностей в~земному житті.
		
		Діячі релігійного руху Реформації, мислителі Північного Відродження розглядають активність, самостійність людини як~знак богообранності. Призначенням людини є праця. Лише в~праці вона знаходить свою справжню сутність, самореалізується, вдосконалюється.
		
		Філософи Нового часу зосереджують увагу на~розумових, пізнавальних здібностях людини. Лише спираючись на~розум людина здатна підкорити, змінити світ, створити розумне і~справедливе суспільство. Філософи утверджують природній потяг людини до добра, щастя, гармонії. Але розвиток механіки, природознавства спричиняють появу в~філософії механістичного погляду на~людину як~своєрідну машину. Людина стала розглядатись як~продукт природи, цілком визначений її законами.
		
		У німецькій  класичній філософії проблема людини перебувала у~центрі філософський пошуків. Зокрема І. Кант вважав питання «що таке людина?» головним питанням філософії, а~саму людину~— «найголовнішим предметом у~світі». Він дотримувався позиції антропологічного дуалізму, але його дуалізм~— це не дуалізм душі і~тіла, як~у~Декарта, а~морально-природний дуалізм. За Кантом людина з~одного боку належить природній необхідності, а~з~іншого~— моральній свободі та~абсолютним цінностям. Відмітною рисою людини є самосвідомість, яка й вирізняє її з-поміж інших живих істот. Г. Гегель у~своїй антропологічній концепції зосередився на~вираженні становища людини як~суб'\-єк\-та духовної діяльності і~носія загально значимого духу і~розуму. Особа, зазначив філософ, починається тільки з~усвідомлення себе як~істоти «нескінченної, загальної та~вільної».
		
		Марксизм наголошує на~соціальній сутності людини. Він розглядає людину як~істоту матеріальну, єдність біологічного і~соціального. «В своїй дійсності вона є сукупність всіх суспільних відносин» і~в той же час їх творець. Людина реалізує себе в~процесі  творчої праці, змінюючи світ, себе і~суспільство.
		
		Філософія ХХ ст. зосередила увагу на~людині як~неповторній індивідуальності. Навіть основним питанням філософії була проголошена проблема людини і~сенсу її життя. Намагаючись наблизитись до окремо взятої живої людини, з~відкриттям глибинної людської ірраціональності, філософи зосереджуються на~її внутрішньому світі, духовності.  Екзистенціалізм бачить в~людині істоту вільну і~трагічну. Вільну, тому що~вона сама обирає і~реалізує проект свого буття. Трагічну, бо людина живе з~гострим відчуттям власної смертності, в~тривозі за непередбачені результати вибору. Екзистенціалісти поставили проблеми вибору і~відповідальності, самотності людини в~світі. 
\end{document}