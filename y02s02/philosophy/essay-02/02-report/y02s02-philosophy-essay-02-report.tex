\documentclass[a4paper,oneside,DIV = 12, 12pt, headings = normal]{scrartcl}

%%% Support for color
\usepackage{xcolor}
\definecolor{lightblue}{HTML}{03A9F4}
\definecolor{red}{HTML}{F44336}
%%%

%%% Font selection
\usepackage{fontspec}

\setromanfont{PT Serif}[
	SmallCapsFeatures = {LetterSpace = 5},
]

% \setsansfont{Source Sans Pro}[
% ]

% \setmonofont{Source Code Pro}[
% ]

% \usepackage{amsmath,unicode-math}
% \setmathfont{STIX Two Math}

%%%

%%% Font settings for different KOMA Script elements
\setkomafont{pagenumber}{\rmfamily}
\setkomafont{disposition}{\rmfamily\bfseries}
% \addtokomafont{section}{\rmfamily\xbseries}
%%%

%%% Typographic enhancements
\usepackage{microtype}
%%%

%%% Language-specific settings
\usepackage{polyglossia}
\setmainlanguage{ukrainian}
%%%

%%% List settings
\usepackage{enumitem}
\setlist[enumerate]{
	leftmargin = *,
}
%%%

%%% Captions
\usepackage{caption}
\usepackage{subcaption}

\DeclareCaptionLabelFormat{closing}{#2)}
\captionsetup[subtable]{labelformat = closing}
\captionsetup[subfigure]{labelformat = closing, position = auto}
%%%

\usepackage[
	style    = gost-numeric,
	language = auto,
	autolang = other,
	sorting  = none,
]{biblatex}
\addbibresource{y02s02-philosophy-bibliography.bib}

%%% Links and hyperreferences
\usepackage{hyperref}
\hypersetup{
	colorlinks      = false,
	linkbordercolor = red,
	urlbordercolor  = lightblue,
	pdfborderstyle  = {/S/U/W 1.5},
}

% Suppress section number
\makeatletter
\renewcommand\@seccntformat[1]{}
\makeatother

%%%


%%% All caps
\newcommand{\allcaps}[1]{{\addfontfeatures{LetterSpace = 3}#1}}
%%%

% \linespread{1.07143}

\begin{document}
	\begin{titlepage}
	\centering
		Міністерство освіти і науки України\\
		Національний авіаційний університет\\
		Навчально-науковий інститут комп'ютерних інформаційних технологій\\
		Кафедра комп'ютеризованих систем управління

		\vspace*{\fill}

		Творче завдання №2\\
		з дисципліни «Філософія»\\
		% Варіант №4

		\vspace*{\fill}
		
		\begin{flushright}
			Виконав:\\
			студент ННІКІТ СП-225\\
			Клокун В.\,Д.\\
			Перевірила:\\
			Матюхіна О.\,А.
		\end{flushright}

		Київ 2018
    \end{titlepage}
	
	\section{Питання}
		Чому соціальний простір і соціальний час є формами буття людей у культурі?
	
	\section{Відповідь}
		Для відповіді на питання наведемо визначення необхідних термінів та встановимо зв'язки між ними. За Дротянко \emph{форми буття} виділяються самими людьми і є проявами людського буття~— процесу становлення й самоствердження людини, реалізації її сутнісних сил~\cite[244--245]{drotyanko-textbook}. За Причепієм \emph{культура}~— це увесь, за винятком природи, мовно та символічно відтворений світ, що охоплює різноманітність видів, засобів і результатів активної творчої діяльності людини, спрямованої на освоєння, пізнання і зміну навколишньої реальності та самої себе~\cite[373]{prichepiy-textbook}. Результати творчої діяльності людини є проявами її самоствердження та сутнісних сил. Вони також спрямовані на освоєння, пізнання та зміну навколишньої реальності, а тому є формою буття людей у культурі.
		
		\emph{Соціальний простір} визначається як простір, що був освоєний людством у процесі його існування~\cite[251]{drotyanko-textbook}. Під час цього процесу люди неминуче створюють знаряддя праці, різноманітні предмети побуту та пам'ятки, що знаменують людське існування, з метою дослідження та адаптації наявного фізичного простору до своїх потреб. Ці предмети є творіннями людей — результатами їх творчої діяльності, а отже й формою буття людей у культурі. Оскільки продукти та прояви соціального простору є складовими соціального простору як єдиного цілого, то й сам соціальний простір є формою буття людей у культурі.
		
		\emph{Соціальний час}~— це тривалість існування людства, його історія~\cite[252]{drotyanko-textbook}. З~плином часу (розгортанням історичного процесу) прояви людської діяльності змінюються: науковий прогрес дозволяє покращувати знаряддя праці, збільшуючи їх ефективність; з появою нових досягнень та ідей переосмислюються предмети побуту для підвищення повсякденного комфорту людей; зводяться монументи для відзначення здобутих досягнень. Тобто кожен момент або епоху існування людини можна розглядати як окремий прояв людського буття~— соціальний простір, тому весь соціальний час є множиною соціальних просторів та~їх~зв'язків. Оскільки соціальний простір є формою буття людей у культурі, як було доведено вище, то й їх множина як більш глобальна сутність також є формою буття людей у культурі.
	
	\printbibliography
	
\end{document}
