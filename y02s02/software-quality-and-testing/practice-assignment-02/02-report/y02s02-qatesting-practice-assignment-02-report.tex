\documentclass[a4paper,oneside,BCOR=1cm,DIV=12,12pt,headings=normal]{scrartcl}

%%% Length calculations
\usepackage{calc}
%%%

%%% Support for color
\usepackage{xcolor}
\definecolor{lightblue}{HTML}{03A9F4}
\definecolor{red}{HTML}{F44336}
%%%

%%% Links and hyperreferences
\usepackage{hyperref}
\hypersetup{
	bookmarksnumbered = true,
	colorlinks        = false,
	linkbordercolor   = red,
	urlbordercolor    = lightblue,
	pdfborderstyle    = {/S/U/W 1.5},
}
%%%

%%% Graphics inclusion
\usepackage{graphicx}
%%%

%%% Support for hyphenated rags
\usepackage{ragged2e}
%%%

%%% Font selection
\usepackage{fontspec}

\setromanfont{STIX Two Text}[
]

\setsansfont{Source Sans Pro}[
]

\setmonofont{Source Code Pro}[
	% Scale = 1.05,
]

% Used for suppressing polyglossia errors regarding missing cyrillic script
% \newfontfamily\cyrillicfonttt[
	% Script = Cyrillic,
	% Scale  = MatchUppercase,
% ]{Inconsolata}

\usepackage{unicode-math}
\setmathfont{STIX Two Math}

%%%

%%% Font settings for different KOMA Script elements
\setkomafont{pagenumber}{\rmfamily}
\setkomafont{disposition}{\rmfamily\bfseries}
%%%

%%% Typographic enhancements
\usepackage{microtype}
%%%

%%% Language-specific settings
\usepackage{polyglossia}
\setmainlanguage{ukrainian}
%%%

%%% List settings
\usepackage{enumitem}
\setlist[enumerate]{
	leftmargin = *,
	label*     = {\arabic*.},
}
%%%

%%% Captions
\usepackage{caption}
%%%

%%% Tables
\usepackage{booktabs}
\usepackage{longtable}

\usepackage{multirow}

\usepackage{cellspace}
% \setlength\cellspacetoplimit{0\baselineskip}
% \setlength\cellspacebottomlimit{0.5\baselineskip}
% \renewcommand{\arraystretch}{1.5}
\addparagraphcolumntypes{v}

\usepackage{array}
\newcolumntype{v}[1]{>{\RaggedRight\arraybackslash\hspace{0pt}}p{#1}}
\newcolumntype{b}[1]{>{\centering\arraybackslash\hspace{0pt}}p{#1}}
\newcolumntype{n}[1]{>{\RaggedLeft\arraybackslash\hspace{0pt}}p{#1}}

% \usepackage{kbordermatrix} % labeling array indices
%%%

%%%

%%% Count figures within sections
\usepackage{chngcntr}
\counterwithin{figure}{section}
%%%

%%% Custom commands
% Print filenames
\newcommand{\filename}[1]{\texttt{#1}}

% Print command names
\newcommand{\progname}[1]{\texttt{#1}}

% Print function names
\newcommand{\funcname}[1]{\texttt{#1}}

% Typeset all caps
\newcommand{\allcaps}[1]{{\addfontfeatures{LetterSpace = 3}#1}}

%%%

%%% Lengths
\setlength{\emergencystretch}{1em}

\newlength{\modulewidth}
\setlength{\modulewidth}{\textwidth / 6}
%%%


\begin{document}
	\begin{titlepage}
	\centering
		Міністерство освіти і науки України\\
		Національний авіаційний університет\\
		Навчально-науковий інститут комп'ютерних інформаційних технологій\\
		Кафедра комп'ютеризованих систем управління

		\vspace*{\fill}

		Завдання №1{.}2\\
		для проходження практики\\
		з дисципліни «Якість програмного забезпечення та тестування»\\
		на тему «Тестування технічного завдання до програмного забезпечення»

		\vspace*{\fill}
		
		\begin{flushright}
			Виконав:\\
			студент \allcaps{ННІКІТ} \allcaps{СП}-225\\
			Клокун В.\,Д.\\
			Перевірив:\\
			Сябрук І.\,М.
		\end{flushright}

		Київ 2018
    \end{titlepage}

	\tableofcontents

	\section{Мета роботи}
		Оволодіти навичками написання та тестування технічного завдання до~програмного забезпечення.

	\section{Хід роботи}
		Процес виконання роботи передбачає складання технічного завдання до~веб-сайту для~дистанційного навчання з~використанням моделі «клієнт~— сервер». В~результаті виконання роботи було створене необхідне технічне завдання~(додаток~\ref{sec:prd}).

	\section{Висновок}
		Під час виконання даного завдання ми оволоділи навичками написання та тестування технічного завдання до~програмного забезпечення.
		\newpage

	\appendix
	\section{Технічне завдання}
	\label{sec:prd}
		\subsection{Загальні відомості}
			Дане технічне завдання описує вимоги та деталі проекту розробки веб-сайту для дистанційного навчання~«RemEDU»~(табл.~\ref{tab:project-summary}).

			\begin{table}[!htbp]
			\caption{Загальні відомості про проект}
			\label{tab:project-summary}
				\begin{tabular}{
				S{v{2\modulewidth - 2\tabcolsep}}
				S{v{4\modulewidth - 2\tabcolsep}}
				}
					\toprule
						Характеристика & Пояснення\\
					\midrule
						Повне найменування системи & Веб-сайт для дистанційного навчання «RemEDU»\\
						Коротке найменування системи & Веб-сайт, веб-додаток, продукт, «RemEDU»\\
						Підстави для~проведення робіт & Угода~№1\\
						Замовник & \allcaps{ТОВ}~«RemEDU»\\
						Розробник & \allcaps{ТОВ}~«Розробник»\\
						Планова дата початку~роботи & 01.01.2019\\
						Планова дата кінця~роботи   & 01.06.2019\\
						Джерела та~порядок фінансування & Вказані в~угоді~№1\\
						Порядок оформлення та~пред'\-яв\-лен\-ня результатів роботи замовнику & Результати роботи зі~створення Продукту передаються Замовнику поетапно відповідно до~календарного плану Проекту\\
					\bottomrule
				\end{tabular}
			\end{table}

		\subsection{Мета створення системи}
			Продукт створюється для надання користувачам можливості зручного дистанційного навчання, потребуючи доступ до мережі Інтернет та сумісний веб-браузер.

		\subsection{Вимоги до програмного продукту}
			\subsubsection{Вимоги до продукту в цілому}
				Веб-сайт повинен надавати користувачу лекційні матеріали у текстовій та~мультимедійній формі, а~також надавати можливість контролю знань, зберігати результати контролю знань, навчальний прогрес та статистику. Використання технології «клієнт~— сервер» розуміється як~доставка контенту та~обмін даними між~веб-сайтом (сервером) та~користувачем (клієнтом).

				Технічна частина веб-сайту повинна складатись з~бекенду та~фронтенду. Бекенд повинен бути написаний на~мові програмування~Python з~використанням фреймворку~Flask. Для управління базами даних повинна бути використана система~MariaDB. Фронтенд повинен бути виконаний з використанням технологій \allcaps{CSS}~3, \allcaps{HTML}~5, ECMAScript~2017 та~фреймворку~Material Web Components.

				Продукт повинен бути розроблений для подальшої легкості підтримки: бути модульним, легко редагованим, розширюваним та оновлюваним.

			\subsubsection{Вимоги до структури веб-сайту}
				Контент веб-сайту повинен бути логічно структурованим та розміщеним у відповідних розділах~(табл.~\ref{tab:site-structure}).

				\begin{table}[!htbp]
					\caption{Опис розділів веб-сайту}
					\label{tab:site-structure}
					\begin{tabular}{
						S{v{2\modulewidth - 2\tabcolsep}}
						S{v{4\modulewidth - 2\tabcolsep}}
					}
						\toprule
							Розділ & Опис\\
						\midrule
							Початкова сторінка & Містить логотип та назву компанії, короткий опис проекту, базову інформацію про користування веб-сайтом, зміст його розділів та посилання на~них\\
							Вхід & Запрошує користувача увійти або зареєструватись у~системі\\
							Сторінка користувача & Містить дані про користувача, курси, які він проходить, та його прогрес\\
							Курси & Містить посилання на~курси, запропоновані на~веб-сайті, відсортовані за категоріями\\
							Про нас & Містить інформацію про компанію та проект\\
							Часті запитання & Містить часті запитання та відповіді на них\\
						\bottomrule
					\end{tabular}
				\end{table}

			\subsubsection{Вимоги до функцій, що виконуються продуктом}
				Розроблений продукт повинен надавати можливість дистанційного навчання. Під дистанційним навчанням розуміється можливість перегляду лекційних матеріалів у текстовій та мультимедійній формах: аудіо- та відеозаписах, графічних ілюстраціях та додаткових прикріплюваних матеріалах. Дані повинні зберігатись у відповідних відкритих форматах~(табл.~\ref{tab:media-formats}).

				\begin{table}[!htbp]
					\caption{Формати для зберігання матеріалів лекції}
					\label{tab:media-formats}
					\begin{tabular}{
							S{v{2\modulewidth - 2\tabcolsep}}
							S{v{2\modulewidth - 2\tabcolsep}}
							S{v{2\modulewidth - 2\tabcolsep}}
					}
						\toprule
							Тип даних & Складова & Допустимі формати\\
						\midrule
							Аудіозапис & Аудіокодек для стиснення без втрат  & \allcaps{FLAC}\\
							           & Аудіокодек для стиснення з втратами & Opus\\
							Відеозапис     & Аудіокодек для стиснення без втрат & \allcaps{FLAC}\\
							               & Аудіокодек для стиснення з втратами & Opus\\
												     & Відеокодек & \allcaps{AV1}, \allcaps{VP9}\\
							               & Контейнер  & Matroska\\
												     & Субтитри   & \allcaps{ASS}\\
							Зображення     &            & \allcaps{PNG}, WebP, \allcaps{JPG}\\
							Текст          & Кодування  & \allcaps{UTF-8}\\
							Інші матеріали &            & Початковий формат\\
						\bottomrule
					\end{tabular}
				\end{table}

				Повинна існувати можливість реєстрації за допомогою e-mail та пароля. Після реєстрації користувачу стає доступним його персональна сторінка, яка дозволяє переглядати курси, які він проходить на даний момент, вже пройшов та рекомендовані курси, а~також результати проходження курсів та загальну статистику: дату реєстрації, кількість часу, проведеного за~навчанням та~за~проходженням тестів.

				Для користувачів з підвищеними привілеями доступу повинна бути розроблена графічна панель керування, яка~дозволить оновлювати зміст сайту, курсів, лекцій, а~також профілі користувачів.

			\subsubsection{Вимоги до інтерфейсу користувача}
				Інтерфейс користувача повинен бути виконаний з дотриманням вимог візуальної мови~Material Design, підтримувати денну та нічну тему оформлення, які можна обирати вручну або встановити автоматичний вибір теми за часом в~обраному або автоматично визначеному часовому поясі. Веб-сайт повинен бути локалізований англійською, російською та українською мовами з можливістю вибору поточної мови інтерфейсу.

			\subsubsection{Вимоги до видів забезпечення}
				Розроблена система повинна підтримувати плавну одночасну роботу з~1000~користувачів. Для забезпечення швидкого завантаження веб-сайту для користувачів по всьому світу повинне бути передбачене використання мереж доставки контенту~(\allcaps{CDN}) на кшталт CloudFlare та~CloudFront. 
				Веб-сайт розміщується на~основному сервері та~кешується системами~\allcaps{CDN}. Основний сервер повинен працювати під~управлінням операційної системи~CentOS, надавати веб-сайт за~допомогою веб-сервера~nginx, системи управління базами даних~MariaDB та~мови програмування Python, також повинні бути встановлені всі необхідні компоненти для~коректної роботи сервісів веб-сайту.

		\subsection{Вимоги до прийому робіт}
			Для прийому результатів роботи визначається чіткий порядок~(табл.~\ref{tab:results-acceptance}).

			\begin{table}[!htbp]
				\caption{Процес прийому результатів робіт за стадіями}
				\label{tab:results-acceptance}
				\begin{tabular}{
						S{v{1\modulewidth - 2\tabcolsep}}
						S{v{2\modulewidth - 2\tabcolsep}}
						S{v{3\modulewidth - 2\tabcolsep}}
				}
					\toprule
						Стадія випробувань & Випробувальники & Порядок домовленості\\
					\midrule
						Попередні & Експертна група & Проведення попередніх випробувань, фіксація виявлених недоліків у~Протоколі випробувань, виправлення недоліків, виправлення виявлених недоліків, прийняття рішення про~передачу систему в~тестову експлуатацію, складання та~підписання Акту прийому систему в~тестову експлуатацію.\\
						Тестові & Група тестування & Проведення тестової експлуатації, фіксація виявлених недоліків у~Протоколі випробувань, виправлення виявлених недоліків, прийняття рішення про~готовність системи до~прийомних випробувань, складання та~підписання Акту про~завершення тестової експлуатації системи.\\
						Прийомні & Комісія з~прийому & Проведення випробувань при~прийомі, фіксація виявлених недоліків у~Протоколі випробувань, виправлення виявлених недоліків, прийняття рішень про~передачу модуля в~експлуатацію, складання та~підписання Акту про~завершення випробувань при~прийомі та~передачі системи в~експлуатацію, оформлення Акту про~завершення робіт.\\
					\bottomrule
				\end{tabular}
			\end{table}

		\subsection{Вимоги до документації}
			У процесі розробки необхідно вести документацію. Документація повинна бути надана у~вигляді \allcaps{PDF}-файлів, створених у~системі комп'\-ютер\-ної верстки~\LaTeX, файлів початкового коду до~них і~роздрукованої документації. Крім того необхідно надати електронну документацію до~розробленої системи в~автономному вигляді у~системах на~кшталт Doxygen або~Sphinx. За етапами розробки документація повинна містити:
			\begin{enumerate}
				\item Проектування.
					\begin{enumerate}
						\item Відомість ескізного проекту.
						\item Пояснювальна записка до ескізного проекту.
						\item Відомість технічного проекту.
						\item Пояснювальна записка до технічного проекту.
						\item Схема функціональної структури.
					\end{enumerate}
				\item Документування.
					\begin{enumerate}
						\item Загальний опис системи.
						\item Технологічна інструкція.
						\item Інструкція користувача.
						\item Програма та методика тестування.
						\item Специфікація.
						\item Текст програм.
					\end{enumerate}
				\item Введення в дію.
					\begin{enumerate}
						\item Акт прийому в експлуатацію.
						\item Протокол тестування.
						\item Акт завершення робіт.
					\end{enumerate}
			\end{enumerate}

		\subsection{Джерела розробки}
			Дане технічне завдання розроблене на основі договору~№1 між~\allcaps{ТОВ}~«RemEDU» та~\allcaps{ТОВ}~«Розробник».

\end{document}

