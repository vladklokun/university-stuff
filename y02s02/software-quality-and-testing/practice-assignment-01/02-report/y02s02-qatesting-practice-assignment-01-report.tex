\documentclass[a4paper,oneside,BCOR=1cm,DIV=12,12pt,headings=normal]{scrartcl}

%%% Support for color
\usepackage{xcolor}
\definecolor{lightblue}{HTML}{03A9F4}
\definecolor{red}{HTML}{F44336}
%%%

%%% Links and hyperreferences
\usepackage{hyperref}
\hypersetup{
	colorlinks      = false,
	linkbordercolor = red,
	urlbordercolor  = lightblue,
	pdfborderstyle  = {/S/U/W 1.5},
}
%%%

%%% Graphics inclusion
\usepackage{graphicx}
%%%

%%% Font selection
\usepackage{fontspec}

\setromanfont{STIX Two Text}[
]

\setsansfont{Source Sans Pro}[
]

\setmonofont{Source Code Pro}[
	% Scale = 1.05,
]

% Used for suppressing polyglossia errors regarding missing cyrillic script
% \newfontfamily\cyrillicfonttt[
	% Script = Cyrillic,
	% Scale  = MatchUppercase,
% ]{Inconsolata}

\usepackage{unicode-math}
\setmathfont{STIX Two Math}

%%%

%%% Font settings for different KOMA Script elements
\setkomafont{pagenumber}{\rmfamily}
\setkomafont{disposition}{\rmfamily\bfseries}
%%%

%%% Typographic enhancements
\usepackage{microtype}
%%%

%%% Language-specific settings
\usepackage{polyglossia}
\setmainlanguage{ukrainian}
%%%

%%% List settings
\usepackage{enumitem}
\setlist[enumerate]{
	leftmargin = *,
}
%%%

%%% Captions
\usepackage{caption}
%%%

%%% Code listings
\usepackage{minted}

%%%

%%% Framing code listings
\usepackage{tcolorbox}
\tcbuselibrary{breakable}
\tcbuselibrary{minted}
\tcbuselibrary{skins}

\newtcbinputlisting[auto counter, list inside, number within = section]{\inputcpp}[4][]{%
	minted language = cpp,
	minted style    = bw,
	minted options  = {
		linenos,
		tabsize = 4,
		breaklines,
		breakbytoken,
		breaksymbol = {},
		fontsize = \small,
	},
	%
	enhanced,
	sharp corners,
	title            = {Лістинг \thetcbcounter: #3},
	list entry       = {\protect\numberline{\thetcbcounter}#3},
	bottom           = 0pt,
	top              = 0pt,
	left             = 0pt,
	right            = 0pt,
	middle           = 0pt,
	colframe         = black!0,
	colback          = black!0,
	colbacktitle     = black!0,
	coltitle         = black,
	colframe         = black,
	titlerule        = 0.5pt,
	leftrule         = 0pt,
	rightrule        = 0pt,
	toptitle         = 2pt,
	bottomtitle      = 1pt,
	before skip      = \intextsep,
	after  skip      = \intextsep,
	%
	listing file={#2},
	listing only,
	breakable,
	%
	label = {#4},
	%
	#1
}

% Customize minted line numbers
\renewcommand{\theFancyVerbLine}{\ttfamily\scriptsize\arabic{FancyVerbLine}}

%%%

%%% Count figures within sections
\usepackage{chngcntr}
\counterwithin{figure}{section}
%%%

%%% Custom commands
% Print filenames
\newcommand{\filename}[1]{\texttt{#1}}

% Print command names
\newcommand{\progname}[1]{\texttt{#1}}

% Print function names
\newcommand{\funcname}[1]{\texttt{#1}}

% Typeset all caps
\newcommand{\allcaps}[1]{{\addfontfeatures{LetterSpace = 3}#1}}

%%%

\setlength{\emergencystretch}{1em}


\begin{document}
	\begin{titlepage}
	\centering
		Міністерство освіти і науки України\\
		Національний авіаційний університет\\
		Навчально-науковий інститут комп'ютерних інформаційних технологій\\
		Кафедра комп'ютеризованих систем управління

		\vspace*{\fill}

		Завдання №1{.}1\\
		для проходження практики\\
		з дисципліни «Якість програмного забезпечення та тестування»\\

		\vspace*{\fill}
		
		\begin{flushright}
			Виконав:\\
			студент \allcaps{ННІКІТ} \allcaps{СП}-225\\
			Клокун В.\,Д.\\
			Перевірив:\\
			Сябрук І.\,М.
		\end{flushright}

		Київ 2018
    \end{titlepage}

	\section{Мета роботи}
		Ознайомитись з видами тестів, оволодіти навичками вибору тестів.

	\section{Хід роботи}
		\subsection{Визначення типу трикутника}
			Створюємо власну версію програми для~перевірки типу трикутника~(ліст.~\ref{lst:01-trianglecheck}). При складанні враховуємо всі тести, що наведені у~тексті завдання. Розроблена програма успішно працює з~усіма наданим тестовими наборами.

			\inputcpp{../01-solution/y02s02-qatesting-practice-assignment-01-p01.cpp}{Програма для перевірки типу трикутника}{lst:01-trianglecheck}
			
		\subsection{Виведення кожної цифри заданого числа}
			Створюємо власну версію програми для почергового виведення кожної цифри заданого числа~(ліст.~\ref{lst:02-numbydigit}). При складанні враховуємо всі тести, що наведені у~тексті завдання. Розроблена програма була успішно протестована.

			\inputcpp{../01-solution/y02s02-qatesting-practice-assignment-01-p02.cpp}{Програма для почергового виведення кожної цифри заданого числа}{lst:02-numbydigit}

	\section{Висновок}
		Під час виконання даного завдання ми ознайомились з видами тестів та оволоділи навичками вибору тестів.

\end{document}

