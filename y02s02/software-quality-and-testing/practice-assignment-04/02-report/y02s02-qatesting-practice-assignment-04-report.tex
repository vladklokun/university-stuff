\documentclass[a4paper,oneside,BCOR=1cm,DIV=12,12pt,headings=normal]{scrartcl}

%%% Length calculations
\usepackage{calc}
%%%

%%% Support for color
\usepackage{xcolor}
\definecolor{lightblue}{HTML}{03A9F4}
\definecolor{red}{HTML}{F44336}
%%%

%%% Links and hyperreferences
\usepackage{hyperref}
\hypersetup{
	bookmarksnumbered = true,
	colorlinks        = false,
	linkbordercolor   = red,
	urlbordercolor    = lightblue,
	pdfborderstyle    = {/S/U/W 1.5},
}
%%%

%%% Graphics inclusion
\usepackage{graphicx}
%%%

%%% Support for hyphenated rags
\usepackage{ragged2e}
%%%

%%% Font selection
\usepackage{fontspec}

\setromanfont{STIX Two Text}[
	SmallCapsFeatures = {LetterSpace = 5},
]

\setsansfont{Source Sans Pro}[
]

\setmonofont{Source Code Pro}[
	% Scale = 1.05,
]

\usepackage{unicode-math}
\setmathfont{STIX Two Math}

%%%

%%% Font settings for different KOMA Script elements
\setkomafont{pagenumber}{\rmfamily}
\setkomafont{disposition}{\rmfamily\bfseries}
%%%

%%% Typographic enhancements
\usepackage{microtype}
%%%

%%% Language-specific settings
\usepackage{polyglossia}
\setmainlanguage{ukrainian}
%%%

%%% List settings
\usepackage{enumitem}
\setlist[enumerate]{
	leftmargin = *,
	label*     = {\arabic*.},
}

\setlist[description]{
	font = \rmfamily\upshape\bfseries,
}
%%%

%%% Captions
\usepackage{caption}
%%%

%%% Tables
\usepackage{booktabs}
\usepackage{longtable}

\usepackage{multirow}

\usepackage{cellspace}
% \setlength\cellspacetoplimit{0\baselineskip}
% \setlength\cellspacebottomlimit{0.5\baselineskip}
% \renewcommand{\arraystretch}{1.5}
\addparagraphcolumntypes{v}

\usepackage{array}
\newcolumntype{v}[1]{>{\RaggedRight\arraybackslash\hspace{0pt}}p{#1}}
\newcolumntype{b}[1]{>{\centering\arraybackslash\hspace{0pt}}p{#1}}
\newcolumntype{n}[1]{>{\RaggedLeft\arraybackslash\hspace{0pt}}p{#1}}

% \usepackage{kbordermatrix} % labeling array indices
%%%

%%%

%%% Count figures within sections
\usepackage{chngcntr}
\counterwithin{figure}{section}
%%%

%%% Custom commands
% Print filenames
\newcommand{\filename}[1]{\texttt{#1}}

% Print command names
\newcommand{\progname}[1]{\texttt{#1}}

% Print function names
\newcommand{\funcname}[1]{\texttt{#1}}

% Typeset all caps
\newcommand{\allcaps}[1]{{\addfontfeatures{LetterSpace = 3}#1}}

%%%

%%% Lengths
\setlength{\emergencystretch}{1em}

\newlength{\modulewidth}
\setlength{\modulewidth}{\textwidth / 12}

% Save baselineskip
\newlength{\oldbaselineskip}
%%%


\begin{document}
	\begin{titlepage}
	\centering
		Міністерство освіти і науки України\\
		Національний авіаційний університет\\
		Навчально-науковий інститут комп'ютерних інформаційних технологій\\
		Кафедра комп'ютеризованих систем управління

		\vspace*{\fill}

		Завдання №1{.}4\\
		для проходження практики\\
		з дисципліни «Якість програмного забезпечення та~тестування»\\
		на тему «Створення плану тестування програми»

		\vspace*{\fill}
		
		%%% This portion of code is used to make the infobox flushleft while touching the right margin
		% Set length to the widest string in the infobox
		% \newlength{\infoboxwidth}
		% \settowidth{\infoboxwidth}{студент \allcaps{ННІКІТ} \allcaps{СП}-225}

		%\hspace{\fill}
		% \begin{minipage}{\infoboxwidth}
		% 	Виконав:\\
		% 	студент \allcaps{ННІКІТ} \allcaps{СП}-225\\
		% 	Клокун В.\,Д.\\
		% 	Перевірив:\\
		% 	Сябрук І.\,М.
		% \end{minipage}
		%\vspace*{\baselineskip}
		%%%

		\begin{flushright}
			Виконав:\\
			студент \allcaps{ННІКІТ} \allcaps{СП}-225\\
			Клокун В.\,Д.\\
			Перевірив:\\
			Сябрук І.\,М.
		\end{flushright}

		Київ 2018
	\end{titlepage}

	\tableofcontents

	\section{Мета роботи}
		Оволодіти навичками створення плану тестування програми.

	\section{Хід роботи}
		Процес виконання роботи передбачає створення плану тестування для програмного продукту~TopSyle~— редактору початкового коду на мовах~\allcaps{CSS}, \allcaps{HTML}, \allcaps{XHTML} та~інших. В результаті виконання роботи був розроблений необхідний план тестування~(додаток~\ref{sec:test-plan}).

	\section{Висновок}
		Під час виконання даного завдання ми оволоділи навичками створення плану тестування на прикладі програмного продукту~TopStyle.

	\newpage
	\appendix

	\section{План тестування}
	\label{sec:test-plan}

	% Macros for the test-plan
	\newcommand{\testplanid}{TOPSTYLE-5.0.0.104-TESTPLAN-000}
	\newcommand{\printtestplanid}{\texttt{\testplanid}}
	%
	\printtestplanid. Версія від 2018-06-20. Усі питання щодо змісту документа надсилати за~адресою~\href{mailto:john@example.com}{\texttt{john@example.com}}.
		
		\subsection{Вступ}
			\subsubsection{Призначення документа}
				Даний документ~(ідентифікатор~\printtestplanid)~— майстер-план тестування, розроблений для~визначення деталей проведення тестових робіт для~програмного продукту~TopStyle.

			\subsubsection{Терміни}
				У~тексті документу використовуються спеціальні терміни. Для~запобігання їх неправильної інтерпретації у~документі наводяться визначення.

				% \begin{table}[!htbp]
				% 	\caption{Терміни, використані в~даному документі, та~їх визначення}
				% 	\label{tab:definitions}
				% 	\begin{tabular}{
				% 			v{2\modulewidth - 2\tabcolsep}
				% 			v{4\modulewidth - 2\tabcolsep}
				% 	}
				% 			\toprule
				% 				Термін & Значення \\
				% 			\midrule
				% 				Веб-двигун & Програмний засіб, який перетворює розмічений зміст у~зручний для~зображення на~екрані вигляд\\
				% 				Встановлення програми & Процес перетворення початкових~(вихідних) ресурсів програми у~робочий програмний продукт\\
				% 				Графічний інтерфейс & Набір об'єктів, що візуально взаємодіють з користувачем за допомогою піктограм, рисунків, клавіш, кнопок тощо\\
				% 				Димове тестування & Мінімальний набір тестів для~виявлення явних помилок, які~заважають найбільш більш базовій роботі програми\\
				% 				Проект & Програмний продукт TopStyle\\
				% 				Робота програми & Процес безпосереднього виконання програмним продуктом функцій, закладених у~нього\\
				% 				Функціональне тестування & Тестування функцій програми на~відповідність вимогам\\
				% 			\bottomrule
				% 	
				% 	\end{tabular}
				% \end{table}

				\begin{description}
					\item [Веб-двигун] Програмний засіб, який перетворює розмічений зміст у~зручний для~зображення на~екрані вигляд.

					\item [Встановлення програми] Процес перетворення початкових~(вихідних) ресурсів програми у~робочий програмний продукт.

					\item [Графічний інтерфейс] Набір об'єктів, що візуально взаємодіють з користувачем за допомогою піктограм, рисунків, клавіш, кнопок тощо.

					\item [Димове тестування] Мінімальний набір тестів для~виявлення явних помилок, які~заважають найбільш більш базовій роботі програми.

					\item [Проект] Програмний продукт TopStyle.

					\item [Робота програми] Процес безпосереднього виконання програмним продуктом функцій, закладених у~нього.

					\item [Функціональне тестування] Тестування функцій програми на~відповідність вимогам.
				\end{description}

			\subsubsection{Мета тестування}
				Метою тестування Проекту є~визначення ступеня його відповідності функціональним вимогам, перевірка правильності роботи на~різних версіях підтримуваних операційних систем, а~також виявлення недоліків і~вразливостей Проекту.
		
			\subsubsection{Предмет тестування}
				Предметом тестування є~програмний продукт TopStyle~(версія 5.0.0.104)~— редактор коду на мовах розмітки~\allcaps{CSS}, \allcaps{HTML}, \allcaps{XHTML} та~інших, який надає користувачу можливість переглядати результат зображення написаного коду веб-двигунами~Gecko, Trident або~WebKit.

		\subsection{Функціонал, що тестується}
			Під час проведення тестування цільового уроекту необхідно протестувати процеси встановлення та~роботи програми, до~тестування процесу роботи програми входить тестування її графічного інтерфейсу. Згідно з~описом програми, суть роботи програми полягає у~редагуванні файлів початкового коду та~зображення результатів їх обробки веб-двигунами.

		\subsection{Функціонал, що не тестується}
			Процес тестування не передбачає тестування швидкодії програми, оскільки метою тестування є визначення ступеня відповідності функціональним вимогам, перевірка правильності роботи та~виявлення недоліків і вразливостей Проекту. Враховуючи специфіку роботи Проекту, швидкодія програми не є критичним фактором і тому не тестується.

		\subsection{Стратегія тестування}
			Тестування даного Проекту передбачає використання автоматичного і ручного тестування. Автоматичне тестування проводиться з використанням автоматизованих та~напівавтоматизованих процедурних засобів, які виконують перевірку роботи програми, знаходять в~ній помилки і неточності з мінімальним наглядом з боку людей. Для тестування Проекту серед автоматичних засобів повинні бути використані засоби налагоджування, статичного аналізу та~тестувальні фреймворки на розсуд Головного тестувальника.

			Ручне тестування полягає у власному огляді Проекту кваліфікованим працівником з метою перевірки критеріїв відповідності очікуванням, використовуючи свої навички та~підходи для оптимального виконання робіт.

			Оскільки метою тестування Проекту є визначення ступеня його відповідності функціональним вимогам, перевірка правильності роботи на різних версіях підтримуваних операційних систем, а також виявлення недоліків і вразливостей Проекту, стратегія сфокусована на \emph{функціональному тестуванні}, однак, не обмежується ним: стратегія також передбачає виконання тестування встановлення та~димового тестування.

			\subsubsection{Функціональне тестування}
				Мета функціонального тестування полягає у виявленні функціональних помилок, невідповідностей до технічного завдання і очікувань користувача шляхом реалізації стандартних, а також нетривіальних тестових сценаріїв.
				
				Розроблена стратегія передбачає виконання функціонального тестування в~контексті розробки та~перевірки тестових сценаріїв у процесі ручного та~автоматичного тестування кожного з елементів, потребуючих тестування. Основними компонентами, що підлягають функціональному тестуванню, є елементи, що відповідають за перевірку та~валідацію початкового коду, його відображення веб-двигунами, а також компоненти інтерфейсу.

			\subsubsection{Тестування встановлення}
				Тестування встановлення використовується для перевірки правильності, цілісності та~завершеності встановлення програмного продукту. У контексті Проекту тестування встановлення полягає в~автоматичному виконанні спроб встановлення на різних тестових конфігураціях та~звітуванні про результат.

			\subsubsection{Димове тестування}
				Димове тестування~— це~короткий цикл тестів, що~виконується для~підтвердження мінімальної робочої здатності програми. Мета димового тестування полягає у~попередженні найбільш серйозних дефектів, які~приводять систему у~критичний стан та~виводять з~ладу навіть базовий функціонал. У~контексті даного Проекту димове тестування полягає у~спробі запуску встановленої програми на~різних тестових конфігураціях. Результат димового тестування вважається успішним, якщо програма нормально починає роботу та~успішно запускається.

		\subsection{Вимоги до середовища тестування}
			Проект розрахований на роботу під управлінням операційної системи~Microsoft Windows, тому тестові конфігурації повинні включати в~себе системи, що працюють під управлінням Microsoft Windows~7, 8, 8.1 та~10~(версій~1511, 1607, 1703, 1709, 1803). Рекомендується віртуалізація тестових стендів для забезпечення зручності управління та~документування.

			Для організації робочого процесу~— документації, звітності та~розподілу часу~— необхідно використовувати систему контролю версій~GitLab, яка надає величезний вибір можливостей для зручного управління розробкою та~тестуванням.

			Усі вищенаведені засоби повинні знаходитись на окремому сервері~(виділеному або віртуальному), призначеному виключно для цілей Проекту. Знімки~(\emph{snapshots}) та~резервні копії даних повинні виконуватись щоденно.

		\subsection{Вимоги до людських ресурсів}
			Для належного виконання завдань процесу тестування необхідний кваліфікований персонал, тому до людських ресурсів, що прийматимуть участь у процесі тестування Проекту, ставляться відповідні вимоги~(табл.~\ref{tab:hr-requirements}).

			\begin{longtable}{
					v{3\modulewidth - 2\tabcolsep}
					v{4\modulewidth - 2\tabcolsep}
					v{5\modulewidth - 2\tabcolsep}
			}
					\caption{Вимоги до людських ресурсів\label{tab:hr-requirements}}\\
					\toprule
						Позиція & Вимоги & Обов'язки\\
					\midrule
				\endfirsthead
					\caption{Вимоги до людських ресурсів~(продовження)\label{tab:hr-requirements}}\\
					\toprule
						Позиція & Вимоги & Обов'язки\\
					\midrule
				\endhead
					\bottomrule
				\endfoot
					%
					Головний тестувальник &
					Стаж не менше 5~років &
					Керування процесом тестування на відповідність планам.\\
					%
					Розробник тестових сценаріїв &
					Наявність досвіду розробки тестових сценаріїв &
					Розробка тестових сценаріїв для виконання функціонального тестування.\\
					%
					Тестувальник &
					Наявність навичок ефективного використання інструментів, які планується використовувати у процесі тестування даного Проекту &
					Виконання тестування, нагляд за автоматизованим та~напівавтоматизованим тестуванням, оформлення результатів тестування.\\
					%
					Системний адміністратор &
					Наявність навичок ефективної роботи з обладнанням, яке планується використовувати у процесі тестування даного Проекту &
					Розгортання, моніторинг та~забезпечення правильної роботи середовища тестування.\\

			\end{longtable}

		\subsection{Етапи тестування}
			\label{subsec:testing-steps}
			Процес тестування цільового продукту поділяється на етапи~(табл.~\ref{tab:schedule}), які вирішують питання організації процесу тестування, безпосереднього проведення тестування необхідних вимог та~оформлення результатів тестування.

			\begin{longtable}{
					v{8\modulewidth - 2\tabcolsep}
					n{2\modulewidth - 2\tabcolsep}
					n{2\modulewidth - 2\tabcolsep}
			}
					\caption{Розклад проведення тестування\label{tab:schedule}}\\
					\toprule
						Етап тестування & \multicolumn{2}{r}{Дата}\\
						\cmidrule(lr){2-3}
						                & Початку & Кінця \\
					\midrule
				\endfirsthead
					\caption{Розклад проведення тестування~(продовження)\label{tab:schedule}}\\
					\toprule
						Етап тестування & \multicolumn{2}{r}{Дата}\\
						\cmidrule(lr){2-3}
						                & Початку & Кінця \\
					\midrule
				\endhead
					\bottomrule
				\endfoot
				Налаштування середовища тестування & 2019.01.01 & 2019.01.07\\
				Тестування встановлення програми   & 2019.01.07 & 2019.01.14\\
				Тестування роботи програми         & 2019.01.14 & 2019.02.14\\
				Оформлення результатів тестування  & 2019.02.14 & 2019.02.28\\
				Завершення тестування              & 2019.03.01 & 2019.03.14\\
			\end{longtable}
			
		\subsection{Критерії початку тестування}
			Для початку процесу тестування даного Проекту необхідно дотримання таких критеріїв:
			\begin{enumerate}
				\item Завершеність розробки функціоналу, що планується тестувати.
				\item Наявність проектної документації.
				\item Готовність людських ресурсів: наявність необхідного персоналу тощо.
				\item Готовність середовища тестування.
			\end{enumerate}
			
		\subsection{Критерії закінчення тестування}
			Для завершення процесу тестування даного Проекту необхідно дотримання таких критеріїв:
			\begin{enumerate}
				\item Завершеність основних етапів тестування~(підрозділ~\ref{subsec:testing-steps}).
				\item Наявність проектної документації.
				\item Прийняття рішення Керівником Проекту про закінчення тестування.
				\item Тривалість періоду без виявлення нових дефектів не менше тижня.
			\end{enumerate}

\end{document}

