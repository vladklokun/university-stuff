\documentclass[a4paper,oneside,BCOR=1cm,DIV=12,12pt,headings=normal]{scrartcl}

%%% Length calculations
\usepackage{calc}
%%%

%%% Support for color
\usepackage{xcolor}
\definecolor{lightblue}{HTML}{03A9F4}
\definecolor{red}{HTML}{F44336}
%%%

%%% Links and hyperreferences
\usepackage{hyperref}
\hypersetup{
	bookmarksnumbered = true,
	colorlinks        = false,
	linkbordercolor   = red,
	urlbordercolor    = lightblue,
	pdfborderstyle    = {/S/U/W 1.5},
}
%%%

%%% Graphics inclusion
\usepackage{graphicx}
%%%

%%% Support for hyphenated rags
\usepackage{ragged2e}
%%%

%%% Font selection
\usepackage{fontspec}

\setromanfont{STIX Two Text}[
]

\setsansfont{Source Sans Pro}[
]

\setmonofont{Source Code Pro}[
	% Scale = 1.05,
]

\usepackage{unicode-math}
\setmathfont{STIX Two Math}

%%%

%%% Font settings for different KOMA Script elements
\setkomafont{pagenumber}{\rmfamily}
\setkomafont{disposition}{\rmfamily\bfseries}
%%%

%%% Typographic enhancements
\usepackage{microtype}
%%%

%%% Language-specific settings
\usepackage{polyglossia}
\setmainlanguage{ukrainian}
%%%

%%% List settings
\usepackage{enumitem}
\setlist[enumerate]{
	leftmargin = *,
	label*     = {\arabic*.},
}
%%%

%%% Captions
\usepackage{caption}
%%%

%%% Tables
\usepackage{booktabs}
\usepackage{longtable}

\usepackage{multirow}

\usepackage{cellspace}
% \setlength\cellspacetoplimit{0\baselineskip}
% \setlength\cellspacebottomlimit{0.5\baselineskip}
% \renewcommand{\arraystretch}{1.5}
\addparagraphcolumntypes{v}

\usepackage{array}
\newcolumntype{v}[1]{>{\RaggedRight\arraybackslash\hspace{0pt}}p{#1}}
\newcolumntype{b}[1]{>{\centering\arraybackslash\hspace{0pt}}p{#1}}
\newcolumntype{n}[1]{>{\RaggedLeft\arraybackslash\hspace{0pt}}p{#1}}

% \usepackage{kbordermatrix} % labeling array indices
%%%

%%%

%%% Count figures within sections
\usepackage{chngcntr}
\counterwithin{figure}{section}
%%%

%%% Custom commands
% Print filenames
\newcommand{\filename}[1]{\texttt{#1}}

% Print command names
\newcommand{\progname}[1]{\texttt{#1}}

% Print function names
\newcommand{\funcname}[1]{\texttt{#1}}

% Typeset all caps
\newcommand{\allcaps}[1]{{\addfontfeatures{LetterSpace = 3}#1}}

%%%

%%% Lengths
\setlength{\emergencystretch}{1em}

\newlength{\modulewidth}
\setlength{\modulewidth}{\textwidth / 6}

% Save baselineskip
\newlength{\oldbaselineskip}
%%%


\begin{document}
	\begin{titlepage}
	\centering
		Міністерство освіти і науки України\\
		Національний авіаційний університет\\
		Навчально-науковий інститут комп'ютерних інформаційних технологій\\
		Кафедра комп'ютеризованих систем управління

		\vspace*{\fill}

		Завдання №1{.}3\\
		для проходження практики\\
		з дисципліни «Якість програмного забезпечення та тестування»\\
		на тему «Створення тестових наборів програми»

		\vspace*{\fill}
		
		\begin{flushright}
			Виконав:\\
			студент \allcaps{ННІКІТ} \allcaps{СП}-225\\
			Клокун В.\,Д.\\
			Перевірив:\\
			Сябрук І.\,М.
		\end{flushright}

		Київ 2018
	\end{titlepage}

	\section{Мета роботи}
		Оволодіти навичками створення тестових наборів програми.

	\section{Хід роботи}
		Процес виконання роботи передбачає створення тестових наборів для тестування програми TopStyle~— редактору початкового коду на мовах~\allcaps{CSS}, \allcaps{HTML}, \allcaps{XHTML}. В результаті виконання роботи були розроблені необхідні тестові набори~(табл.~\ref{tab:test-cases}).

		\setlength{\oldbaselineskip}{\baselineskip}
		\begin{longtable}{
				S{v{2\modulewidth - 2\tabcolsep}}
				S{v{3\modulewidth - 2\tabcolsep}}
				S{v{1\modulewidth - 2\tabcolsep}}
		}
			\caption{Тестові набори для тестування програми TopStyle\label{tab:test-cases}}\\
			\toprule
				Перевірочна дія & Очікуваний результат & Фактичний рез.\\
			\midrule
			\endfirsthead
			\caption{Тестові набори для тестування програми TopStyle~(продовження)}\\
			\toprule
				Перевірочна дія & Очікуваний результат & Фактичний рез.\\
			\midrule
			\endhead
			\bottomrule
			\endfoot
			%
			Відкрити існуючий файл, формат якого підтримується &
			Файл відкривається і~на~екран виводиться його вміст &
			Виконано \\
			\addlinespace[\oldbaselineskip]
			%
			Відкрити існуючий файл, формат якого не~підтримується &
			На~екран виводиться повідомлення, що~перегляд файлів, формат яких не~підтримується, не~рекомендується. На~екран виводиться зміст файлу. &
			Виконано\\
			\addlinespace[\oldbaselineskip]
			%
			Зберегти наразі відкритий файл &
			Файл зберігається &
			Виконано\\
			\addlinespace[\oldbaselineskip]
			%
			Внести зміни у~відкритий файл та~зберегти його &
			Внесені зміни збережені &
			Виконано \\
			\addlinespace[\oldbaselineskip]
			%
			Змінити файл та~повернутись до~попереднього стану зміни за~допомогою клавіш Ctrl+Z &
			Файл повертається до~попереднього стану &
			Виконано \\
			\addlinespace[\oldbaselineskip]
			%
			Змінити файл, повернутись до~попереднього стану зміни~(Ctrl+Z) та~відмінити повернення до~попереднього стану~(Ctrl+Shift+Z) &
			Файл повертається до~попереднього стану, а~потім відміняє повернення до~попереднього стану &
			Виконано \\
			% \addlinespace[\oldbaselineskip]
			%
			У~полі редагування файлу ввести існуючу CSS-властивість~(property) &
			Одразу з'являється вікно підказки для~вибору властивості &
			Помилка\\
			\addlinespace[\oldbaselineskip]
			%
			Перейти до~наступного правила~\allcaps{CSS}~(Next \allcaps{CSS} Rule) &
			У~вікні редагування файлу виконується перехід до~наступного правила~\allcaps{CSS} &
			Виконано\\
			\addlinespace[\oldbaselineskip]
			%
			Перейти до~попереднього правила~\allcaps{CSS}~(Previous \allcaps{CSS} Rule) &
			У~вікні редагування файлу виконується перехід до~попереднього правила~\allcaps{CSS} &
			Виконано\\
			\addlinespace[\oldbaselineskip]
			%
			Додати коментар &
			До~файлу у~поточній позиції курсору додаються символи, які~позначають коментар &
			Виконано\\
			\addlinespace[\oldbaselineskip]
			%
			Виконати перевірку стилю~(F6) файлу~\texttt{sample2.css} з~комплекту поставки &
			У~відповідному вікні з'являється результат перевірки: 0~помилок, 0~попереджень &
			Виконано \\
			\addlinespace[\oldbaselineskip]
			%
			Додати невірне правило~\allcaps{CSS}, виконати перевірку стилю~(F6) файлу &
			У~відповідному вікні з'являється результат перевірки: наявні помилки &
			Виконано \\
			\addlinespace[\oldbaselineskip]
			%
			Виконати перевірку стилю усіх відкритих файлів~(Ctrl+Shift+F6) &
			У~відповідному вікні з'являється результат перевірки &
			Виконано \\
			\addlinespace[\oldbaselineskip]
			%
			Розділити вікно редактора на~дві вертикальні частини &
			Вікно редактора розділяється на~дві вертикальні частини &
			Виконано\\
			\addlinespace[\oldbaselineskip]
			%
			Створити нову робочу область~(New Workspace...)~«Testing» &
			Створюється та~відкривається нова робоча область~«Testing» &
			Виконано \\
			% \addlinespace[\oldbaselineskip]
			%
			Відкрити робочу область~(Open Workspace)~«Testing» &
			Після вибору відкривається робоча область~«Testing» &
			Виконано\\
			\addlinespace[\oldbaselineskip]
			%
			Закрити поточну робочу область~(Close Workspace) &
			Поточна робоча область закривається &
			Виконано\\
			\addlinespace[\oldbaselineskip]
			%
			Закрити існуючий файл, формат якого підтримується &
			Файл закривається, його зміст зникає з~екрану &
			Виконано \\
		\end{longtable}

	\section{Висновок}
		Під час виконання даного завдання ми оволоділи навичками створення тестових наборів для тестування~програмного забезпечення.

\end{document}

