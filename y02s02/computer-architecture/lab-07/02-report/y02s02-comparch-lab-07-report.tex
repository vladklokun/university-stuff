\documentclass[a4paper,oneside,DIV=12,12pt,headings=normal]{scrartcl}

%%% Length calculations
\usepackage{calc}
%%%

%%% Support for color
\usepackage{xcolor}
\definecolor{lightblue}{HTML}{03A9F4}
\definecolor{red}{HTML}{F44336}
%%%

%%% Graphics inclusion
\usepackage{graphicx}
%%%

%%% Font selection
\usepackage{fontspec}

\setromanfont{STIX Two Text}[
	SmallCapsFeatures = {LetterSpace = 5},
]

\setsansfont{Source Sans Pro}[
]

\setmonofont{Source Code Pro}[
]
%%%

%%% Math settings
\usepackage{amsmath,unicode-math}
\setmathfont{STIX Two Math}

% \usepackage{IEEEtrantools}
\usepackage{mleftright}
%%%

%%% Font settings for different KOMA Script elements
\setkomafont{pagenumber}{\rmfamily}
\setkomafont{disposition}{\rmfamily\bfseries}
%%%

%%% Typographic enhancements
\usepackage{microtype}
%%%

%%% Language-specific settings
\usepackage{polyglossia}
\setmainlanguage{ukrainian}
%%%

%%% List settings
\usepackage{enumitem}
 \setlist[enumerate]{
	 noitemsep,
% 	leftmargin = *,
 }
%%%

%%% Captions
\usepackage{caption}
\usepackage{subcaption}

\DeclareCaptionLabelFormat{closing}{#2)}
\captionsetup[subtable]{labelformat = closing}
\captionsetup[subfigure]{labelformat = closing, position = auto}
%%%

%%% Tables
\usepackage{booktabs}
\usepackage{longtable}

\usepackage{multirow}

\usepackage{array}
\newcolumntype{v}[1]{>{\raggedright\arraybackslash\hspace{0pt}}p{#1}}
\newcolumntype{b}[1]{>{\centering\arraybackslash\hspace{0pt}}p{#1}}
\newcolumntype{n}[1]{>{\raggedleft\arraybackslash\hspace{0pt}}p{#1}}

\usepackage{kbordermatrix} % labeling array indices
%%%

%%% Floats on a single row
\usepackage{floatrow}
\newfloatcommand{capbtabbox}{table}[][\FBwidth]
%%%

%%% Physical units
\usepackage{siunitx}
%%%

%%% Set counters withing subsections
\usepackage{chngcntr}
\counterwithin{table}{section}
\counterwithin{figure}{section}
%%%

%%%
\usepackage{pdflscape}
%%%

%%% Links and hyperreferences
\usepackage{hyperref}
\hypersetup{
	colorlinks      = false,
	linkbordercolor = red,
	urlbordercolor  = lightblue,
	pdfborderstyle  = {/S/U/W 1.5},
}
%%%

%%% All caps
\newcommand{\allcaps}[1]{{\addfontfeatures{LetterSpace = 3}#1}}
%%%

%%% Ceiling function typesetting
\newcommand{\ceil}[1]{\mleft\lceil#1\mright\rceil}
%%%

%%% Typesetting schematic elements
\newcommand{\schel}[1]{\textit{#1}}
%%%

%%% Shift operators
\DeclareMathOperator{\L1}{L1}
\DeclareMathOperator{\R1}{R1}
%%%

%%% Address intervals
\newcommand{\addrinterval}[2]{(#1{:}#2)}
%%%

\setlength{\emergencystretch}{1em}

\begin{document}
	\begin{titlepage}
	\centering
		Міністерство освіти і науки України\\
		Національний авіаційний університет\\
		Навчально-науковий інститут комп'ютерних інформаційних технологій\\
		Кафедра комп'ютеризованих систем управління

		\vspace*{\fill}

		Лабораторна робота №7\\
		з дисципліни «Архітектура комп'ютерів»\\
		на тему «Блок мікропрограмованого керування»\\
		% Варіант №4

		\vspace*{\fill}
		
		\begin{flushright}
			Виконав:\\
			студент ННІКІТ СП-225\\
			Клокун В.\,Д.\\
			Перевірив:\\
			Зіньков Ю.\,Г.
		\end{flushright}

		Київ 2018
    \end{titlepage}

		\section{Мета роботи}
			Вивчення схемотехніки та системи інструкцій (команд) схеми керування послідовністю мікрокоманд~К1894ВУ4, побудова блоку мікропрограмованого керування на її основі та розробка мікропрограм.

		\section{Завдання}
			При виконанні роботи ставиться завдання розробити мікропрограму для блоку мікропрограмованого керування на основі заданих даних~(табл.~\ref{tab:task}).
			\begin{table}[!htbp]
				\centering
				\begin{tabular}{llv{4.5em}v{4.5em}v{16.9em}}
					\toprule
						№ & МП i & Вершин циклу & Кількість циклів & Взаємозв'язок МП та~МПП\\
					\midrule
						4 & МП3(0010) & 3–7 & 2 & МП3№3(50)~$\rightarrow$ МПП4(100)№3(200)~$\rightarrow$ МПП1(300) \\
					\bottomrule
				\end{tabular}
				\caption{Завдання}
				\label{tab:task}
			\end{table}

		\section{Хід роботи}
			Для виконання роботи надані такі початкові умови:
			\begin{enumerate}
				\item Основна мікропрограма та її початкова адреса: МП3(0010).
				\item Номери вершин, які складають циклічну ділянку мікропрограми: 3–7.
				\item Число повторень в циклі: 2.
				\item Взаємозв'язки основної МП та МПП: МП3№3(50)~$\rightarrow$ МПП4(100)№3(200)~$\rightarrow$ МПП1(300).
			\end{enumerate}
			Звідси маємо дані про номери та адреси місць звернення для кожної програми~(табл.~\ref{tab:mp-details}). Складаємо закодовану мікропрограму для блока мікпропрограмованого керування~(табл.~\ref{tab:coded-microprogram}).
			\begin{table}[!htbp]
				\begin{tabular}{ll}
					\toprule
						Символ & Значення \\
					\midrule
						МП3       & Ім'я основної мікропрограми\\
						№3(50)    & Номер та адреса місця звернення в МП3 до МПП4\\
						МПП4(100) & Ім'я та початкова адреса МПП4\\
						№3(200)   & Номер та адреса місця звернення в МПП4 до МПП1\\
						МПП1(300) & Ім'я та початкова адреса МПП1\\
					\bottomrule
				\end{tabular}
				\caption{Дані про мікропрограми}
				\label{tab:mp-details}
			\end{table}

			\begin{landscape}
				\begin{table}[p]
				\centering
					\begin{tabular}{rrl*{12}{r}}
						\toprule
							 &  &  & \multicolumn{7}{c}{Мікрокоманда} & \multicolumn{4}{c}{Вхідні дані} &  \\
							 \cmidrule(lr){4-10} \cmidrule(lr){11-14}
							 № & Адреса & Мнемоніка & $I$ & $P$ & $MS$ & $\overline{CC}$ & $COM$ & $\overline{RLD}$ & $CI$ & $M$ & $V$ & $LSI$ & $\overline{EN}$ & $DO$ \\
						\midrule
							1  & 0204 & LDCT & 14 & 0002 & x & x & x & 1 & 1 & xxxx & xxxx & xxxx & 0 & 0205 \\
							2  & 0205 & CONT & 16 & xxxx & x & x & x & 1 & 1 & xxxx & xxxx & xxxx & 0 & 020 \\
							3  & 0206 & CJP  & 03 & 4777 & 1 & 0 & 1 & 1 & 1 & xxxx & xxxx & 0xxx & 0 & 4777 \\
							4  & 4777 & CJS  & 01 & 3401 & x & 1 & x & 1 & 1 & xxxx & xxxx & xxxx & 0 & 3401 \\
							5  & 5000 & CJP  & 03 & 0207 & x & 1 & x & 1 & 1 & xxxx & xxxx & xxxx & 0 & 0207 \\
							6  & 0207 & RPCT & 11 & 0205 & x & x & x & 1 & 1 & xxxx & xxxx & xxxx & 0 & 0207 \\
							7  & 3401 & CONT & 16 & xxxx & x & x & x & 1 & 1 & xxxx & xxxx & xxxx & 0 & 3402 \\
							8  & 3402 & CJV  & 06 & xxxx & 2 & 0 & 1 & 1 & 1 & xxxx & 7020 & x0xx & 0 & 7020 \\
							9  & 3403 & SJP  & 02 & 7080 & x & x & x & 1 & 1 & xxxx & xxxx & xxxx & 0 & 7020 \\
							10 & 7020 & CRTN & 12 & xxxx & x & 1 & x & 1 & 1 & xxxx & xxxx & xxxx & 0 & 5000 \\
						\bottomrule
					\end{tabular}
				\caption{Закодована мікропрограма для блоку мікропрограмного керування}
				\label{tab:coded-microprogram}
				\end{table}
			\end{landscape}

			Перевіряємо правильність мікропрограми за допомогою програмного продукту «Емулятор К1894ВУ4». Результати виконання мікропрограми для трьох значень логічних сигналів~$LS(x_1)$ та~$LS(x_2)$.
			\begin{table}[!htbp]
				\centering
				\begin{subtable}[t]{0.31\linewidth}
					\vspace{0em}
					\centering
					\begin{tabular}{*{3}{n{3em}}}
						\toprule
							Число виконаних МК & Номер МК в~МП & Адреса наступної МК\\
						\midrule
							1  & 1  & 0205 \\
							2  & 2  & 0206 \\
							3  & 3  & 4777 \\
							4  & 4  & 3401 \\
							5  & 7  & 3402 \\
							6  & 8  & 7020 \\
							7  & 10 & 5000 \\
							8  & 5  & 0207 \\
							9  & 6  & 0205 \\
							10 & 2  & 0206 \\
							11 & 3  & 4777 \\
							12 & 4  & 3401 \\
							13 & 7  & 3402 \\
							14 & 8  & 7020 \\
							15 & 10 & 5000 \\
							16 & 5  & 0207 \\
							17 & 6  & 0207 \\
							18 & 2  & 0206 \\
							19 & 3  & 4777 \\
							20 & 4  & 3401 \\
							21 & 7  & 3402 \\
							22 & 8  & 7020 \\
							23 & 10 & 5000 \\
							24 & 51 & 0207 \\
							25 & 8  & 0210 \\
						\bottomrule
					\end{tabular}
					\caption{}
					\label{subtab:mp-res-ls00}
				\end{subtable}
				~
				\begin{subtable}[t]{0.31\linewidth}
					\vspace{0em}
					\centering
					\begin{tabular}{*{3}{n{3em}}}
						\toprule
							Число виконаних МК & Номер МК в~МП & Адреса наступної МК\\
						\midrule
							1  & 1  & 0205 \\
							2  & 2  & 0206 \\
							3  & 3  & 4777 \\
							4  & 4  & 3401 \\
							5  & 7  & 3402 \\
							6  & 8  & 3403 \\
							7  & 9 & 7020 \\
							8  & 10 & 5000 \\
							9  & 5  & 0207 \\
							10 & 6  & 0205 \\
							11 & 2  & 0206 \\
							12 & 3  & 4777 \\
							13 & 4  & 3401 \\
							14 & 7  & 3402 \\
							15 & 8  & 3403 \\
							16 & 9  & 7020 \\
							17 & 10 & 5000 \\
							18 & 5  & 0207 \\
							19 & 6  & 0205 \\
							20 & 2  & 0206 \\
							21 & 3  & 4777 \\
							22 & 4  & 3401 \\
							23 & 7  & 3402 \\
							24 & 8  & 3403 \\
							25 & 9  & 7020 \\
							26 & 10 & 5000 \\
							27 & 6  & 0207 \\
							28 & 6  & 0210 \\
						\bottomrule
					\end{tabular}
					\caption{}
					\label{subtab:mp-res-ls01}
				\end{subtable}
				~
				\begin{subtable}[t]{0.31\linewidth}
					\vspace{0em}
					\centering
					\begin{tabular}{*{3}{n{3em}}}
						\toprule
							Число виконаних МК & Номер МК в~МП & Адреса наступної МК\\
						\midrule
							1  & 1  & 0205 \\
							2  & 2  & 0206 \\
							3  & 3  & 0207 \\
							4  & 6  & 0205 \\
							5  & 2  & 0206 \\
							6  & 3  & 0207 \\
							7  & 6  & 0205 \\
							8  & 2  & 0206 \\
							9  & 3  & 0207 \\
							10 & 8  & 0210 \\
						\bottomrule
					\end{tabular}
					\caption{}
					\label{subtab:mp-res-ls10}
				\end{subtable}
				\caption{Результат виконання мікропрограми: \subref{subtab:mp-res-ls00}~— при $LS(x_1) = 0$, $LS(x_2) = 0$, \subref{subtab:mp-res-ls01}~— при $LS(x_1) = 0$, $LS(x_2) = 1$, \subref{subtab:mp-res-ls10}~— при $LS(x_1) = 1$, $LS(x_2) = 0$}
			\end{table}

		\section{Висновок}
			Під час виконання даної лабораторної роботи ми вивчили схемотехніку та систему інструкцій (команд) схеми керування послідовністю мікрокоманд~К1894ВУ4, побудували блок мікропрограмованого керування на її основі та розробили мікропрограму.


\end{document}

