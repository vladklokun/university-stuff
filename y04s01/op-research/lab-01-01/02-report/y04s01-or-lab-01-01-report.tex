\documentclass[
  a4paper,
  oneside,
  BCOR = 10mm,
  DIV = 12,
  12pt,
  headings = normal,
]{scrartcl}

%%% Length calculations
\usepackage{calc}
%%%

%%% Support for color
\usepackage{xcolor}
\definecolor{lightblue}{HTML}{03A9F4}
\definecolor{red}{HTML}{F44336}
%%%

%%% Including graphics
\usepackage{graphicx}
%%%

%%% Font selection
\usepackage{fontspec}

\setromanfont{STIX Two Text}[
  SmallCapsFeatures = {LetterSpace = 8},
]

\setsansfont{IBM Plex Sans}[
  Scale = MatchUppercase,
]

\setmonofont{IBM Plex Mono}[
  Scale = MatchUppercase,
]
%%%

%%% Math typesetting
\usepackage{amsmath}

\usepackage{unicode-math}
\setmathfont{STIX Two Math}

\usepackage{IEEEtrantools}
%%%

%%% List settings
\usepackage{enumitem}
\setlist[enumerate]{
  label*      = {\arabic*.},
  left        = \parindent,
  topsep      = 0\baselineskip,
  parsep      = 0\baselineskip,
  noitemsep, % override itemsep
}
% List settings for levels 2–4
\setlist[enumerate, 2, 3, 4]{
  label*      = {\arabic*.},
  left        = 0em,
  topsep      = 0\baselineskip,
  parsep      = 0\baselineskip,
  noitemsep, % override itemsep
}

\setlist[itemize]{
  label*      = {—},
  left        = \parindent,
  topsep      = 0\baselineskip,
  parsep      = 0\baselineskip,
  itemsep     = 1\baselineskip,
  noitemsep, % override itemsep
}

\setlist[description]{
  font        = {\rmfamily\upshape\bfseries},
  topsep      = 1\baselineskip,
  parsep      = 0\baselineskip,
  itemsep     = 0\baselineskip,
}

%%%

%%% Structural elements typesetting
\setkomafont{pagenumber}{\rmfamily\upshape}
\setkomafont{disposition}{\rmfamily\bfseries}

% Sectioning
\RedeclareSectionCommand[
  beforeskip = -1\baselineskip,
  afterskip  = 1\baselineskip,
  font       = {\normalsize\bfseries\scshape},
]{section}

\RedeclareSectionCommand[
  beforeskip = -1\baselineskip,
  afterskip  = 1\baselineskip,
  font       = {\normalsize\bfseries\itshape},
]{subsection}

\RedeclareSectionCommand[
  beforeskip = -1\baselineskip,
  afterskip  = 1\baselineskip,
  font       = {\normalsize\bfseries},
]{subsubsection}

\RedeclareSectionCommand[
  beforeskip = -1\baselineskip,
  afterskip  = -0.5em,
  font       = {\normalsize\mdseries\scshape\addfontfeatures{Letters = {UppercaseSmallCaps}}},
]{paragraph}
%%%

%%% Typographic enhancements
\usepackage{microtype}
%%%

%%% Language-specific settings
\usepackage{polyglossia}
\setmainlanguage{ukrainian}
\setotherlanguages{english}
%%%

%%% Captions
\usepackage{caption}
\usepackage{subcaption}

%\DeclareCaptionLabelFormat{closing}{#2)}
%\captionsetup[subtable]{labelformat = closing}

%\captionsetup[subfigure]{labelformat = closing}

\captionsetup[table]{
  aboveskip = 0\baselineskip,
  belowskip = 0\baselineskip,
}

\captionsetup[figure]{
  aboveskip = 1\baselineskip,
  belowskip = 0\baselineskip,
}

\captionsetup[subfigure]{
  labelformat = simple,
  labelformat = brace,
  justification = RaggedRight,
  singlelinecheck = false,
}
%%%

%%% Hyphenated ragged typesetting
\usepackage{ragged2e}
%%%

%%% Table typesetting
\usepackage{booktabs}
\usepackage{longtable}

\usepackage{multirow}

\usepackage{array}
\newcolumntype{v}[1]{>{\RaggedRight\arraybackslash\hspace{0pt}}p{#1}}
\newcolumntype{b}[1]{>{\Centering\arraybackslash\hspace{0pt}}p{#1}}
\newcolumntype{n}[1]{>{\RaggedLeft\arraybackslash\hspace{0pt}}p{#1}}
%%%

%%% Drawing
\usepackage{tikz}
\usepackage{tikzscale}
\usetikzlibrary{positioning}
\usetikzlibrary{arrows.meta} % Stealth arrow tips
%%%

%%% SI units typesetting
\usepackage{siunitx}
\sisetup{
  output-decimal-marker = {,},
  exponent-product      = {\cdot},
  inter-unit-product    = \ensuremath{{} \cdot {}},
  per-mode              = symbol,
}
%%%

% Code Highlighting
\usepackage{minted}
\setmintedinline{
  style = bw,
  breaklines,
}

\newminted[bashterm]{text}{%
  autogobble,%
  breaklines,%
  style=bw,%
}

\newminted[codegeneric]{text}{%
  autogobble,%
  style=bw,%
  breaklines,%
  fontsize=\small,%
}

\newmintinline{bash}{%
}

\newmintinline[minttext]{text}{%
  breaklines,%
  breakanywhere,%
}

%%% Framing code listings
\usepackage{tcolorbox}
\tcbuselibrary{breakable}
\tcbuselibrary{minted}
\tcbuselibrary{skins}

% Text file listing
\newtcblisting[
  auto counter,
  list inside,
  number within = section,
]{listingplaintext}[3][]{%
  minted language = text,
  minted style    = bw,
  minted options  = {
    autogobble,
    linenos,
    tabsize = 4,
    breaklines,
    breakanywhere,
    fontsize = \footnotesize,
  },
  empty,
  sharp corners,
  coltitle = black,
  borderline horizontal = {1pt}{0pt}{black},
  titlerule = {0.5pt},
  titlerule style = {
    black,
  },
  toptitle = 0.3em,
  bottomtitle = 0.3em,
  before skip      = \intextsep,
  after  skip      = \intextsep,
  title            = {Лістинг \thetcbcounter: #2},
  list entry       = {\protect\numberline{\thetcbcounter}#2},
  left = 0em,
  right = 0em,
  %
  listing only,
  breakable,
  %
  label = {#3},%
}

\newtcblisting[
  use counter from = listingplaintext,
  list inside,
  number within = section,
]{listingpython}[3][]{%
  minted language = python,
  minted style    = bw,
  minted options  = {
    autogobble,
    linenos,
    tabsize = 4,
    breaklines,
    breakanywhere,
    fontsize = \footnotesize,
  },
  empty,
  sharp corners,
  coltitle = black,
  borderline horizontal = {1pt}{0pt}{black},
  titlerule = {0.5pt},
  titlerule style = {
    black,
  },
  toptitle = 0.3em,
  bottomtitle = 0.3em,
  before skip      = \intextsep,
  after  skip      = \intextsep,
  title            = {Лістинг \thetcbcounter: #2},
  list entry       = {\protect\numberline{\thetcbcounter}#2},
  left = 0em,
  right = 0em,
  %
  listing only,
  breakable,
  %
  label = {#3},
  %
  #1%
}

\newtcbinputlisting[
  use counter from = listingplaintext,
  list inside,
  number within = section
]{\inputpython}[4][]{%
  minted language = python,
  minted style    = bw,
  minted options  = {
    autogobble,
    linenos,
    tabsize = 4,
    breaklines,
    breakanywhere,
    fontsize = \footnotesize,
  },
  empty,
  sharp corners,
  coltitle = black,
  borderline horizontal = {1pt}{0pt}{black},
  titlerule = {0.5pt},
  titlerule style = {
    black,
  },
  toptitle = 0.3em,
  bottomtitle = 0.3em,
  before skip      = \intextsep,
  after  skip      = \intextsep,
  title            = {Лістинг \thetcbcounter: #3},
  list entry       = {\protect\numberline{\thetcbcounter}#3},
  left = 0em,
  right = 0em,
  %
  listing file={#2},
  listing only,
  breakable,
  %
  label = {#4}
}

% Linux command-line listing
\newtcblisting{linuxterm}%
{%
  % Syntax highlighing options
  listing only,%
  minted language = bash,%
  minted options={%
    autogobble,%
    linenos%
  },%
  % Presentation options
  empty,%
  %% Margins
  sharp corners,%
  toptitle = 0.0em,%
  bottomtitle = 0.0em,%
  left = 0em,%
  right = 0em,%
  before skip = \intextsep,%
  after skip = \intextsep,%
}

\newtcblisting{linuxtermout}%
{%
  % Syntax highlighing options
  listing only,%
  minted language = text,%
  minted options={%
    autogobble,%
    linenos%
  },%
  % Presentation options
  empty,%
  %% Margins
  sharp corners,%
  toptitle = 0.0em,%
  bottomtitle = 0.0em,%
  left = 0em,%
  right = 0em,%
  before skip = \intextsep,%
  after skip = \intextsep,%
}

% Dockerfile listings
\newtcblisting[
  use counter from = listingplaintext,
  list inside,
  number within = section,
]{listingdocker}[3][]{%
  minted language = dockerfile,
  minted style    = bw,
  minted options  = {
    autogobble,%
    linenos,
    tabsize = 4,
    breaklines,
    breakanywhere,
    fontsize = \footnotesize,
  },
  empty,
  sharp corners,
  coltitle = black,
  borderline horizontal = {1pt}{0pt}{black},
  titlerule = {0.5pt},
  titlerule style = {
    black,
  },
  toptitle = 0.3em,
  bottomtitle = 0.3em,
  before skip      = \intextsep,
  after  skip      = \intextsep,
  title            = {Лістинг \thetcbcounter: #2},
  list entry       = {\protect\numberline{\thetcbcounter}#2},
  left = 0em,
  right = 0em,
  %
  listing only,
  breakable,
  %
  label = {#3},%
}

% Docker Compose listings
\newtcblisting[
  use counter from = listingplaintext,
  list inside,
  number within = section,
]{listingdockercompose}[3][]{%
  minted language = yaml,
  minted style    = bw,
  minted options  = {
    autogobble,%
    linenos,
    tabsize = 4,
    breaklines,
    breakanywhere,
    fontsize = \footnotesize,
  },
  empty,
  sharp corners,
  coltitle = black,
  borderline horizontal = {1pt}{0pt}{black},
  titlerule = {0.5pt},
  titlerule style = {
    black,
  },
  toptitle = 0.3em,
  bottomtitle = 0.3em,
  before skip      = \intextsep,
  after  skip      = \intextsep,
  title            = {Лістинг \thetcbcounter: #2},
  list entry       = {\protect\numberline{\thetcbcounter}#2},
  left = 0em,
  right = 0em,
  %
  listing only,
  breakable,
  %
  label = {#3},%
}


% Customize minted line numbers
\renewcommand{\theFancyVerbLine}{\ttfamily\scriptsize\arabic{FancyVerbLine}}

%%%

%%% Typeset menus and keys
\usepackage{menukeys}[
  os=win,
]
%%%

%%% Links and hyperreferences
\usepackage{hyperref}
\hypersetup{
  bookmarksnumbered = true,
  colorlinks      = false,
  linkbordercolor = red,
  urlbordercolor  = lightblue,
  pdfborderstyle  = {/S/U/W 1.5},
}
%%%

%%% Length adjustment

% Set baselineskip, default is 14.5 pt
\linespread{1.068966} % ~15.5 pt
\setlength{\emergencystretch}{1em}
\setlength{\parindent}{1.5em}
\newlength{\gridunitwidth}
\setlength{\gridunitwidth}{\textwidth / 12}
%%%

%%% Custom commands
\newcommand{\allcaps}[1]{%
  {%
    \addfontfeatures{%
      Letters = UppercaseSmallCaps,
      LetterSpace = 8,%
    }%
    #1%
  }%
}
\newcommand{\filename}[1]{\texttt{#1}}
\newcommand{\progname}[1]{\texttt{#1}}
\newcommand{\commandname}[1]{\texttt{#1}}
\newcommand{\modulename}[1]{\texttt{#1}}
\newcommand{\transeng}[1]{{англ.}~\textit{\textenglish{#1}}}
%%%

%%% Custom math commands
\newcommand{\longvar}[1]{\mathit{#1}}
%%%

\begin{document}

\begin{titlepage}
    \begin{center}
      Міністерство освіти і~науки України\\
      Національний авіаційний університет\\
      Факультет кібербезпеки, комп'ютерної та програмної інженерії\\
      Кафедра комп'ютеризованих систем управління

      \vspace{\fill}
        Лабораторна робота №~1.1\\
        з~дисципліни «Дослідження операцій»\\
        на~тему «Предмет, методи та~завдання курсу „Дослідження операцій“. Побудова оптимізаційних економіко-математичних моделей»

      \vspace{\fill}

      \begin{flushright}
        Виконав:\\
        студент \allcaps{ННІКІТ}\\
        групи \allcaps{СП}-425\\
        Клокун В.\,Д.\\
        Перевірила:\\
        Яковенко Л.\,В.
      \end{flushright}

      Київ 2019
    \end{center}
  \end{titlepage}

  \section{Завдання роботи}
    Підприємство може виготовляти чотири види продукції: $T_{1}$, $T_2$, $T_3$, $T_4$. Норми витрат ресурсів і~прибуток від~одиниці кожного виду ресурсу наведені в~таблиці~\ref{tab:task}.

    \begin{table}[!htbp]
      \centering
      \caption{Норми витрат ресурсів і~прибуток від~одиниці ресурсу}
      \label{tab:task}
      \begin{tabular}{
        v{4\gridunitwidth}
        *{4}{S[
            table-format=3.1,
            table-column-width=1\gridunitwidth,
          ]
        }
        n{2\gridunitwidth}
      }
        \toprule
          Показники & \multicolumn{4}{c}{Продукція} & Ресурси \\
          \cmidrule(lr){2-5}
          & $T_{1}$ & $T_{2}$ & $T_{3}$ & $T_{4}$\\
        \midrule
          Трудові ресурси, людино-змін & 2.5 & 2.5 & 2   & 1.5 & 100\\
          Напівфабрикати, кг~          & 4   & 10  & 4   & 6   & 260\\
          Обладнання, станко-зміни     & 8   & 7   & 4   & 10  & 370\\
          Прибуток від~од. продукції   & 40  & 50  & 100 & 80  & \\
        \bottomrule
      \end{tabular}
    \end{table}

    Записати математичну модель випуску продукції, яка~максимізує прибутки, якщо:
    \begin{enumerate}
      \item Кількість одиниць третьої продукції повинна бути в~3~рази більшою за~кількість одиниць першої продукції.
      \item Першої продукції слід випускати не~менше 25~одиниць, третьої~— не~більше~30, а~другої і~четвертої~— у~співвідношенні~$1:3$.
    \end{enumerate}

  \section{Хід~роботи}
    Усі~моделі дослідження операцій складаються з~3 базових компонентів:
    \begin{enumerate}
      \item Керованих змінних, значення яких необхідно визначити.
      \item Цілі (або~цільової функції), яку~необхідно оптимізувати.
      \item Обмежень, які~має~задовольняти розв'язок.
    \end{enumerate}

    За~умовою задачі необхідно максимізувати прибуток підприємства, яке~виготовляє продукцію декількох видів. Підприємство отримує прибуток, коли виготовляє і~продає певну кількість продукції, отже його прибуток залежить від~кількості проданої продукції кожного виду. Підприємство може регулювати, скільки продукції виготовляти, тому ці~кількості і~будуть керованими змінними. Якщо позначити~$i$-й~вид~продукції як~$T_{i}$, а~кількість виготовлених одиниць продукції~$i$-го~виду як~$x_{i}$, то~$x_{1}$, $x_{2}$, $x_{3}$ та~$x_{4}$ і~будуть керованими змінними.

    Ціль задачі~— максимізувати прибутки. Позначимо прибуток від~продажу однієї одиниці продукції виду~$T_{i}$ як~$p_{i}$, тоді прибуток від~продажу~$x_{i}$ одиниць товару цього ж~типу виражається так: $P_{i} = p_{i} x_{i}$.

    Прибутки від~продажу одиниці кожного з~видів продукції наведені в~умові задачі, тому загальний прибуток від~виготовлення довільної кількості продукції кожного виду виражається такою функцією:
    \begin{IEEEeqnarray*}{rCl}
      P &=& p_{1} x_{1} + p_{2} x_{2} + p_{3} x_{3} + p_{4} x_{4}
         = 40 x_{1} + 50 x_2 + 100 x_3 + 90 x_{4}.
    \end{IEEEeqnarray*}
    Ціль поставленої задачі максимізувати прибутки, тому її~можна записати так:
    \begin{IEEEeqnarray*}{rCl}
      P &=& 40 x_{1} + 50 x_2 + 100 x_3 + 90 x_{4} \to \max.
    \end{IEEEeqnarray*}

    Також задача накладає певні обмеження. Наприклад, підприємство може виробляти лише невід'ємну кількість кожного з~видів продукції, тобто:
    \begin{IEEEeqnarray*}{rCl}
      x_{1}, x_{2}, x_{3}, x_{4} \geqslant 0.
    \end{IEEEeqnarray*}
    Крім цього кількість виготовленої продукції певних типів має~задовольняти такі вимоги щодо пропорцій одиниць продукції відносно одна одної:
    \begin{IEEEeqnarray*}{rCl}
      x_{3} = 3 x_{1}, \quad
      x_{1} \geqslant 25, \quad
      x_{3} \leqslant 30, \quad
      x_{2} = 3 x_{4}.
    \end{IEEEeqnarray*}
    Також сказано, що~кількість кожного ресурсу обмежена, тобто:
    \begin{IEEEeqnarray*}{rCrCrCrCl.s}
          \num{2.5} x_1 &+& \num{2.5} x_2 &+& 2 x_3 &+& \num{1.5} x_4 &\leqslant& 100  & — трудові ресурси,\\
          4 x_1   &+& 10 x_2  &+& 4 x_3 &+& 6 x_4   &\leqslant& 260 & — напівфабрикати, \\
          8 x_1   &+& 7 x_2   &+& 4 x_3 &+& 10 x_4  &\leqslant& 370 & — обладнання.
    \end{IEEEeqnarray*}

    Отже, враховуючи усі~вищезазначені деталі, математичну модель задачі можна записати так:
    \begin{IEEEeqnarray*}{l}
      z = 40 x_1 + 50 x_2 + 100 x_3 + 80 x_4 \to \max,\\
      \left\{ \,
        \begin{IEEEeqnarraybox}[
          \IEEEeqnarraystrutmode
          \IEEEeqnarraystrutsizeadd{2pt}{2pt}
        ][c]{rCrCrCrCl}
          \num{2.5} x_1 &+& \num{2.5} x_2 &+& 2 x_3 &+& \num{1.5} x_4 &\leqslant& 100 \\
          4 x_1   &+& 10 x_2  &+& 4 x_3 &+& 6 x_4   &\leqslant& 260, \\
          8 x_1   &+& 7 x_2   &+& 4 x_3 &+& 10 x_4  &\leqslant& 370, \\
          3 x_1   &=        & \multicolumn{7}{l}{$x_3,$}   \\
            x_1   &\geqslant& \multicolumn{7}{l}{$25,$}    \\
            x_3   &\leqslant& \multicolumn{7}{l}{$30,$}    \\
            x_2   &=        & \multicolumn{7}{l}{$3 x_4,$}\\
            x_1  &\multicolumn{8}{l}{$, x_2, x_3, x_4 \geqslant 0.$}
        \end{IEEEeqnarraybox}
      \right.
    \end{IEEEeqnarray*}

  \section{Висновок}
    Виконуючи дану лабораторну роботу, ми~ознайомились з~предметом, методами та~завданням курсу «Дослідження операцій», а~також побудували оптимізаційну економіко-математичну модель поставленої задачі.

\end{document}
