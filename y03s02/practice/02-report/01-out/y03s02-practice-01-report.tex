\documentclass[
	a4paper,
	oneside,
	BCOR = 10mm,
	DIV = 12,
	12pt,
	headings = normal,
]{scrartcl}

%%% Length calculations
\usepackage{calc}
%%%

%%% Support for color
\usepackage{xcolor}
\definecolor{lightblue}{HTML}{03A9F4}
\definecolor{red}{HTML}{F44336}
%%%

%%% Including graphics
\usepackage{graphicx}
%%%

%%% Font selection
\usepackage{fontspec}

\setromanfont{STIX Two Text}[
	SmallCapsFeatures = {LetterSpace = 8},
]

\setsansfont{IBM Plex Sans}[
	Scale = MatchUppercase,
]

\setmonofont{IBM Plex Mono}[
	Scale = MatchUppercase,
]
%%%

%%% Math typesetting
\usepackage{amsmath}

\usepackage{unicode-math}
\setmathfont{STIX Two Math}

\usepackage{IEEEtrantools}
%%%

%%% List settings
\usepackage{enumitem}
\setlist[enumerate]{
	label*      = {\arabic*.},
	left        = \parindent,
	topsep      = 0\baselineskip,
	parsep      = 0\baselineskip,
	noitemsep, % override itemsep
}
% List settings for levels 2–4
\setlist[enumerate, 2, 3, 4]{
	label*      = {\arabic*.},
	left        = 0em,
	topsep      = 0\baselineskip,
	parsep      = 0\baselineskip,
	noitemsep, % override itemsep
}

\setlist[itemize]{
	label       = {—},
	left        = \parindent,
	topsep      = 0\baselineskip,
	parsep      = 0\baselineskip,
	itemsep     = 1\baselineskip,
	noitemsep, % override itemsep
}

\setlist[itemize, 2, 3, 4]{
	label       = {—},
	left        = 0em,
	topsep      = 0\baselineskip,
	parsep      = 0\baselineskip,
	itemsep     = 1\baselineskip,
	noitemsep, % override itemsep
}

\setlist[description]{
	font        = {\rmfamily\upshape\bfseries},
	topsep      = 1\baselineskip,
	parsep      = 0\baselineskip,
	itemsep     = 0\baselineskip,
}

%%%

%%% Structural elements typesetting
\setkomafont{pagenumber}{\rmfamily\upshape}
\setkomafont{disposition}{\rmfamily\bfseries}

% Sectioning
\RedeclareSectionCommand[
	beforeskip = -1\baselineskip,
	afterskip  = 1\baselineskip,
	font       = {\normalsize\bfseries\scshape},
]{section}

\RedeclareSectionCommand[
	beforeskip = -1\baselineskip,
	afterskip  = 1\baselineskip,
	font       = {\normalsize\bfseries\itshape},
]{subsection}

\RedeclareSectionCommand[
	beforeskip = -1\baselineskip,
	afterskip  = 1\baselineskip,
	font       = {\normalsize\bfseries},
]{subsubsection}

\RedeclareSectionCommand[
	beforeskip = -1\baselineskip,
	afterskip  = -0.5em,
	font       = {\normalsize\mdseries\scshape\addfontfeatures{Letters = {UppercaseSmallCaps}}},
]{paragraph}
%%%

%%% Typographic enhancements
\usepackage{microtype}
%%%

%%% Language-specific settings
\usepackage{polyglossia}
\setmainlanguage{ukrainian}
\setotherlanguages{english}
%%%

%%% Captions
\usepackage{caption}
\usepackage{subcaption}

%\DeclareCaptionLabelFormat{closing}{#2)}
%\captionsetup[subtable]{labelformat = closing}

%\captionsetup[subfigure]{labelformat = closing}

\captionsetup[table]{
	aboveskip = 0\baselineskip,
	belowskip = 0\baselineskip,
}

\captionsetup[figure]{
	aboveskip = 1\baselineskip,
	belowskip = 0\baselineskip,
}

\captionsetup[subfigure]{
	labelformat = simple,
	labelformat = brace,
}
%%%

%%% Hyphenated ragged typesetting
\usepackage{ragged2e}
%%%

%%% Table typesetting
\usepackage{booktabs}
\usepackage{longtable}

\usepackage{multirow}

\usepackage{array}
\newcolumntype{v}[1]{>{\RaggedRight\arraybackslash\hspace{0pt}}p{#1}}
\newcolumntype{b}[1]{>{\Centering\arraybackslash\hspace{0pt}}p{#1}}
\newcolumntype{n}[1]{>{\RaggedLeft\arraybackslash\hspace{0pt}}p{#1}}
%%%

%%% Drawing
\usepackage{tikz}
\usepackage{tikzscale}
\usetikzlibrary{positioning}
\usetikzlibrary{arrows.meta} % Stealth arrow tips

\usepackage{tikz-uml}
%%%

%%% SI units typesetting
\usepackage{siunitx}
\sisetup{
	output-decimal-marker = {,},
	exponent-product      = {\cdot},
	inter-unit-product    = \ensuremath{{} \cdot {}},
	per-mode              = symbol,
}
%%%

% Code Highlighting
\usepackage{minted}
\setmintedinline{
	style = bw,
	breaklines,
}

\newminted[bashterm]{bash}{%
	autogobble,%
	style=bw,%
}

\newmintinline{bash}{%
}

%%% Framing code listings
\usepackage{tcolorbox}
\tcbuselibrary{breakable}
\tcbuselibrary{minted}
\tcbuselibrary{skins}

% Text file listing
\newtcblisting[
	auto counter,
	list inside, 
	number within = section,
]{listingplaintext}[3][]{%
	minted language = text,
	minted style    = bw,
	minted options  = {
		autogobble,
		linenos,
		tabsize = 4,
		breaklines,
		breakanywhere,
		fontsize = \footnotesize,
	},
	empty,
	sharp corners,
	coltitle = black,
	borderline horizontal = {1pt}{0pt}{black},
	titlerule = {0.5pt},
	titlerule style = {
		black,
	},
	toptitle = 0.3em,
	bottomtitle = 0.3em,
	before skip      = \intextsep,
	after  skip      = \intextsep,
	title            = {Лістинг \thetcbcounter: #2},
	list entry       = {\protect\numberline{\thetcbcounter}#2},
	left = 0em,
	right = 0em,
	%
	listing only,
	breakable,
	%
	label = {#3},%
}

\newtcblisting[
	use counter from = listingplaintext,
	list inside, 
	number within = section,
]{listingpython}[3][]{%
	minted language = python,
	minted style    = bw,
	minted options  = {
		autogobble,
		linenos,
		tabsize = 4,
		breaklines,
		breakanywhere,
		fontsize = \footnotesize,
	},
	empty,
	sharp corners,
	coltitle = black,
	borderline horizontal = {1pt}{0pt}{black},
	titlerule = {0.5pt},
	titlerule style = {
		black,
	},
	toptitle = 0.3em,
	bottomtitle = 0.3em,
	before skip      = \intextsep,
	after  skip      = \intextsep,
	title            = {Лістинг \thetcbcounter: #2},
	list entry       = {\protect\numberline{\thetcbcounter}#2},
	left = 0em,
	right = 0em,
	%
	listing only,
	breakable,
	%
	label = {#3},
	%
	#1%
}

\newtcbinputlisting[
	use counter from = listingplaintext,
	list inside,
	number within = section
]{\inputpython}[4][]{%
	minted language = python,
	minted style    = bw,
	minted options  = {
		autogobble,
		linenos,
		tabsize = 4,
		breaklines,
		breakanywhere,
		fontsize = \footnotesize,
	},
	empty,
	sharp corners,
	coltitle = black,
	borderline horizontal = {1pt}{0pt}{black},
	titlerule = {0.5pt},
	titlerule style = {
		black,
	},
	toptitle = 0.3em,
	bottomtitle = 0.3em,
	before skip      = \intextsep,
	after  skip      = \intextsep,
	title            = {Лістинг \thetcbcounter: #3},
	list entry       = {\protect\numberline{\thetcbcounter}#3},
	left = 0em,
	right = 0em,
	%
	listing file={#2},
	listing only,
	breakable,
	%
	label = {#4}
}

% Linux command-line listing
\newtcblisting{linuxterm}%
{%
	% Syntax highlighing options
	listing only,%
	minted language = bash,%
	minted options={%
		autogobble,%
		linenos%
	},%
	% Presentation options
	empty,%
	%% Margins
	sharp corners,%
	toptitle = 0.0em,%
	bottomtitle = 0.0em,%
	left = 0em,%
	right = 0em,%
	before skip = \intextsep,%
	after skip = \intextsep,%
}

\newtcblisting{linuxtermout}%
{%
	% Syntax highlighing options
	listing only,%
	minted language = text,%
	minted options={%
		autogobble,%
		linenos%
	},%
	% Presentation options
	empty,%
	%% Margins
	sharp corners,%
	toptitle = 0.0em,%
	bottomtitle = 0.0em,%
	left = 0em,%
	right = 0em,%
	before skip = \intextsep,%
	after skip = \intextsep,%
}

% Dockerfile listings
\newtcblisting[
	use counter from = listingplaintext,
	list inside, 
	number within = section,
]{listingdocker}[3][]{%
	minted language = dockerfile,
	minted style    = bw,
	minted options  = {
		autogobble,%
		linenos,
		tabsize = 4,
		breaklines,
		breakanywhere,
		fontsize = \footnotesize,
	},
	empty,
	sharp corners,
	coltitle = black,
	borderline horizontal = {1pt}{0pt}{black},
	titlerule = {0.5pt},
	titlerule style = {
		black,
	},
	toptitle = 0.3em,
	bottomtitle = 0.3em,
	before skip      = \intextsep,
	after  skip      = \intextsep,
	title            = {Лістинг \thetcbcounter: #2},
	list entry       = {\protect\numberline{\thetcbcounter}#2},
	left = 0em,
	right = 0em,
	%
	listing only,
	breakable,
	%
	label = {#3},%
}

% Docker Compose listings
\newtcblisting[
	use counter from = listingplaintext,
	list inside, 
	number within = section,
]{listingdockercompose}[3][]{%
	minted language = yaml,
	minted style    = bw,
	minted options  = {
		autogobble,%
		linenos,
		tabsize = 4,
		breaklines,
		breakanywhere,
		fontsize = \footnotesize,
	},
	empty,
	sharp corners,
	coltitle = black,
	borderline horizontal = {1pt}{0pt}{black},
	titlerule = {0.5pt},
	titlerule style = {
		black,
	},
	toptitle = 0.3em,
	bottomtitle = 0.3em,
	before skip      = \intextsep,
	after  skip      = \intextsep,
	title            = {Лістинг \thetcbcounter: #2},
	list entry       = {\protect\numberline{\thetcbcounter}#2},
	left = 0em,
	right = 0em,
	%
	listing only,
	breakable,
	%
	label = {#3},%
}

% Customize minted line numbers
\renewcommand{\theFancyVerbLine}{\ttfamily\scriptsize\arabic{FancyVerbLine}}

%%%

%%% Links and hyperreferences
\usepackage{hyperref}
\hypersetup{
	bookmarksnumbered = true,
	colorlinks      = false,
	linkbordercolor = red,
	urlbordercolor  = lightblue,
	pdfborderstyle  = {/S/U/W 1.5},
}
%%%

%%% Length adjustment

% Set baselineskip, default is 14.5 pt
\linespread{1.068966} % ~15.5 pt
\setlength{\emergencystretch}{1em}
\setlength{\parindent}{1.5em}
\newlength{\gridunitwidth}
\setlength{\gridunitwidth}{\textwidth / 12}
%%%

%%% Custom commands
\newcommand{\allcaps}[1]{%
	{%
		\addfontfeatures{%
			Letters = UppercaseSmallCaps,
			LetterSpace = 8,%
			Kerning = Off%
		}%
		#1%
	}%
}
\newcommand{\makeallcaps}[1]{%
	\allcaps{\MakeUppercase{#1}}
}
\newcommand{\filename}[1]{\texttt{#1}}
\newcommand{\progname}[1]{\texttt{#1}}
\newcommand{\commandname}[1]{\texttt{#1}}
\newcommand{\modulename}[1]{\texttt{#1}}
\newcommand{\transeng}[1]{{англ.}~\textit{\textenglish{#1}}}
\newcommand{\elemtitle}[1]{%
	{\centering%
		\makeallcaps{#1}%
		\par
	}
}
\newcommand{\elemtitleext}[2]{%
	{\centering%
		\makeallcaps{#1}\\%
		#2%
		\par
	}
}
\newcommand{\blank}[1]{\rule[-0.0em]{#1}{0.4pt}}
\newcommand{\lastblank}{\hrulefill}
%%%

%%% Custom math commands
\newcommand{\longvar}[1]{\mathit{#1}}
%%%

\begin{document}

\begin{titlepage}
		\begin{center}
			Міністерство освіти і~науки України\\
			Національний авіаційний університет\\
			Навчально-науковий інститут комп'ютерних інформаційних технологій\\
			Кафедра комп'ютеризованих систем управління

			\vspace{\fill}
				Звіт\\
				з~проектно-технологічної практики

			\vspace{\fill}

			\begin{flushleft}
				студента 3~курсу \allcaps{СП}-325~групи\\
				напряму~123~«Комп'ютерна інженерія»\\
				Клокуна Владислава Денисовича\\[1\baselineskip]

				База практики: \allcaps{ТОВ} «Смарт Медіа Інвест»\\[1\baselineskip]

				Керівник практики:\\
				від університету — старший викладач Кашкевич Іван Фуркатович\\
				від бази практики — керівник проектів Чезганов Олександр Сергійович\\
			\end{flushleft}
			
			% \vspace{\fill}

			Київ 2019
		\end{center}
	\end{titlepage}

	\thispagestyle{empty}%
	{%
		\setlength{\parindent}{0em}%
		\setlength{\parskip}{1\baselineskip}%
		Підстава для проходження практики: робочий навчальний план №~РС-4-6.05010202/12 підготовки фахівців за напрямом підготовки 6.050102~«Комп'\-ю\-те\-рна інженерія», договір на проведення практики між університетом та \allcaps{ТОВ} «Смарт Медіа Інвест», наказ ректора №~687/ст від 10.04.2019.

		% \vspace{1\baselineskip}
		Термін проходження практики: з 03.06.2019 по 23.06.2019

		% \vspace{1\baselineskip}
		Індивідуальні завдання: розробити програмний продукт~— чат-бота для платформи~\textenglish{Telegram}.

		\elemtitleext{Відмітка}{про проходження практики}
		Прибув на базу практики\\
		«\blank{3em}» \blank{9em} 2019~р. (підпис) \lastblank{} М.\,П.

		% \vspace{1\baselineskip}
		Вибув з бази практики\\
		«\blank{3em}» \blank{9em} 2019~р. (підпис) \lastblank{} М.\,П.

		\elemtitle{Календарний план-графік практики}
		\begin{tabular}{
			|n{1\gridunitwidth - 2\tabcolsep}
			|v{6.5\gridunitwidth - 2\tabcolsep}
			|n{2.25\gridunitwidth - 2\tabcolsep}
			|n{2.25\gridunitwidth - 2\tabcolsep}
			|
		}
			\hline
				\multirow{2}{\linewidth}{№ п/п} & \multirow{2}{*}{Об'єкт практики та види робіт} & \multicolumn{2}{c|}{Термін виконання}\\
					\cline{3-4}
					&  & початок & закінчення\\
			\hline
				 1 & Постановка і аналіз задачі. Вибір необхідних інструментів для розробки. Затвердження специфікації & 03.06.2019 & 04.06.2019 \\
				\hline
				 2 & Знайомство з інфраструктурою і забезпеченням організації & 05.06.2019 & 07.06.2019 \\
				\hline
				 3 & Розробка програмний продукт & 10.06.2019 & 14.06.2019 \\
				\hline
				 4 & Розробка рішення для розгортання програми & 17.06.2019 & 18.06.2019 \\
				\hline
				 5 & Тестування і запуск розробленого програмного продукту & 19.06.2019 & 21.06.2019 \\
			\hline
		\end{tabular}

		Керівник практики від університету: \lastblank
	}

	% Finish introductory pages
	\newpage
	\tableofcontents

	\newpage
	\section{Характеристика бази практики}
		\subsection{Загальні відомості про організацію}
			Організація \allcaps{ТОВ}~«Смарт Медіа Інвест» зареєстрована за~адресою 47201, Тернопільська~область, Зборівський~район, місто~Зборів, вулиця~Б.\,Хмельницького. За класифікатором видів економічної діяльності вона є рекламним агентством, займається створенням, запуском, супроводом, просуванням і підтримкою медіапроектів у~соціальних мережах на кшталт «Вконтакте», «\textenglish{Instagram}» і~«\textenglish{Telegram}».
			
		\subsection{Сфера діяльності}
			Щоб підтримувати медіапроект, необхідно збирати, створювати, оформлювати і публікувати інформацію, на якій він спеціалізується, розважати користувачів проекту, а також тримати його у належному стані. Для досягнення цих цілей і покращення результатів використовують різні інструменти: для розваги користувачів~— мультимедійні матеріали і інтерактивні заходи, для підтримки проекту~— спеціальні програмні продукти.
		
		За кожну зі складових відповідають певні структурні підрозділи, які тісно взаємодіють між собою. Пошуком і створенням інформації, розробкою ідей, креативу, а також інтерактивних заходів займаються редактори. Мультимедійні матеріали для оформлення розроблених ідей надають дизайнери. За створення і запуск інструментів, необхідних для організації інтерактивних розважальних заходів і підтримки медіапроектів, відповідають розробники. Крім цього, розробники відповідають за налаштування і підтримку апаратно-програмної інфраструктури, яка необхідна для правильної роботи вже запущених продуктів, у~справному стані.
		
		Інструменти, створенням і запуском яких займаються розробники, можна класифікувати так:
		\begin{itemize}
			\item розважальні інструменти:
				\begin{itemize}
					\item чат-боти,
					\item боти-коментатори,
				\end{itemize}
			\item супровідні інструменти:
				\begin{itemize}
					\item очищувачі,
					\item спеціалізовані сценарії (живі обкладинки тощо),
					\item інструменти збору статистики.
				\end{itemize}
		\end{itemize}
		Щоб зрозуміти, що входить в обов'язки розробників, розглянемо кожен клас цих інструментів детальніше. \emph{Розважальні інструменти} призначені для того, щоб розважити користувача, безпосередньою взаємодією з ним. Прикладом розважального інструменту є \emph{чат-бот}~— програмний продукт, який веде діалог (чат) з користувачем відповідно до заданого сценарію або специфікації. Наприклад, за допомогою чат-бота можна реалізувати текстову пригоду, головним героєм якого буде кожен окремий користувач, організувати лотерею або віртуального помічника, який відповідатиме на питання користувача.

		\emph{Бот-коментатор}~— це програмний продукт, який додає коментарі у спеціальних дошках для обговорення у соціальних мережах залежно від певної умови. Наприклад, якщо користувач залишає коментар до запису про товар, в якому питає про деталь, відому боту-коментатору, цей бот може залишити коментар з відповіддю на поставлене питання. 

		Наступною категорією є \emph{супровідні інструменти}, тобто інструменти, призначені для автоматизованої підтримки представництв медіапроекту у певному стані. Першим представником супровідних інструментів є \emph{очищувач}~— програмний продукт який очищує ту чи іншу складову присутності медіапроекту у соціальній мережі: записи на дошці обговорень та її коментарі, зміст фото-, аудіо- і відеоальбомів тощо.

		\emph{Спеціалізований сценарій} (або \emph{скрипт})~— це програмний продукт, який виконує різноманітні вузькоспеціалізовані дії в залежності від свого призначення. Поширеним прикладом спеціалізованого сценарію є «жива обкладинка»~— програмний продукт, який змінює обкладинку представництва медіапроекту у соціальній мережі в залежності від певних параметрів: кількості записів, створених користувачами, проведених дій, переглядів тощо.

		\emph{Інструмент збору статистики}~— це програмний продукт, який збирає статистику різних представництв медіапроекту у соціальних мережах, оброблює зібрані дані та експортує їх у потрібний формат. Результати роботи цих інструментів використовуються для аналізу, відстеження і планування стану та життєздатності проекту, а також для звітності.

		Отже, як бачимо, компанія займається повним циклом управління медіапроектами. Для цього вона використовує набір різноманітних інструментів, до яких також входять програмні інструменти, за створення і запуск яких відповідає підрозділ розробників, в якому автор проходив практику.

	\section{Зміст та результати виконаних робіт}
		\subsection{Постановка і аналіз задачі}
			Для проходження проектно-виробничої практики була поставлена така задача: розробити чат-бота для платформи~«\textenglish{Telegram}», який надсилає казки своїм користувачам. Коли користувач запускає бота, він повинен привітати його і надіслати випадкову казку. Якщо користувач просить надіслати ще, бот надсилає наступну випадкову казку. Також, бот повинен регулярно розсилати казки своїм користувачам: щодня в певний час.

			\begin{figure}[!htbp]
				\centering
				\begin{tikzpicture}
					\tikzumlset{fill usecase=white}
					\begin{umlsystem}{Чат-бот}
						\umlusecase[x = 04, y = 00, width = 2\gridunitwidth, name = start-bot]{Підписатись на розсилку}
						\umlusecase[x = 04, y = 02, width = 2\gridunitwidth, name = request-tale]{Отримати казку}
						\umlusecase[x = 10.5, y = 02, width = 2\gridunitwidth, name = request-tale-reg]{Отримати щоденну казку}
						\umlusecase[x = 04, y = 04.5, width = 2\gridunitwidth, name = rate-tale]{Оцінити отриману казку}
					\end{umlsystem}

					% Define actors
					\umlactor[x = 00, y = 02, name = user]{Користувач}

					\umlassoc{Користувач}{start-bot}
					\umlassoc{Користувач}{request-tale}
					\umlinclude{request-tale}{request-tale-reg}

				\end{tikzpicture}
				\caption{Діаграма варіантів використання чат-бота}
				\label{fig:bot-diag-use-case}
			\end{figure}

		\subsection{Знайомство з інфраструктурою і забезпеченням організації}

\end{document}

