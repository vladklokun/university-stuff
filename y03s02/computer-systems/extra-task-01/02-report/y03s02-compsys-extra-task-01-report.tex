\documentclass[
	a4paper,
	oneside,
	BCOR = 10mm,
	DIV = 12,
	12pt,
	headings = normal,
	bibliography = totoc,
]{scrartcl}

%%% Length calculations
\usepackage{calc}
%%%

%%% Support for color
\usepackage{xcolor}
\definecolor{lightblue}{HTML}{03A9F4}
\definecolor{red}{HTML}{F44336}
%%%

%%% Including graphics
\usepackage{graphicx}
%%%

%%% Font selection
\usepackage{fontspec}

\setromanfont{STIX Two Text}[
	SmallCapsFeatures = {LetterSpace = 8},
]

\setsansfont{IBM Plex Sans}[
	Scale = MatchUppercase,
]

\setmonofont{IBM Plex Mono}[
	Scale = MatchUppercase,
]
%%%

%%% Math typesetting
\usepackage{amsmath}

\usepackage{unicode-math}
\setmathfont{STIX Two Math}

\usepackage{IEEEtrantools}
%%%

%%% List settings
\usepackage{enumitem}
\setlist[enumerate]{
	label*      = {\arabic*.},
	leftmargin  = *,
	labelindent = \parindent,
	topsep      = 1\baselineskip,
	parsep      = 0\baselineskip,
	itemsep     = 1\baselineskip,
	noitemsep, % override itemsep
}

\setlist[itemize]{
	label*      = {—},
	leftmargin  = *,
	labelindent = \parindent,
	topsep      = 1\baselineskip,
	parsep      = 0\baselineskip,
	itemsep     = 1\baselineskip,
	noitemsep, % override itemsep
}

\setlist[description]{
	font        = {\rmfamily\upshape\bfseries},
	topsep      = 1\baselineskip,
	parsep      = 0\baselineskip,
	itemsep     = 0\baselineskip,
}

\newlist{steps}{enumerate}{10}
\setlist[steps]{
	label*      = {\arabic*.},
	leftmargin  = 0em,
	labelindent = \parindent,
	topsep      = 1\baselineskip,
	parsep      = 0\baselineskip,
	itemsep     = 1\baselineskip,
	noitemsep, % override itemsep
}


%%%

%%% Structural elements typesetting
\setkomafont{pagenumber}{\rmfamily\upshape}
\setkomafont{disposition}{\rmfamily\bfseries}

% Sectioning
\RedeclareSectionCommand[
	beforeskip = -1\baselineskip,
	afterskip  = -0.5em,
	font       = {\normalsize\bfseries\scshape},
]{section}

\RedeclareSectionCommand[
	beforeskip = -1\baselineskip,
	afterskip  = 1\baselineskip,
	font       = {\normalsize\bfseries\itshape},
]{subsection}

\RedeclareSectionCommand[
	beforeskip = -1\baselineskip,
	afterskip  = 1\baselineskip,
	font       = {\normalsize\bfseries},
]{subsubsection}

\RedeclareSectionCommand[
	beforeskip = -1\baselineskip,
	afterskip  = -0.5em,
	font       = {\normalsize\mdseries\scshape\addfontfeatures{Letters = {UppercaseSmallCaps}}},
]{paragraph}
%%%

%%% Typographic enhancements
\usepackage{microtype}
%%%

%%% Language-specific settings
\usepackage{polyglossia}
\setmainlanguage{ukrainian}
\setotherlanguages{english}
%%%

%%% Captions
\usepackage{caption}
\usepackage{subcaption}

%\DeclareCaptionLabelFormat{closing}{#2)}
%\captionsetup[subtable]{labelformat = closing}

%\captionsetup[subfigure]{labelformat = closing}

\captionsetup[table]{
	aboveskip = 0\baselineskip,
	belowskip = 0\baselineskip,
}

\captionsetup[figure]{
	aboveskip = 1\baselineskip,
	belowskip = 0\baselineskip,
}

\captionsetup[subfigure]{
	labelformat = simple,
	labelformat = brace,
}
%%%

%%% Hyphenated ragged typesetting
\usepackage{ragged2e}
%%%

%%% Table typesetting
\usepackage{booktabs}
\usepackage{longtable}

\usepackage{multirow}

\usepackage{array}
\newcolumntype{v}[1]{>{\RaggedRight\arraybackslash\hspace{0pt}}p{#1}}
\newcolumntype{b}[1]{>{\Centering\arraybackslash\hspace{0pt}}p{#1}}
\newcolumntype{n}[1]{>{\RaggedLeft\arraybackslash\hspace{0pt}}p{#1}}
%%%

%%% Drawing
\usepackage{tikz}
\usepackage{tikzscale}
\usetikzlibrary{positioning}
\usetikzlibrary{arrows.meta} % Stealth arrow tips
%%%

%%% SI units typesetting
\usepackage{siunitx}
\sisetup{
	output-decimal-marker = {,},
	exponent-product      = {\cdot},
	inter-unit-product    = \ensuremath{{} \cdot {}},
	per-mode              = symbol,
}
%%%

%%% Framing code listings
\usepackage{tcolorbox}
\tcbuselibrary{breakable}
\tcbuselibrary{minted}
\tcbuselibrary{skins}

\newtcblisting[
	auto counter, 
	list inside, 
	number within = section,
]{listingprolog}[3][]{%
	minted language = prolog,
	minted style    = bw,
	minted options  = {
		linenos,
		tabsize = 4,
		breaklines,
		% breakanywhere,
		fontsize = \footnotesize,
		autogobble
	},
	%
	empty,
	sharp corners,
	coltitle              = black,
	borderline horizontal = {1pt}{0pt}{black},
	titlerule             = 0.5pt,
	titlerule style       = {
		black,
	},
	toptitle         = 0.3em,
	bottomtitle      = 0.1em,
	before skip      = \intextsep,
	after  skip      = \intextsep,
	title            = {Лістинг \thetcbcounter: #2},
	list entry       = {\protect\numberline{\thetcbcounter}#2},
	left = 0em,
	right = 0em,
	%
	listing only,
	breakable,
	%
	label = {#3},
	%
	#1
}

\newtcbinputlisting[auto counter, list inside, number within = section]{\inputprolog}[4][]{%
	minted language = prolog,
	minted style    = bw,
	minted options  = {
		linenos,
		tabsize = 4,
		breaklines,
		breakbytokenanywhere,
		fontsize = \footnotesize,
	},
	%
	empty,
	sharp corners,
	coltitle              = black,
	borderline horizontal = {1pt}{0pt}{black},
	titlerule             = 0.5pt,
	titlerule style       = {
		black,
	},
	toptitle         = 0.3em,
	bottomtitle      = 0.1em,
	before skip      = \intextsep,
	after  skip      = \intextsep,
	title            = {Лістинг \thetcbcounter: #3},
	list entry       = {\protect\numberline{\thetcbcounter}#3},
	left = 0em,
	right = 0em,
	%
	listing file={#2},
	listing only,
	breakable,
	%
	label = {#4},
	%
	#1
}

% Customize minted
\usepackage{minted}
\setminted{
	style = bw,
	breaklines,
	linenos,
}

\setmintedinline{
	style = bw,
	breaklines,
}

% Customize minted line numbers
\renewcommand{\theFancyVerbLine}{\ttfamily\scriptsize\arabic{FancyVerbLine}}

%%%

%%% Keystroke typesetting
% \usepackage[
% 	os = win,
% ]
% {menukeys}
%%%

%%% Bibliography
\usepackage[
	style    = gost-numeric,
	language = auto,
	autolang = other,
	sorting  = none,
]{biblatex}
\addbibresource{y03s02-compsys-extra-task-01-bibliography.bib}
%%%

%%% Links and hyperreferences
\usepackage{hyperref}
\hypersetup{
	% bookmarksnumbered = true,
	colorlinks      = false,
	linkbordercolor = red,
	urlbordercolor  = lightblue,
	pdfborderstyle  = {/S/U/W 1.5},
}
%%%

%%% Length adjustments
% Set baselineskip, default is 14.5 pt
\linespread{1.068966} % ~15.5 pt
\setlength{\emergencystretch}{1em}
\setlength{\parindent}{1.5em}
\newlength{\gridunitwidth}
\setlength{\gridunitwidth}{\textwidth / 12}
%%%

%%% Custom commands
\newcommand{\allcaps}[1]{{\addfontfeatures{LetterSpace = 8, Kerning = Off}#1}}
\newcommand{\filename}[1]{\texttt{#1}}
\newcommand{\progname}[1]{\texttt{#1}}
\newcommand{\modulename}[1]{\texttt{#1}}

\newcommand{\userinput}[1]{\texttt{#1}}

\newcommand{\Mytextrightarrow}{$\rightarrow$}
%%%

%%% Custom math commands
\newcommand{\longvar}[1]{\mathit{#1}}
%%%

\begin{document}

\begin{titlepage}
		\begin{center}
			Міністерство освіти і науки України\\
			Національний авіаційний університет\\
			Навчально-науковий інститут комп'ютерних інформаційних технологій\\
			Кафедра комп'ютеризованих систем управління

			\vspace{\fill}
				Додаткове завдання~№1\\
				з~дисципліни «Комп'ютерні системи»\\
				% на~тему «Структура програми на~\textenglish{Turbo Prolog}»\\

			\vspace{\fill}

			\begin{flushright}
				Виконав:\\
				студент \allcaps{ННІКІТ}\\
				групи СП-325\\
				Клокун В.\,Д.\\
				Перевірив:\\
				Жуков І.\,А.
			\end{flushright}

			Київ 2019
		\end{center}
	\end{titlepage}

	\section{Хід роботи}
		\subsection{Поясніть, чому~мультипроцесорні обчислювальні системи доцільніші при~меншій кількості процесорів}
			% \paragraph{Завдання}
			% 	Поясніть, чому мультипроцесорні обчислювальні системи доцільніше використовувати при~невеликій кількості процесорів.

			Однією з~найважливіших технік прискорення роботи сучасних обчислювальних систем є~\emph{паралелізм}~— виконання двох і~більше задач одночасно. Паралелізм впроваджується на~декількох рівнях, одним з~яких є~процесорний. Існує 2~види комп'ютерних систем з~процесорним паралелізмом: 
			\begin{enumerate}[itemsep = 1\baselineskip]
				\item Мультипроцесор~— комп'ютерна система з~декількома процесорами, які~мають спільну пам'ять і~ділять~її між~собою. 
				\item Мультикомп'ютер~— комп'ютерна система з~декількома процесорами, кожен з~яких має власну пам'ять. 
			\end{enumerate}

			У~мультипроцесорах кожен процесор має доступ до~всієї пам'\-я\-ті і~може читати та~записувати дані лише за~допомогою інструкцій~\texttt{LOAD} і~\texttt{STORE}~\cite[586]{tanenbaum-structured-comp-org}, тобто всі процесори мають бути підключені до~пам'яті певним чином~— шиною. Ця~особливість зумовлює декілька недоліків: 
			\begin{enumerate}[itemsep = 1\baselineskip]
				\item Коли велика кількість процесорів намагаються отримати доступ до~пам'\-я\-ті однією шиною, з'я\-вля\-ються конфлікти, більше процесорів~— більше конфліктів. 
				\item Розробка великих мультипроцесорів є~складною задачею, оскільки важко під'\-єд\-на\-ти велику кількість процесорів до~пам'\-ят\-і~\cite[73]{tanenbaum-structured-comp-org}. 
			\end{enumerate}
			Тим не~менш, відносно легко побудувати мультипроцесор з~числом процесорів не~більше~256~\cite[73]{tanenbaum-structured-comp-org}. 
			
			Отже, мультипроцесори доцільніше використовувати виключно до~певної кількості процесорів, так~як зі~збільшенням кількості об'\-єд\-на\-них процесорів збільшується ризик появи конфліктів, а~також ускладнюється процес під'\-єд\-на\-ння процесорів до~пам'яті. 
		
		\subsection{Визначте компонент сучасних персональних комп'ютерів, який~споживає найбільше енергії}
			% \paragraph{Завдання}
			% 	Визначте компонент сучасних персональних комп'ютерів, який споживає найбільше енергії. 

			Розглянемо споживання енергії на~прикладі компонентів сучасних персональних комп'\-ю\-те\-рів~(табл.~\ref{tab:pc-parts-power-consumption}). Оскільки блок живлення забезпечує електроенергією інші компоненти ком\-п'\-ю\-те\-ра, вважати його споживачем енергії в~контексті персонального ком\-п'\-ю\-те\-ра недоцільно. Тим не менш зазначимо, що~типові сучасні блоки живлення можуть надавати від~300 до~2000~\si{\watt} енергії. Це~означає, що~такі блоки живлення споживатимуть від~375 до~2500~\si{\watt}, оскільки більшість сучасних блоків живлення для~персональних комп'ютерів сертифіковані за~стандартом~\textenglish{80 PLUS}, який вимагає, щоб~коефіцієнт корисної дії при~навантаженні~50\% був більшим за~\num{0.8}. 

			\begin{table}[!htbp]
				\centering
				\caption{Типове споживання енергії компонентами персональних ком\-п'\-ю\-те\-рів~\cites{buildcomputers-overall-power-consumption}{thw-r9-fury-x}{thw-gtx-980-ti}{thw-rx-580}{thw-gtx-1060}{anandtech-tr-2970wx-2920x}{anandtech-tr-2990wx-2950wx}{anandtech-ryzen-2-tested}}
				\label{tab:pc-parts-power-consumption}
				\begin{tabular}{
					v{6\gridunitwidth - 2\tabcolsep}
					n{6\gridunitwidth - 2\tabcolsep}
				}
					\toprule
						Компонент & Приблизне споживання енергії, \si{\watt} \\
					\midrule
						Блок живлення & — \\
						Материнська плата & 25–80 \\
						Центральний процесор & 25–250 \\
						Відеокарта & ${\leqslant} 500$ \\
						Оперативна пам'ять & 2–6 \\
						Жорсткий диск & 1–10 \\
						Твердотільний накопичувач~(\textenglish{\allcaps{SSD}}) & \num{0.025}–6\\
						Оптичний дисковод & 15–30 \\
						Повітряне охолодження & 1–6 \\
					\bottomrule
				\end{tabular}
			\end{table}

			Як~видно з~приблизних даних, найбільшу кількість енергії споживають процесор та~відеокарта. Розглянемо точніші значення для~цих компонентів~(табл.~\ref{tab:cpu-gpu-power-consumption}). 

			\begin{table}[!htbp]
				\centering
				\caption{Споживання енергії деякими процесорами та~відеокартами персональних комп'ютерів}
				\label{tab:cpu-gpu-power-consumption}
				\begin{tabular}{
						v{3\gridunitwidth - 2\tabcolsep}
						v{4\gridunitwidth - 2\tabcolsep}
						n{2.5\gridunitwidth - 2\tabcolsep}
						n{2.5\gridunitwidth - 2\tabcolsep}
				}
					\toprule
						Компонент & Модель & \multicolumn{2}{n{5\gridunitwidth - 2\tabcolsep}}{Макс. споживання енергії, \si{\watt}} \\
						\cmidrule(lr){3-4}
						& & У~простої & Під~навантаженням \\
					\midrule
						Відеокарта & \textenglish{AMD Radeon~R9 Fury~X} & \num{21.96}~\cite{thw-r9-fury-x} & \num{453.58}~\cite{thw-r9-fury-x}\\
											 & \textenglish{Nvidia 980~Ti} & \num{38.33}~\cite{thw-gtx-980-ti} & \num{428.38}~\cite{thw-gtx-980-ti}\\
											 & \textenglish{Sapphire RX~580 Nitro+} & \num{16.30}~\cite{thw-rx-580} & \num{237.40}~\cite{thw-rx-580}\\
											 & \textenglish{Nvidia GTX~1060} & \num{9.00}~\cite{thw-gtx-1060} & \num{122.00}~\cite{thw-gtx-1060}\\
						%
						Процесор & \textenglish{AMD TR2 2990WX} & \num{50.15}~\cite{anandtech-tr-2990wx-2950wx} & \num{248.51}~\cite{anandtech-tr-2970wx-2920x}\\
																 & \textenglish{Intel Core i9-7980XE} & \num{32.21}~\cite{anandtech-tr-2990wx-2950wx} & \num{182.69}~\cite{anandtech-tr-2970wx-2920x}\\
																 & \textenglish{AMD Ryzen 2600X} & & \num{107.41}~\cite{anandtech-ryzen-2-tested}\\
																 & \textenglish{Intel Core i5-8600K} & & \num{107.41}~\cite{anandtech-ryzen-2-tested}\\
					\bottomrule
				\end{tabular}
			\end{table}

			Видно, що~найбільше енергії споживають пристрої верхнього сегменту: процесори~\textenglish{AMD TR~2990WX} і~\textenglish{Intel Core i9-7990XE} та~відеокарти~\textenglish{AMD Radeon~R9 Fury~X} і~її конкурентка~\textenglish{Nvidia 980~Ti}. Серед цих пристроїв показники максимального споживання енергії під~навантаженням відеокарт перевищують показники процесорів більш ніж на~200~\si{\watt}, тому робимо висновок, що~компонентом персонального комп'ютера, який споживає найбільшу кількість енергії, є~відеокарта. 

	\printbibliography

\end{document}

