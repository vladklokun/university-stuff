\documentclass[
	a4paper,
	oneside,
	BCOR = 10mm,
	DIV = 12,
	12pt,
	headings = normal,
]{scrartcl}

%%% Length calculations
\usepackage{calc}
%%%

%%% Support for color
\usepackage{xcolor}
\definecolor{lightblue}{HTML}{03A9F4}
\definecolor{red}{HTML}{F44336}
%%%

%%% Including graphics
\usepackage{graphicx}
%%%

%%% Font selection
\usepackage{fontspec}

\setromanfont{STIX Two Text}[
	SmallCapsFeatures = {LetterSpace = 8},
]

\setsansfont{IBM Plex Sans}[
	Scale = MatchUppercase,
]

\setmonofont{IBM Plex Mono}[
	Scale = MatchUppercase,
]
%%%

%%% Math typesetting
\usepackage{amsmath}

\usepackage{unicode-math}
\setmathfont{STIX Two Math}

\usepackage{IEEEtrantools}
%%%

%%% List settings
\usepackage{enumitem}
\setlist[enumerate]{
	label*      = {\arabic*.},
	leftmargin  = *,
	labelindent = \parindent,
	topsep      = 1\baselineskip,
	parsep      = 0\baselineskip,
	itemsep     = 1\baselineskip,
	noitemsep, % override itemsep
}

\setlist[itemize]{
	label*      = {—},
	leftmargin  = *,
	labelindent = \parindent,
	topsep      = 1\baselineskip,
	parsep      = 0\baselineskip,
	itemsep     = 1\baselineskip,
	noitemsep, % override itemsep
}

\setlist[description]{
	font        = {\rmfamily\upshape\bfseries},
	topsep      = 1\baselineskip,
	parsep      = 0\baselineskip,
	itemsep     = 0\baselineskip,
}

%%%

%%% Structural elements typesetting
\setkomafont{pagenumber}{\rmfamily\upshape}
\setkomafont{disposition}{\rmfamily\bfseries}

% Sectioning
\RedeclareSectionCommand[
	beforeskip = -1\baselineskip,
	afterskip  = 1\baselineskip,
	font       = {\normalsize\bfseries\scshape},
]{section}

\RedeclareSectionCommand[
	beforeskip = -1\baselineskip,
	afterskip  = 1\baselineskip,
	font       = {\normalsize\bfseries\itshape},
]{subsection}

\RedeclareSectionCommand[
	beforeskip = -1\baselineskip,
	afterskip  = 1\baselineskip,
	font       = {\normalsize\bfseries},
]{subsubsection}

\RedeclareSectionCommand[
	beforeskip = -1\baselineskip,
	afterskip  = -0.5em,
	font       = {\normalsize\mdseries\scshape\addfontfeatures{Letters = {UppercaseSmallCaps}}},
]{paragraph}
%%%

%%% Typographic enhancements
\usepackage{microtype}
%%%

%%% Language-specific settings
\usepackage{polyglossia}
\setmainlanguage{ukrainian}
\setotherlanguages{english}
%%%

%%% Captions
\usepackage{caption}
\usepackage{subcaption}

%\DeclareCaptionLabelFormat{closing}{#2)}
%\captionsetup[subtable]{labelformat = closing}

%\captionsetup[subfigure]{labelformat = closing}

\captionsetup[table]{
	aboveskip = 0\baselineskip,
	belowskip = 0\baselineskip,
}

\captionsetup[figure]{
	aboveskip = 1\baselineskip,
	belowskip = 0\baselineskip,
}

\captionsetup[subfigure]{
	labelformat = simple,
	labelformat = brace,
}
%%%

%%% Hyphenated ragged typesetting
\usepackage{ragged2e}
%%%

%%% Table typesetting
\usepackage{booktabs}
\usepackage{longtable}

\usepackage{multirow}

\usepackage{array}
\newcolumntype{v}[1]{>{\RaggedRight\arraybackslash\hspace{0pt}}p{#1}}
\newcolumntype{b}[1]{>{\Centering\arraybackslash\hspace{0pt}}p{#1}}
\newcolumntype{n}[1]{>{\RaggedLeft\arraybackslash\hspace{0pt}}p{#1}}
%%%

%%% Drawing
\usepackage{tikz}
\usepackage{tikzscale}
\usetikzlibrary{positioning}
\usetikzlibrary{arrows.meta} % Stealth arrow tips
%%%

%%% SI units typesetting
\usepackage{siunitx}
\sisetup{
	output-decimal-marker = {,},
	exponent-product      = {\cdot},
	inter-unit-product    = \ensuremath{{} \cdot {}},
	per-mode              = symbol,
}
%%%

%%% Framing code listings
\usepackage{tcolorbox}
\tcbuselibrary{breakable}
\tcbuselibrary{minted}
\tcbuselibrary{skins}

\newtcblisting[
	auto counter, 
	list inside, 
	number within = section,
]{listingpython}[3][]{%
	minted language = python,
	minted style    = bw,
	minted options  = {
		linenos,
		tabsize = 4,
		breaklines,
		% breakanywhere,
		fontsize = \footnotesize,
		autogobble
	},
	%
	% empty,
	sharp corners,
	colframe         = black,
	colback          = black!0,
	leftrule         = 0em,
	rightrule        = 0em,
	toprule          = 1pt, % orig = 0pt
	bottomrule       = 1pt, % orig = 0pt
	titlerule        = 0.5pt,
	colbacktitle     = black!0,
	coltitle         = black,
	toptitle         = 0.3em,
	bottomtitle      = 0.1em,
	borderline north = {1pt}{0pt}{black},
	borderline south = {1pt}{0pt}{black},
	before skip      = \intextsep,
	after  skip      = \intextsep,
	title            = {Лістинг \thetcbcounter: #2},
	list entry       = {\protect\numberline{\thetcbcounter}#2},
	left = 0em,
	right = 0em,
	%
	listing only,
	breakable,
	%
	label = {#3},
	%
	#1
}

\newtcbinputlisting[auto counter, list inside, number within = section]{\inputpython}[4][]{%
	minted language = python,
	minted style    = bw,
	minted options  = {
		linenos,
		tabsize = 4,
		breaklines,
		breakbytokenanywhere,
		fontsize = \footnotesize,
	},
	%
	% empty,
	sharp corners,
	colframe         = black,
	colback          = black!0,
	leftrule         = 0em,
	rightrule        = 0em,
	toprule          = 0pt, % orig = 0pt
	bottomrule       = 0pt, % orig = 0pt
	titlerule        = 0.5pt,
	colbacktitle     = black!0,
	coltitle         = black,
	toptitle         = 0.3em,
	bottomtitle      = 0.1em,
	borderline north = {1pt}{0pt}{black},
	borderline south = {1pt}{0pt}{black},
	before skip      = \intextsep,
	after  skip      = \intextsep,
	title            = {Лістинг \thetcbcounter: #3},
	list entry       = {\protect\numberline{\thetcbcounter}#3},
	left = 0em,
	right = 0em,
	%
	listing file={#2},
	listing only,
	breakable,
	%
	label = {#4},
	%
	#1
}

% Customize minted
\usepackage{minted}
\setmintedinline{
	style = bw,
	breaklines,
}

% Customize minted line numbers
\renewcommand{\theFancyVerbLine}{\ttfamily\scriptsize\arabic{FancyVerbLine}}

%%%

%%% Links and hyperreferences
\usepackage{hyperref}
\hypersetup{
	bookmarksnumbered = true,
	colorlinks      = false,
	linkbordercolor = red,
	urlbordercolor  = lightblue,
	pdfborderstyle  = {/S/U/W 1.5},
}
%%%

%%% Length adjustments
% Set baselineskip, default is 14.5 pt
\linespread{1.068966} % ~15.5 pt
\setlength{\emergencystretch}{1em}
\setlength{\parindent}{1.5em}
\newlength{\gridunitwidth}
\setlength{\gridunitwidth}{\textwidth / 12}
%%%

%%% Custom commands
\newcommand{\allcaps}[1]{{\addfontfeatures{LetterSpace = 8, Kerning = Off}#1}}
\newcommand{\filename}[1]{\texttt{#1}}
\newcommand{\progname}[1]{\texttt{#1}}
\newcommand{\modulename}[1]{\texttt{#1}}

\newcommand{\schel}[1]{\textit{#1}}
%%%

%%% Custom math commands
\newcommand{\longvar}[1]{\mathit{#1}}
%%%

\begin{document}

\begin{titlepage}
		\begin{center}
			Міністерство освіти і науки України\\
			Національний авіаційний університет\\
			Навчально-науковий інститут комп'ютерних інформаційних технологій\\
			Кафедра комп'ютеризованих систем управління

			\vspace{\fill}
				Лабораторна робота №2\\
				з~дисципліни «Діагностика та~експлуатація комп'ютера»\\
				на~тему «Діагностика оперативної пам'яті комп'ютера»\\

			\vspace{\fill}

			\begin{flushright}
				Виконав:\\
				студент \allcaps{ННІКІТ}\\
				групи СП-325\\
				Клокун В.\,Д.\\
				Перевірила:\\
				Голего Н.\,М.
			\end{flushright}

			Київ 2019
		\end{center}
	\end{titlepage}

	\section{Мета роботи}
		Ознайомлення з~методами виявлення несправностей оперативної пам'\-я\-ті ком\-п'\-ют\-е\-ра. 

	\section{Хід роботи}
		\subsection{Перевірка пам'яті за допомогою програми~\textenglish{Cache Burst}}
			Перевіряємо пам'ять за допомогою програми~\textenglish{Cache Burst}. Для цього запускаємо програму і~встановлюємо такі налаштування: 
			\begin{enumerate}
				\item \textenglish{Select 32-bit Transfer}.
					\begin{enumerate}[topsep = 0em]
						\item \textenglish{Read}.
						\item \textenglish{Write}.
					\end{enumerate}
				\item \textenglish{Select 64-bit Transfer}.
					\begin{enumerate}[topsep = 0em]
						\item \textenglish{MMX Read}.
						\item \textenglish{MMX Write}.
					\end{enumerate}
				\item \textenglish{Select 128-bit Transfer}.
					\begin{enumerate}[topsep = 0em]
						\item \textenglish{SSE Read}.
						\item \textenglish{SSE Write}.
					\end{enumerate}
			\end{enumerate}
			Коли налаштування встановлені, натискаємо кнопку~«\textenglish{Start}» і~спостерігаємо за~процесом проходження тесту~(рис.~\ref{fig:cburst-32-test}). 

			\begin{figure}[!htbp]
				\centering
				\includegraphics[height = 10\baselineskip]{./assets/y03s02-pcdiag-lab-02-p01-01.png}
				\caption{Результат тестування програмою~\textenglish{Cache Burst 32}}
				\label{fig:cburst-32-test}
			\end{figure}

			В~результаті проходження тесту отримали графік пропускної здатності пам'яті для~різних видів доступу до~кеш-пам'яті процесора. 

		\subsection{Перевірка пам'яті за~допомогою програми~\textenglish{Memory Benchmark v0.1.4}}
			Перевіряємо пам'ять за допомогою програми~\textenglish{Memory Benchmark v0.1.4}. Для~цього запускаємо програму, яка автоматично після запуску розпочне тестування пам'яті. Після завершення тестів отримали результати у~вигляді графіків~(рис.~\ref{fig:membench-test}). 

			\begin{figure}[!htbp]
				\centering
				\begin{subfigure}[t]{\columnwidth / 3}
					\centering
					\includegraphics[width = \columnwidth]{./assets/y03s02-pcdiag-lab-02-p02-01.png}
					\caption{}
					\label{subfig:membench-test-01}
				\end{subfigure}%
				\begin{subfigure}[t]{\columnwidth / 3}
					\centering
					\includegraphics[width = \columnwidth]{./assets/y03s02-pcdiag-lab-02-p02-02.png}
					\caption{}
					\label{subfig:membench-test-02}
				\end{subfigure}%
				\begin{subfigure}[t]{\columnwidth / 3}
					\centering
					\includegraphics[width = \columnwidth]{./assets/y03s02-pcdiag-lab-02-p02-03.png}
					\caption{}
					\label{subfig:membench-test-03}
				\end{subfigure}%
				\caption{Результат тестування програмою~\textenglish{Cache Burst 32}: \subref{subfig:membench-test-01}~— графік пропускної здатності кеш-пам'\-я\-ті процесора, \subref{subfig:membench-test-02}~— графік швидкості зчитування з~оперативної пам'\-я\-ті, \subref{subfig:membench-test-01}~— графік швидкості запису в~оперативну пам'\-ять}
				\label{fig:membench-test}
			\end{figure}

			В~результаті проходження тесту отримали графіки пропускної здатності пам'\-я\-ті для~різних видів доступу до~кеш- і~оперативної пам'\-я\-ті. Як~бачимо, дані незначно відрізняються від~результатів тестування програмою~\textenglish{Cache Burst}, однак загалом графіки ведуть себе однаково. 

		\subsection{Перевірка пам'яті за~допомогою програми~\textenglish{Memtest86+}}
			
			Перевіряємо пам'ять за допомогою програми~\textenglish{Memtest86+}. Для~перевірки пам'\-я\-ті програма~\textenglish{Memtest86+} використовує дуже низький рівень абстракції від~обладнання, тому розповсюджується у~вигляді образів завантажувальних носіїв. 
			
			Щоб виконати перевірку пам'яті програмою~\textenglish{Memtest86+}, необхідно завантажити тестовий комп'ютер у~її~середовище. Після завантаження через декілька секунд програма~\textenglish{Memtest86+} автоматично розпочне тестування пам'яті. Тестування відбувається у~вигляді проходів. Кожен прохід складається з~декількох тестів на запис певних даних у~пам'ять, їх зчитування з~неї та порівняння результатів. Для виконання завдання даної лабораторної роботи знадобиться 1~прохід. Після завершення першого проходу тесту отримали результати у~вигляді текстового звіту: «\textenglish{Pass complete, no errors, press Esc to exit}»~(рис.~\ref{fig:membench-test}). 

			\begin{figure}[!htbp]
				\centering
				\includegraphics[height = 10\baselineskip]{./assets/y03s02-pcdiag-lab-02-p03-01-bw.png}
				\caption{Результат тестування програмою~\textenglish{Memtest86+}}
				\label{fig:memtest86-test}
			\end{figure}

			Таке повідомлення інформує, що~перший прохід тесту успішно завершився і~помилки оперативної пам'яті не~були виявлені. 

	\section{Висновок}
		Виконуючи дану лабораторну роботу, ми ознайомились з~методами виявлення несправностей оперативної пам'\-я\-ті ком\-п'\-ю\-те\-ра.

\end{document}

