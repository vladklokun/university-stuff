\documentclass[
	a4paper,
	oneside,
	BCOR = 10mm,
	DIV = 12,
	12pt,
	headings = normal,
]{scrartcl}

%%% Length calculations
\usepackage{calc}
%%%

%%% Support for color
\usepackage{xcolor}
\definecolor{lightblue}{HTML}{03A9F4}
\definecolor{red}{HTML}{F44336}
%%%

%%% Including graphics
\usepackage{graphicx}
%%%

%%% Font selection
\usepackage{fontspec}

\setromanfont{STIX Two Text}[
	SmallCapsFeatures = {LetterSpace = 8},
]

\setsansfont{IBM Plex Sans}[
	Scale = MatchUppercase,
]

\setmonofont{IBM Plex Mono}[
	Scale = MatchUppercase,
]
%%%

%%% Math typesetting
\usepackage{amsmath}

\usepackage{unicode-math}
\setmathfont{STIX Two Math}

\usepackage{IEEEtrantools}
%%%

%%% List settings
\usepackage{enumitem}
\setlist[enumerate]{
	label*      = {\arabic*.},
	leftmargin  = *,
	labelindent = \parindent,
	topsep      = 1\baselineskip,
	parsep      = 0\baselineskip,
	itemsep     = 1\baselineskip,
	noitemsep, % override itemsep
}

\setlist[itemize]{
	label*      = {—},
	leftmargin  = *,
	labelindent = \parindent,
	topsep      = 1\baselineskip,
	parsep      = 0\baselineskip,
	itemsep     = 1\baselineskip,
	noitemsep, % override itemsep
}

\setlist[description]{
	font        = {\rmfamily\upshape\bfseries},
	topsep      = 1\baselineskip,
	parsep      = 0\baselineskip,
	itemsep     = 0\baselineskip,
}

%%%

%%% Structural elements typesetting
\setkomafont{pagenumber}{\rmfamily\upshape}
\setkomafont{disposition}{\rmfamily\bfseries}

% Sectioning
\RedeclareSectionCommand[
	beforeskip = -1\baselineskip,
	afterskip  = 1\baselineskip,
	font       = {\normalsize\bfseries\scshape},
]{section}

\RedeclareSectionCommand[
	beforeskip = -1\baselineskip,
	afterskip  = 1\baselineskip,
	font       = {\normalsize\bfseries\itshape},
]{subsection}

\RedeclareSectionCommand[
	beforeskip = -1\baselineskip,
	afterskip  = 1\baselineskip,
	font       = {\normalsize\bfseries},
]{subsubsection}

\RedeclareSectionCommand[
	beforeskip = -1\baselineskip,
	afterskip  = -0.5em,
	font       = {\normalsize\mdseries\scshape\addfontfeatures{Letters = {UppercaseSmallCaps}}},
]{paragraph}
%%%

%%% Typographic enhancements
\usepackage{microtype}
%%%

%%% Language-specific settings
\usepackage{polyglossia}
\setmainlanguage{ukrainian}
\setotherlanguages{english}
%%%

%%% Captions
\usepackage{caption}
\usepackage{subcaption}

%\DeclareCaptionLabelFormat{closing}{#2)}
%\captionsetup[subtable]{labelformat = closing}

%\captionsetup[subfigure]{labelformat = closing}

\captionsetup[table]{
	aboveskip = 0\baselineskip,
	belowskip = 0\baselineskip,
}

\captionsetup[figure]{
	aboveskip = 1\baselineskip,
	belowskip = 0\baselineskip,
}

\captionsetup[subfigure]{
	labelformat = simple,
	labelformat = brace,
}
%%%

%%% Hyphenated ragged typesetting
\usepackage{ragged2e}
%%%

%%% Table typesetting
\usepackage{booktabs}
\usepackage{longtable}

\usepackage{multirow}

\usepackage{array}
\newcolumntype{v}[1]{>{\RaggedRight\arraybackslash\hspace{0pt}}p{#1}}
\newcolumntype{b}[1]{>{\Centering\arraybackslash\hspace{0pt}}p{#1}}
\newcolumntype{n}[1]{>{\RaggedLeft\arraybackslash\hspace{0pt}}p{#1}}
%%%

%%% Drawing
\usepackage{tikz}
\usepackage{tikzscale}
\usetikzlibrary{positioning}
\usetikzlibrary{arrows.meta} % Stealth arrow tips
\usetikzlibrary{shapes.geometric} % Stealth arrow tips
%%%

%%% SI units typesetting
\usepackage{siunitx}
\sisetup{
	output-decimal-marker = {,},
	exponent-product      = {\cdot},
	inter-unit-product    = \ensuremath{{} \cdot {}},
	per-mode              = symbol,
}
%%%

%%% Framing code listings
\usepackage{tcolorbox}
\tcbuselibrary{breakable}
\tcbuselibrary{minted}
\tcbuselibrary{skins}

\newtcblisting[
	auto counter, 
	list inside, 
	number within = section,
]{listingpython}[3][]{%
	minted language = python,
	minted style    = bw,
	minted options  = {
		linenos,
		tabsize = 4,
		breaklines,
		% breakanywhere,
		fontsize = \footnotesize,
		autogobble
	},
	%
	% empty,
	sharp corners,
	colframe         = black,
	colback          = black!0,
	leftrule         = 0em,
	rightrule        = 0em,
	toprule          = 1pt, % orig = 0pt
	bottomrule       = 1pt, % orig = 0pt
	titlerule        = 0.5pt,
	colbacktitle     = black!0,
	coltitle         = black,
	toptitle         = 0.3em,
	bottomtitle      = 0.1em,
	borderline north = {1pt}{0pt}{black},
	borderline south = {1pt}{0pt}{black},
	before skip      = \intextsep,
	after  skip      = \intextsep,
	title            = {Лістинг \thetcbcounter: #2},
	list entry       = {\protect\numberline{\thetcbcounter}#2},
	left = 0em,
	right = 0em,
	%
	listing only,
	breakable,
	%
	label = {#3},
	%
	#1
}

\newtcbinputlisting[auto counter, list inside, number within = section]{\inputpython}[4][]{%
	minted language = python,
	minted style    = bw,
	minted options  = {
		linenos,
		tabsize = 4,
		breaklines,
		breakbytokenanywhere,
		fontsize = \footnotesize,
	},
	%
	% empty,
	sharp corners,
	colframe         = black,
	colback          = black!0,
	leftrule         = 0em,
	rightrule        = 0em,
	toprule          = 0pt, % orig = 0pt
	bottomrule       = 0pt, % orig = 0pt
	titlerule        = 0.5pt,
	colbacktitle     = black!0,
	coltitle         = black,
	toptitle         = 0.3em,
	bottomtitle      = 0.1em,
	borderline north = {1pt}{0pt}{black},
	borderline south = {1pt}{0pt}{black},
	before skip      = \intextsep,
	after  skip      = \intextsep,
	title            = {Лістинг \thetcbcounter: #3},
	list entry       = {\protect\numberline{\thetcbcounter}#3},
	left = 0em,
	right = 0em,
	%
	listing file={#2},
	listing only,
	breakable,
	%
	label = {#4},
	%
	#1
}

% Customize minted
\usepackage{minted}
\setmintedinline{
	style = bw,
	breaklines,
}

\newmintinline[proinline]{prolog}{%
	style = bw,
	breaklines,
}

\newminted[prologcode]{prolog}{%
	style = bw,
	breaklines,
	linenos,
	autogobble,
	tabsize = 2,
}

% Customize minted line numbers
\renewcommand{\theFancyVerbLine}{\ttfamily\scriptsize\arabic{FancyVerbLine}}

%%%

%%% Date and time typesetting
\usepackage[
	ukrainian,
	useregional,
	showdow,
]{datetime2}
%%%

%%% Links and hyperreferences
\usepackage{hyperref}
\hypersetup{
	bookmarksnumbered = true,
	colorlinks      = false,
	linkbordercolor = red,
	urlbordercolor  = lightblue,
	pdfborderstyle  = {/S/U/W 1.5},
}
%%%

%%% Length adjustments
% Set baselineskip, default is 14.5 pt
\linespread{1.068966} % ~15.5 pt
\setlength{\emergencystretch}{1em}
\setlength{\parindent}{1.5em}
\newlength{\gridunitwidth}
\setlength{\gridunitwidth}{\textwidth / 12}
% \setlength{\RaggedRightParindent}{\parindent}
%%%

%%% Custom commands
\newcommand{\allcaps}[1]{{\addfontfeatures{LetterSpace = 8, Kerning = Off}#1}}
\newcommand{\filename}[1]{\texttt{#1}}
\newcommand{\progname}[1]{\texttt{#1}}
\newcommand{\modulename}[1]{\texttt{#1}}
%%%

%%% Custom math commands
\newcommand{\longvar}[1]{\mathit{#1}}
%%%

\begin{document}

% \RaggedRight

\begin{titlepage}
		\begin{center}
			Міністерство освіти і~науки України\\
			Національний авіаційний університет\\
			Навчально-науковий інститут комп'ютерних інформаційних технологій\\
			Кафедра комп'ютеризованих систем управління

			\vspace{\fill}
				Конспект\\
				з~дисципліни «Функціональне і логічне програмування»\\

			\vspace{\fill}

			\begin{flushright}
				Виконав:\\
				студент \allcaps{ННІКІТ}\\
				групи СП-325\\
				Клокун В.\,Д.\\
			\end{flushright}

			Київ 2019
		\end{center}
	\end{titlepage}

	\section{\DTMDate{2019-02-08}}
		В логічній програмі алгоритм розв'язання задачі не~описується. Описується умова задачі, а~саме вказується, які є~об'єкти предметної області, які відношення між ними, а~також описується мета, якої необхідно досягти. Основою опису є~відношення між об'єктами. 

		Логічна програма складається з~двох типів висловлювань:
		\begin{enumerate}
			\item Факти, які описують властивості об'єктів. 
			\item Правила, які описують відношення між об'єктами. 
		\end{enumerate}

		Використовуючи факти і~правила, наведені в~програмі, можна отримати або~вивести з~наявної інформації нові факти або~правила. Крім цього, в~логічній програмі вказується мета пошуку. Тобто логічна програма є~базою даних про~умову задачі. Розв'язання задачі полягає у~пошуку способу задоволення заданої мети на~основі наявних фактів і~правил. Алгоритм цього пошуку реалізований в~інтерпретаторі або~компіляторі мови програмування. Найбільш поширеною мовою логічного програмування є~Prolog, у~якого є~численні спадкоємці. 

		Функційне програмування. Єдина управляюча конструкція в~мові функційного програмування — це~виклик функції. У функційній мові задана деяка множина базових функцій, а~решта функцій будується з~них за допомогою композиції. Наприклад, ми можемо мати $\max (a, b)$, тоді $\max (a, b, c) = \max (a, \max(b, c) )$. Найбільш відомі мови функційного програмування: \textenglish{Lisp}, \textenglish{Haskell}, \textenglish{Scheme}, \textenglish{Standard~ML}. 

		Концептуальне програмування. Це автоматизований синтез програм на~основі формальних специфікацій задачі. Специфікація задачі — це~її формальний опис: блок-схема, словесний опис, словесний опис і~так далі. 

		Імперативне і~декларативне програмування. 

	\section{\DTMDate{2019-02-11}}
		\subsection{Мова Prolog. Терм}
			Терм~— це~об'єкт даних, може бути константою змінною або~структурою (складним термом). Константи в~Пролозі поділяються на~2 види:
			\begin{enumerate}
				\item Константи-числа, які позначають числові значення. 
				\item Константи-атоми, які позначають елементарні об'єкти предметної області. 
			\end{enumerate}

			Числа бувають цілі і~дійсні (\textenglish{integer}, \textenglish{real}). 

			Атом — будь-яка послідовність символів, взятих у~подвійні лапки — «\proinline{""}». В більшості випадків атом іменує об'єкт предметної області. Наприклад, атом може позначати людину, тварину або~предмет. Якщо за контекстом програми атом можна відрізнити від інших об'єктів, наприклад, від імен змінних, то~лапки можна не~використовувати. Але це~накладає свої обмеження: не~можно ставити пробіли, спеціальні символи. Всі інші типи даних в~Пролозі складаються із сполучень констант і~змінних. 

		\subsection{Змінні}
			Змінна в~Пролозі може набувати значення константи, тобто атому або~числа. Імена змінних в~програмі починаються з~великої літери або~символу підкреслення. Якщо ім'я об'єкта в~програмі починається з~великої літери, вважається, що~це змінна. \proinline{X} — змінна, \proinline{x} — атом, незмінна величина. 

			В лапки можна включати спеціальні символи за допомогою «\textbackslash». Існує спеціальна змінна, яка~називається анонімною і~позначається символом підкреслення~«\proinline{_}». Вона успішно зіставляється з~будь-яким значенням, але~не пов'язується з~ним, тобто отримати з~неї будь-яке значення неможливо. Її використовують в~програмах для заповнення місця. В тому разі, якщо це~індекси в~програмі допомогає вказати аргумент, але~значення, з~яким пов'язаний аргумент, неважливе для програміста. 

		\subsection{Складні терми (структури)}
			Запис структур в~програмах на~Пролозі виглядає подібно до записів у~мовах C~або~Pascal. В Пролозі кажуть, що~структура складається з~головного функтора та компонентів. Головним функтором називається ім'я функції (атом), а~компонентами — те, що~стоїть в~дужках. Компоненти структури наводять в~дужках і~через кому: \proinline{dog(rex), father(ivan, anton)}. 

			Арність — кількість компонентів структури. \proinline{dog/1, father/2} або~\proinline{father(i, o)} — позначення арності, де \proinline{i} — input, \proinline{o} — output. 

			В Пролозі використовується 2 види коментарів:
			\begin{enumerate}
				\item Блочні — \proinline{/* comment text */}.
				\item Строчні — \proinline{% comment text}.
			\end{enumerate}

		\subsection{Розділи програми} 
			Програма на~Пролозі може включати такі розділи: 
			\begin{enumerate}
				\item Domains — оголошуються і~описуються домени — типи, визначені програмістом. Тут треба оголошувати структури.
					\begin{prologcode}
					domains
					\end{prologcode}
				\item Predicates. 
					\begin{prologcode}
					predicates
						dog(symbol)
						father(symbol, symbol)
					\end{prologcode}
				\item  Clauses. Тут розміщують факти і~правила, що~описують зміст програми. 
				\item  Goal. Ціль програми. 
				\item  Database — база знань, динамічні типи даних. 
			\end{enumerate}

		Програма на~Пролозі — це~сукупність тверджень або~висловлювань (речень, клаузів), які за своєю суттю близькі до алгебри логіки. Сукупність тверджень утворює базу даних (базу знань) програми. Це статична база даних, тобто вона не~змінюється під час роботи програми. Також може використовуватись динамічна база даних, твердження якої можуть доповнюватись або~видалятись з~неї під час роботи програми. 

		Твердження в~розділі clauses бувають 2 типів:
		\begin{enumerate}
			\item Факти. 
			\item Правила. 
		\end{enumerate}

		Кожне твердження (факт або~правило) закінчується крапкою. В ході виконання (або розгляду) програми інтерпретатор Прологу виконує доведення цілей, які входять у~ці твердження. 

		Факт — це~одиночна ціль, яка~вважається за визначенням істинною. Він є~описом заданих відомостей про~об'єкти предметної області. Наприклад:
		\begin{prologcode}
			dog(rex).
			father(ivan, petro).
		\end{prologcode}
		Правила дозволяють вивести з~наявних фактів деякий інший факт. Правила складаються із голови і~хвоста. Наприклад:
		\begin{prologcode}
		animal(X) if dog(X).

		animal(X) :- dog(X).
		\end{prologcode}

		Голова правила істинна, якщо істинні всі цілі, вказані у~хвості. 

		Змінні у~Пролозі не~оголошуються. Відповідно тип змінної в~програмі на~Пролозі ніяк не~вказується. Більш того, одна і~та ж змінна в~різних реченнях може набувати значення різних типів. Але про~тип змінної, використаної в~конкретному випадку, можна дізнатись із розділу \proinline{predicates}, де вказується тип компоненту структури. 

		В Пролозі головними сутностями вважаються не~об'єкти предметної області, а~відношення між ними. Об'єкти, які вступають в~ці відношення, розглядаються як другорядні сутності, порівняно з~самими відношеннями. З цього підходу випливають такі наслідки:
		\begin{enumerate}
			\item Типи аргументів приписують не~до змінних, які представляють об'єкти, а~до відношень, в~яких ці об'єкти використовуються. 
			\item Замість секції оголошення змінних, яка~входить до імперативної програми, в~Пролог-програмі присутня секція оголошень відношень — \proinline{predicates} 
			\item Областю дії змінної є~одне речення. В рамках одного речення одне і~те ж ім'я змінної означає одну і~ту ж змінну. Однак, в~іншому твердженні змінна з~тим же ім'ям (\proinline{X}) може мати інший сенс і~навіть містити значення іншого типу. 
		\end{enumerate}

		Написання програми на~Пролозі полягає в~тому, що~програміст описує об'єкти предметної області, описує відношення між ними і~визначає правила, які описують умови, за яких ці відношення дійсні. 

		Співставлення. Це процес, на~вхід якого подаємо 2 терми. Результатом співставлення є~висновок, що~ці 2 терми співставимі, тобто відповідні один одному або~неспівставимі, тобто невідповідні. Якщо виявилось, що~терми співставимі, то~кажуть, що~співставлення завершилось успіхом. Якщо неспівставимі, то~кажуть, що~процес співставлення завершилось невдачею — \proinline{fail}. 

		Правила співставлення: 
		\begin{enumerate}
			\item Якщо 2 об'єкти — константи або~конкретизовані змінні, то~вони співставимі, якщо їх значення рівні. 
			\item Якщо один із об'єктів не~конкретизована змінна, то~співставлення можливе. При цьому якщо другий об'єкт — константа або~конкретизована змінна, то~першу неконкретизовану змінну конкретизує значення другого об'єкта. Якщо другий об'єкт теж неконкретизована змінна, то~2 змінні пов'язуються між собою, тимчасово ототожнюються. Після цього, якщо одна із них буде конкретизована певним значенням, то~таку ж конкретизацію отримає друга пов'язана з~нею змінна. 
			\item Якщо 2 об'єкти — структури, то~вони співставимі, якщо вони, по-перше, мають однакові головні функтори, і, по-друге, всі їх компоненти співставимі. 
		\end{enumerate}

\end{document}

