\documentclass[a4paper,oneside,DIV=12,12pt]{scrartcl}

\usepackage{fontspec}
\setmainfont{STIX Two Text}
\setsansfont{Roboto}
\setmonofont{PT Mono}

\usepackage{microtype}

\usepackage{polyglossia}
\setmainlanguage{ukrainian}

\usepackage{amsmath}
\usepackage{unicode-math}
\setmathfont{STIX Two Math}

\usepackage{siunitx}
\sisetup{output-decimal-marker = {,},
exponent-product = {\cdot}}

% Problem-solution typesetting
\usepackage{xsim}
\loadxsimstyle{runin}
\DeclareExerciseTranslations{exercise}{
	Ukrainian	=	завдання ,
}

\DeclareExerciseTranslations{solution}{
	Ukrainian	=	розв'язання ,
}

\xsimsetup{
	solution/print = true,
	exercise/template = runin,
	solution/template = runin,
}

\begin{document}
	\begin{exercise}
		Ефект Комптона полягає:
	\end{exercise}
	\begin{solution}
		У розсіюванні атомами речовини випромінювання і збільшенні довжини хвилі розсіяного випромінювання.
	\end{solution}
	
	\begin{exercise}
		Зміна довжини хвилі при ефекті Комптона визначається так:
	\end{exercise}
	\begin{solution}
		\[
			\Delta \lambda = \lambda' - \lambda = 2 \lambda_{\text{к}} \sin^2 \frac{\theta}{2}.
		\]
	\end{solution}
	
	\begin{exercise}
		Максимальна зміна довжини хвилі фотона при розсіюванні рентгенівських променів на електронах речовини визначається так:
	\end{exercise}
	\begin{solution}
		Коли $\theta = \pi$, і $m_e$~— маса електрона, тоді
		\[
			\Delta \lambda_{\text{max}}
			= 2 \lambda_{\text{к}}
			= 2 \frac{h}{m_e c}
			= \SI{4,86e-12}{\metre}.
		\]
	\end{solution}
	
	\begin{exercise}
		Дописати закони збереження, які виконуються при ефекті Комптона.
	\end{exercise}
	\begin{solution}
		\[
			h \nu + m_0 c^2 = h \nu' + mc^2.
		\]
		\[
			\vec{p}_{\text{ф}} = \vec{p}_{\text{ф}}' + \vec{p}_{\text{е}}.
		\]
	\end{solution}
	
	\begin{exercise}
		Комптонівська довжина хвилі електрона визначається так:
	\end{exercise}
	\begin{solution}
		\[
			\lambda_{\text{к}}
			= \frac{h}{m_e c}
			= \SI{2,42621e-12}{\metre}.
		\]
	\end{solution}
	
	\begin{exercise}
		Максимальна зміна довжини хвилі фотона при розсіюванні рентгенівських променів на протонах визначається так:
	\end{exercise}
	\begin{solution}
		Коли $\theta = \pi$, і $m_p$~— маса протона, тоді
		\[
			\Delta \lambda_{\text{max}}
			= 2 \lambda_{\text{к}}
			= 2 \frac{h}{m_p c}
			= \SI{4,86e-12}{\metre}.
		\]
	\end{solution}
	
	\begin{exercise}
		Комптонівська довжина хвилі протона визначається так:
	\end{exercise}
	\begin{solution}
		\[
			\lambda_{\text{к}}
			= \frac{h}{m_p c}
			= \SI{2,42621e-12}{\metre}.
		\]
	\end{solution}
	
	\begin{exercise}
		На рис. 4 подано квантову схему ефекту Комптона. Показати: частинку, на якій відбувається ефект; напрям руху падаючого фотона.
	\end{exercise}
	\begin{solution}
		Частинка, на якій відбувається ефект — 2. Напрям руху падаючого фотона — 1.
	\end{solution}
	
	\begin{exercise}
		На рис. 4 подано квантову схему ефекту Комптона. Показати: кут розсіювання фотона; напрям руху розсіяного фотона.
	\end{exercise}
	\begin{solution}
		Кут розсіювання фотона~— 4. Напрям руху розсіяного фотона~— 3.
	\end{solution}
	
	\begin{exercise}
		На рис. 4 подано квантову схему ефекту Комптона. Показати: кут, під яким рухається електрон віддачі; кут розсіяння фотона; напрям руху електрона віддачі.
	\end{exercise}
	\begin{solution}
		Кут, під яким рухається електрон віддачі~— 6. Кут розсіювання фотона~— 4. Напрям руху електрона віддачі~— 5.
	\end{solution}
	
	\begin{exercise}
		На рис. 4 подано квантову схему ефекту Комптона. Показати: кут розсіювання фотона; напрям руху розсіяного фотона.
	\end{exercise}
	\begin{solution}
		Кут розсіювання фотона~— 4. Напрям руху розсіяного фотона~— 3.
	\end{solution}
	
	\begin{exercise}
		Показати формули, які визначають енергію розсіяного фотона.
	\end{exercise}
	\begin{solution}
		\[
			h \nu' = \frac{hc}{\lambda'}.
		\]
	\end{solution}
	
	\begin{exercise}
		Показати формули, які визначають кінетичну енергію електрона віддачі.
	\end{exercise}
	\begin{solution}
		???
	\end{solution}
	
	\begin{exercise}
		Показати формули, які визначають імпульс фотона, що падає на електрон речовини.
	\end{exercise}
	\begin{solution}
		\[
			p_\text{ф} = \frac{h \nu}{c} = \frac{h}{\lambda}.
		\]
	\end{solution}
	
	\begin{exercise}
		Показати формули, які визначають імпульс розсіяного фотона.
	\end{exercise}
	\begin{solution}
		\[
			p_\text{ф}' = \frac{h \nu'}{c} = \frac{h}{\lambda'}.
		\]
	\end{solution}
	
	\begin{exercise}
		Показати формули, які визначають імпульс електрона віддачі.
	\end{exercise}
	\begin{solution}
		\[
			p_\text{е} = \vec{p}_{\text{ф}} - \vec{p}_{\text{ф}}'.
		\]
	\end{solution}
	
	\begin{exercise}
		Тиск світла полягає…
	\end{exercise}
	\begin{solution}
		У тому, що під час зіткнення з поверхнею тіла, фотон передає їй свій імпульс.
	\end{solution}
	
	\begin{exercise}
		Показати формулу, як визначає тиск світла на дзеркальну поверхню.
	\end{exercise}
	\begin{solution}
		\[
			p = \frac{n h \nu}{c} (1 + \rho), \, \rho = 1 \implies p = 2 \frac{n h \nu}{c}.
		\]
		\[
			p = 2 \frac{n h \nu}{c} = 2 \frac{h \nu}{c}.
		\]
	\end{solution}
	
	\begin{exercise}
		Показати формулу, як визначає тиск світла на чорну поверхню.
	\end{exercise}
	\begin{solution}
		\[
			p = \frac{n h \nu}{c} (1 + \rho), \, \rho = 0 \implies p = \frac{n h \nu}{c}.
		\]
	\end{solution}
	
	\begin{exercise}
		Зазначити властивості випромінювання, які спричиняють тиск світла.
	\end{exercise}
	\begin{solution}
		\begin{enumerate}
			\item Світло — це потік фотонів.
			\item Світло — це потік фотонів, маса спокою яких дорівнює нулю. ???
		\end{enumerate}
	\end{solution}
\end{document}