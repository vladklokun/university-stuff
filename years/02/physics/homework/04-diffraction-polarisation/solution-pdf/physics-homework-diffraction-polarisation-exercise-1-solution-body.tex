% ------------------------------------------------------------------------
% file `physics-homework-diffraction-polarisation-exercise-1-solution-body.tex'
%
%     solution of type `exercise' with id `1'
%
% generated by the `solution' environment of the
%   `xsim' package v0.10 (2017/09/19)
% from source `physics-homework-diffraction-polarisation' on 2017/11/06 on line 54
% ------------------------------------------------------------------------
^^I^^I\given{$b = \SI{1}{\metre}$\\
^^I^^I$\lambda = \SI{5e-7}{\metre}$.}
^^I^^I^^I
^^I^^I^^I
^^I
^^I^^IНеобхідно розглянути два випадки:
^^I^^I\begin{enumerate}
^^I^^I^^I\item Якщо джерело світла точкове, тоді використовуємо формулу для однорідного середовища:
^^I^^I^^I\[
^^I^^I^^I^^Ir_6 = \sqrt{\frac{ab}{a+b} m \lambda}
^^I^^I^^I^^I    = \sqrt{\frac{0{,}5}{1{,}5}} \times 6 \times \num{5e-7}
^^I^^I^^I^^I^^I= \sqrt{\num{10e-7}} = \SI{1e-3}{\metre}.
^^I^^I^^I\]
^^I^^I^^I
^^I^^I^^I\item Хвильовий фронт, що падає на отвір, плоский, падіння світла нормальне. Тоді використовуємо формулу:
^^I^^I^^I\[
^^I^^I^^I^^Ir_6 = \sqrt{b m \lambda}
^^I^^I^^I^^I    = \sqrt{6 \times \num{5e-7}}
^^I^^I^^I^^I^^I= \sqrt{\num{30e-7}}
^^I^^I^^I^^I^^I= \sqrt{\num{3e-6}}
^^I^^I^^I^^I^^I= \SI{1.73e-3}{\metre}.
^^I^^I^^I\]
^^I^^I\end{enumerate}
^^I^^I
