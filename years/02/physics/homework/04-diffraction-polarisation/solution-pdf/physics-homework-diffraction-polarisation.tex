\documentclass[a4paper,oneside,DIV=10,12pt]{scrartcl}

\usepackage{graphicx}
\usepackage{float}

\usepackage{fontspec}
\setmainfont{STIX Two Text}
%\setsansfont{Roboto}
\newfontfamily{\cyrillicfontsf}{Roboto}

\usepackage{microtype}

\usepackage{polyglossia}
\setmainlanguage{ukrainian}

\usepackage{amsmath,amsthm}
\usepackage{unicode-math}
\setmathfont{STIX Two Math}

\usepackage{booktabs}

\usepackage{tikz}
\usetikzlibrary{arrows,automata,positioning}

\usepackage{siunitx}
\sisetup{output-decimal-marker = {,},
exponent-product = {\cdot}}

% Problem-solution typesetting
\usepackage{xsim}
\DeclareExerciseTranslations{exercise}{
	Ukrainian	=	завдання ,
}

\DeclareExerciseTranslations{solution}{
	Ukrainian	=	розв'язання ,
}

\xsimsetup{
	solution/print = true,
}

\newcommand{\given}[1]{\begin{trivlist}\item[\hskip \labelsep\textbf{Дано:}]#1\end{trivlist}}

\begin{document}
	\begin{exercise}
		Відстані між екраном з отвором і точкою спостереження дорівнює \SI{1}{\metre}. На отвір падає світло з довжиною хвилі \SI{5e-7}{\metre}. Визначити радіус шостої зони Френеля, якщо:
		\begin{enumerate}
			\item Джерело світла точкове і відстань до нього до отвору дорівнює \SI{0.5}{\metre}.
			\item Хвильовий фронт, що падає на отвір, плоский, падіння світла нормальне.
		\end{enumerate}
	\end{exercise}
	
	\begin{solution}
		\given{$b = \SI{1}{\metre}$\\
		$\lambda = \SI{5e-7}{\metre}$.}
			
			
	
		Необхідно розглянути два випадки:
		\begin{enumerate}
			\item Якщо джерело світла точкове, тоді використовуємо формулу для однорідного середовища:
			\[
				r_6 = \sqrt{\frac{ab}{a+b} m \lambda}
				    = \sqrt{\frac{0{,}5}{1{,}5}} \times 6 \times \num{5e-7}
					= \sqrt{\num{10e-7}} = \SI{1e-3}{\metre}.
			\]
			
			\item Хвильовий фронт, що падає на отвір, плоский, падіння світла нормальне. Тоді використовуємо формулу:
			\[
				r_6 = \sqrt{b m \lambda}
				    = \sqrt{6 \times \num{5e-7}}
					= \sqrt{\num{30e-7}}
					= \sqrt{\num{3e-6}}
					= \SI{1.73e-3}{\metre}.
			\]
		\end{enumerate}
		
	\end{solution}
	
	\begin{exercise}
		На щілину нормально падає монохроматичне світло. Кут дифракції другого максимуму дорівнює \SI{1}{\degree}. Скільком довжинам хвиль падаючого світла дорівнює ширина щілини?
	\end{exercise}
	
	\begin{solution}
	\end{solution}
	
	\begin{exercise}
		Яке найбільше значення номера дифракційного максимуму, що відповідає жовтій лінії довжиною хвилі \SI{598}{\nano\metre} при нормальному падінні світла на щілину шириною \SI{2}{\micro\metre}? Скільки загалом спостерігається максимумів?
	\end{exercise}
	
	\begin{solution}
	\end{solution}
	
	\begin{exercise}
		Яким має бути період дифракційної решітки, щоб у спектрі першого порядку були роздільні лінії з довжинами хвилі \SI{589}{\nano\metre} і \SI{589.6}{\nano\metre}? Довжина решітки — \SI{2.5}{\centi\metre}.
	\end{exercise}
	
	\begin{solution}
	\end{solution}
	
	\begin{exercise}
		Кут між площинами пропускання поляроїдів дорівнює \SI{50}{\degree}. Природне світло, проходячи через таку систему, послаблюється у 8 раз. Нехтуючи втратою світла при відбиванні, визначити коефіцієнт поглинання світла в поляроїдах.
	\end{exercise}
	
	\begin{solution}
	\end{solution}
	
	\begin{exercise}
		Дві призми Ніколя розташовані так, що кут між їх головними площинами дорівнює \SI{60}{\degree}. У скільки разів зменшиться інтенсивність природного світла при проходженні призми через:
		\begin{enumerate}
			\item одну призму;
			\item дві призми.
		\end{enumerate}
		Втрати інтенсивності світла на відбивання і заломлення становлять 8\%.
	\end{exercise}
	
	\begin{solution}
	\end{solution}
	
	% No. 185
	\begin{exercise}
		
	\end{exercise}
	
	\begin{solution}
	\end{solution}
\end{document}