\documentclass[a4paper,oneside,DIV=9,12pt]{scrartcl}

\usepackage{fontspec}
\setmainfont{STIX Two Text}

\usepackage{amsmath}
\usepackage{unicode-math}
\setmathfont{STIX Two Math}

\usepackage{ieeetrantools}

\usepackage{microtype}

\usepackage{polyglossia}
\setmainlanguage{ukrainian}

\usepackage{siunitx}
\sisetup{output-decimal-marker = {,},
		inter-unit-product = \ensuremath{{}\cdot{}},
		exponent-product = \cdot}

\newcommand\given{\noindent\textbf{Дано:}}
\newcommand\find{\noindent\textbf{Знайти:}}
\newcommand\solution{\noindent\textbf{Розв'язання:}}
\newcommand\answer[1]{\noindent\textbf{Відповідь:} #1}

\begin{document}
	
	\given\par
		$v_{\text{дж}} = \SI{200}{\metre\per\second}$;
		
		$v = \SI{333}{\metre\per\second}$.
		
	\find\par
		$\frac{\nu}{\nu_{0}}$ --- ?.
		
	\solution\par
		Формула, що описує ефект Доплера:
		\[
			\nu = \nu_0 \frac{v + v_{\text{пр}}}{v + v_{\text{дж}}}.
		\]
		
		Оскільки спостерігач (приймач) нерухомий, а куля рухається йому назустріч, то формула приймає такий вигляд ($v_{\text{пр}} = 0, \, v_{\text{дж}} < 0$):
		\[
			\nu = \nu_0 \frac{v}{v - v_{\text{дж}}} = \nu_0 \frac{333}{333 - 200} \approx \num{2,5}\nu_0.
		\]
		
		Якщо врахувати, що політ кулі складається з польоту до спостерігача (частота тону $\nu_1$) та від нього (частота тону $\nu_2$), то різниця між цими тонами:
		\[
			\nu_1 = \nu_0 \frac{v}{v - v_{\text{дж}}}, \quad \nu_2 = \nu_0 \frac{v}{v + v_{\text{дж}}}.
		\]
		
		Отже:
		\[
			\frac{\nu_1}{\nu_2} = \frac{\nu_0 \frac{v}{v - v_{\text{дж}}}}{\nu_0 \frac{v}{v + v_{\text{дж}}}} = \frac{v + v_{\text{дж}}}{v - v_{\text{дж}}} =\frac{333 + 200}{333 - 200} \approx 4.
		\]
	\answer{\num{2,5}; \num{4}}.
	
\end{document}