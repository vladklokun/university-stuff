\documentclass[a4paper,oneside,DIV=9,12pt]{scrartcl}

\usepackage{fontspec}
\setmainfont{STIX Two Text}

\usepackage{amsmath}
\usepackage{unicode-math}
\setmathfont{STIX Two Math}

\usepackage{microtype}

\usepackage{polyglossia}
\setmainlanguage{ukrainian}

\usepackage{siunitx}
\sisetup{output-decimal-marker = {,},
		inter-unit-product = \ensuremath{{}\cdot{}},
		exponent-product = \cdot}

\newcommand\given{\noindent\textbf{Дано:}}
\newcommand\find{\noindent\textbf{Знайти:}}
\newcommand\solution{\noindent\textbf{Розв'язання:}}
\newcommand\answer[1]{\noindent\textbf{Відповідь: #1}}

\begin{document}
	%\noindent\begin{tabular}{ll}
	%	Дано & Знайти \\
	%	$S = \num{0,6} \cos (1800t - \num{5,3}x)\, \si{\metre}$ & $\varepsilon_{\text{max}}$ --- ?\\
%		$S_0 = \SI{0,6}{\metre}$ & $\dfrac{\varepsilon_{\text{max}}}{v}$ --- ?\\
%		$\omega = \SI{1800}{\per\second}$ & \\
%		$k = \SI{5,3}{\per\metre}$ & \\
%	\end{tabular}
	
	\given\par
	$S = \num{60} \cos (1800t - \num{5,3}x)\, \si{\centi\metre}$
	
	
	$S_0 = \SI{60}{\centi\metre} = \SI{0,6}{\metre}$
	
	
	$\omega = \SI{1800}{\per\second}$
	
	
	$k = \SI{5,3}{\per\centi\metre} = \SI{0,053}{\per\metre}$
	
	\find\par
	$\varepsilon_{\text{max}}$ --- ?
	
	$\frac{\varepsilon_{\text{max}}}{v}$ --- ?
	
	\solution\par
	%Щоб знайти відносну деформацію середовища, продиференціюємо рівняння плоскої хвилі по $x$:
	%\[
	%	\varepsilon = \frac{\partial (S_{0} \cos (\omega t - kx))}{\partial x} = -S_{0}k \sin (wt - kx).
	%\]
	
	Знайдемо амплітуду коливань відносної деформації середовища:
	\[
		\varepsilon_{\text{max}} = S_{0}k = \SI{0,6}{\metre} \cdot \SI{0,053}{\per\metre} = \num{0,0318}.
	\]
	
	Знайдемо ампілтуду коливань швидкості частинок:
	\[
		v_{\text{чmax}} = S_0 \omega = \SI{0,6}{\metre} \cdot \SI{1800}{\per\second} = \SI{1080}{\metre\per\second}.
	\]
	
	Знайдемо відношення амплітуди коливань відносної деформації середовища до амплітуди коливань швидкості частинок:
	\[
		\frac{\varepsilon_{\text{max}}}{v_{\text{чmax}}} = \frac{\num{0,0318}}{\SI{1080}{\metre\per\second}} = \SI{0,0000294}{\second\per\metre}.
	\]
	
	\answer{\SI{0,0000294}{\second\per\metre}}.
	
\end{document}