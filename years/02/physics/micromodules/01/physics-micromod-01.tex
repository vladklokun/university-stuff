\documentclass[a4paper,oneside,DIV=10,12pt]{scrartcl}

\usepackage{fontspec}
\setmainfont{STIX Two Text}

\usepackage{microtype}

\usepackage{amsmath}
\usepackage{unicode-math}
\setmathfont{STIX Two Math}

\usepackage{polyglossia}
\setmainlanguage{ukrainian}

\begin{document}
	\begin{enumerate}
		\item Механічною хвилею називається процес поширення коливань будь-якої природи в просторі або середовищі.
		\item Поперечні хвилі виникають тільки у таких речовинах:
			
			Поперечні хвилі виникають тільки у твердих тілах, а в рідинах та газах вони не утворюються, оскільки під час зсуву шарів дрідини або газу один відносно одного пружні сили не виникаються.
		\item Поздовжні хвилі виникають тільки у таких речовинах:
			
			Газ.
		\item Фазова швидкість або швидкість поширення хвилі --- це швидкість, з якою переміщується у просторі фаза коливань. Фазова швидкість залежить від густини середовища і його пружних властивостей.
		
		\item Довжина хвилі --- це відстань, на яку поширюється хвиля за один період.
		
		\item Довжина хвилі визначається за формулами:
		\[
			\lambda = vT = \frac{v}{\nu} = \frac{2\pi}{\omega}.
		\]
		
		\item Хвильове число визначається за формулами:
		\[
			k = \frac{\omega}{v} = \frac{2\pi}{vT} = \frac{2\pi}{\lambda}.
		\]
		
		\item Рівняння механічної хвилі має такий вигляд:
		\[
			A\cos\left(\omega t - kx\right).
		\]
		
		\item Електромагнітною хвилею називається електромагнітне поле, яке поширюється в просторі.
		
		\item Електромагнітні хвилі бувають такі: поперечні.
	\end{enumerate}
\end{document}