\documentclass[a4paper,oneside,12pt,DIV=9,titlepage,toc]{scrartcl}

\usepackage{fontspec}
\setmainfont{STIX Two Text}
\setsansfont{PT Sans}

\usepackage{microtype}

\usepackage{polyglossia}
\setdefaultlanguage{ukrainian}

% Math typeset
\usepackage{amsmath}
\usepackage{unicode-math}
\setmathfont{STIX Two Math}

\begin{document}
	\begin{titlepage}
		\begin{center}
			Конспект
		\end{center}
	\end{titlepage}
	
	\tableofcontents
	
	\section{Механічні та електромагнітні коливання}
		\emph{Коливаннями} називають процеси, коли значення фізичної величини повторюються в часі.
		
		\emph{Вільними} називають коливання, коли коливальна система виведена зі стану рівноваги, а потім надана сама собі.
		
		Коливання будуть \emph{загасаючими}, коли енергія коливальної системи витрачається на подолання сил опору або тертя.
		
		Коливання можуть бути \emph{незагасаючими}, коли коливальна система виконує коливання під дією періодично змінюваної зовнішньої сили, тобто коливальна система весь час отримує зовні енергію, яку вона витратила на подолання сил опору або тертя.
		
		Приклади коливальних систем:
		\begin{itemize}
			\item дитяча гойдалка (качеля) --- механічні загасаючі коливання. Якщо розкачуємо --- незагасаючі;
			\item серце --- рухається за законом вимушених коливань.
		\end{itemize}
		
		\subsection{Вільні незагасаючі механічні коливання}
			Пружинний маятник являє собою пружину, до якої прикріплено вантаж масою $m$. Нехтуємо силою тертя.
			
			$x$ --- зміщення коливальної системи від положення рівноваги. Це зміщення залежить від часу, тобто: $x(t)$.
			
			$T$ (період коливань) --- час, за який коливальна система виконує одне повне коливання. Вимірюється у секундах.
			
			$\nu$ (частота коливань) --- кількість повних коливань, які виконала коливальна система.
			\[
				\nu = \frac{1}{T}, \quad
				\left[ \frac{1}{\textrm{с}} \right] =  \left[ \textrm{Гц} \right]
			\]
			
			Під час коливання кінетична енергія перетворюється у потенціальну і навпаки.
			
			\[
				\frac{dp}{dt} = \sum F_i; \quad
				p = m \cdot v; \quad
				v  = \frac{dx}{dt} \\
				m = \mathrm{const};
			\]
			
			В цій коливальній системі $\sum F_i = F_{\textrm{пр}} = -kx$, $k$ --- коефіціент пружності пружини.
			
			\[
				m \frac{dv}{dt} = -kx;
			\]
			
			\[
				m \frac{d^2x}{dt^2} = -kx;
			\]
			
			\[
				m \frac{d^2x}{dt^2} + kx = 0 \quad | : m
			\]
			
			\[
				\frac{d^2x}{dt^2} + \frac{k}{m}x = 0;
			\]
			
			\[
				\frac{d^2x}{dt^2} + \omega_0^2 x = 0;
			\]
			Для зручності вводимо $\omega_0^2$. Таким чином отримуємо диференціальне рівняння незагасаючих механічних коливань.
\end{document}