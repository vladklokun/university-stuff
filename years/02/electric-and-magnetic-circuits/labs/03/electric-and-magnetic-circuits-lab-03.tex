\documentclass[a4paper,oneside,DIV=10,12pt]{scrartcl}

\usepackage{graphicx}
\usepackage{float}

\usepackage{fontspec}
\setmainfont{STIX Two Text}
%\setsansfont{Roboto}
\newfontfamily{\cyrillicfontsf}{Roboto}

\usepackage{microtype}

\usepackage{polyglossia}
\setmainlanguage{ukrainian}

\usepackage{amsmath}
\usepackage{unicode-math}
\setmathfont{STIX Two Math}
\usepackage[retainorgcmds]{IEEEtrantools}

\usepackage{booktabs}

\usepackage{siunitx}
\sisetup{output-decimal-marker = {,},
exponent-product = {\cdot}}

\newcommand\schel[1]{\textit{#1}}

\begin{document}
	\begin{titlepage}
		\begin{center}
			Міністерство освіти і науки України\\
			Національний авіаційний університет\\
			Навчально-науковий інститут комп'ютерних інформаційних технологій\\
			Кафедра комп'ютеризованих систем управління
			
			\vspace{\fill}
				Лабораторна робота №3\\
				з дисципліни «Теорія електричних та магнітних кіл»\\
				на тему: «Дослідження нерозгалуженого електричного~кола синусоїдного~струму»\\
				Варіант №
				
			\vspace{\fill}
			
			\begin{flushright}
				Виконав:\\
				студент ННІКІТ СП-225\\
				Клокун Владислав\\
				Перевірив:\\
				Молчанов О.~В.
			\end{flushright}
			Київ 2017
		\end{center}
	\end{titlepage}
	
	\section{Мета роботи}
		\begin{enumerate}
			\item Використовуючи вимірювальні прилади, набути навички визначення параметрів ланцюга змінного струму, а саме: активного опору резистора, активного і реактивного опорів реальної котушки індуктивності і реального конденсатора.
			\item Дослідити різні комбінації послідовного включення в ланцюг активного резистора, котушки індуктивності і конденсатора.
			\item Дослідити резонанс у послідовному контурі.
		\end{enumerate}
		
	\section{Короткі теоретичні відомості}
		Для того, щоб визначити значення опорів різних елементів електричних ланцюгів, необхідно виміряти за допомогою приладів значення напруги, прикладеної до елемента, значення струму, який по ньому протікає, а також активну потужність, що виділяється, та кут зсуву фази. Ці величини вимірюються за допомогою вольтметра, амперметра, ватметра, фазометра.
		
		Значення активного опору резистора визначається за законом Ома:
		\[
			R = \frac{U}{I}.
		\]
		
		Потужність, споживана елементом, виділяється у вигляді тепла тільки на активних резисторах і вимірюється ватметром. Тому опір активного резистора можна визначити ще й за формулою:
		\[
			R = \frac{P}{I^2}.
		\]
		
		Щоб визначити значення активного опору реальних котушки індуктивності і конденсатора за допомогою вольтметра, амперметра і ватметра, використовуємо формули, що отримуємо з трикутника опорів:
		\[
			Z = \sqrt{R^2 + X^2},
		\]
		де $Z = \frac{U}{I}$ --- модуль повного опору кола (\si{\ohm}), $R$ --- повний активний опір кола (\si{\ohm}), $X$ --- повний реактивний опір кола (\si{\ohm}), $U$ --- діюче значення синусоїдної напруги (\si{\volt}), $I$ --- діюче значення синусоїдного струму (\si{\ampere}).
		
		\[
			X = X_K - X_C = \omega L - \frac{1}{LC},
		\]
		де $X_K$ --- реактивний індуктивний опір кола (\si{\ohm}), $X_C$ --- реактивний ємністний опір кола (\si{\ohm}), $L$ --- індуктивність котушок кола (\si{\henry}), $C$ --- ємність конденсаторів кола {\si{\farad}}, $\omega$ --- кутова частота (\si{\radian\per\second}).
		
		\[
			\omega = 2 \pi f,
		\]
		$f$ --- циклічна частота (\si{\hertz}).
		
	\section{Порядок виконання роботи}
		Зібрати вимірювальну частину схеми (рис.~\ref{fig:schematic}), використовуючи амперметр, фазометр, мультиметр і, підключаючи по черзі (лабораторний блок №8) резистор, котушку індуктивності і конденсатор, зробити необхідні вимірювання і занести їх в табл~\ref{tab:measurements1}. 
		
		\begin{figure}[!htbp]
			\caption{Вимірювальна частина схеми}
			\label{fig:schematic}
		\end{figure}
		
		\begin{table}[!htbp]
		\centering
			\begin{tabular}{
				l
				S
				S
				S
				S
				S
				S
				S
				S
				S
				S
				S
				S
			}
			\toprule
				{Коло} & \multicolumn{6}{c}{Виміряти} & \multicolumn{5}{c}{Обчислити опір, \si{\ohm}} \\
				\cmidrule(lr){2-7} \cmidrule(lr){8-12}
				& {$U, \si{\volt}$} & {$I, \si{\ampere}$} & {$I, \si{\degree}$} & {$U_R, \si{\volt}$} & {$U_K, \si{\volt}$} & {$U_C, \si{\volt}$} & {$R$} & {$R_{K}$} & {$R_C$} & {$X_K$} & {$X_C$} \\
			\midrule
				
			\bottomrule
			\end{tabular}
		\caption{Вимірювання 1}
		\label{tab:measurements1}
		\end{table}
		
		Використовуючи виміряні величини, обчислити значення активного опору резистора, активного і реактивного опорів котушки індуктивності і конденсатора. Отримані значення занести в табл.~\ref{tab:measurements1}.
		
		Підключаючи послідовно до вимірювальної частини схеми комбінації елементів \schel{RL}, \schel{RC}, \schel{RLC}, зробити необхідні вимірювання та занести їх в табл.~\ref{tab:measurements2}.
		
		\begin{table}[!htbp]
		\centering
			\begin{tabular}{
				l
				S
				S
				S
				S
				S
				S
				S
				S
			}
				\toprule
					{Коло} & {$U, \si{\volt}$} & {$I, \si{\ampere}$} & {$\varphi, \si{\degree}$} & {$U_R, \si{\volt}$} & {$U_K, \si{\volt}$} & {$U_C, \si{\volt}$} & {$U_R + U_K, \si{\volt}$} & {$U_R + U_C, \si{\volt}$} \\
				\midrule
					\schel{RL} & & & & & & {—} & {—} & {—} \\
					\schel{RC} & & & & & {—} & & {—} & {—} \\
					\schel{RLC} & & & & & & & & \\
				\bottomrule
			\end{tabular}
		\caption{Вимірювання 2}
		\label{tab:measurements2}
		\end{table}
		
		Підключити до вимірювальної частини схеми тільки котушку індуктивності (лабораторний блок №8) і конденсатор (магазин ємності). Знаючи величину реактивного опору котушки, визначити значення резонансної ємності, встановити на вході схеми напругу \SIrange[range-phrase = --]{5}{7}{\volt} і, змінюючи ємність конденсатора у діапазоні \SIrange[range-phrase = --]{0}{99,5}{\micro\farad}, виміряти величини, вказані в табл.~\ref{tab:measurements3}.
		
		\begin{table}[!htbp]
		\centering
			\begin{tabular}{
				l
				S
				S
				S
				S
				S
				S
			}
				\toprule
					№ & {$U, \si{\volt}$} & {$I, \si{\ampere}$} & {$\varphi, \si{\degree}$} & {$U_K, \si{\volt}$} & {$U_C, \si{\volt}$} & {$C, \si{\micro\farad}$} \\
				\midrule
					1 & & & & & & 0\\
					2 & & & & & & \\
					3 & & & & & & \\
					4 & & & & & & \\
					5 & & & & & & {$C = C_0$}\\
					6 & & & & & & \\
					7 & & & & & & \\
					8 & & & & & & \\
					9 & & & & & & \\
					10 & & & & & & 99,5\\
				\bottomrule
			\end{tabular}
		\caption{Вимірювання 3}
		\label{tab:measurements3}
		\end{table}
		
		Кількість змін значення ємності дорівнює десяти, причому п'яте значення ємності змінного конденсатора має дорівнювати значенню резонансної ємності.
		
		Побудувати в масштабі векторні діаграми напруг для кожної комбінації включення елементів. Побудувати в масштабі трикутники напруг і опорів для кожного випадку.
		
		Побудувати в масштабі характеристики $I = f(C)$, $U_K = f(C)$, $U_C = f(C)$, $\varphi = f(C)$ в одній координатній сітці.
		
	\section{Висновки}
		Під час виконання даної лабораторної роботи ми набули навички визначення параметрів ланцюга змінного струму за допомогою вимірювальних приладів, а саме: активного опору резистора, активного і реактивного опорів реальної котушки індуктивності і реального конденсатора; дослідили різні комбінації послідовного включення в ланцюг активного резистора, котушки індуктивності і конденсатора; дослідили резонанс у послідовному контурі.
\end{document}
