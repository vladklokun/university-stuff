\documentclass[a4paper,oneside,DIV=12,12pt]{scrartcl}

\usepackage{graphicx}

%%% Load fonts
\usepackage{fontspec}
\setromanfont[
	SmallCapsFeatures = {LetterSpace = 5}
]{STIX Two Text}
\setsansfont{Roboto}
\setmonofont{PT Mono}

\usepackage{amsmath,unicode-math}
\setmathfont{STIX Two Math}
%%%

%%% Language-specific setup
\usepackage{polyglossia}
\setmainlanguage{ukrainian}
%%%

%%% Microtypographic enhancements
\usepackage{microtype}
%%%

%%% Table typesetting
\usepackage{booktabs}
\usepackage{longtable}
\usepackage{array}
\newcolumntype{v}[1]{>{\raggedright\arraybackslash\hspace{0pt}}p{#1}}
\newcolumntype{n}[1]{>{\raggedleft\arraybackslash\hspace{0pt}}p{#1}}

\usepackage{multirow}
% \renewcommand{\arraystretch}{1.4}

\usepackage{cellspace}
\setlength\cellspacetoplimit{3pt}
\setlength\cellspacebottomlimit{3pt}
%%%

%%% Captions
\usepackage{caption}
\usepackage{subcaption}
%%%

%%% Math typesetting
\usepackage{ieeetrantools}
%%%

\begin{document}
	\begin{titlepage}
    \begin{center}
	Міністерство освіти і науки України\\
	Національний авіаційний університет\\
	Навчально-науковий інститут комп'ютерних інформаційних технологій\\
	Кафедра комп'ютеризованих систем управління

	\vspace{\fill}

	Лабораторна робота №7\\
	з дисципліни «Якість програмного забезпечення та тестування»\\
	на тему «Експертні методи прогнозування»

	\vspace{\fill}
	
	\begin{flushright}
		Виконав:\\
		студент ННІКІТ СП-225\\
		Клокун Владислав\\
		Перевірила:\\
		Апенько Н. В.
	\end{flushright}

	Київ 2017

    \end{center}
    \end{titlepage}
	
	\section{Експертні оцінки}
		Припустимо, що чотири експерти оцінили п'ять напрямків. Дані оцінки наведені у табл.~\ref{tab:direction-marks}.
		
		\begin{longtable}[c]{lrrrr}
			\toprule
				\multirow{2}{*}[-2pt]{Напрямок} & \multicolumn{4}{c}{Експерт} \\
				\cmidrule(lr){2-5}
				 & \multicolumn{1}{c}{1} & \multicolumn{1}{c}{2} & \multicolumn{1}{c}{3} & \multicolumn{1}{c}{4} \\
			\midrule
			\endhead
			\bottomrule
			\caption{Таблиця балів оцінки напрямів розвитку}
			\endfoot
			\label{tab:direction-marks}
			
				1 & 100 & 100 & 100 & 100\\
				2 & 90  & 90  & 90  & 90\\
				3 & 80  & 80  & 80  & 80\\
				4 & 70  & 70  & 70  & 70\\
				5 & 60  & 60  & 60  & 60\\
		\end{longtable}
		
		За отриманими оцінками складаємо матрицю рангів~(табл.~\ref{tab:direction-ranks}).
		
		\begin{longtable}[c]{lrrrr}
			\toprule
				\multirow{2}{*}[-2pt]{Напрямок} & \multicolumn{4}{c}{Експерт} \\
				\cmidrule(lr){2-5}
				 & \multicolumn{1}{c}{1} & \multicolumn{1}{c}{2} & \multicolumn{1}{c}{3} & \multicolumn{1}{c}{4} \\
			\midrule
			\endhead
			\bottomrule
			\caption{Таблиця рангів оцінки напрямів розвитку}
			\endfoot
			\label{tab:direction-ranks}
			
				1 & 1 & 1 & 1 & 1\\
				2 & 2 & 2 & 2 & 2\\
				3 & 3 & 3 & 3 & 3\\
				4 & 4 & 4 & 4 & 4\\
				5 & 5 & 5 & 5 & 5\\
		\end{longtable}
		
		Оскільки значення оцінок і рангів однакові, то середні значення рангів~$S_j$ і середні значення балів~$M_j$ будуть рівні самим значенням рангів та оцінок відповідно.
		
		Максимальна оцінка була виставлена лише напрямку~1, тому лише вона має ненульовий показник частоти максимально можливих оцінок:
		\[
			K_{100_{1}} = \frac{4}{4} = 1.
		\]
		
		Середня вага кожного напрямку (нормована оцінка) розраховується за формулою:
		\[
			w_{ij} = \frac{C_{ij}}{\displaystyle\sum^m_{j = 1}{C_{ij}}}.
		\]
		
		Розраховуємо значення для кожного напрямку і заносимо їх в таблицю~\ref{tab:direction-relative-value}.
		\begin{IEEEeqnarray*}{rCl}
			W_{11} = W_{12} = W_{13} = W_{14} &=& \frac{100}{100 + 90 + 80 + 70 + 60} = \frac{100}{400} = 0{,}25
		\end{IEEEeqnarray*}
		\begin{IEEEeqnarray*}{rCl}
			W_{21} = W_{22} = W_{23} = W_{24} &=& \frac{90}{100 + 90 + 80 + 70 + 60} = \frac{90}{400} = 0{,}225\\[2\jot]
			W_{31} = W_{32} = W_{33} = W_{34} &=& \frac{80}{100 + 90 + 80 + 70 + 60} = \frac{80}{400} = 0{,}2\\[2\jot]
			W_{41} = W_{42} = W_{43} = W_{44} &=& \frac{70}{100 + 90 + 80 + 70 + 60} = \frac{70}{400} = 0{,}175\\[2\jot]
			W_{51} = W_{52} = W_{53} = W_{54} &=& \frac{60}{100 + 90 + 80 + 70 + 60} = \frac{60}{400} = 0{,}15.\\[2\jot]
		\end{IEEEeqnarray*}
		
		\begin{longtable}[c]{lrrrr}
			\toprule
				\multirow{2}{*}[-2pt]{Напрямок} & \multicolumn{4}{c}{Експерт} \\
				\cmidrule(lr){2-5}
				 & \multicolumn{1}{c}{1} & \multicolumn{1}{c}{2} & \multicolumn{1}{c}{3} & \multicolumn{1}{c}{4} \\
			\midrule
			\endhead
			\bottomrule
			\caption{Таблиця відносних значень напрямів розвитку}
			\endfoot
			\label{tab:direction-relative-value}
			
				1 & 0{,}250 & 0{,}250 & 0{,}250 & 0{,}250\\
				2 & 0{,}225 & 0{,}225 & 0{,}225 & 0{,}225\\
				3 & 0{,}200 & 0{,}200 & 0{,}200 & 0{,}200\\
				4 & 0{,}175 & 0{,}175 & 0{,}175 & 0{,}175\\
				5 & 0{,}150 & 0{,}150 & 0{,}150 & 0{,}150\\
		\end{longtable}
		
		Оскільки значення для кожного експерта однакові, то і середні значення будуть рівні значенням для кожного напрямку.
		
		Будуємо матрицю переваг~(табл.~\ref{tab:advantage-matrix}).
		\begin{longtable}[c]{lScScScScScScScScScSc}
			\toprule
				Напрямок & 1 & 2 & 3 & 4 & 5 \\
			\midrule
			\endhead
			\bottomrule
			\caption{Матриця переваг}
			\endfoot
			\label{tab:advantage-matrix}
			
				1 & $1$             & $\dfrac{10}{ 9}$ & $\dfrac{10}{ 8}$ & $\dfrac{10}{ 7}$ & $\dfrac{10}{ 6}$ \\
				2 & $\dfrac{9}{10}$ & $1$              & $\dfrac{ 9}{ 8}$ & $\dfrac{ 9}{ 7}$ & $\dfrac{ 9}{ 6}$ \\
				3 & $\dfrac{8}{10}$ & $\dfrac{ 8}{ 9}$ & $1$              & $\dfrac{ 8}{ 7}$ & $\dfrac{ 8}{ 6}$ \\
				4 & $\dfrac{7}{10}$ & $\dfrac{ 7}{ 9}$ & $\dfrac{ 7}{ 8}$ & $1$              & $\dfrac{ 7}{ 6}$ \\
				5 & $\dfrac{6}{10}$ & $\dfrac{ 6}{ 9}$ & $\dfrac{ 6}{ 8}$ & $\dfrac{ 6}{ 7}$ & $1$ \\
		\end{longtable}
		
		За отриманими даними складаємо таблицю показників порівняльної важливості напрямків~(табл.~\ref{tab:direction-importance}).
		
		\begin{longtable}[c]{lSlSrSrSrSrSr}
			\toprule
				Найменування & Позначення & \multicolumn{5}{c}{Напрямок}\\
				\cmidrule(lr){3-7}
				             &            & 1 & 2 & 3 & 4 & 5 \\
			\midrule
			\endhead
			\bottomrule
			\caption{Таблиця показників порівняльної важливості напрямків}
			\endfoot
			\label{tab:direction-importance}
			
				Сума рангів                          & $S_j$              & $4$      & $8$       & $12$    & $16$      & $20$\\
				Середній ранг                        & $\overline{S}_j$   & $1$      & $2$       & $3$     & $4$       & $5$\\
				Середнє значення в~балах             & $\overline{\mu}_k$ & $100$    & $90$      & $80$    & $70$      & $60$\\
				Частота макс. можливих оцінок  & $K_{100_j}$        & $1$      & $0$       & $0$     & $0$       & $0$\\
				Середня вага                         & $W_j$              & $0{,}25$ & $0{,}225$ & $0{,}2$ & $0{,}175$ & $0{,}15$\\
				Розмах                               & $L_j$              & $0$      & $0$       & $0$     & $0$       & $0$\\
				Коефіцієнт активності~експертів      & $K_{\text{act}_j}$ & $1$      & $1$       & $1$     & $1$       & $1$\\
		\end{longtable}
		
		Також, оскільки оцінки експертів однакові, коефіцієнт конкордації дорівнює~$1$.
		
	\section{Моделі якості ПЗ}
		\subsection{Модель МакКола}
			Модель МакКола в основному призначена для визначення повної характеристики якості програмного продукту його різними характеристиками. Модель якості МакКола має три головних напрями для визначення якості програмного забезпечення:
			\begin{enumerate}
				\item Функціональність.
				\item Модифікованість.
				\item Переносність.
			\end{enumerate}
			
		\subsection{Модель Боема}
			Модель якості Боема має недоліки сучасних моделей, які автоматично і якісно оцінюють якість програмного забезпечення. Модель Боема намагається якісно визначити якість програмного забезпечення заданим набором показників і метрик. Ця модель представляє характеристики програмного забезпечення у більш крупному масштабі, порівняно з моделлю МакКола. Модель Боема схода на модель якості МакКола тим, що вона також є ієрархічною, структурованою навколо високорівневих, проміжних і примітивних характеристик, кожна з яких вносить свій вклад в рівень якості програмного забезпечення.
			
		\subsection{Модель якості Дромі}
			Модель якості Дромі заснована на критеріях оцінки. Модель Дромі намагається оцінити якість системи, тоді як кожний програмний продукт має якість, що відрізняється від інших. Модель Дромі допомагає передбачати дефекти ПЗ і вказує на ті властивості ПЗ, недбалість до яких може призвести до появи дефектів. Ця модель заснована на відношеннях між характеристиками якості і підхарактеристиками, між властивостями програмного забезпечення і характеристиками якості ПЗ.
\end{document}