\documentclass[a4paper,oneside,DIV=12,12pt]{scrartcl}

\usepackage{graphicx}

%%% Load fonts
\usepackage{fontspec}
\setromanfont[
	SmallCapsFeatures = {LetterSpace = 5}
]{STIX Two Text}
\setsansfont{Roboto}
\setmonofont{PT Mono}

\usepackage{amsmath,unicode-math}
\setmathfont{STIX Two Math}
%%%

%%% Language-specific setup
\usepackage{polyglossia}
\setmainlanguage{ukrainian}
%%%

%%% Microtypographic enhancements
\usepackage{microtype}
%%%

%%% Table typesetting
\usepackage{booktabs}
\usepackage{longtable}
\usepackage{array}
\newcolumntype{v}[1]{>{\raggedright\arraybackslash\hspace{0pt}}p{#1}}
\newcolumntype{n}[1]{>{\raggedleft\arraybackslash\hspace{0pt}}p{#1}}

\usepackage{multirow}
% \renewcommand{\arraystretch}{1.4}

\usepackage{cellspace}
% \setlength\cellspacetoplimit{3pt}
% \setlength\cellspacebottomlimit{3pt}
%%%

%%% Captions
\usepackage{caption}
\usepackage{subcaption}
%%%

%%% Math typesetting
\usepackage{ieeetrantools}
%%%

%%%
\makeatletter
\@addtoreset{section}{part}
\makeatother
%%%

\newcommand{\progname}{\texttt{addrbook}}
\newcommand{\theprogcode}{ADRB}

\newcommand{\thetestplan}{TSTPLN}
\newcommand{\thetestplanhash}{c900c0}

\newcommand{\theqaplan}{SQAPLN}
\newcommand{\theqaplanhash}{69632c}

\newcommand{\thegenplan}{GENPLN}
\newcommand{\thegenplanhash}{ccb263}

\newcommand{\theprd}{PRD}
\newcommand{\theprdhash}{06809b}

\newcommand{\printtestplan}{\texttt{\theprogcode-\thetestplan-\thetestplanhash}}
\newcommand{\printqaplan}{\texttt{\theprogcode-\theqaplan-\theqaplanhash}}
\newcommand{\printgenplan}{\texttt{\theprogcode-\thegenplan-\thegenplanhash}}
\newcommand{\printprd}{\texttt{\theprogcode-\theprd-\theprdhash}}

\begin{document}
	\begin{titlepage}
    \begin{center}
	Міністерство освіти і науки України\\
	Національний авіаційний університет\\
	Навчально-науковий інститут комп'ютерних інформаційних технологій\\
	Кафедра комп'ютеризованих систем управління

	\vspace{\fill}

	Лабораторна робота №8\\
	з дисципліни «Якість програмного забезпечення та тестування»\\
	на тему «Техніки управління якістю ПЗ»

	\vspace{\fill}
	
	\begin{flushright}
		Виконав:\\
		студент ННІКІТ СП-225\\
		Клокун Владислав\\
		Перевірила:\\
		Апенько Н. В.
	\end{flushright}

	Київ 2017

    \end{center}
    \end{titlepage}
	
	\part{План якості програмного продукту}
		\section{Мета}
			Цей документ створений для управління якістю програмного продукту~\progname \ у~процесі життєвого циклу.
			
		\section{Довідкові документи}
			У даному розділі наведений повний список документів, на які спирається даний документ.
			
			\begin{longtable}[c]{ll}
				\toprule
				Тип документу & Ідентифікатор \\
				\midrule
				\endhead
				\bottomrule
				\caption{Документи, на які спирається \printqaplan}
				\endfoot
				\label{tab:reference-docs}
				
				Технічне завдання & \printprd \\
				План проекту & \printgenplan \\
				% Внутрішні стандарти розробки проектів & \texttt{INT-PROJGUIDE-642F1D}\\
			
			\end{longtable}
		
		\section{Управління}
			\subsection{Організація}
				У забезпеченні якості програмного продукту~\progname \ приймають участь такі структурні підрозділи:
				\begin{enumerate}
					\item Відділ розробки.
					\item Відділ тестування.
					\item Відділ документації.
				\end{enumerate}
				
			\subsection{Завдання}
				Завдання організації якості програмного забезпечення такі:
				\begin{enumerate}
					\item Розробка якісного ПЗ.
					\item Тестування якості розробленого ПЗ.
					\item Розробка якісної документації до розробленого ПЗ.
				\end{enumerate}
				
			\subsection{Обов'язки}
				Обов'язки кожного структурного підрозділу розподілені таким чином:
				\begin{enumerate}
					\item Відділ розробки~— розробка та супровід програмного продукту.
					\item Відділ тестування~— тестування програмного продукту.
					\item Відділ тестування~— оформлення документації до програмного продукту.
				\end{enumerate}
				
		\section{Документація}
			\subsection{Ціль}
				Ціль розробки документації у процесі забезпечення якості програмного продукту є збереження та організація у зручному вигляді відомостей про процеси життєвого циклу програмного продукту.
				
			\subsection{Мінімальні вимоги до документації}
				\subsubsection{Software Requirements Specification}
					Software Requirements Specification (SRS) повинна точно і ясно описувати кожну з основних вимог до програмного продукту і зовнішніх інтерфейсів. Кожна вимога повинна бути визначена так, що відповідність їй може бути об'єктивно підтверджена та валідована описаним методом.
					
				\subsubsection{Software Design Description}
					Software Design Description (SDD) повинен описувати, як буде структурований програмний продукт задля відповідності вимогам, описаним в~SRS. SDD описує компоненти та підкомпоненти проекту програмного продукту, включаючи бази даних та внутрішні інтерфейси.
					
				\subsubsection{Software Verification and Validation Plan}
					Software Verification and Validation Plan~(SVVP) повинен описувати методи, що використовуються для:
					\begin{enumerate}
						\item Підтвердження, що:
							\begin{enumerate}
								\item Вимоги, що описані в~SRS, затверджені відповідною структурою.
								\item Вимоги, що описані в~SRS, розроблені відповідно до дизайну, що описаний в~SDD.
								\item Дизайн, що описаний в~SDD, розроблений у вигляді коду.
							\end{enumerate}
						\item Валідації, що код, що виконується, відповідає вимогам, описаним в~SRS.
					\end{enumerate}
					
				\subsubsection{Software Configuration and Management Plan}
					Software Configuration and Management Plan описує результати виконання дій, описаних в~SVVP.
					
			\section{Стандарти, практики та метрики}
				Під час процесів життєвого циклу програмного продукту повинні збиратись такі метрики:
				\begin{enumerate}
					\item Кількість рядків початкового коду.
					\item Кількість дефектів.
					\item Кількість дефектів на рядок коду.
					\item Кількість дефектів на версію програми.
					\item Відсоток покриття коду тестами.
				\end{enumerate}
				
			\section{Аудит і перевірка}
				\subsection{Software Requirements Review (SRR)}
					SRR або перевірка вимог до програмного забезпечення проводиться для переконання в адекватності вимог, описаних в~SRS.
					
				\subsection{Preliminary Design Review (PDR)}
					PDR або попередня перевірка дизайну проводиться з метою оцінювання технічної адекватності поверхневого дизайну.
					
				\subsection{Critical Design Review (CDR)}
					CRD або детальна перевірка дизайну проводиться для більш детальної оцінки технічної якості дизайну.
					
			\section{Повідомлення про проблеми та виправні дії}
				Для забезпечення повідомлення, відстеження та вирішення проблем використовується інструменти платформи GitLab, що розгортається на обладнанні організації розробника.
					
	\part{Статичний аналіз якості ПЗ}
		\section{Результати}
			За результатами виконання статичного аналізу проекту~\progname , а саме детального дослідження проектної документації, програмного забезпечення та іншої інформації про програмний продукт, не прибігаючи до його виконання, недоліки не були виявлені.
\end{document}