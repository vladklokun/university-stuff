\documentclass[a4paper,oneside,DIV=12,12pt]{scrartcl}


\usepackage{fontspec}
\setmainfont{PT Serif}
\setsansfont{PT Sans}
\setmonofont{PT Mono}

\usepackage{microtype}

\usepackage{polyglossia}
\setmainlanguage{ukrainian}

\usepackage{enumitem,calc}
\DeclareDocumentEnvironment{steps}%
{O{}}% If no argument is given the label defaults to 'Step'
{\begin{enumerate}[leftmargin = *]}% Tune labelindent to set hanging step number
{\end{enumerate}}

\usepackage{minted}

\makeatletter
\@addtoreset{section}{part}
\makeatother

\newcommand\testplanid{\texttt{INAP-TST-MASTER-A47E40}}
\newcommand\projname{\texttt{INAP}}
\newcommand\filename[1]{\texttt{#1}}
\newcommand\error[1]{\texttt{#1}}

\begin{document}
    \begin{titlepage}
    \begin{center}
	Міністерство освіти і науки України\\
	Національний авіаційний університет\\
	Навчально-науковий інститут комп'ютерних інформаційних технологій\\
	Кафедра комп'ютеризованих систем управління

	\vspace{\fill}

	Лабораторна робота №5\\
	з дисципліни «Якість програмного забезпечення та тестування»\\
	на тему «Автоматизація процесу тестування»

	\vspace{\fill}
	
	\begin{flushright}
				Виконав:\\
				студент ННІКІТ СП-225\\
				Клокун Владислав\\
				Перевірила:\\
				Апенько Н. В.
	\end{flushright}

	Київ 2017

    \end{center}
    \end{titlepage}
	
	%\section{Звіт про помилки}
	\part{Програма та тести}
	\section{Код програми}
		%\begin{mdframed}
			\inputminted[
				mathescape,
				tabsize=4,
				linenos,
				breaklines,
				breakbytokenanywhere,
				breaksymbol={},
				style=bw,
				fontsize=\small
				]{python}{../01-solution/address_book.py}
		%\end{mdframed}
		
	\section{Код тестів}
		%\begin{mdframed}
			\inputminted[
				mathescape,
				tabsize=4,
				linenos,
				breaklines,
				breakbytokenanywhere,
				breaksymbol={},
				style=bw,
				fontsize=\small
				]{python}{../01-solution/test_address_book.py}
		%\end{mdframed}
		
	\newpage
	\part{Звіт про помилки}
	\section{Повідомлення про помилку під час спроби перегляду адресної книги, якщо вона не існує}
		
		\subsection{Короткий опис}
				Якщо видалити файл адресної книги спробувати запустити програму для виведення змісту адресної книги, програма повідомляє про помилку.
				
		\subsection{Детальний опис}
				Якщо в директорії відсутній файл адресної книги \filename{address-book}, то при запуску програми командою \verb+address-book.py -p+ виводиться така помилка: «\error{OS error: [Errno 2] No such file or directory: 'address-book'}».
				
		\subsection{Кроки для відтворення}
			\begin{steps}
				\item Видалити, перемістити або перейменувати файл \filename{address-book} у папці, що містить програму.
				
				\item Запустити програму для перегляду адресної книги (\verb+address-book.py -p+).
				
				\item Спостерігати результат.
			\end{steps}
			
		\subsection{Відтворюваність}
			Проблема спостерігається завжди.
			
		\subsection{Важливість}
			S4 — Незначна.
			
		\subsection{Статус}
			Вирішений.
			
		\subsection{Резолюція}
			Не буде виправлений (WONTFIX): очікувана поведінка. У разі відсутності файла адресної книги користувачу виводиться повідомлення про помилку і програма нормально завершує роботу.
			
			Перед тим, як переглядати адресну книгу, користувач повинен занести туди хоча б один запис.
\end{document}