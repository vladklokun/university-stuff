\documentclass[a4paper,oneside,DIV=12,12pt]{scrartcl}

\usepackage{fontspec}
\setmainfont{PT Serif}
\setsansfont{PT Sans}
\setmonofont{PT Mono}

\usepackage{microtype}

\usepackage{polyglossia}
\setmainlanguage{ukrainian}

%Table typesetting
\usepackage{booktabs}
\usepackage{longtable}
\usepackage{array}
\newcolumntype{x}[1]{>{\raggedright\arraybackslash\hspace{0pt}}p{#1}}
\renewcommand{\arraystretch}{1.4}

\newcommand\testplanid{\texttt{INAP-TST-MASTER-A47E40}}
\newcommand\projname{\texttt{INAP}}

\begin{document}
    \begin{titlepage}
    \begin{center}
	Міністерство освіти і науки України\\
	Національний авіаційний університет\\
	Навчально-науковий інститут комп'ютерних інформаційних технологій\\
	Кафедра комп'ютеризованих систем управління

	\vspace{\fill}

	Лабораторна робота №2\\
	з дисципліни «Якість програмного забезпечення та тестування»\\
	на тему «Написання та оформлення тестового плану»

	\vspace{\fill}
	
	\begin{flushright}
				Виконав:\\
				студент ННІКІТ СП-225\\
				Клокун Владислав\\
				Перевірила:\\
				Апенько Н. В.
	\end{flushright}

	Київ 2017

    \end{center}
    \end{titlepage}

	\noindent\testplanid. Версія 28~вересня 2017~року. Усі питання за змістом документа надсилати за адресою: \texttt{john@example.com}.
	
    \section{Введення}
		\testplanid --- мастер-план тестування, розроблений для визначення деталей проведення тестових робіт для проекту під назвою \projname. Даний план складений за принципами, що викладені у вищих проектних документах (табл. \ref{tab:referencedoc}).
	
		\begin{table}[h]
			\label{tab:referencedoc}
			\centering
			\begin{tabular}{x{0.3\textwidth}x{0.4\textwidth}}
				\toprule
				Тип документу & Ідентифікатор \\
				\midrule
				Технічне завдання & \texttt{INAP-PRD-CA5B62}\\
				План проекту & \texttt{INAP-GENPLAN-C0C6DB}\\
				Внутрішні стандарти розробки проектів & \texttt{INT-PROJGUIDE-642F1D}\\
				\bottomrule
			\end{tabular}
			\caption{Документи, на які спирається \testplanid}
		\end{table}
    \section{Вимоги, що тестуються}
		Під час проведення тестування цільового програмного забезпечення необхідно протестувати такі процеси програмного продукту:
		\begin{enumerate}
			\item Розповсюдження.
			\item Встановлення.
			\item Робота.
		\end{enumerate}
		
    \section{Вимоги, що не тестуються}
		Під час проведення тестування цільового програмного забезпечення \emph{не} тестуються такі процеси програмного продукту:
		\begin{enumerate}
			\item Швидкодія: неможливо визначити чітку метрику через велику різноманітність систем, на яких має працювати програмний продукт. Крім того, враховуючи невеликий передбачений розмір програмного продукту та швидкодію сучасних апаратних систем, доцільність аналізу критерію сумнівна.
			\item Робота використаних системних викликів: за умови коректного використання, не тестується інформація та результати виконання системних викликів, оскільки підтримкою таких засобів займаються спеціалізовані команди відповідних операційних систем.
		\end{enumerate}
    \section{Методи тестування}
		\subsection{Розповсюдження}
			Розповсюдження програми --- це процес, що описує появу ресурсів, що необхідні для встановлення цільового програмного продукту, на апаратному забезпеченні кінцевого користувача. Згідно з \texttt{INAP-PRD-CA5B62}, програмний продукт розповсюджується у вигляді початкового коду та конфігураційних файлів, які використовуються для збору проекту користувачем.
			
			Тестування процесу розповсюдження програмного продукту полягає в аналізі активного робочого часу серверу системи контролю версій, яка надає користувачу файли після відповідного запиту. Такий робочий час називатимемо «аптаймом» (англ. \emph{uptime}).
			
			Даний етап тестування вважається успішним, коли аптайм перевищує 99\% від усього очікуваного часу роботи системи розповсюдження.
			
		\subsection{Встановлення}
			Встановлення програми --- це процес перетворення початкових (вихідних) ресурсів встановлення програми у робочий програмний продукт. Встановлення продукту проводиться за допомогою виконання інструкцій, описаних у спеціальних файлах --- Makefile (за текстом \texttt{INAP-PRD-CA5B62}).
			
			Тестування процесу встановлення програмного продукту полягає у спробі виконання процесу встановлення на різноманітних комбінаціях апаратних систем та їх програмного оточення.
			
			Даний етап тестування вважається успішним, коли програмний продукт успішно встановлюється на усіх наявних тестових стендах, описаних у розділі \ref{sec:testenvreq}).
			
		\subsection{Робота}
			Робота --- це процес безпосереднього виконання програмним продуктом функцій, закладених в нього. Суть роботи продукту полягяє у генерації паролів для користувачів за допомогою ентропії, наданої системними викликами, та обробки заданих текстових файлів, що містять словники, з елементів яких складається пароль. Тому, тестування роботии програмного продукту поділяється на декілька етапів.
			
			\subsection{Тестування обробки словників}
				Першим етапом є тестування обробки словників. Обробка словників має \emph{високий ризик}, оскільки вони є довільними файлами, які можуть бути змінені користувачами. Таким чином, виникає велика вірогідність помилок, що мають фатальні наслідки: пошкодження пам'яті тощо.
				
				Для виконання тестування обробки словників перш за все використовуються спеціально сформульовані набори тестових даних, направлені на перевірку роботи логіки програмного продукту у крайніх випадках: пустих файлах, завеликих файлах, файлах, до яких відсутній доступ, пошкоджених файлів тощо. Розробкою таких наборів займається відповідальний тестувальник (див. розділ \ref{sec:resourcesreq}).
			\subsection{Тестування обробки ентропії}
				Тестування математичного підгрунтя при генерації рівномірно розподілених випадкових елементів.
				
			\subsection{Тестування обробки параметрів}
				Тестування обробки параметрів командного рядка.
				
    \section{Вимоги до середовища тестування}
		\label{sec:testenvreq}
		Windows 7, Windows 8.1, Windows 10; macOS; Debian.
    \section{Необхідні ресурси}
		\label{sec:resourcesreq}
		\begin{table}[h]
			\centering
			\begin{tabular}{x{0.2\textwidth}x{0.25\textwidth}x{0.4\textwidth}}
				\toprule
				Позиція & Вимоги & Завдання \\
				\midrule
				Головний тестувальник & Не менше 5~років стажу & Керування процесом тестування на відповідність даному мастер-плану. \\
				Розробник наборів тестових даних & Розробив набори тестових даних як мінімум для 1 проекту & Розробка тестових наборів для тестування правильності обробки словників. \\
				\bottomrule
			\end{tabular}
		\end{table}
    \section{Етапи тестування}
    \section{Критерії тестування}
\end{document}
