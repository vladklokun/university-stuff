\documentclass[a4paper,oneside,DIV=12,12pt]{scrartcl}

\usepackage{fontspec}
\setmainfont{PT Serif}
\setsansfont{PT Sans}
\setmonofont{PT Mono}

\usepackage{microtype}

\usepackage{polyglossia}
\setmainlanguage{ukrainian}

%Table typesetting
\usepackage{booktabs}
\usepackage{longtable}
\usepackage{array}
\newcolumntype{v}[1]{>{\raggedright\arraybackslash\hspace{0pt}}p{#1}}
\newcolumntype{n}[1]{>{\raggedleft\arraybackslash\hspace{0pt}}p{#1}}
\renewcommand{\arraystretch}{1.4}

\newcommand\testplanid{\texttt{INAP-TST-MASTER-A47E40}}
\newcommand\projname{\texttt{INAP}}

\begin{document}
    \begin{titlepage}
    \begin{center}
	Міністерство освіти і науки України\\
	Національний авіаційний університет\\
	Навчально-науковий інститут комп'ютерних інформаційних технологій\\
	Кафедра комп'ютеризованих систем управління

	\vspace{\fill}

	Лабораторна робота №2\\
	з дисципліни «Якість програмного забезпечення та тестування»\\
	на тему «Написання та оформлення тестового плану»

	\vspace{\fill}
	
	\begin{flushright}
				Виконав:\\
				студент ННІКІТ СП-225\\
				Клокун Владислав\\
				Перевірила:\\
				Апенько Н. В.
	\end{flushright}

	Київ 2017

    \end{center}
    \end{titlepage}

	\noindent\testplanid. Версія 28~вересня 2017~року. Усі питання за змістом документа надсилати за адресою: \texttt{john@example.com}.
	
    \section{Введення}
		\testplanid --- мастер-план тестування, розроблений для визначення деталей проведення тестових робіт для проекту під назвою \projname. Даний план складений за принципами, що викладені у вищих проектних документах (табл. \ref{tab:referencedoc}).
	
		\begin{table}[h]
			\label{tab:referencedoc}
			\centering
			\begin{tabular}{v{0.3\textwidth}v{0.4\textwidth}}
				\toprule
				Тип документу & Ідентифікатор \\
				\midrule
				Технічне завдання & \texttt{INAP-PRD-CA5B62}\\
				План проекту & \texttt{INAP-GENPLAN-C0C6DB}\\
				Внутрішні стандарти розробки проектів & \texttt{INT-PROJGUIDE-642F1D}\\
				\bottomrule
			\end{tabular}
			\caption{Документи, на які спирається \testplanid}
		\end{table}
    \section{Вимоги, що тестуються}
		Під час проведення тестування цільового програмного забезпечення необхідно протестувати такі процеси програмного продукту:
		\begin{enumerate}
			\item Розповсюдження.
			\item Встановлення.
			\item Робота.
		\end{enumerate}
		
    \section{Вимоги, що не тестуються}
		Під час проведення тестування цільового програмного забезпечення не тестуються такі процеси програмного продукту:
		\begin{enumerate}
			\item Швидкодія: неможливо визначити чітку метрику через велику різноманітність систем, на яких має працювати програмний продукт. Крім того, враховуючи невеликий передбачений розмір програмного продукту та швидкодію сучасних апаратних систем, доцільність аналізу критерію сумнівна.
			\item Робота використаних системних викликів: за умови коректного використання, не тестується інформація та результати виконання системних викликів, оскільки підтримкою таких засобів займаються спеціалізовані команди відповідних операційних систем.
		\end{enumerate}
		
		\subsection{Розповсюдження}
			Розповсюдження програми --- це процес, що описує появу ресурсів, що необхідні для встановлення цільового програмного продукту, на апаратному забезпеченні кінцевого користувача. Згідно з \texttt{INAP-PRD-CA5B62}, програмний продукт розповсюджується у вигляді початкового коду та конфігураційних файлів, які використовуються для збору проекту користувачем.
			
			Тестування процесу розповсюдження програмного продукту полягає в аналізі активного робочого часу серверу системи контролю версій, яка надає користувачу файли після відповідного запиту. Такий робочий час називатимемо «аптаймом» (англ. \emph{uptime}).
			
			Даний етап тестування вважається успішним, коли аптайм перевищує 99\% від усього очікуваного часу роботи системи розповсюдження.
			
		\subsection{Встановлення}
			Встановлення програми --- це процес перетворення початкових (вихідних) ресурсів встановлення програми у робочий програмний продукт. Встановлення продукту проводиться за допомогою виконання інструкцій, описаних у спеціальних файлах --- Makefile (за текстом \texttt{INAP-PRD-CA5B62}).
			
			Тестування процесу встановлення програмного продукту полягає у спробі виконання процесу встановлення на різноманітних комбінаціях апаратних систем та їх програмного оточення.
			
			Даний етап тестування вважається успішним, коли програмний продукт успішно встановлюється на усіх наявних тестових стендах, описаних у розділі \ref{sec:testenvreq}).
			
		\subsection{Робота}
			Робота --- це процес безпосереднього виконання програмним продуктом функцій, закладених в нього. Суть роботи продукту полягає у генерації паролів для користувачів за допомогою ентропії, наданої системними викликами, та обробки заданих текстових файлів, що містять словники, з елементів яких складається пароль. Тому, тестування роботи програмного продукту поділяється на декілька етапів.
			
			\subsubsection{Тестування обробки словників}
				Першим етапом є тестування обробки словників. Обробка словників має високий ризик, оскільки вони є довільними файлами, які можуть бути змінені користувачами. Таким чином, виникає велика вірогідність помилок, що мають фатальні наслідки: пошкодження пам'яті тощо.
				
				Для виконання тестування обробки словників перш за все використовуються спеціально сформульовані набори тестових даних, направлені на перевірку роботи логіки програмного продукту у крайніх випадках: пустих файлах, завеликих файлах, файлах, до яких відсутній доступ, пошкоджених файлів тощо. Розробкою таких наборів займається відповідальний тестувальник (див. розділ \ref{sec:resourcesreq}).
			\subsubsection{Тестування обробки ентропії}
				Тестування обробки ентропії полягає в перевірці математичного підґрунтя при генерації рівномірно розподілених випадкових елементів.
				
			\subsubsection{Тестування обробки параметрів}
				Тестування обробки параметрів полягає у перевірці коректності встановлення та обробки параметрів командного рядка, що передаються програмі.
				
    \section{Методи тестування}
		Тестування даного продукту передбачає перевірку теоретичних основ та практичного процесу роботи програми. Тому під час тестування використовується автоматичне і ручне тестування.
		
		Автоматичне тестування проводиться з використанням автоматизованих процедурних засобів, що виконують автоматичну перевірку роботи програми, знаходять в ній помилки і неточності. До таких засобів відносяться засоби налагоджування (debug), статичного аналізу та тестувальні фреймворки, які перевіряють роботу програми на сформованих тест-кейсах.
		
		Ручне тестування полягає у перевірці певних критерії кваліфікованим працівником, наприклад, перевірки математичного підґрунтя роботи програми, ручний пошук помилок у програмі тощо.
		
    \section{Вимоги до середовища тестування}
		\label{sec:testenvreq}
		Оскільки цільовий продукт повинен бути кросплатформеним, то його тестування необхідно проводити на різних платформах з різними конфігураціями.
		
		Підтримка продукту запланована x86-сумісних системах під управлінням таких операційних систем:
		\begin{enumerate}
			\item Windows:
				\begin{enumerate}
					\item 7;
					\item 8.1;
					\item 10.
				\end{enumerate}
			\item macOS:
				\begin{enumerate}
					\item Yosemite;
					\item El Capitan;
					\item Sierra;
					\item High Sierra.
				\end{enumerate}
			\item Linux:
				\begin{enumerate}
					\item Debian Stretch;
					\item Fedora 26;
					\item Void Linux.
				\end{enumerate}
		\end{enumerate}
		
		Таким чином середовище тестування повинно включати в себе x86-сумісні конфігурації під управлінням кожної з операційних систем, перерахованих вище.
		
    \section{Необхідні ресурси}
		\label{sec:resourcesreq}
		
		Необхідні ресурси поділяються на людські та матеріальні. Деталі вимог до людських ресурсів наведені у табл.~\ref{tab:human-resources-requirements}.
		
		\begin{table}[h]
			\centering
			\begin{tabular}{v{0.2\textwidth}v{0.25\textwidth}v{0.4\textwidth}}
				\toprule
				Позиція & Вимоги & Завдання \\
				\midrule
				Головний тестувальник &
				Не менше 5~років стажу. &
				Керування процесом тестування на відповідність даному мастер-плану. \\
				
				Консультант-математик &
				Високий рівень знань в області криптографії, випадкових чисел та теорії обчислювальної складності. &
				Перевірка теоретичної правильності роботи програми. \\
				
				Розробник наборів тестових даних &
				Розробив набори тестових даних як мінімум для одного проекту. &
				Розробка тестових наборів для тестування правильності обробки словників. \\

				Тестувальник &
				Навички ефективного використання інструментів, що використовуються під час тестування даного продукту. &
				Нагляд та оформлення результатів автоматичного тестування\\
				
				Системний адміністратор &
				Навички ефективної роботи з обладнанням, що використовується в середовищі тестування. &
				Розгортання, моніторинг та забезпечення правильної роботи середовища тестування. \\
				\bottomrule
			\end{tabular}
		\caption{Вимоги до людських ресурсів}
		\label{tab:human-resources-requirements}
		\end{table}
		
		До матеріальних ресурсів відноситься апаратне забезпечення, необхідне для успішного виконання тестування. Для успішного виконання тестування необхідно обладнання, що відповідає вимогам, описаним у розділі~\ref{sec:testenvreq}. Крім того, необхідне обладнання, що забезпечить роботи інфраструктури для роботи команди тестування (git-репозиторій, система сумісної роботи тощо).
		
    \section{Етапи тестування}
		\label{sec:test-schedule}
		
		Процес тестування цільового продукту поділяється на етапи, які вирішують питання організації процесу тестування, безпосереднього проведення тестування необхідних вимог та оформлення результатів тестування. Деталі проведення тестування наведені у табл.~\ref{tab:test-schedule}.
		
		\begin{table}[!htbp]
		\centering
			\begin{tabular}{v{0.4\textwidth}n{0.5\textwidth}}
				\toprule
					Етап тестування & Тривалість\\
				\midrule
					Тестування математичного підґрунтя &
					2017-01-23~—~2017-03-01\\
					Налаштування середовища тестування &
					2017-03-02~—~2017-03-09\\
					Тестування розповсюдження програми &
					2017-03-10~—~2017-03-17\\
					Тестування встановлення програми &
					2017-03-18~—~2017-03-25\\
					Тестування роботи програми &
					2017-03-26~—~2017-04-14\\
					Оформлення результатів тестування &
					2017-04-15~—~2017-04-29\\
					Завершення тестування &
					2017-04-30~—~2017-05-10\\
				\bottomrule
			\end{tabular}
		\caption{Розклад проведення тестування}
		\label{tab:test-schedule}
		\end{table}
		
    \section{Критерії тестування}
		\label{sec:test-criteria}
		
		\subsection{Критерії початку тестування}
			Для початку тестування необхідно виконання таких критеріїв:
			\begin{enumerate}
				\item Завершеність розробки необхідного функціоналу.
				\item Наявність проектної документації
				\item Готовність середовища тестування.
			\end{enumerate}
			
		\subsection{Критерії завершення тестування}
			Тестування завершується при відповідності таким критеріям:
			\begin{enumerate}
				\item Усі етапи, описані в розділі~\ref{sec:test-schedule}, виконані успішно.
				\item Кількість відкритих багів не~більше~5.
				\item Тривалість періоду без виявлення нових багів не менше двох тижнів.
			\end{enumerate}
			
\end{document}
