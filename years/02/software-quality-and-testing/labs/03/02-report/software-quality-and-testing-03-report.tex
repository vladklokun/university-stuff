\documentclass[a4paper,oneside,DIV=12,12pt]{scrartcl}

\usepackage{fontspec}
\setmainfont{PT Serif}
\setsansfont{PT Sans}
\setmonofont{PT Mono}

\usepackage{microtype}

\usepackage{polyglossia}
\setmainlanguage{ukrainian}

%Table typesetting
\usepackage{booktabs}
\usepackage{longtable}
\usepackage{array}
\newcolumntype{v}[1]{>{\raggedright\arraybackslash\hspace{0pt}}p{#1}}
\newcolumntype{b}[1]{>{\centering\arraybackslash\hspace{0pt}}p{#1}}
\newcolumntype{n}[1]{>{\raggedleft\arraybackslash\hspace{0pt}}p{#1}}
\renewcommand{\arraystretch}{1.4}

\newcommand\testplanid{\texttt{MFS-TST-MASTER-A47E40}}
\newcommand\projname{\texttt{MFS}}

\newcommand{\program}[1]{\texttt{#1}}
\newcommand{\rawdata}[1]{\texttt{#1}}

\usepackage{unicode-math}
\setmathfont{PT Serif}


\begin{document}
	\begin{titlepage}
    \begin{center}
	Міністерство освіти і науки України\\
	Національний авіаційний університет\\
	Навчально-науковий інститут комп'ютерних інформаційних технологій\\
	Кафедра комп'ютеризованих систем управління

	\vspace{\fill}

	Лабораторна робота №3\\
	з дисципліни «Якість програмного забезпечення та тестування»\\
	на тему «Написання тестових випадків (test case) згідно плану тестування»

	\vspace{\fill}
	
	\begin{flushright}
				Виконав:\\
				студент ННІКІТ СП-225\\
				Клокун Владислав\\
				Перевірила:\\
				Апенько Н. В.
	\end{flushright}

	Київ 2017

    \end{center}
    \end{titlepage}
	
	\noindent\testplanid. Версія 28~вересня 2017~року. Усі питання за змістом документа надсилати за адресою: \texttt{john@example.com}.
	
    \section{Введення}
		\testplanid --- мастер-план тестування, розроблений для визначення деталей проведення тестових робіт для проекту під назвою \projname. Даний план складений за принципами, що викладені у вищих проектних документах (табл. \ref{tab:referencedoc}).
	
		\begin{table}[h]
			\label{tab:referencedoc}
			\centering
			\begin{tabular}{v{0.3\textwidth}v{0.4\textwidth}}
				\toprule
				Тип документу & Ідентифікатор \\
				\midrule
				Технічне завдання & \texttt{MFS-PRD-CA5B62}\\
				План проекту & \texttt{MFS-GENPLAN-C0C6DB}\\
				Внутрішні стандарти розробки проектів & \texttt{INT-PROJGUIDE-642F1D}\\
				\bottomrule
			\end{tabular}
			\caption{Документи, на які спирається \testplanid}
		\end{table}
		
	\section{Вимоги, що тестуються}
		Під час проведення тестування цільового програмного забезпечення тестується виключно коректність роботи програми на можливих введених даних.
		
	\section{Вимоги, що не тестуються}
		Під час проведення тестування цільового програмного забезпечення не тестуються такі вимоги:
		\begin{enumerate}
			\item Розповсюдження.
			\item Встановлення.
			\item Швидкодія.
		\end{enumerate}
		
	\section{Методи тестування}
		Для проведення тестування використовуються засоби автоматичного тестування, що включені у використану мову програмування.
		Тестування виконується за допомогою спеціально сформованих тест-кейсів.
		
	\section{Вимоги до середовища тестування}
		\label{sec:testing-environment}
		
		Середовище тестування повинно мати налагоджене середовище збірки проектів, написаних мовою програмування Rust, зокрема \program{cargo} та \program{rustc}.
		
		
	\section{Необхідні ресурси}
		Для проведення тестування необхідно робоче середовище тестування (див. розділ~\ref{sec:testing-environment}) та одна особа на роль тестувальника.
		
	\section{Етапи тестування}
		Графік проведення тестування наведений у табл.~\ref{tab:testing-schedule}.
		
		\begin{table}[!htbp]
		\centering
			\begin{tabular}{v{0.3\textwidth}n{0.4\textwidth}}
				\toprule
					Етап тестування & Тривалість\\
				\midrule
					Налаштування середовища тестування &
					2017-03-02~—~2017-03-09\\
					Тестування роботи програми &
					2017-03-26~—~2017-04-14\\
					Оформлення результатів тестування &
					2017-04-15~—~2017-04-29\\
					Завершення тестування &
					2017-04-30~—~2017-05-10\\
				\bottomrule
			\end{tabular}
		\caption{Графік проведення тестування}
		\label{tab:testing-schedule}
		\end{table}
		
	\section{Критерії тестування}
		Тестування завершується, коли розроблений продукт успішно проходить написані тест-кейси.
		
	\appendix
	\section{Тест кейси}
		Опис тест-кейсів наведений у табл.~\ref{tab:test-cases-description}.
		
		\begin{table}[!htbp]
		\centering
			\begin{tabular}{v{0.6\textwidth}v{0.15\textwidth}n{0.15\textwidth}}
				\toprule
					Дія & Очікуваний результат & Результат тесту\\
				\midrule
					Пошук найчастішого знаку у наборі \linebreak $[]$ &
					\rawdata{'0'} &
					Passed.
					\\
					Пошук найчастішого знаку у наборі \linebreak $\left[1,2,3,4,5 \right]$ &
					\rawdata{'+'} &
					Passed.
					\\
					Пошук найчастішого знаку у наборі $[-5, -4, -3, -2, -1]$ &
					\rawdata{'-'} &
					Passed.
					\\
					Пошук найчастішого знаку у наборі $[-1, 1, -2, 2, -3, 3]$ &
					\rawdata{'-'} &
					Passed.
					\\
				\bottomrule
			\end{tabular}
		\caption{Опис тест-кейсів}
		\label{tab:test-cases-description}
		\end{table}
		
\end{document}