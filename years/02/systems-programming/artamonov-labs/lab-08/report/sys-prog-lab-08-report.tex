\documentclass[a4paper,oneside,DIV=12,12pt]{scrartcl}

\usepackage{fontspec}
\setmainfont{STIX Two Text}
\setsansfont{Roboto}
\setmonofont{PT Mono}

\usepackage{microtype}

\usepackage{polyglossia}
\setmainlanguage{ukrainian}

\usepackage{unicode-math}
\setmathfont{STIX Two Math}

%%% Source code listing
\usepackage{listings}
\lstset{%
	basicstyle = \small\ttfamily,%
	breaklines = true,%
	captionpos = b,%
	language = python,%
}

\renewcommand{\lstlistingname}{Лістинг}
%%%

%%% Caption setup (workaround for wrong caption style for listing)
\usepackage{caption}
\captionsetup[lstlisting]{font={rm}}
%%%

\begin{document}
	\begin{titlepage}
	\centering
		Міністерство освіти і науки України\\
		Національний авіаційний університет\\
		Навчально-науковий інститут комп'ютерних інформаційних технологій\\
		Кафедра комп'ютеризованих систем управління

		\vspace{\fill}

		Лабораторна робота №8\\
		з дисципліни «Системне програмування»\\
		на тему «Керування відеоадаптером»

		\vspace{\fill}
		
		\begin{flushright}
			Виконав:\\
			студент ННІКІТ СП-225\\
			Клокун Владислав\\
			Перевірив:\\
			Артамонов Є.~Б.
		\end{flushright}

		Київ 2017
    \end{titlepage}
	
	\section{Мета роботи}
		Вивчення особливостей функціонування відеосистеми в текстовому режимі й одержання практичних навичок роботи з відеомонітором у цьому режимі.
		
	\section{Постановка задачі}
		Одержати на екрані оригінальний, бажано динамічний відеоефект. 
		
	\section{Розв'язання}
		Під час виконання лабораторної роботи була розроблена програма, яка показує коло, що відбивається від границь вікна, в якому вона знаходиться. Початковий код розробленої програми наведений у~лістингу~\ref{lst:lab-08-listing}.
		
		\lstinputlisting[
			caption = {Початковий код розробленої програми},%
			label = {lst:lab-08-listing},%
		]
		{../solution/lab-08.py}
	
\end{document}