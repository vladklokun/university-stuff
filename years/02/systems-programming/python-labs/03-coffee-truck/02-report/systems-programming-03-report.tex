\documentclass[a4paper,oneside,DIV=12,12pt]{scrartcl}

\usepackage{fontspec}
\setmainfont{PT Serif}
\setsansfont{PT Sans}
\setmonofont{PT Mono}

\usepackage{microtype}

\usepackage{polyglossia}
\setmainlanguage{ukrainian}

\usepackage{minted}

\usepackage{mdframed}
\usepackage{xcolor}
\mdfsetup{%
%backgroundcolor = black!1,%
linecolor = black!0,%
%innerleftmargin = 0.5em, innerrightmargin = 0.5em,innertopmargin = 0.5em, %innerbottommargin = 0.5em%
}

\begin{document}
	\begin{titlepage}
    \begin{center}
	Міністерство освіти і науки України\\
	Національний авіаційний університет\\
	Навчально-науковий інститут комп'ютерних інформаційних технологій\\
	Кафедра комп'ютеризованих систем управління

	\vspace{\fill}

	Лабораторна робота №3\\
	з дисципліни «Системне програмування»\\
	на тему «Класи»

	\vspace{\fill}
	
	\begin{flushright}
				Виконав:\\
				студент ННІКІТ СП-225\\
				Клокун Владислав\\
				Перевірив:\\
				Радченко П.~В.
	\end{flushright}

	Київ 2017

    \end{center}
    \end{titlepage}
	
	\section{Завдання}
		Припустимо, що є вантажівка, у якій знаходяться коробки, кожна з яких містить кавові боби певного сорту. Використовуючи класи, підрахувати загальну вартість усієї кави у вантажівці.
		
	\section{Розв'язання}
		%\begin{listing}[htbp]
			\begin{mdframed}
			\inputminted[
				mathescape,
				tabsize=4,
				linenos,
				breaklines,
				breakbytokenanywhere,
				breaksymbol={},
				style=bw,
				%fontsize=\small
				]{python}{../01-code/03-coffee-truck.py}
			\end{mdframed}
		%\caption{Початковий код програми «Хрестики-нулики»}
		%\end{listing}
		
\end{document}