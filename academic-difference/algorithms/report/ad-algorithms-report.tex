\documentclass[a4paper,oneside,DIV=12,12pt]{scrartcl}

\usepackage{graphicx}

%%% Load fonts
\usepackage{fontspec}
\setromanfont[
	SmallCapsFeatures = {LetterSpace = 5}
]{STIX Two Text}
\setsansfont{Roboto}
\setmonofont{PT Mono}

\usepackage{amsmath,unicode-math}
\setmathfont{STIX Two Math}
%%%

%%% Language-specific setup
\usepackage{polyglossia}
\setmainlanguage{ukrainian}
%%%

%%% Microtypographic enhancements
\usepackage{microtype}
%%%

%%% Table typesetting
\usepackage{booktabs}
\usepackage{longtable}
\usepackage{array}
\newcolumntype{v}[1]{>{\raggedright\arraybackslash\hspace{0pt}}p{#1}}
\newcolumntype{n}[1]{>{\raggedleft\arraybackslash\hspace{0pt}}p{#1}}
% \renewcommand{\arraystretch}{1.4}
%%%

%%% Math typesetting
\usepackage{ieeetrantools}

\usepackage{systeme}
%%%

%%% Problem-solution typesetting
\usepackage{xsim}
\loadxsimstyle{runin}
\DeclareExerciseTranslations{exercise}{
	Ukrainian	=	завдання ,
}

\DeclareExerciseTranslations{solution}{
	Ukrainian	=	розв'язання ,
}

\xsimsetup{
	solution/print = true,
	exercise/template = runin,
	solution/template = runin,
}
%%%

%%% Steps environment
\usepackage{enumitem,calc}
\DeclareDocumentEnvironment{steps}%
{O{}}% If no argument is given the label defaults to 'Step'
{\begin{enumerate}[leftmargin = *]}% Tune labelindent to set hanging step number
{\end{enumerate}}
%%%

%%%
\newcommand{\sheetno}[1]{{\centering\scshape\bfseries Білет~№{#1}\par}}
\newcommand{\subproblem}[1]{\textit{#1}.}
%%%

\begin{document}
	\begin{titlepage}
		\begin{center}
			Міністерство освіти і науки України\\
			Національний авіаційний університет\\
			Навчально-науковий інститут комп'ютерних інформаційних технологій\\
			Кафедра комп'ютеризованих систем управління
			
			\vspace{\fill}
				Робота\\
				для ліквідації академічної різниці\\
				з дисципліни «Алгоритми та методи обчислень»\\
				Варіант 3
				
			\vspace{\fill}
			
			\begin{flushright}
				Виконав:\\
				студент ННІКІТ СП-225\\
				Клокун Владислав\\
			\end{flushright}
			Київ~2017
		\end{center}
	\end{titlepage}
	
	\begin{exercise}
		Дано масив цілих чисел $A = \left\{ 3, 1, -13, 8, 15, 63,21, 17, -4, 2\right\}$. Необхідно відсортувати його в порядку зростання методами вставки, бульбашки та методом шейкера. Намалювати блок-схему для методу шейкера.
	\end{exercise}
	
	\begin{solution}
		\subproblem{Сортування бульбашкою}
			За один повний цикл проходу по масиву виконує порівняння одного елемента з іншим. Якщо вони знаходяться в неправильному порядку, міняє їх місцями.
			
			\begin{steps}
				\item $A = [3, 1, -13, 8, 15, 63, 21, 17, -4, 2]$.
				\item $A = [1, -13, 3, 8, 15, 21, 17, -4, 2, 63]$.
				\item $A = [-13, 1, 3, 8, 15, 17, -4, 2, 21, 63]$.
				\item $A = [-13, 1, 3, 8, -4, 2, 15, 17, 21, 63]$.
				\item $A = [-13, 1, 3, 8, -4, 2, 15, 17, 21, 63]$.
				\item $A = [-13, 1, 3, -4, 2, 8, 15, 17, 21, 63]$.
				\item $A = [-13, 1, -4, 2, 3, 8, 15, 17, 21, 63]$.
				\item $A = [-13, -4, 1, 2, 3, 8, 15, 17, 21, 63]$.
			\end{steps}
			
		\subproblem{Сортування вставками}
			На кожному кроці обираємо один з елементів і вставляємо його на потрібну позицію, доки вхідні дані не закінчаться.
			\begin{steps}
				\item $A = [3, 1, -13, 8, 15, 63, 21, 17, -4, 2]$.  
				\item $A = [-13, 1, 3, 8, 15, 63, 21, 17, -4, 2]$.
				\item $A = [-13, -4, 3, 8, 15, 63, 21, 17, 1, 2]$.
				\item $A = [-13, -4, 1, 8, 15, 63, 21, 17, 3, 2]$.
				\item $A = [-13, -4, 1, 2, 15, 63, 21, 17, 3, 8]$.
				\item $A = [-13, -4, 1, 2, 3, 63, 21, 17, 15, 8]$.
				\item $A = [-13, -4, 1, 2, 3, 8, 21, 17, 15, 63]$.
				\item $A = [-13, -4, 1, 2, 3, 8, 15, 17, 21, 63]$.
			\end{steps}
			
		\subproblem{Сортування методом шейкера}
			Двосторонній алгоритм сортування бульбашкою. 
			\begin{steps}
				\item $A = [3, 1, -13, 8, 15, 63, 21, 17, -4, 2]$.  
				\item $A = [1, -13, 3, 8, 15, 21, 17, -4, 2, 63]$.
				\item $A = [-13, 1, -4, 3, 8, 15, 21, 17, 2, 63]$.
				\item $A = [-13, -4, 1, 3, 8, 15, 17, 2, 21, 63]$.
				\item $A = [-13, -4, 1, 2, 3, 8, 15, 17, 21, 63]$.
			\end{steps}
			
	\end{solution}
	
	\begin{exercise}
		Довести, що задану програму можна застосовувати до наданих станів машини Поста. Вказати результат роботи машини Поста, якщо початковий стан стрічки: $111001$.
		
		\begin{longtable}[c]{lll}
			\endhead
			\endfoot
			1. $?~3;~2$. & 4. $?~6;~5$.       & 7. $?~8;~9$ \\ 
			2. $\to 1$.  & 5. $\leftarrow 1$. & 8. $?$ \\
			3. $\to 4$.  & 6. $\rightarrow 7$ & 9. $\rightarrow 4$
		\end{longtable}
		
	\end{exercise}
	
	\begin{solution}
		Припустимо, що каретка знаходиться на першій комірці. Розберемо програму покроково.
		
		\begin{longtable}[c]{llv{0.6\textwidth}}
			\toprule
				Команда  & Стан & Пояснення\\
			\midrule
			\endhead
			\bottomrule
			\endfoot
			
				$1~?~3;~2$          & 1110011    & Мітка є, отже перехід до~команди~2\\
				$2~\rightarrow 1$   & 1110011    & Зсув каретки вправо і перехід до~команди~1\\
				$1~?~3;~2$          & 1110011    & Мітка є, отже перехід до~команди~2\\
				$2~\rightarrow 1$   & 1110011    & Зсув каретки вправо і перехід до~команди~1\\
				$1~?~3;~2$          & 1110011    & Мітка є, отже перехід до~команди~2\\
				$2~\rightarrow 1$   & 1110011    & Зсув каретки вправо і перехід до~команди~1\\
				$1~?~3;~2$          & 1110011    & Мітки немає, отже перехід до~команди~3\\
				$3~\rightarrow 4$   & 1110011    & Зсув каретки вправо перехід до~команди~4\\
				$4~?~6;~5$          & 1110011    & Мітки немає отже перехід до~команди~6\\
				$6~\rightarrow 7$   & 1110011    & Зсув каретки вправо перехід до~команди~7\\
				$7~?~8;~9$          & 1110011    & Мітка є, отже перехід до~команди~9\\
				$9~\rightarrow 4$   & 1110011    & Зсув каретки вправо перехід до~команди~4\\
				$4~?~6;~5$          & 1110011    & Мітка є отже перехід до~команди~5\\
				$5~\leftarrow 1$    & 1110011    & Зсув каретки вліво перехід до~команди~1\\
				$1~?~3;~2$          & 1110011    & Мітка є, отже перехід до~команди~2\\
				$2~\rightarrow 1$   & 1110011    & Зсув каретки вправо і перехід до~команди~1\\
				$1~?~3;~2$          & 1110011    & Мітка є, отже перехід до~команди~2\\
				$2~\rightarrow 1$   & 1110011    & Зсув каретки вправо і перехід до~команди~1\\
				$1~?~3;~2$          & 11100110   & Мітки немає, отже перехід до~команди~3\\
				\multicolumn{3}{c}{\emph{Далі вважаємо, що справа нескінченно багато порожніх комірок}}\\
				$3~\rightarrow 4$   & 11100110   & Зсув каретки вправо, перехід до~команди~4\\
				$4~?~6;~5$          & 111001100  & Мітки немає, отже перехід до~команди~6\\
				$6~\rightarrow 7$   & 111001100  & Зсув каретки вправо, перехід до~команди~7\\
				$7~?~8;~9$          & 1110011000 & Мітки немає, отже перехід до~команди~8\\
				$8~!$               & 1110011000 & Завершення програми\\

		\end{longtable}
	\end{solution}
	
	\begin{exercise}
		Є 6 міст. Необхідно розвезти товар у кожне місто таким чином, що побувати у кожному місті лише один раз. Вартість перевезення із одного міста в інше відома і задана таблицею. Знайти найменшу ціну, за яку можна здійснити таке перевезення.
		
		\begin{longtable}[c]{lcccccc}
			\toprule
				Місто & 1 & 2 & 3 & 4 & 5 & 6 \\
			\midrule
			\endhead
			\bottomrule
			\endfoot
			
				1 & $\infty$ & $8$      & $3$      & $5$      & $7$      & $4$ \\
				2 & $3$      & $\infty$ & $7$      & $8$      & $9$      & $4$ \\
				3 & $4$      & $9$      & $\infty$ & $6$      & $8$      & $9$ \\
				4 & $9$      & $5$      & $8$      & $\infty$ & $4$      & $10$ \\
				5 & $2$      & $1$      & $15$     & $6$      & $\infty$ & $3$ \\
				6 & $10$     & $2$      & $5$      & $4$      & $9$      & $\infty$\\
		\end{longtable}
	\end{exercise}
	
	\begin{solution}
		Складемо таблицю відстаней, визначивши мінімальні значення в кожному рядку.
		
		\begin{longtable}[c]{lccccccr}
			\toprule
				Місто & 1 & 2 & 3 & 4 & 5 & 6 & $d_i$\\
			\midrule
			\endhead
			\bottomrule
			\endfoot
			
				1 & $\infty$ & $8$      & $3$      & $5$      & $7$      & $4$      & $3$\\
				2 & $3$      & $\infty$ & $7$      & $8$      & $9$      & $4$      & $3$\\
				3 & $4$      & $9$      & $\infty$ & $6$      & $8$      & $9$      & $4$\\
				4 & $9$      & $5$      & $8$      & $\infty$ & $4$      & $10$     & $4$\\
				5 & $2$      & $1$      & $15$     & $6$      & $\infty$ & $3$      & $1$\\
				6 & $10$     & $2$      & $5$      & $4$      & $9$      & $\infty$ & $2$\\
		\end{longtable}
		
		Виконуємо редукцію рядків, віднімаючи від кожного елемента рядку значення~$d_i$.
		
		\begin{longtable}[c]{lccccccr}
			\toprule
				Місто & 1 & 2 & 3 & 4 & 5 & 6 & $d_i$\\
			\midrule
			\endhead
			\bottomrule
			\endfoot
			
				1 & $\infty$ & $5$      & $0$      & $2$      & $4$      & $1$      & $3$\\
				2 & $0$      & $\infty$ & $4$      & $5$      & $6$      & $1$      & $3$\\
				3 & $0$      & $5$      & $\infty$ & $2$      & $4$      & $5$      & $4$\\
				4 & $5$      & $1$      & $4$      & $\infty$ & $0$      & $6$      & $4$\\
				5 & $1$      & $0$      & $14$     & $5$      & $\infty$ & $2$      & $1$\\
				6 & $8$      & $0$      & $3$      & $2$      & $7$      & $\infty$ & $2$\\
		\end{longtable}
		
		Визначаємо мінімальні значення в кожному стовпчику.
		
		\begin{longtable}[c]{lccccccr}
			\toprule
				Місто & 1 & 2 & 3 & 4 & 5 & 6\\
			\midrule
			\endhead
			\bottomrule
			\endfoot
			
				1     & $\infty$ & $5$      & $0$      & $2$      & $4$      & $1$\\
				2     & $0$      & $\infty$ & $4$      & $5$      & $6$      & $1$\\
				3     & $0$      & $5$      & $\infty$ & $2$      & $4$      & $5$\\
				4     & $5$      & $1$      & $4$      & $\infty$ & $0$      & $6$\\
				5     & $1$      & $0$      & $14$     & $5$      & $\infty$ & $2$\\
				6     & $8$      & $0$      & $3$      & $2$      & $7$      & $\infty$\\
				$d_j$ & $0$      & $0$      & $0$      & $2$      & $0$      & $1$\\
		\end{longtable}
		
		Виконуємо редукцію стовпчиків.
		\begin{longtable}[c]{lccccccr}
			\toprule
				Місто & 1 & 2 & 3 & 4 & 5 & 6\\
			\midrule
			\endhead
			\bottomrule
			\endfoot
			
				1     & $\infty$ & $5$      & $0$      & $0$      & $4$      & $0$\\
				2     & $0$      & $\infty$ & $4$      & $3$      & $6$      & $0$\\
				3     & $0$      & $5$      & $\infty$ & $0$      & $4$      & $4$\\
				4     & $5$      & $1$      & $4$      & $\infty$ & $0$      & $5$\\
				5     & $1$      & $0$      & $14$     & $3$      & $\infty$ & $1$\\
				6     & $8$      & $0$      & $3$      & $0$      & $7$      & $\infty$\\
				$d_j$ & $0$      & $0$      & $0$      & $2$      & $0$      & $1$\\
		\end{longtable}
		
		Знайдемо оцінки для комірок, що містять значення~$0$.
		
		\begin{longtable}[c]{lccccccr}
			\toprule
				Місто & 1 & 2 & 3 & 4 & 5 & 6\\
			\midrule
			\endhead
			\bottomrule
			\endfoot
			
				1     & $\infty$ & $5$      & $0^3$    & $0^0$    & $4$            & $0^0$\\
				2     & $0^0$    & $\infty$ & $4$      & $3$      & $6$            & $0^0$\\
				3     & $0^0$    & $5$      & $\infty$ & $0^0$    & $4$            & $4$\\
				4     & $5$      & $1$      & $4$      & $\infty$ & $\mathbf{0^5}$ & $5$\\
				5     & $1$      & $0^1$    & $14$     & $3$      & $\infty$       & $1$\\
				6     & $8$      & $0^0$    & $3$      & $0^0$    & $7$            & $\infty$\\
		\end{longtable}
		
		Знайшовши частину найкоротшого шляху: $4 \to 5 = 5$, виконуємо редукцію матриці та знаходимо мінімальні значення у рядках.
		
		\begin{longtable}[c]{lccccccr}
			\toprule
				Місто & 1 & 2 & 3 & 4 & 6 & $d_i$\\
			\midrule
			\endhead
			\bottomrule
			\endfoot
			
				1     & $\infty$ & $5$      & $0$      & $0$      & $0$      & $0$\\
				2     & $0$      & $\infty$ & $4$      & $3$      & $0$      & $0$\\
				3     & $0$      & $5$      & $\infty$ & $0$      & $4$      & $0$\\
				5     & $1$      & $0$      & $14$     & $\infty$ & $1$      & $0$\\
				6     & $8$      & $0$      & $3$      & $0$      & $\infty$ & $0$\\
		\end{longtable}
		
		Оскільки всі мінімальні елементи дорівнюють~$0$, редукція не внесе жодних змін. Також видно, що мінімальні елементи у рядках також дорівнюють~$0$, тому одразу знаходимо оцінки і шукаємо комірку з максимальною оцінкою.
		\begin{longtable}[c]{lccccccr}
			\toprule
				Місто & 1 & 2 & 3 & 4 & 6 \\
			\midrule
			\endhead
			\bottomrule
			\endfoot
			
				1     & $\infty$ & $5$      & $\mathbf{0^3}$ & $0^0$    & $0^0$\\
				2     & $0^0$    & $\infty$ & $4$            & $3$      & $0^0$\\
				3     & $0^0$    & $5$      & $\infty$       & $0^0$    & $4$\\
				5     & $1$      & $0^1$    & $14$           & $\infty$ & $1$\\
				6     & $8$      & $0^0$    & $3$            & $0^0$    & $\infty$\\
		\end{longtable}
		
		Отримали ділянку оптимального шляху: $1 \to 3 = 3$. Виконуємо редукцію і виключаємо зворотний шлях, встановлюючи значення~$\infty$ у відповідній комірці.
		
		\begin{longtable}[c]{lccccccr}
			\toprule
				Місто & 1 & 2 & 4 & 6 \\
			\midrule
			\endhead
			\bottomrule
			\endfoot
			
				2     & $0$      & $\infty$ & $3$      & $0$\\
				3     & $\infty$ & $5$      & $0$      & $4$\\
				5     & $1$      & $0$      & $\infty$ & $1$\\
				6     & $8$      & $0$      & $0$      & $\infty$\\
		\end{longtable}
		
		Мінімальні значення у рядках і стовпчиках~— $0$, тому переходимо до пошуку максимальних оцінок.
		
		\begin{longtable}[c]{lccccccr}
			\toprule
				Місто & 1 & 2 & 4 & 6 \\
			\midrule
			\endhead
			\bottomrule
			\endfoot
			
				2     & $0^1$    & $\infty$ & $3$            & $0^1$\\
				3     & $\infty$ & $5$      & $\mathbf{0^4}$ & $4$\\
				5     & $1$      & $0^1$    & $\infty$       & $1$\\
				6     & $8$      & $0^0$    & $0^0$          & $\infty$\\
		\end{longtable}
		
		Отримали ділянку шляху $3 \to 4 = 6$. Виконуємо редукцію.
		
		\begin{longtable}[c]{lccccccr}
			\toprule
				Місто & 1 & 2 & 6 \\
			\midrule
			\endhead
			\bottomrule
			\endfoot
			
				2     & $0$ & $\infty$ & $0$\\
				5     & $1$ & $0$      & $1$\\
				6     & $8$ & $0$      & $\infty$\\
		\end{longtable}
		
		Редукція стовпчиків і рядків знов не внесе змін, тому переходимо до оцінок.
		
		\begin{longtable}[c]{lccccccr}
			\toprule
				Місто & 1 & 2 & 6 \\
			\midrule
			\endhead
			\bottomrule
			\endfoot
			
				2     & $0^1$ & $\infty$       & $0^1$\\
				5     & $1$   & $0^1$          & $1$\\
				6     & $8$   & $\mathbf{0^8}$ & $\infty$\\
		\end{longtable}
		
		Отримали відрізок $6 \to 2 = 2$. Виконуємо редукцію та виключаємо зворотній шлях.
		
		\begin{longtable}[c]{lccccccr}
			\toprule
				Місто & 1 & 6 \\
			\midrule
			\endhead
			\bottomrule
			\endfoot
			
				2     & $0$ & $\infty$\\
				5     & $1$   & $1$\\
		\end{longtable}
		
		Бачимо, що у рядку~$5$ мінімальний елемент~— $1$. Проведемо редукцію рядків і перейдемо до оцінок.
		
		\begin{longtable}[c]{lccccccr}
			\toprule
				Місто & 1 & 6 \\
			\midrule
			\endhead
			\bottomrule
			\endfoot
			
				2     & $0^{\infty}$ & $\infty$\\
				5     & $0$   & $0^{\infty}$\\
		\end{longtable}
		
		Після редукції отримаємо дві останні ділянки $5 \to 6 = 3$, $2 \to 1 = 3$. Таким чином маємо повний шлях: $1 \to 3 \to 4 \to 5 \to 6 \to 1 = 21$.
	\end{solution}
	
	
	\begin{exercise}
		Дано систему чотирьох лінійних рівнянь з чотирма невідомими, необхідно знайти її розв'язки методом Гауса. Обчислити визначник та знайти обернену матрицю.
		\[
			\systeme{
				2x_1 - x_2 - x_3 + x_4 = 4,
				2x_1 + 3x_2 - x_3 + 2x_4 = 1,
				2x_1 + 5x_2 - 3x_3 + 4x_4 = 3,
				x_1 - x_2 - 2x_3 + 2x_4 = 4}
		\]
	\end{exercise}
	
	\begin{solution}
		\subproblem{Розв'язання системи лінійних алгебраїчних рівнянь}
		Для розв'язання СЛАР, необхідно звести її до одиничної матриці. Для представимо систему у вигляді матриці та будемо виконувати перетворення рядків:
		\[
		\left(
			\begin{array}{rrrr|r}
				2 & -1 & -1 & 1 & 4 \\
				2 & 3  & -1 & 2 & 1 \\
				2 & 5  & -3 & 4 & 3 \\
				1 & -1 & -2 & 2 & 4 \\
			\end{array}
		\right)
		\]
		
		Щоб отримати $a_{11} = 1$, розділимо перший рядок на~$2$.
		
		\begin{IEEEeqnarray*}{rCl}
		\left(
			\begin{array}{rrrr|r}
				1 & -0{,}5 & -0{,}5 & 0{,}5 & 2 \\
				2 & 3      & -1     & 2     & 1 \\
				2 & 5      & -3     & 4     & 3 \\
				1 & -1     & -2     & 2     & 4 \\
			\end{array}
		\right)
		&=&
		\left(
			\begin{array}{rrrr|r}
				1 & -0{,}5 & -0{,}5 & 0{,}5 & 2 \\
				0 & 4      & 0      & 1     & -3 \\
				0 & 6      & -2     & 3     & -1 \\
				0 & -0{,}5 & -1{,}5 & 1{,}5 & 2 \\
			\end{array}
		\right)
		=
		\end{IEEEeqnarray*}
		
		\begin{IEEEeqnarray*}{rCl}
		=
		\left(
			\begin{array}{rrrr|r}
				1 & -0{,}5 & -0{,}5 & 0{,}5  & 2 \\
				0 & 1      & 0      & 0{,}25 & -0{,}75 \\
				0 & 6      & -2     & 3      & -1 \\
				0 & -0{,}5 & -1{,}5 & 1{,}5  & 2 \\
			\end{array}
		\right)
		&=&
		\left(
			\begin{array}{rrrr|r}
				1 & 0 & -0{,}5 & 0{,}625  & 1{,}625 \\
				0 & 1 & 0      & 0{,}25   & -0{,}75 \\
				0 & 0 & -2     & 1{,}5    & 3{,}5 \\
				0 & 0 & -1{,}5 & 1{,}625  & 1{,}625 \\
			\end{array}
		\right)
		=\\[2\jot]
		=
		\left(
			\begin{array}{rrrr|r}
				1 & 0 & -0{,}5 & 0{,}625  & 1{,}625 \\
				0 & 1 & 0      & 0{,}25   & -0{,}75 \\
				0 & 0 & 1      & -0{,}75  & -1{,}75 \\
				0 & 0 & -1{,}5 & 1{,}625  & 1{,}625 \\
			\end{array}
		\right)
		&=&
		\left(
			\begin{array}{rrrr|r}
				1 & 0 & 0      & 0{,}25   & 0{,}75 \\
				0 & 1 & 0      & 0{,}25   & -0{,}75 \\
				0 & 0 & 1      & -0{,}75  & -1{,}75 \\
				0 & 0 & 0      & 0{,}5    & -1 \\
			\end{array}
		\right)
		=\\[2\jot]
		=
		\left(
			\begin{array}{rrrr|r}
				1 & 0 & 0      & 0{,}25   & 0{,}75 \\
				0 & 1 & 0      & 0{,}25   & -0{,}75 \\
				0 & 0 & 1      & -0{,}75  & -1{,}75 \\
				0 & 0 & 0      & 1        & -2 \\
			\end{array}
		\right)
		&=&
		\left(
			\begin{array}{rrrr|r}
				1 & 0 & 0      & 0 & 1{,}25 \\
				0 & 1 & 0      & 0 & -0{,}25 \\
				0 & 0 & 1      & 0 & -3{,}75 \\
				0 & 0 & 0      & 1 & -2 \\
			\end{array}
		\right)
		\end{IEEEeqnarray*}
		
		З отриманої матриці очевидно: $x_1 = 1{,}25$, $x_2 = -0{,}25$, $x_3 = -3{,}25$, $x_4 = -2$.
		
		\subproblem{Обчислення визначника}
		\[
			\left|
			\begin{array}{rrrr}
				2 & -1 & -1 & 1 \\
				2 & 3  & -1 & 2 \\
				2 & 5  & -3 & 4 \\
				1 & -1 & -2 & 2 \\
			\end{array}
			\right|
			=
			\left|
			\begin{array}{rrrr}
				2 & -1     & -1     & 1 \\
				0 & 4      & 0      & 1 \\
				0 & 6      & -2     & 3 \\
				0 & -0{,}5 & -1{,}5 & 1{,}5 \\
			\end{array}
			\right|=
			\left|
			\begin{array}{rrrr}
				2 & -1     & -1     & 1 \\
				0 & 4      & 0      & 1 \\
				0 & 0      & -2     & 1{,}5 \\
				0 & 0      & -1{,}5 & 1{,}625 \\
			\end{array}
			\right|
			=
		\]
		
		\[
			=
			\left|
			\begin{array}{rrrr}
				2 & -1     & -1 & 1 \\
				0 & 4      & 0  & 1 \\
				0 & 0      & -2 & 1{,}5 \\
				0 & 0      & 0  & 0{,}5 \\
			\end{array}
			\right|
			= 2 \cdot 4 \cdot (-2) \cdot 0{,}5 = -8.
		\]
		
		\subproblem{Обчислення оберненої матриці}
			Для обчислення оберненої матриці запишемо задану систему у матричному вигляді, записавши допоміжну одиничну матрицю:
			\begin{IEEEeqnarray*}{c}
				\left(
				\begin{array}{rrrr|rrrr}
					2 & -1 & -1 & 1 & 1 & 0 & 0 & 0 \\
					2 & 3  & -1 & 2 & 0 & 1 & 0 & 0 \\
					2 & 5  & -3 & 4 & 0 & 0 & 1 & 0 \\
					1 & -1 & -2 & 2 & 0 & 0 & 0 & 1 \\
				\end{array}
				\right)
				=
				\left(
				\begin{array}{rrrr|rrrr}
					1 & -0{,}5 & -0{,}5 & -0{,}5 & 0{,}5 & 0 & 0 & 0 \\
					2 & 3      & -1     & 2      & 0     & 1 & 0 & 0 \\
					2 & 5      & -3     & 4      & 0     & 0 & 1 & 0 \\
					1 & -1     & -2     & 2      & 0     & 0 & 0 & 1 \\
				\end{array}
				\right)
				=\\[2\jot]
				=
				\left(
				\begin{array}{rrrr|rrrr}
					1 & -0{,}5 & -0{,}5 & -0{,}5 & 0{,}5  & 0 & 0 & 0 \\
					0 & 4      & 0      & 1      & -1     & 1 & 0 & 0 \\
					0 & 6      & -2     & 3      & -1     & 0 & 1 & 0 \\
					0 & -0{,}5 & -1{,}5 & 1{,}5  & -0{,}5 & 0 & 0 & 1 \\
				\end{array}
				\right)
				=\\[2\jot]
				=
				\left(
				\begin{array}{rrrr|rrrr}
					1 & -0{,}5 & -0{,}5 & -0{,}5 & 0{,}5   & 0      & 0 & 0 \\
					0 & 1      & 0      & 0{,}25 & -0{,}25 & 0{,}25 & 0{,}25 & 0 \\
					0 & 6      & -2     & 3      & -1      & 0      & 1 & 0 \\
					0 & -0{,}5 & -1{,}5 & 1{,}5  & -0{,}5  & 0      & 0 & 1 \\
				\end{array}
				\right)
				=\\[2\jot]
			\end{IEEEeqnarray*}
			
			\begin{IEEEeqnarray*}{c}
				=
				\left(
				\begin{array}{rrrr|rrrr}
					1 & 0 & -0{,}5 & 0{,}625 & 0{,}375   & 0{,}125 & 0 & 0 \\
					0 & 1 & 0      & 0{,}25  & -0{,}25   & 0{,}25  & 0 & 0 \\
					0 & 0 & -2     & 1{,}5   & 0{,}5     & -1{,}5  & 1 & 0 \\
					0 & 0 & -1{,}5 & 1{,}625 & -0{,}625  & 0{,}125 & 0 & 1 \\
				\end{array}
				\right)
				=\\[2\jot]
				=
				\left(
				\begin{array}{rrrr|rrrr}
					1 & 0 & -0{,}5 & 0{,}625 & 0{,}375   & 0{,}125 & 0      & 0 \\
					0 & 1 & 0      & 0{,}25  & -0{,}25   & 0{,}25  & 0      & 0 \\
					0 & 0 & 1      & -0{,}75 & -0{,}25   & 0{,}75  & -0{,}5 & 0 \\
					0 & 0 & -1{,}5 & 1{,}625 & -0{,}625  & 0{,}125 & 0      & 1 \\
				\end{array}
				\right)
				=\\[2\jot]
				=
				\left(
				\begin{array}{rrrr|rrrr}
					1 & 0 & 0 & 0{,}25  & 0{,}25  & 0{,}5  & -0{,}25 & 0 \\
					0 & 1 & 0 & 0{,}25  & -0{,}25 & 0{,}25 & 0       & 0 \\
					0 & 0 & 1 & -0{,}75 & -0{,}25 & 0{,}75 & -0{,}5  & 0 \\
					0 & 0 & 0 & 0{,}5   & -1      & 1{,}25 & -0{,}75 & 1 \\
				\end{array}
				\right)
				=\\[2\jot]
				=
				\left(
				\begin{array}{rrrr|rrrr}
					1 & 0 & 0 & 0{,}25  & 0{,}25  & 0{,}5  & -0{,}25 & 0 \\
					0 & 1 & 0 & 0{,}25  & -0{,}25 & 0{,}25 & 0       & 0 \\
					0 & 0 & 1 & -0{,}75 & -0{,}25 & 0{,}75 & -0{,}5  & 0 \\
					0 & 0 & 0 & 1       & -2      & 2{,}5  & -1{,}5  & 2 \\
				\end{array}
				\right)
				=\\[2\jot]
				=
				\left(
				\begin{array}{rrrr|rrrr}
					1 & 0 & 0 & 0 & 0{,}75  & -0{,}125 & 0{,}125  & -0{,}5 \\
					0 & 1 & 0 & 0 & 0{,}25  & -0{,}375 & 0{,}375  & -0{,}5 \\
					0 & 0 & 1 & 0 & -1{,}75 & 2{,}625  & -1{,}625 & 1{,}5 \\
					0 & 0 & 0 & 1 & -2      & 2{,}5    & -1{,}5   & 2 \\
				\end{array}
				\right).
			\end{IEEEeqnarray*}
			
			Бачимо, що зліва утворилась одинична матриця, тобто ми знайшли шукану обернену матрицю~$A^{-1}$.
			\[
				A^{-1} = 
				\left(
					\begin{array}{rrrr}
						0{,}75  & -0{,}125 & 0{,}125  & -0{,}5 \\
						0{,}25  & -0{,}375 & 0{,}375  & -0{,}5 \\
						-1{,}75 & 2{,}625  & -1{,}625 & 1{,}5 \\
						-2      & 2{,}5    & -1{,}5   & 2 \\
					\end{array}
				\right).
			\]
		
	\end{solution}
	
	\begin{exercise}
		Необхідно здійснити~$k$-ту ітерацію для знаходження розв'язку рівняння $2x^2 + 5x - 1 = 0$ методом дотичних на проміжку~$\left[ -2, 10\right]$, якщо $x^{(k)} = 1$.
	\end{exercise}
	
	\begin{solution}
		Процес пошуку розв'язку можна поділити на такі кроки:
		\begin{enumerate}
			\item Пошук відрізків, що містять єдиний корінь.
			\item Власне ітерація~— пошук більш точного значення кореня.
		\end{enumerate}
		
		Нехай є точки $A = F(x_0)$, $B = F(x_1)$, тоді:
		\[
			\frac{y - f(x_1)}{f(x_1) - f(x_0)} = \frac{x - x_1}{(x_1 - x_0)}.
		\]
		
		Точка перетину прямої з віссю~$OX$:
		\[
			x = x_1 - \frac{x_1 - x_0}{f(x_1) - f(x_0)} \cdot f(x_1).
		\]
		
		Уточнюємо розв'язок: $F(X(k)) = 1$, $X(k) - X(k - 1)$ — приріст $X(k)$ в результаті ітерації попереднім значенням $X(k)$.
		
		Складаємо таблицю для першої ітерації:
		\begin{longtable}[c]{lcc}
			\toprule
				& $X$ & $F(x)$\\
			\midrule
			\endhead
			\bottomrule
			\endfoot
			
			$X_2$ & $-1{,}85$ & $3{,}38$ \\
			$X_3$ & $-1$ & $3{,}72$ \\
		\end{longtable}
		
		Отже $X_2 = x_1 - x \left( x_1 - 1\right) = 11{,}85$.
	\end{solution}
	
	\begin{exercise}
		Нехай потрібно розв'язати систему нелінійних рівнянь методом простих ітерацій до четвертого наближення. Знайти область визначення для невідомих, якщо задано: $x_1^{(0)} = 1{,}5$, $x_2^{(0)} = 0$, $x_3^{(0)} = 2$.
		\[
			\begin{cases}
				x_1^2 - \frac{x_2}{3} = 2,\\
				x_2 + 5x = -1,\\
				x_3 - x_1^3 + 3x_2^2 = 2.
			\end{cases}
		\]
	\end{exercise}
	
	\begin{solution}
		Спочатку виразимо кожну змінну через іншу таким чином:
		\begin{IEEEeqnarray*}{rCl}
			x_1 &=& \sqrt{\frac{x_2}{3} + 2},\\[2\jot]
			x_2 &=& 5x_1 - 1,\\[2\jot]
			x_3 &=& x_1^3 - 3x_2^2 + 2.
		\end{IEEEeqnarray*}
		
		З отриманих виразів очевидна область визначення~$D(y)$ кожної змінної: $x_1 \geqslant 1$, $x_2,x_3 \in \mathbb{R}$.
		
		Підставляємо у праві частини рівнянь початкові умови і знаходимо значення невідомих на першій ітерації:
		\begin{IEEEeqnarray*}{l}
			x_1 = \sqrt{0 + 2}      = 1{,}4742,\\
			x_2 = 5 \cdot 1{,}5 - 1 = 6{,}5,\\
			x_3 = {1{,}5}^3 - 0 + 2 = 5{,}375.\\
		\end{IEEEeqnarray*}
		
		Аналогічно для подальших ітерацій, для другої ітерації:
		\begin{IEEEeqnarray*}{l}
			x_1 = \sqrt{4{,}83}                      = 2{,}2,\\
			x_2 = 5 \cdot 1{,}47 - 1                 = 6{,}35,\\
			x_3 = {1{,}47}^3 - 3 \cdot {8{,}5}^2 + 2 = -211{,}58.\\
		\end{IEEEeqnarray*}
		
		Третя ітерація:
		\begin{IEEEeqnarray*}{l}
			x_1 = \sqrt{4{,}12}                      = 2{,}03,\\
			x_2 = 5 \cdot 2{,}2 - 1                 = 10,\\
			x_3 = {2{,}2}^3 - 3 \cdot {6{,}35}^2 + 2 = -108{,}312.\\
		\end{IEEEeqnarray*}
		
		Четверта ітерація:
		\begin{IEEEeqnarray*}{l}
			x_1 = \sqrt{5{,}33}                   = 2{,}3,\\
			x_2 = 5 \cdot 2{,}03 - 1              = 9{,}2,\\
			x_3 = {2{,}03}^3 - 3 \cdot {10}^2 + 2 = -293{,}9.\\
		\end{IEEEeqnarray*}
		
	\end{solution}
	
	\begin{exercise}
		Нехай дано дослідні дані попиту та пропозиції, а також ціну на певну продукцію. Необхідно оцінити похибку розв'язку методом найменших квадратів, якщо зроблено припущення, що залежність попиту та ціни визначається за формулою $S = 3c^2 + 1$, а пропозиції від ціни $P = 4c + 5$.
		
		\begin{longtable}[c]{lcccccc}
			\toprule
				      & $i = 1$ & $i = 2$ & $i = 3$ & $i = 4$ & $i = 5$ & $i = 6$\\
			\midrule
			\endhead
			\bottomrule
			\endfoot
			
				$c_i$ & $2$ & $3$ & $7$ & $2$ & $5$  & $3$ \\
				$P_i$ & $3$ & $3$ & $6$ & $8$ & $10$ & $2$ \\
				$c_i$ & $3$ & $6$ & $4$ & $2$ & $9$  & $4$ \\
		\end{longtable}
	\end{exercise}
	
	\begin{solution}
		Запишемо рівняння трендів для функцій $P(c)$ та~$S(c)$:
		\begin{equation}
		\label{eqn:ex7-mnk-system}
			\begin{cases}
				an + b \sum{c} = \sum{P},\\
				a \sum{c} + b \sum{c^2} = \sum{P \cdot c},
			\end{cases}
		\end{equation}
		де $n$~— кількість пар $P~—~c$, $a$ і~$b$ — коефіцієнти лінійного рівняння:
		\begin{equation}
		\label{eqn:ex7-linear-eqn}
			P = ac + b.
		\end{equation}
		
		Складаємо таблицю.
		
		\begin{longtable}[c]{lccccccr}
			\toprule
				& $i = 1$ & $i = 2$ & $i = 3$ & $i = 4$ & $i = 5$ & $i = 6$ & Сума \\
			\midrule
			\endhead
			\bottomrule
			\endfoot
			
			$c$         & $2$ & $3$ & $7$  & $2$  & $5$   & $3$ & $22$ \\
			$P$         & $3$ & $3$ & $6$  & $8$  & $10$  & $2$ & $32$ \\
			$c^2$       & $4$ & $9$ & $49$ & $4$  & $25$  & $9$ & $100$ \\
			$P^2$       & $9$ & $9$ & $36$ & $64$ & $100$ & $4$ & $222$ \\
			$P \cdot c$ & $6$ & $9$ & $42$ & $16$ & $50$  & $6$ & $129$ \\
		\end{longtable}
		
		Підставляємо дані таблиці в систему~\eqref{eqn:ex7-mnk-system}:
		\[
			\systeme{
				6a  + 22b  = 32,
				22a + 100b = 129.
			}
		\]
		
		Розв'язавши отриману СЛАР отримаємо: $a = 3{,}121$, $b = 0{,}603$. Підставимо отримані значення у лінійне рівняння~\eqref{eqn:ex7-linear-eqn}: $P = 0{,}603c + 3{,}121$.
		
		Обчислимо похибку початкового припущення $P = 4c + 5$:
		\begin{IEEEeqnarray*}{rCcCl}
			\Delta a &=& 4 - 0{,}603 &=& 3{,}397,\\
			\Delta b &=& 5 - 3{,}121 &=& 1{,}879.
		\end{IEEEeqnarray*}
		
		Оскільки функція~$S = 3c^2 + 1$ квадратична, її СЛАР має такий вигляд:
		\begin{equation}
		\label{eqn:ex7-mnk-quadradic-system}
			\begin{cases}
				an + b \sum{t} + c \sum{t^2} = \sum{S},\\
				a \sum{t} + b \sum{t^2} + c \sum{t^3} = \sum{S \cdot t}\\
				a \sum{t^2} + b \sum{t^3} + c \sum{t^4} = \sum{S \cdot t^2},
			\end{cases}
		\end{equation}
		де $a, b, c$~— коефіцієнти квадратного рівняння, $t$~— ціна, $n$~— кількість пар~$S~—~t$.
		
		Складаємо таблицю для функції~$S(c)$.
		\begin{longtable}[c]{lccccccr}
			\toprule
				& $i = 1$ & $i = 2$ & $i = 3$ & $i = 4$ & $i = 5$ & $i = 6$ & Сума \\
			\midrule
			\endhead
			\bottomrule
			\endfoot
			
			$t$           & $2$  & $3$  & $7$    & $2$  & $5$   & $3$  & $22$ \\
			$S$           & $3$  & $3$  & $4$    & $2$  & $9$   & $4$  & $28$ \\
			$t^2$         & $4$  & $9$  & $49$   & $4$  & $25$  & $9$  & $100$ \\
			$t^3$         & $8$  & $27$ & $343$  & $8$  & $125$ & $27$ & $538$ \\
			$t^4$         & $16$ & $81$ & $2401$ & $16$ & $625$ & $81$ & $3220$ \\
			$S^2$         & $9$  & $36$ & $16$   & $4$  & $81$  & $16$ & $162$ \\
			$S \cdot t$   & $6$  & $18$ & $28$   & $4$  & $45$  & $12$ & $113$ \\
			$S \cdot t^2$ & $12$ & $54$ & $196$  & $8$  & $225$ & $36$ & $531$ \\
		\end{longtable}
		
		За даними таблиці складаємо систему:
		\[
			\systeme{
				6a + 22b + 100c = 28,
				22a + 100b + 538c = 113,
				100a + 538b + 3220c = 531.
			}
		\]
		
		Розв'язавши її, отримаємо: $a = -9{,}624$, $b = 7{,}439$, $c = -0{,}779$.
		
		Підставляємо у рівняння:
		\[
		S = -0{,}779t^2 + 7{,}439t - 9{,}624.
		\]
		
		Визначаємо похибку початкового припущення:
		\begin{IEEEeqnarray*}{rCrCl}
			\Delta a &=& -0{,}779 - 3 &=& 3{,}779,\\
			\Delta b &=&  7{,}439 - 1 &=& 7{,}439,\\
			\Delta c &=&  9{,}624 + 1 &=& 8{,}624.
		\end{IEEEeqnarray*}
		
	\end{solution}
	
	\begin{exercise}
		Знайти значення функції в точці~$x = 2$ за заданою таблицею, використовуючи формулу Лагранжа.
		
		\begin{longtable}[c]{lccc}
			\toprule
				           & $i = 0$ & $i = 1$ & $i = 2$ \\
			\midrule
			\endhead
			\bottomrule
			\endfoot
			
				$x_i$      & $1$     & $3$     & $4$ \\
				$y_i(x_i)$ & $5$     & $3$     & $2$ \\
		\end{longtable}
	\end{exercise}
	
	\begin{solution}
		Будуємо поліном Лагранжа за заданою таблицею:
		\[
			L(x) = y_0 \cdot \frac{(x - x_1)(x - x_2)}{(x_0 - x_1)(x_0 - x_2)}
			     + y_1 \cdot \frac{(x - x_0)(x - x_2)}{(x_1 - x_0)(x_1 - x_2)}
			     + y_2 \cdot \frac{(x - x_0)(x - x_1)}{(x_2 - x_0)(x_2 - x_0)}.
		\]
		
		Підставляємо значення з таблиці:
		\begin{IEEEeqnarray*}{rCl}
			L(x) &=& 5 \cdot \frac{(x - 3)(x - 4)}{(1 - 3)(1 - 4)}
			     + 3 \cdot \frac{(x - 1)(x - 4)}{(3 - 1)(3 - 4)}
			     + 2 \cdot \frac{(x - 1)(x - 3)}{(4 - 1)(4 - 1)}\\[2\jot]
				 &=& 5 \cdot \frac{x^2 - 7x + 12}{6}
				  +  3 \cdot \frac{x^2 - 5x + 4}{-2}
				  +  2 \cdot \frac{x^2 - 4x + 3}{9}\\[2\jot]
				 &=& \frac{15 \cdot (x^2 - 7x + 12)
						 - 27 \cdot (x^2 - 5x + 4)
						 + 4  \cdot (x^2 - 4x + 3)}{18}\\
				 &=& \frac{15x^2 - 105x + 180
				         - 27x^2 + 135x - 108
				         + 4x^2 - 16x + 12
				 }{18}\\
				 &=& \frac{-8x^2 + 14x + 84}{18}.
		\end{IEEEeqnarray*}
		
		Знаходимо значення функції в точці $x = 2$:
		\[
			L(2) = \frac{-8 \cdot 4 + 14 \cdot 2 + 84}{18} = 4{,}44.
		\]
	\end{solution}
	
	\begin{exercise}
		Побудувати інтерполяційний поліном Ньютона за заданою таблицею.
		
		\begin{longtable}[c]{lccc}
			\toprule
				           & $i = 0$ & $i = 1$ & $i = 2$ \\
			\midrule
			\endhead
			\bottomrule
			\endfoot
			
				$x_i$      & $1$     & $3$     & $4$ \\
				$y_i(x_i)$ & $5$     & $3$     & $2$ \\
		\end{longtable}
	\end{exercise}
		
	\begin{solution}
		Складемо систему розділених різниць:
		\begin{IEEEeqnarray*}{rCcCl}
			f_{10} &=& \frac{y_1 - y_0}{x_1 - x_0}       &=& -1,\\[2\jot]
			f_{11} &=& \frac{y_2 - y_1}{x_2 - x_1}       &=& -1,\\[2\jot]
			f_{20} &=& \frac{f_{11} - f_{10}}{x_2 - x_0} &=& 0.\\[2\jot]
		\end{IEEEeqnarray*}
		
		Запишемо формулу інтерполяційного поліному Ньютона і підставимо туди отримані значення:
		\begin{IEEEeqnarray*}{rCcCl}
			P(x) &=& y_0 + f_{10}(x - x_0) + f_{20} \left( x - x_0 \right) \left( x - x_1 \right) && \\
			     &=& 5 - 1 \cdot (x - 1) + 0 \cdot \left( x - x_0 \right) \left( x - x_1 \right) &=& 6 - x.
		\end{IEEEeqnarray*}
	\end{solution}
	
\end{document}