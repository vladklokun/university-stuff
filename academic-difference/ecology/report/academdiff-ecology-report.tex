\documentclass[a4paper,oneside,DIV=10,12pt]{scrreprt}

\usepackage{fontspec}
\setmainfont{PT Serif}[Ligatures=TeX]
\setsansfont{PT Sans}[Ligatures=TeX]
\setmonofont{PT Mono}[Ligatures=TeX]

\usepackage{microtype}

\usepackage{polyglossia}
\setmainlanguage{ukrainian}

%Table typesetting
\usepackage{booktabs}
\usepackage{longtable}
\usepackage{array}
\newcolumntype{z}[1]{>{\raggedright\arraybackslash\hspace{0pt}}p{#1}}
\newcolumntype{x}[1]{>{\raggedleft\arraybackslash\hspace{0pt}}p{#1}}
\renewcommand{\arraystretch}{1.4}

\def\labelitemi{—}
\def\labelitemii{—}

\newcommand\defterm[1]{\textbf{#1}}

\newcommand\langorigin[1]{\textit{#1}}

\begin{document}
	\begin{titlepage}
		\begin{center}
			Міністерство освіти і науки України\\
			Національний авіаційний університет\\
			Навчально-науковий інститут комп'ютерних інформаційних технологій\\
			Кафедра комп'ютеризованих систем управління
			
			\vspace{\fill}
				Реферат\\
				на тему:\\
				«Природоохоронне~законодавство України.\\
				Природоохоронні~території та~їх значення»\\
				Академічна різниця
				
			\vspace{\fill}
			
			\begin{flushright}
				Виконав:\\
				студент ННІКІТ СП-225\\
				Клокун Владислав\\
			\end{flushright}
			Київ 2017
		\end{center}
	\end{titlepage}
	
	\tableofcontents
	
	\chapter{Природоохоронне законодавство України}
		Природоохоронна діяльність в Україні забезпечується Конституцією, яка зазначає, що забезпечення екологічної безпеки і підтримання екологічної рівноваги на території країни є обов'\-яз\-ком держави, а кожен громадянин зобов'язаний не завдавати шкоди природі та компенсувати завдані ним збитки. Організацію охорони довкілля і загальне керівництво природоохоронними заходами здійснює Міністерство охорони навколишнього природного середовища і ядерної безпеки. Державний нагляд контроль у природоохоронній сфері наразі здійснюють 5 органів виконавчої влади:
		\begin{enumerate}
			\item Держ\-еко\-інспекція;
			\item Держ\-гео\-надра;
			\item Держ\-ліс\-агентство;
			\item Держ\-риб\-а\-гент\-ство;
			\item Держ\-прод\-спожив\-служба.
		\end{enumerate}
		
		Для виконання зобов'язань держави з охорони природи були прийняті такі Закони України: «Про охорону навколишнього природного середовища», «Про природно-заповідний фонд України», «Про Червону книгу України», «Про тваринний світ», «Про рослинний світ», а також багато інших законодавчих актів. 
		
		\section{Закон України «Про охорону навколишнього природного середовища»}
			Основним документом, що характеризує природоохоронне законодавство України, є закон України №1264-12 «Про охорону навколишнього природного середовища», прийнятий у~1991~році. Цей закон визначає правові, економічні та соціальні основи організації охорони навколишнього природного середовища в інтересах нинішнього і майбутніх поколінь.
			
			Закон визначає, що завданням законодавства про охорону навколишнього природного є регулювання відносин у галузі охорони, використання і відтворення природних ресурсів, забезпечення екологічної безпеки, запобігання і ліквідації негативного впливу господарської та іншої діяльності на навколишнє природне середовище, збереження природних ресурсів, генетичного фонду живої природи, ландшафтів та інших природних комплексів, унікальних територій та природних об'єктів, пов'язаних з історико-культурною спадщиною.
			
			Крім того, закон встановлює, що відносини у галузі охорони навколишнього середовища в Україні регулюються цим Законом, а також земельним, водним, лісовим законодавством, законодавством про надра, про охорону атмосферного повітря, про охорону і використання рослинного і тваринного світу та іншим спеціальним законодавством.
			
			За текстом Закону України «Про охорону навколишнього природного середовища» основними принципами охорони навколишнього природного середовища є:
			\begin{itemize}
				\item пріоритетність вимог екологічної безпеки, обов'язковість додержання екологічних стандартів, нормативів та лімітів використання природних ресурсів при здійсненні господарської, управлінської та іншої діяльності;
				\item гарантування екологічно безпечного середовища для життя і здоров'я людей;
				\item запобіжний характер заходів щодо охорони навколишнього природного середовища;
				\item екологізація матеріального виробництва на основі комплексності рішень у питаннях охорони навколишнього природного середовища, використання та відтворення відновлюваних природних ресурсів, широкого впровадження новітніх технологій;
				\item збереження просторової та видової різноманітності і цілісності природних об'єктів і комплексів;
				\item науково обгрунтоване узгодження екологічних, економічних та соціальних інтересів суспільства на основі поєднання міждисциплінарних знань екологічних, соціальних, природничих і технічних наук та прогнозування стану навколишнього природного середовища;
				\item обов’язковість надання висновків державної екологічної експертизи;
				\item гласність і демократизм при прийнятті рішень, реалізація яких впливає на стан навколишнього природного середовища, формування у населення екологічного світогляду;
				\item науково обгрунтоване нормування впливу господарської та іншої діяльності на навколишнє природне середовище;
				\item безоплатність загального та оплату спеціального використання природних ресурсів для господарської діяльності;
				\item компенсація шкоди, заподіяної порушенням законодавства про охорону навколишнього природного середовища;
				\item вирішення питань охорони навколишнього природного середовища та використання природних ресурсів з урахуванням ступеня антропогенної зміненості територій, сукупної дії факторів, що негативно впливають на екологічну обстановку;
				\item поєднання заходів стимулювання і відповідальності у справі охорони навколишнього природного середовища;
				\item вирішення проблем охорони навколишнього природного середовища на основі широкого міждержавного співробітництва;
				\item встановлення екологічного податку, рентної плати за спеціальне використання води, рентної плати за спеціальне використання лісових ресурсів, рентної плати за користування надрами відповідно до Податкового кодексу України.
			\end{itemize}
		
		\section{Закон України «Про природно-заповідний фонд України»}
			Закон України №2456-12 «Про природно-заповідний фонд України» визначає правові основи організації, охорони, ефективного використання при\-род\-но\-за\-по\-від\-но\-го фонду України, відтворення його природних комплексів та об'єктів.
			
			Цей закон встановлює, що природно-заповідний фонд становлять ділянки суші і водного простору, природні комплекси та об'єкти яких мають особливу природоохоронну, наукову, естетичну, рекреаційну та іншу цінність і виділені з метою збереження природної різноманітності ландшафтів, генофонду тваринного і рослинного світу, підтримання загального екологічного балансу та забезпечення фонового моніторингу навколишнього природного середовища.
			
			Завданням законодавства України про природно-заповідний фонд України є регулювання суспільних відносин щодо організації, охорони і використання територій та об'єктів природно-заповідного фонду, відтворення їх природних комплексів, управління у цій галузі.
			
			До природно-заповідного фонду України належать:
			\begin{itemize}
				\item природні території та об'\-єк\-ти --- природні заповідники, біосферні заповідники, національні природні парки, регіональні ландшафтні парки, заказники, пам'\-ят\-ки природи, заповідні урочища;
				\item штучно створені об'\-єк\-ти --- ботанічні сади, дендрологічні парки, зоологічні парки, пам\-'ят\-ки природи, парки-пам'\-ят\-ки садово-паркового мистецтва.
			\end{itemize}
			
			Для збереження територій та об'\-єк\-тів  природно-заповідного фонду Закон встановлює такі шляхи:
			\begin{itemize}
				\item встановлення заповідного режиму;
				\item організація систематичних спостережень за станом заповідних природних комплексів та об'\-єк\-тів;
				\item проведення комплексних досліджень для розробки наукових основ їх збереження та ефективного використання;
				\item додерження вимог щодо охорони територій та об'\-єк\-тів при\-род\-но-за\-по\-від\-ного фонду під час здійснення господарської, управлінської та іншої діяльності, розробки проектної і проектно-планувальної документації, землевпорядкування, лісовпорядкування, проведення екологічних експертиз;
				\item запровадження економічних важелів стимулювання їх охорони;
				\item здійснення державного та громадського контролю за додержанням режиму їх охорони та використання, а також за знищення та пошкодження заповідних природних комплексів та об'\-єк\-тів;
				\item проведення широкого міжнародного співробітництва у цій сфері;
				\item проведення інших заходів з метою збереження територій та об'\-єк\-тів природно-заповідного фонду.
			\end{itemize}
			
			
		\section{Закон України «Про Червону книгу України»}
			Закон України «Про Червону книгу України» №3055-III від 07.02.2002 регулює відносини, пов'язані з веденням Червоної книги України, охороною, використанням та відтворенням рідкісних і таких, що перебувають під загрозою зникнення, видів тваринного і рослинного світу, занесених до Червоної книги України, з метою попередження зникнення таких видів із природи, забезпечення збереження їх генофонду.
			
			Червона книга України --- це офіційний державний документ, який містить перелік рідкісних і таких, що перебувають під загрозою зникнення, видів тваринного і рослинного світу у межах території України, її континентального шельфу та виключної (морської) економічної зони, а також узагальнені відомості про сучасний стан цих видів та заходи щодо їх відтворення.
			
			Червона книга є основою для розробки та реалізації програм і планів дій, спрямованих на охорону та відтворення видів, занесених до неї.
			
			Об'єкти Червоної книги України, надані відповідно до закону з дозволу центрального органу виконавчої влади, що реалізує державну політику у сфері охорони навколишнього природного середовища, у приватну власність, розведені (отримані) у штучних умовах від законно  набутих  у  приватну  власність  об'єктів  Червоної книги України, а також ввезені в  Україну з-за кордону або набуті в Україні в осіб, які мають право приватної власності на ці об'єкти, є приватною власністю юридичних або фізичних осіб. Законність набуття у приватну власність об'єктів Червоної книги України повинна бути підтверджена відповідними документами.
			
			Ведення Червоної книги України покладене на колишнє Міністерство охорони навколишнього природного середовища України, що було 9 грудня 2010 року реорганізоване в Міністерство екології та природних ресурсів України.
			
			Для наукового забезпечення Червоної книги України, підготовки пропозицій про занесення до Червоної книги України та вилучення з неї видів тварин і рослин, організації наукових досліджень, розробки заходів щодо охорони рідкісних і таких, що перебувають під загрозою зникнення, видів тварин і рослин, контролю за їх виконанням, координації діяльності державних органів та громадських організацій створена Національна комісія з питань Червоної книги України.
			
			Види тварин і рослин, занесені до Червоного списку Міжнародного союзу охорони природи та природних ресурсів і Європейського Червоного списку, які зустрічаються на території України, заносяться до Червоної книги України або одержують інший особливий статус відповідно до законодавства України про охорону та використання тваринного і рослинного світу.
			
			Кабінет Міністрів України забезпечує офіційне видання та розповсюдження Червоної книги України не рідше одного разу на 10 років.
			
			Наукові та інші установи, підприємства, організації та громадяни повідомляють Міністерству охорони навколишнього природного середовища України наявну у них інформацію про поширення, чисельність, стан видів тварин і рослин, занесених до Червоної книги України, та негайно інформує про факти їх знищення, пошкодження, загибелі чи захворювання.
			
		\section{Закон України «Про тваринний світ»}
			Закон України «Про тваринний світ» 2894-14 регулює відносини у галузі охорони, використання і відтворення сільськогосподарських, свійських тварин, а також діяльність пов'язану з охороною і використанням залишків викопних тварин.
			
			Завданнями законодавства України про тваринний світ є
			\begin{itemize}
				\item регулювання відносин у галузі охорони, використання і відтворення об'єктів тваринного світу;
				\item збереження та поліпшення середовища існування диких тварин;
				\item забезпечення умов збереження всього видового і популяційного різноманіття тварин.
			\end{itemize}
			
			До об'єктів тваринного світу, на які поширюється дія закону «Про тваринний світ» відносяться:
			\begin{itemize}
				\item дикі тварини;
				\item частини диких тварин (роги, шкіра тощо);
				\item продукти життєдіяльності диких тварин (мед, віск тощо).
			\end{itemize}
			
			Крім того підлягають охороні нори, хатки, лігва, мурашники, боброві загати та інше житло і споруди тварин, місця токування, линяння, гніздових колоній птахів, постійних чи тимчасових скупчень тварин, нерестовищ, а також інші території, що є середовищем їх існування та шляхами міграції.
			
			На об'єкти тваринного світу поширюється право громадян на приватну власність за умови, що ці об'єкти були отримані законним шляхом. Однак, право власності може бути припинене у разі жорстокого поводження з даними об'єктами або встановлення законодавчими актами заборони щодо перебування у приватній власності окремих об'єктів тваринного світу.
			
			Під час проведення заходів щодо охорони, раціонального використання і відтворення тваринного світу, а також під час здійснення будь-якої діяльності, яка може вплинути на середовище існування диких тварин та стан тваринного світу, повинно забезпечуватися додержання таких основних вимог і принципів:
			\begin{itemize}
				\item збереження умов існування видового і популяційного різноманіття тваринного світу в стані природної волі;
				\item недопустимість погіршення середовища існування, шляхів міграції та умов розмноження диких тварин;
				\item збереження цілісності природних угруповань диких тварин;
				\item додержання науково обгрунтованих нормативів і лімітів використання об'єктів тваринного світу, забезпечення невиснажливого їх використання, а також відтворення;
				\item раціональне використання корисних властивостей і продуктів життєдіяльності диких тварин;
				\item оплату спеціального використання об'єктів тваринного світу;
				\item регулювання чисельності диких тварин в інтересах охорони здоров'я населення і запобігання заподіянню шкоди довкіллю, господарській та іншій діяльності;
				\item урахування висновків екологічної експертизи щодо об'єктів господарської та іншої діяльності, які можуть негативно впливати на стан тваринного світу.
				
			\end{itemize}
			
		\section{Закон України «Про рослинний світ»}
			Закон України «Про рослинний світ» №591-14 регулює відносини у сфері охорони, використання та відтворення рослин і багаторічних насаджень сільськогосподарського призначення.
			
			Завданням законодавство України про рослинний світ є регулювання суспільних відносин у сфері охорони, використання та відтворення дикорослих та інших несільськогосподарського призначення судинних рослин, мохоподібних, водоростей, лишайників, а також грибів, їх угруповань і місцезростань.
			
			За текстом закону приймаються такі визначення:
			\begin{itemize}
				\item \defterm{рослинний світ} --- сукупність усіх видів рослин, а також грибів та утворених ними угруповань на певній території;
				\item \defterm{дикорослі рослини} --- рослини, що природно зростають на певній території;
				\item \defterm{природні рослинні угруповання} --- сукупність видів рослин, що зростають в межах певних ділянок та перебувають у тісній взаємодії як між собою, так і з умовами довкілля;
				\item \defterm{акліматизація} --- пристосування (адаптація) виду до нових умов існування у зв'язку зі штучним його переселенням;
				\item \defterm{інтродукція} --- штучне введення виду до складу рослинного світу поза межами його природного ареалу;
				\item \defterm{об'єкти рослинного світу} --- дикорослі та інші несільськогосподарського призначення судинні рослини, мохоподібні, водорості, лишайники, а також гриби на всіх стадіях розвитку та утворені ними природні угруповання;
				\item \defterm{природні рослинні ресурси} --- об'єкти рослинного світу, що використовуються або можуть бути використані населенням, для потреб виробництва та інших потреб.
			\end{itemize}
			
			Закон встановлює такі основні вимоги, яких необхідно дотримуватись під час здійснення діяльності, яка впливає на стан охорони, використання та відтворення рослинного світу:
			\begin{itemize}
				\item збереження природної просторової, видової, популяційної та ценотичної різноманітності об'єктів рослинного світу;
				\item збереження умов місцезростання дикорослих рослин і природних рослинних угруповань;
				\item науково обгрунтованого невиснажливого використання природних рослинних ресурсів;
				\item здійснення заходів щодо запобігання негативному впливу господарської діяльності на рослинний світ;
				\item охорони об'єктів рослинного світу від пожеж, захист від шкідників і хвороб;
				\item здійснення заходів щодо відтворення об'єктів рослинного світу;
				\item регулювання поширення та чисельності дикорослих рослин і використання їх запасів з врахуванням інтересів охорони здоров'я населення.
			\end{itemize}
			
			Вказані вимоги враховуються під час розробки проектів законодавчих актів, загальнодержавних, міждержавних, регіональних програм та здійснення заходів з охорони, використання та відтворення рослинного світу.
			
	\chapter{Природоохоронні території та їх значення}
		За визначенням Міжнародного союзу охорони природи (МСОП) \defterm{природоохоронна територія}~--- це територія або акваторія, призначена для захисту і підтримки біологічної різноманітності та природних і пов'язаних із ними культурних ресурсів, захист якої встановлюється законом або іншими засобами.
		
		Закон України №2456-12 «Про природно-заповідний фонд України» встановлює, що природно-заповідний фонд (тобто природоохоронні території) становлять ділянки суші і водного простору, природні комплекси та об'єкти яких мають особливу природоохоронну, наукову, естетичну, рекреаційну та іншу цінність і виділені з метою збереження природної різноманітності ландшафтів, генофонду тваринного і рослинного світу, підтримання загального екологічного балансу та забезпечення фонового моніторингу навколишнього природного середовища.
		
		Основними типами природоохоронних територій є:
		\begin{itemize}
			\item заповідники;
				\begin{itemize}
					\item природні;
					\item біосферні;
				\end{itemize}
			\item національні природні парки;
			\item заказники.
		\end{itemize}
		
		%Природоохоронні території мають виключне екологічне та наукове значення. Екологічне значення полягає в збереженні біологічного різноманіття. Науковим значенням є можливість проведення досліджень видів, що населяють дану територію; впливу людської господарської діяльності на природу за допомогою порівяння природних об'\-єк\-тів, що надійно охороняються, та об'\-єк\-тів, на яких ведеться активна людська діяльність.
		
		\section{Заповідники}
			\defterm{Заповідники} --- це природоохоронні науково-дослідницькі установи, створені з метою збереження в природному стані типових для даної місцевості чи унікальних природних комплексів; вивчення природних процесів або явищ, які в них відбуваються; розроблення наукових основ охорони природи. На території заповідників заборонена будь-яка господарська діяльність, а землі назавжди вилучені з будь-яких форм використання. Заповідники поділяються на \emph{природні} та \emph{біосферні}.
			
			\subsection{Природні}
				\defterm{Природні заповідники} --- це заповідники \emph{державного} значення. Перелік природних заповідників України наведений у табл.~\ref{tab:ukrnaturepreserves}
				
				\begin{table}[!htbp]
					\centering
					\begin{tabular}{z{0.25\textwidth}z{0.27\textwidth}x{0.16\textwidth}x{0.17\textwidth}}
						\toprule
							Заповідник & Область & Площа, га & Заснований \\
						\midrule
							Ґорґани & Івано-Франківська & 5344 & 1996\\
							Дніпровсько-Орільский & Дніпропетровська & 3766 & 1990\\
							Древлянський & Житомирська & 30873 & 2009\\
							Єланецький степ & Миколаївська & 1676 & 1996\\
							Казантипський & АР Крим & 450 & 1998\\
							Канівський & Черкаська & 2027 & 1923\\
							Карадазький & АР Крим & 2872 & 1979\\
							Кримський & АР Крим & 44175 & 1923\\
							Луганський & Луганська & 2122 & 1968\\
							Медобори & Тернопільська & 10521 & 1990\\
							Мис Мартьян & АР Крим & 240 & 1973\\
							Михайлівська цілина & Сумська & 883 & 2009\\
							Опуцький & АР Крим & 1592 & 1998\\
							Поліський & Житомирська & 20104 & 1968\\
							Природний & Львівська & 2085 & 1984\\
							Рівненський & Рівненська & 42289 & 1999\\
							Черемський & Волинська & 2976 & 2001\\
							Український степовий & Донецька, Запорізька, Сумська & 3336 & 1961\\
							Ялтинський гірсько-лісовий & АР Крим & 14523 & 1973\\
						\bottomrule
					\end{tabular}
					\caption{Природні заповідники України}
					\label{tab:ukrnaturepreserves}
				\end{table}
			
			\subsection{Біосферні}
				\defterm{Біосферні заповідники} --- природоохоронні науково-дос\-лід\-ні установи \emph{міжнародного} значення, що створюються з метою збереження в природному стані найбільш типових природних комплексів біосфери, здійснення фонового екологічного моніторингу, вивчення спонтанних та глобально-ан\-тро\-по\-ген\-них змін, що відбуваються в біосфері. В таких заповідниках реалізують міжнародні наукові і природоохоронні програми.
				
				Перелік біосферних заповідників, що функціонують в Україні, наведений у табл.~\ref{tab:ukrbiospherereserve}.
				\begin{table}[!htbp]
					\centering
					\begin{tabular}{z{0.25\textwidth}z{0.27\textwidth}x{0.16\textwidth}x{0.17\textwidth}}
						\toprule
						Заповідник & Область & Площа, га & Заснований \\
						\midrule
						Асканія-Нова ім.~Фальц-Фейна & Херсонська	& 33308	& 1983\\
						Чорноморський	& Одеська	& 50252	& 1998\\
						Карпатський	& Закарпатська	& 57880	& 1968\\
						Дунайський	& Херсонська, Миколаївська	& 109255	& 1983\\
						\bottomrule
					\end{tabular}
					\caption{Біосферні заповідники України}
					\label{tab:ukrbiospherereserve}
				\end{table}
				
				Концепція біосферного заповідника була розроблена у~1974~році робочою групою в рамках програми ЮНЕСКО «Людина та біосфера» (англ. \langorigin{Man and the Biosphere}). Через два роки почалось формування Всесвітньої мережі біосферних заповідників. Згідно з Положенням про Всесвітню мережу біосферних резерватів «Мережа є інструментом збереження біологічного різноманіття та стійкого користування його компонентів, що таким чином вносить вклад в досягнення цілей Конвенції про біологічне різноманіття та інших відповідних конвенцій та актів».
				
				За даними ЮНЕСКО, у світі налічують 699 біосферних заповідників:
				\begin{itemize}
					\item 70 у 28 країнах Африки;
					\item 30 в 11 арабських країнах;
					\item 142 в 24 країнах Азії та Тихого океану;
					\item 302 в 26 країнах Європи та Північної Америки;
					\item 125 в 21 країні Латинської Америки та Карибського басейну.
				\end{itemize}
				
				Біосферні заповідники мають три взаємо\-зв'я\-за\-ні зони, що мають виконувати три доповнюючі та посилюючі функції:
				\begin{enumerate}
					\item Основна зона --- надійно захищена екосистема для збереження ландшафтів, екосистем, видів та генетичного різноманіття.
					\item Буферна зона --- ця зона оточує або прилягає до основної зони заповідника. Вона використовується для діяльності, сумісної з необхідною екологічної практикою, направленою на укріплення наукової, моніторингової та навчальної діяльності.
					\item Перехідна зона --- це зона, у межах якої дозволено вести різноманітну діяльність, що сприяє розвитку економіки та людського потенціалу.
				\end{enumerate}
				
		\section{Національні природні парки}
			Національні природні парки --- це природоохоронні установи, покликані зберігати цінні природні, історико-культурні комплекси та об'єкти. На відміну від заповідників, на території національних природних парків крім зон повної заповідності є певні місця, відкриті для організованого відвідування, наприклад, туризму. Також, на території таких парків може бути дозволена обмежена господарська діяльність.
			
		\section{Заказники}
			Заказники --- це природні території, створені для збереження і відтворення природних комплексів або окремих видів організмів. На їхній території наукову та інші види діяльності здійснюють з дотриманням вимог охорони довкілля. На відміну від заповідників, заказники можуть бути постійними або тимчасовими.
			
			Особливе місце у здійсненні природоохоронних заходів належить ботанічним садам і зоопаркам, де вивчають, зберігають, акліматизують та ефективно використовують представників рідкісних і типових видів місцевої і світової фауни та флори. Але головне призначення цих закладів~--- проведення просвітницько-виховної роботи, формування у людей дбайливого ставлення до природи.
\end{document}

%http://edufuture.biz/index.php?title=%D0%9F%D1%80%D0%B8%D1%80%D0%BE%D0%B4%D0%BE%D0%BE%D1%85%D0%BE%D1%80%D0%BE%D0%BD%D0%BD%D0%B5_%D0%B7%D0%B0%D0%BA%D0%BE%D0%BD%D0%BE%D0%B4%D0%B0%D0%B2%D1%81%D1%82%D0%B2%D0%BE_%D0%A3%D0%BA%D1%80%D0%B0%D1%97%D0%BD%D0%B8