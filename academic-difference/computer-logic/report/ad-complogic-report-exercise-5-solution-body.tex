% ------------------------------------------------------------------------
% file `ad-complogic-report-exercise-5-solution-body.tex'
%
%     solution of type `exercise' with id `5'
%
% generated by the `solution' environment of the
%   `xsim' package v0.10 (2017/09/19)
% from source `ad-complogic-report' on 2017/11/22 on line 431
% ------------------------------------------------------------------------
^^I^^IПобудуємо таблицю, що описує заданий автомат~(табл.~\ref{tab:task-5-automata-table}). Для цього пронумеруємо стани за принципом кодування Грея: $a_1$~— $00$, $a_2$~— $01$, $a_3$~— $11$. Оскільки для побудови функціональної схеми необхідно використати T-тригери, також наведемо таблицю їх переходів~(табл.~\ref{tab:task-5-t-flipflop-excitation-table}).
^^I^^I
^^I^^I\begin{table}[!htbp]
^^I^^I\centering
^^I^^I^^I\begin{tabular}{ccc}
^^I^^I^^I^^I\toprule
^^I^^I^^I^^I^^I$Q_t$ & $Q_{t+1}$ & T\\
^^I^^I^^I^^I\midrule
^^I^^I^^I^^I^^I0     & 0         & 0\\
^^I^^I^^I^^I^^I1     & 1         & 0\\
^^I^^I^^I^^I^^I0     & 1         & 1\\
^^I^^I^^I^^I^^I1     & 0         & 1\\
^^I^^I^^I^^I\bottomrule
^^I^^I^^I\end{tabular}
^^I^^I\caption{Таблиця переходів T-тригера}
^^I^^I\label{tab:task-5-t-flipflop-excitation-table}
^^I^^I\end{table}
^^I^^I
^^I^^I\begin{table}[!htbp]
^^I^^I\centering
^^I^^I^^I\begin{tabular}{cccccc}
^^I^^I^^I^^I\toprule
^^I^^I^^I^^I^^I$(A_1 A_2)_t$ & $(A_1 A_2)_{t+1}$ & $x_1$ & $y_1 y_2$ & $T_1$ & $T_2$\\
^^I^^I^^I^^I\midrule
^^I^^I^^I^^I^^I$00$  & $01$      & —     & $01$      & 0     & 1\\
^^I^^I^^I^^I^^I$01$  & $11$      & 0     & $00$      & 1     & 0\\
^^I^^I^^I^^I^^I$11$  & $00$      & 1     & $00$      & 1     & 1\\
^^I^^I^^I^^I^^I$11$  & $10$      & 0     & $10$      & 0     & 1\\
^^I^^I^^I^^I\bottomrule
^^I^^I^^I\end{tabular}
^^I^^I\caption{Таблиця заданого автомата}
^^I^^I\label{tab:task-5-automata-table}
^^I^^I\end{table}
^^I^^I
^^I^^IСкладемо та мінімізуємо функції, що описують залежності станів:
^^I^^I\begin{IEEEeqnarray*}{rCl}
^^I^^I^^IT_1 (A_1, A_2, x) &=& \neg{A_1} \land A_2 \lor A_2 \land x,\\
^^I^^I^^IT_2 (A_1, A_2, x) &=& \neg{A_1} \land \neg{A_2} \lor A_1 \land A_2.\\
^^I^^I^^Iy(A_1, A_2, x) &=& A_1 \land \neg{x},\\
^^I^^I^^Iy(A_1, A_2, x) &=& \neg{A_2}.\\
^^I^^I\end{IEEEeqnarray*}
^^I^^I
