% ------------------------------------------------------------------------
% file `ad-complogic-report-exercise-3-solution-body.tex'
%
%     solution of type `exercise' with id `3'
%
% generated by the `solution' environment of the
%   `xsim' package v0.10 (2017/09/19)
% from source `ad-complogic-report' on 2017/11/21 on line 178
% ------------------------------------------------------------------------
^^I^^IСпростимо логічний вираз $L$, використовуючи закони Де~Моргана, подвійного заперечення, ідемпотенції та дистрибутивності.
^^I^^I\begin{IEEEeqnarray*}{rCl}
^^I^^I^^I%L &=& x_3x_2 \lor x_3 \barneg{x_2} \lor \barneg{\barneg{x_1} \lor \barneg{x_1 \lor x_2} }\\
^^I^^I^^I%  &=& x_3 x_2 \lor x_3 \barneg{x_2} \lor \barneg{\barneg{x_1}} \barneg{\barneg{(x_1 \lor x_2)}}\\
^^I^^I^^I%  &=& x_3 x_2 \lor x_3 \barneg{x_2} \lor x_1 x_1 \lor x_2\\
^^I^^I^^I%  &=& x_3 x_2 \lor x_3 \barneg{x_2} \lor x_1 x_1 \lor x_1 x_2\\
^^I^^I^^I%  &=& x_3 (x_2 \lor \barneg{x_2}) \lor x_1 \lor x_1 x_2\\
^^I^^I^^I%  &=& x_3 \lor x_2 \lor x_1(1 + x_2)\\
^^I^^I^^I%  &=& x_3 \lor x_1.
^^I^^I^^I%
^^I^^I^^IL &=& x_3 \land x_2 \lor x_3 \land \neg x_2 \lor \neg (\neg x_1 \lor \neg(x_1 \lor x_2)\\
^^I^^I^^I  &=& x_3 \land x_2 \lor x_3 \land \neg x_2 \lor \neg (\neg x_1) \land \neg(\neg (x_1 \lor x_2))\\
^^I^^I^^I  &=& x_3 \land x_2 \lor x_3 \land \neg x_2 \lor x_1 \land (x_1 \lor x_2)\\
^^I^^I^^I  &=& x_3 \land x_2 \lor x_3 \land \neg x_2 \lor x_1 \land x_1 \lor x_1 \land x_2\\
^^I^^I^^I  &=& x_3 \land (x_2 \lor \neg x_2) \lor x_1 \lor x_1 \land x_2\\
^^I^^I^^I  &=& x_3 \lor x_1 \lor x_1 \land x_2\\
^^I^^I^^I  &=& x_3 \lor x_1 \land (1 \lor x_2)\\
^^I^^I^^I  &=& x_3 \lor x_1.
^^I^^I\end{IEEEeqnarray*}
^^I^^I
^^I^^IМінімізуємо логічний вираз $F$. Для цього представимо його у двійковому вигляді:
^^I^^I\begin{IEEEeqnarray*}{rCl}
^^I^^I^^IF &=& 0 \lor 4 \lor 7 \lor 8 \lor 11 \lor 12 \lor 13 \lor 15\\
^^I^^I^^I  &=& 0000 \lor 0100 \lor 0111 \lor 1000 \lor 1011 \lor 1100 \lor 1101 \lor 1111\\
^^I^^I^^I  &=& \neg{A} \neg{B} \neg{C} \neg{D}
^^I^^I^^I      \lor \neg{A}      B \neg{C} \neg{D}
^^I^^I^^I      \lor \neg{A}      B  \neg{C} \neg{D}
^^I^^I^^I      \lor \neg{A}      B       C       D \\
^^I^^I^^I  &&  \lor      A  \neg{B} \neg{C} \neg{D}
^^I^^I^^I      \lor      A  \neg{B}      C       D
^^I^^I^^I      \lor      A       B  \neg{C}      D
^^I^^I^^I      \lor      A       B       C       D.
^^I^^I\end{IEEEeqnarray*}
^^I^^I
^^I^^IПобудуємо карту Карно (рис.~\ref{fig:task3-karnaugh-map}).
^^I^^I\begin{figure}
^^I^^I\centering
^^I^^I^^I\begin{karnaugh-map}*[4][4][1][$CD$][$AB$]
^^I^^I^^I^^I\minterms{0,4,7,8,11,12,13,15}
^^I^^I^^I^^I\implicant{0}{8}
^^I^^I^^I^^I\implicant{13}{15}
^^I^^I^^I^^I\implicant{ 7}{15}
^^I^^I^^I^^I\implicant{15}{11}
^^I^^I^^I\end{karnaugh-map}
^^I^^I\caption{Карта Карно логічного виразу $F$}
^^I^^I\label{fig:task3-karnaugh-map}
^^I^^I\end{figure}
^^I^^IЗвідси маємо:
^^I^^I\[
^^I^^I^^IF = \neg{C} \neg{D} \lor ABD \lor BCD \lor ACD.
^^I^^I\]
