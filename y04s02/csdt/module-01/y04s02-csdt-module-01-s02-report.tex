\documentclass[
  a4paper,
  oneside,
  BCOR = 10mm,
  DIV = 12,
  12pt,
  headings = normal,
]{scrartcl}

%%% Length calculations
\usepackage{calc}
%%%

%%% Support for color
\usepackage{xcolor}
\definecolor{lightblue}{HTML}{03A9F4}
\definecolor{red}{HTML}{F44336}
%%%

%%% Including graphics
\usepackage{graphicx}
%%%

%%% Font selection
\usepackage{fontspec}

\setromanfont{STIX Two Text}[
  SmallCapsFeatures = {LetterSpace = 8},
]

\setsansfont{IBM Plex Sans}[
  Scale = MatchUppercase,
]

\setmonofont{IBM Plex Mono}[
  Scale = MatchUppercase,
]
%%%

%%% Math typesetting
\usepackage{amsmath}
\usepackage{mathtools}

\usepackage{unicode-math}
\setmathfont{STIX Two Math}

\usepackage{IEEEtrantools}

% Typeset matrices with row labels
\usepackage{kbordermatrix}
%%%

%%% List settings
\usepackage{enumitem}
\setlist[enumerate]{
  label*      = {\arabic*.},
  left        = \parindent,
  topsep      = 0\baselineskip,
  parsep      = 0\baselineskip,
  noitemsep, % override itemsep
}
% List settings for levels 2–4
\setlist[enumerate, 2, 3, 4]{
  label*      = {\arabic*.},
  left        = 0em,
  topsep      = 0\baselineskip,
  parsep      = 0\baselineskip,
  noitemsep, % override itemsep
}

\setlist[itemize]{
  label*      = {—},
  left        = \parindent,
  topsep      = 0\baselineskip,
  parsep      = 0\baselineskip,
  itemsep     = 1\baselineskip,
  noitemsep, % override itemsep
}

\setlist[description]{
  font        = {\rmfamily\upshape\bfseries},
  topsep      = 1\baselineskip,
  parsep      = 0\baselineskip,
  itemsep     = 0\baselineskip,
}

%%%

%%% Structural elements typesetting
\setkomafont{pagenumber}{\rmfamily\upshape}
\setkomafont{disposition}{\rmfamily\bfseries}

% Sectioning
\RedeclareSectionCommand[
  beforeskip = -1\baselineskip,
  afterskip  = 1\baselineskip,
  font       = {\normalsize\bfseries\scshape},
]{section}

\RedeclareSectionCommand[
  beforeskip = -1\baselineskip,
  afterskip  = 1\baselineskip,
  font       = {\normalsize\bfseries\itshape},
]{subsection}

\RedeclareSectionCommand[
  beforeskip = -1\baselineskip,
  afterskip  = 1\baselineskip,
  font       = {\normalsize\bfseries},
]{subsubsection}

\RedeclareSectionCommand[
  beforeskip = -1\baselineskip,
  afterskip  = -0.5em,
  font       = {\normalsize\mdseries\scshape\addfontfeatures{Letters = {UppercaseSmallCaps}}},
]{paragraph}
%%%

%%% Typographic enhancements
\usepackage{microtype}
%%%

%%% Language-specific settings
\usepackage{polyglossia}
\setmainlanguage{ukrainian}
\setotherlanguages{english}
%%%

%%% Captions
\usepackage{caption}
\usepackage{subcaption}

%\DeclareCaptionLabelFormat{closing}{#2)}
%\captionsetup[subtable]{labelformat = closing}

%\captionsetup[subfigure]{labelformat = closing}

\captionsetup[table]{
  aboveskip = 0\baselineskip,
  belowskip = 0\baselineskip,
}

\captionsetup[figure]{
  aboveskip = 1\baselineskip,
  belowskip = 0\baselineskip,
}

\captionsetup[subfigure]{
  labelformat = simple,
  labelformat = brace,
  justification = RaggedRight,
  singlelinecheck = false,
}
%%%

%%% Hyphenated ragged typesetting
\usepackage{ragged2e}
%%%

%%% Table typesetting
\usepackage{booktabs}
\usepackage{longtable}

\usepackage{multirow}

\usepackage{array}
\newcolumntype{v}[1]{>{\RaggedRight\arraybackslash\hspace{0pt}}p{#1}}
\newcolumntype{b}[1]{>{\Centering\arraybackslash\hspace{0pt}}p{#1}}
\newcolumntype{n}[1]{>{\RaggedLeft\arraybackslash\hspace{0pt}}p{#1}}
\newcolumntype{V}[1]{>{$}v{#1}<{$}}
\newcolumntype{B}[1]{>{$}b{#1}<{$}}
\newcolumntype{N}[1]{>{$}n{#1}<{$}}
%%%

%%% Drawing
\usepackage{tikz}
\usepackage{tikzscale}
\usetikzlibrary{datavisualization}
\usetikzlibrary{datavisualization.formats.functions}
\usetikzlibrary{positioning}
\usetikzlibrary{patterns}
\usetikzlibrary{intersections}
\usetikzlibrary{arrows.meta} % Stealth arrow tips
\usetikzlibrary{graphs}
\usetikzlibrary{graphdrawing}
\usegdlibrary{layered}
\usetikzlibrary{quotes}

\usepackage{pgfplots}
\usepgfplotslibrary{fillbetween}
%%%

%%% SI units typesetting
\usepackage{siunitx}
\sisetup{
  output-decimal-marker = {,},
  exponent-product      = {\cdot},
  inter-unit-product    = \ensuremath{{} \cdot {}},
  per-mode              = symbol,
}
%%%

% Code Highlighting
\usepackage{minted}
\setmintedinline{
  style = bw,
  breaklines,
}

\newminted[bashterm]{text}{%
  autogobble,%
  breaklines,%
  style=bw,%
}

\newminted[codegeneric]{text}{%
  autogobble,%
  style=bw,%
  breaklines,%
  fontsize=\small,%
}

\newmintinline{bash}{%
}

\newmintinline[minttext]{text}{%
  breaklines,%
  breakanywhere,%
}

%%% Framing code listings
\usepackage{tcolorbox}
\tcbuselibrary{breakable}
\tcbuselibrary{minted}
\tcbuselibrary{skins}

% Text file listing
\newtcblisting[
  auto counter,
  list inside,
  number within = section,
]{listingplaintext}[3][]{%
  minted language = text,
  minted style    = bw,
  minted options  = {
    autogobble,
    linenos,
    tabsize = 4,
    breaklines,
    breakanywhere,
    fontsize = \footnotesize,
  },
  empty,
  sharp corners,
  coltitle = black,
  borderline horizontal = {1pt}{0pt}{black},
  titlerule = {0.5pt},
  titlerule style = {
    black,
  },
  toptitle = 0.3em,
  bottomtitle = 0.3em,
  before skip      = \intextsep,
  after  skip      = \intextsep,
  title            = {Лістинг \thetcbcounter: #2},
  list entry       = {\protect\numberline{\thetcbcounter}#2},
  left = 0em,
  right = 0em,
  %
  listing only,
  breakable,
  %
  label = {#3},%
}

\newtcbinputlisting[
  use counter from = listingplaintext,
  list inside,
  number within = section
]{\inputplaintext}[4][]{%
  minted language = text,
  minted style    = bw,
  minted options  = {
    autogobble,
    linenos,
    tabsize = 4,
    breaklines,
    breakanywhere,
    fontsize = \footnotesize,
  },
  empty,
  sharp corners,
  coltitle = black,
  borderline horizontal = {1pt}{0pt}{black},
  titlerule = {0.5pt},
  titlerule style = {
    black,
  },
  toptitle = 0.3em,
  bottomtitle = 0.3em,
  before skip      = \intextsep,
  after  skip      = \intextsep,
  title            = {Лістинг \thetcbcounter: #3},
  list entry       = {\protect\numberline{\thetcbcounter}#3},
  left = 0em,
  right = 0em,
  %
  listing file={#2},
  listing only,
  breakable,
  %
  label = {#4}
}

\newtcblisting[
  use counter from = listingplaintext,
  list inside,
  number within = section,
]{listingpython}[3][]{%
  minted language = python,
  minted style    = bw,
  minted options  = {
    autogobble,
    linenos,
    tabsize = 4,
    breaklines,
    breakanywhere,
    fontsize = \footnotesize,
  },
  empty,
  sharp corners,
  coltitle = black,
  borderline horizontal = {1pt}{0pt}{black},
  titlerule = {0.5pt},
  titlerule style = {
    black,
  },
  toptitle = 0.3em,
  bottomtitle = 0.3em,
  before skip      = \intextsep,
  after  skip      = \intextsep,
  title            = {Лістинг \thetcbcounter: #2},
  list entry       = {\protect\numberline{\thetcbcounter}#2},
  left = 0em,
  right = 0em,
  %
  listing only,
  breakable,
  %
  label = {#3},
  %
  #1%
}

\newtcbinputlisting[
  use counter from = listingplaintext,
  list inside,
  number within = section
]{\inputpython}[4][]{%
  minted language = python,
  minted style    = bw,
  minted options  = {
    autogobble,
    linenos,
    tabsize = 4,
    breaklines,
    breakanywhere,
    fontsize = \footnotesize,
  },
  empty,
  sharp corners,
  coltitle = black,
  borderline horizontal = {1pt}{0pt}{black},
  titlerule = {0.5pt},
  titlerule style = {
    black,
  },
  toptitle = 0.3em,
  bottomtitle = 0.3em,
  before skip      = \intextsep,
  after  skip      = \intextsep,
  title            = {Лістинг \thetcbcounter: #3},
  list entry       = {\protect\numberline{\thetcbcounter}#3},
  left = 0em,
  right = 0em,
  %
  listing file={#2},
  listing only,
  breakable,
  %
  label = {#4}
}

% Linux command-line listing
\newtcblisting{linuxterm}%
{%
  % Syntax highlighing options
  listing only,%
  minted language = bash,%
  minted options={%
    autogobble,%
    linenos%
  },%
  % Presentation options
  empty,%
  %% Margins
  sharp corners,%
  toptitle = 0.0em,%
  bottomtitle = 0.0em,%
  left = 0em,%
  right = 0em,%
  before skip = \intextsep,%
  after skip = \intextsep,%
}

\newtcblisting{linuxtermout}%
{%
  % Syntax highlighing options
  listing only,%
  minted language = text,%
  minted options={%
    autogobble,%
    linenos%
  },%
  % Presentation options
  empty,%
  %% Margins
  sharp corners,%
  toptitle = 0.0em,%
  bottomtitle = 0.0em,%
  left = 0em,%
  right = 0em,%
  before skip = \intextsep,%
  after skip = \intextsep,%
}

% Dockerfile listings
\newtcblisting[
  use counter from = listingplaintext,
  list inside,
  number within = section,
]{listingdocker}[3][]{%
  minted language = dockerfile,
  minted style    = bw,
  minted options  = {
    autogobble,%
    linenos,
    tabsize = 4,
    breaklines,
    breakanywhere,
    fontsize = \footnotesize,
  },
  empty,
  sharp corners,
  coltitle = black,
  borderline horizontal = {1pt}{0pt}{black},
  titlerule = {0.5pt},
  titlerule style = {
    black,
  },
  toptitle = 0.3em,
  bottomtitle = 0.3em,
  before skip      = \intextsep,
  after  skip      = \intextsep,
  title            = {Лістинг \thetcbcounter: #2},
  list entry       = {\protect\numberline{\thetcbcounter}#2},
  left = 0em,
  right = 0em,
  %
  listing only,
  breakable,
  %
  label = {#3},%
}

% Docker Compose listings
\newtcblisting[
  use counter from = listingplaintext,
  list inside,
  number within = section,
]{listingdockercompose}[3][]{%
  minted language = yaml,
  minted style    = bw,
  minted options  = {
    autogobble,%
    linenos,
    tabsize = 4,
    breaklines,
    breakanywhere,
    fontsize = \footnotesize,
  },
  empty,
  sharp corners,
  coltitle = black,
  borderline horizontal = {1pt}{0pt}{black},
  titlerule = {0.5pt},
  titlerule style = {
    black,
  },
  toptitle = 0.3em,
  bottomtitle = 0.3em,
  before skip      = \intextsep,
  after  skip      = \intextsep,
  title            = {Лістинг \thetcbcounter: #2},
  list entry       = {\protect\numberline{\thetcbcounter}#2},
  left = 0em,
  right = 0em,
  %
  listing only,
  breakable,
  %
  label = {#3},%
}

% SWI Prolog listings
\newtcblisting[
  use counter from = listingplaintext,
  list inside,
  number within = section,
]{listingprolog}[3][]{%
  minted language = prolog,
  minted style    = bw,
  minted options  = {
    autogobble,%
    linenos,
    tabsize = 4,
    breaklines,
    breakanywhere,
    fontsize = \footnotesize,
  },
  empty,
  sharp corners,
  coltitle = black,
  borderline horizontal = {1pt}{0pt}{black},
  titlerule = {0.5pt},
  titlerule style = {
    black,
  },
  toptitle = 0.3em,
  bottomtitle = 0.3em,
  before skip      = \intextsep,
  after  skip      = \intextsep,
  title            = {Лістинг \thetcbcounter: #2},
  list entry       = {\protect\numberline{\thetcbcounter}#2},
  left = 0em,
  right = 0em,
  %
  listing only,
  breakable,
  %
  label = {#3},%
}

\newtcbinputlisting[
  use counter from = listingplaintext,
  list inside,
  number within = section
]{\inputprolog}[4][]{%
  minted language = prolog,
  minted style    = bw,
  minted options  = {
    autogobble,
    linenos,
    tabsize = 4,
    breaklines,
    breakanywhere,
    fontsize = \footnotesize,
  },
  empty,
  sharp corners,
  coltitle = black,
  borderline horizontal = {1pt}{0pt}{black},
  titlerule = {0.5pt},
  titlerule style = {
    black,
  },
  toptitle = 0.3em,
  bottomtitle = 0.3em,
  before skip      = \intextsep,
  after  skip      = \intextsep,
  title            = {Лістинг \thetcbcounter: #3},
  list entry       = {\protect\numberline{\thetcbcounter}#3},
  left = 0em,
  right = 0em,
  %
  listing file={#2},
  listing only,
  breakable,
  %
  label = {#4}
}

% JSON

\newtcbinputlisting[
  use counter from = listingplaintext,
  list inside,
  number within = section
]{\inputjson}[4][]{%
  minted language = json,
  minted style    = bw,
  minted options  = {
    autogobble,
    linenos,
    tabsize = 4,
    breaklines,
    breakanywhere,
    fontsize = \footnotesize,
  },
  empty,
  sharp corners,
  coltitle = black,
  borderline horizontal = {1pt}{0pt}{black},
  titlerule = {0.5pt},
  titlerule style = {
    black,
  },
  toptitle = 0.3em,
  bottomtitle = 0.3em,
  before skip      = \intextsep,
  after  skip      = \intextsep,
  title            = {Лістинг \thetcbcounter: #3},
  list entry       = {\protect\numberline{\thetcbcounter}#3},
  left = 0em,
  right = 0em,
  %
  listing file={#2},
  listing only,
  breakable,
  %
  label = {#4}
}


% Customize minted line numbers
\renewcommand{\theFancyVerbLine}{\ttfamily\scriptsize\arabic{FancyVerbLine}}

%%%

%%% Typeset menus and keys
\usepackage{menukeys}[
  os=win,
]
%%%

%%% Links and hyperreferences
\usepackage{hyperref}
\hypersetup{
  bookmarksnumbered = true,
  colorlinks      = false,
  linkbordercolor = red,
  urlbordercolor  = lightblue,
  pdfborderstyle  = {/S/U/W 1.5},
}
%%%

%%% Length adjustment

% Set baselineskip, default is 14.5 pt
\linespread{1.068966} % ~15.5 pt
\setlength{\emergencystretch}{1em}
\setlength{\parindent}{1.5em}
\newlength{\gridunitwidth}
\setlength{\gridunitwidth}{\textwidth / 12}
%%%

%%% Custom commands
\newcommand{\allcaps}[1]{%
  {%
    \addfontfeatures{%
      Letters = UppercaseSmallCaps,
      LetterSpace = 8,%
    }%
    #1%
  }%
}
\newcommand{\filename}[1]{\texttt{#1}}
\newcommand{\progname}[1]{\texttt{#1}}
\newcommand{\commandname}[1]{\texttt{#1}}
\newcommand{\modulename}[1]{\texttt{#1}}
\newcommand{\transeng}[1]{{англ.}~\textit{\textenglish{#1}}}
%%%

%%% Custom math commands
\newcommand{\longvar}[1]{\mathit{#1}}
\newcommand{\vect}[1]{\mathbfit{#1}}
\newcommand{\matr}[1]{\mathbfit{#1}}

\newcommand{\logequiv}{\mathrel{\Longleftrightarrow}} % Logically equivalent

\newcommand{\ssep}{\mid} % set builder separator

\DeclareMathOperator*{\minimize}{min} % minimize for linear programs
\DeclareMathOperator*{\rand}{rand} % rand()

\DeclarePairedDelimiter{\setpower}{\lvert}{\rvert} % set power
%%%

\begin{document}

\begin{titlepage}
    \begin{center}
      Міністерство освіти і~науки України\\
      Національний авіаційний університет\\
      Факультет кібербезпеки, комп'ютерної та~програмної інженерії\\
      Кафедра комп'ютеризованих систем управління

      \vspace{\fill}
        Модульна контрольна робота №~3\\
        з~дисципліни «Технології проектування комп'ютерних систем»\\
        Варіант №~3

      \vspace{\fill}

      \begin{flushright}
        Виконав:\\
        студент \allcaps{ФККПІ}\\
        групи \allcaps{СП}-425\\
        Клокун В.\,Д.\\
        Перевірив:\\
        Масловський Б.\,Г.
      \end{flushright}

      Київ 2020
    \end{center}
  \end{titlepage}

  \section{Хід~роботи}
    \subsection{Принципи управління розробкою ПЗ}
      Існує декілька різних методологій управління розробкою програмного забезпечення, кожна з~яких має~власні принципи управління. Розглянемо деякі з~цих~методологій.

      Принципи каскадної методології:
      \begin{enumerate}
        \item Завершувати проектування програми до~початку аналізу і~написання коду.
        \item Вести документацію своєчасно і~у~повному обсязі.
        \item Якщо це~можливо, необхідно виконувати роботу двічі.
        \item Планувати, контролювати та~наглядати за~тестуванням.
        \item Залучати замовника до~роботи.
      \end{enumerate}

      Каскадна методологія була популярною у~минулому, а~зараз більш популярною стала методологія~\textenglish{Agile}~(\transeng{гнучка}). Вона побудована на~таких принципах:
      \begin{enumerate}
        \item Найвищий пріоритет~— задоволення потреб замовника за~допомогою частих та~неперервних поставок цінного програмного забезпечення.
        \item Зміни у~вимогах характерні навіть на~пізніх етапах реалізації проекту.
        \item Працюючий продукт варто випускати кожні декілька тижнів, у~крайньому випадку~— кожні декілька місяців.
        \item Найбільш ефективний спосіб передачі інформації~— зустріч членів команди розробки програмного забезпечення.
        \item Представники бізнесу і~команда розробки повинні працювати над~проектом разом.
        \item Головна міра прогресу проекту — працююче програмне забезпечення.
        \item Гнучкі процеси сприяють неперервному розвитку.
        \item Постійна увага до~технічної досконалості та~якісної архітектури сприяють гнучкості.
        \item Найкраща архітектура, вимоги та~дизайн створюються у~самоорганізовуючихся командах.
        \item Команди повинні постійно шукати способи стати більш ефективними, налаштовуючи та~коригуючи свої дії.
      \end{enumerate}

    \subsection{Визначення та~графічне зображення зв'язків та~кратності в~асоціаціях}
      \subsubsection{Асоціація}
        Асоціація~(рис.~\ref{fig:assoc}) показує відношення між~двома класами об'єктів, яке~дозволяє одному екземпляру об'єкта викликати інший, щоб~виконати певну дію~від~його імені. В~загальному сенсі, причина використання відношення асоціації називається, на~кшталт «відправка повідомлення», «виклик методу» або~«виклик функції-члена» у~контрольованого об'єкта.
        \begin{figure}[!htbp]
          \centering
          \begin{subfigure}[t]{6\gridunitwidth - 0.5\gridunitwidth}
            \includegraphics[width = \columnwidth]{./assets/01-assoc-bidi.png}
            \caption{Двонаправлена асоціація}
            \label{subfig:assoc-bidi}
          \end{subfigure}%
          \hspace{1\gridunitwidth}%
          \begin{subfigure}[t]{6\gridunitwidth - 0.5\gridunitwidth}
            \includegraphics[width = \columnwidth]{./assets/02-assoc-unidi.png}
            \caption{Однонаправлена асоціація}
            \label{subfig:assoc-unidi}
          \end{subfigure}
          \caption{Графічне зображення відношення асоціації}
          \label{fig:assoc}
        \end{figure}

      Відношення асоціації також характеризуються кратністю~(рис.~\ref{fig:assoc-graphical}), тобто кількістю зв'язків між~кожним екземпляром класу на~початку лінії з~екземпляром класу на~її~кінці. Кратність може бути будь-якою, навіть нескінченною, підмножиною додатних цілих чисел: «1» — рівно 1, «1..*»~— 1 або~більше, «3..5»~— від~3 до~5. Спеціальний символ~«*» означає «багато», тобто 0 або~більше.

      \begin{table}[!htbp]
        \centering
        \begin{tabular}{
          v{2\gridunitwidth - 2\tabcolsep}
          v{3\gridunitwidth - 2\tabcolsep}
          v{7\gridunitwidth - 2\tabcolsep}
        }
          \toprule
            Нотація & Значення & Приклад \\
          \midrule
            0 .. 1 & Нуль або~один & До~робочої станції може бути підключений консольний термінал або~ні~\\
            1 & Рівно 1 & У~країни є~одна столиця \\
            0 .. * & Нуль або~більше екземплярів & На~одному екрані може бути відкрито багато вікон або~жодного \\
            1 .. * & Один або~більше & На~працюючому комп'ютері може бути встановлена одна операційна система або~більше\\
          \bottomrule
        \end{tabular}
      \end{table}

      \begin{figure}[!htbp]
        \centering
        \includegraphics[width = 4\gridunitwidth]{./assets/03-assoc-multiplicity.png}
        \caption{Графічне позначення кратності асоціацій}
        \label{fig:assoc-graphical}
      \end{figure}

    \subsection{Охарактеризувати СМО, що~в~нотації Кендалла позначено як~M/D/2/3 з~дисципліно обслуговування LIFO}
      Система масового обслуговування, яка~в~нотації Кендалла позначена як~\textenglish{M/D/2/3} з~дисципліною обслуговування~\textenglish{\allcaps{LIFO}} має~такі характеристики:
      \begin{enumerate}
        \item Перший символ описує процес прибуття до~черги. Літера~\textenglish{M} означає, що~система є~Марковською~(або~без~пам'яті), тобто час~прибуття нових заявок у~систему описується розподілом~Пуассона: \begin{IEEEeqnarray*}{c}
            f(k; \lambda) = P(X = k) = \frac{\lambda^k e^{-\lambda}}{k!}.
          \end{IEEEeqnarray*}

        \item Другий символ описує розподіл часу на~кожну заявку. Літера~\textenglish{D} показує, що~час~розподілений вирождено або~фіксовано. Наприклад, на~обробку кожної заявки йде~$t$~часу.
        \item Третій символ вказує кількість каналів у~системі. В~заданій системі 2~канали.
        \item Четвертий символ показує кількість місць у~черзі або~максимальний обсяг заявок, які~можуть знаходитись у~системі. Дана система може одночасно вмістити в~черзі 3 заявки.
        \item Дисципліна обслуговування~\textenglish{\allcaps{LIFO}~(last in, first out}, останній прибув~— перший вибув) означає, що~заявки оброблюються у~зворотньому порядку: у~першу чергу обслуговують ті~заявки, які~прибули останніми.
      \end{enumerate}

\end{document}
