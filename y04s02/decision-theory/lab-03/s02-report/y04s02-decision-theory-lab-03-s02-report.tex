\documentclass[
  a4paper,
  oneside,
  BCOR = 10mm,
  DIV = 12,
  12pt,
  headings = normal,
]{scrartcl}

%%% Length calculations
\usepackage{calc}
%%%

%%% Support for color
\usepackage{xcolor}
\definecolor{lightblue}{HTML}{03A9F4}
\definecolor{red}{HTML}{F44336}
%%%

%%% Including graphics
\usepackage{graphicx}
%%%

%%% Font selection
\usepackage{fontspec}

\setromanfont{STIX Two Text}[
  SmallCapsFeatures = {LetterSpace = 8},
]

\setsansfont{IBM Plex Sans}[
  Scale = MatchUppercase,
]

\setmonofont{IBM Plex Mono}[
  Scale = MatchUppercase,
]
%%%

%%% Math typesetting
\usepackage{amsmath}
\usepackage{mathtools}

\usepackage{unicode-math}
\setmathfont{STIX Two Math}

\usepackage{IEEEtrantools}

% Typeset matrices with row labels
\usepackage{kbordermatrix}
%%%

%%% List settings
\usepackage{enumitem}
\setlist[enumerate]{
  label*      = {\arabic*.},
  left        = \parindent,
  topsep      = 0\baselineskip,
  parsep      = 0\baselineskip,
  noitemsep, % override itemsep
}
% List settings for levels 2–4
\setlist[enumerate, 2, 3, 4]{
  label*      = {\arabic*.},
  left        = 0em,
  topsep      = 0\baselineskip,
  parsep      = 0\baselineskip,
  noitemsep, % override itemsep
}

\setlist[itemize]{
  label*      = {—},
  left        = \parindent,
  topsep      = 0\baselineskip,
  parsep      = 0\baselineskip,
  itemsep     = 1\baselineskip,
  noitemsep, % override itemsep
}

\setlist[description]{
  font        = {\rmfamily\upshape\bfseries},
  topsep      = 1\baselineskip,
  parsep      = 0\baselineskip,
  itemsep     = 0\baselineskip,
}

%%%

%%% Structural elements typesetting
\setkomafont{pagenumber}{\rmfamily\upshape}
\setkomafont{disposition}{\rmfamily\bfseries}

% Sectioning
\RedeclareSectionCommand[
  beforeskip = -1\baselineskip,
  afterskip  = 1\baselineskip,
  font       = {\normalsize\bfseries\scshape},
]{section}

\RedeclareSectionCommand[
  beforeskip = -1\baselineskip,
  afterskip  = 1\baselineskip,
  font       = {\normalsize\bfseries\itshape},
]{subsection}

\RedeclareSectionCommand[
  beforeskip = -1\baselineskip,
  afterskip  = 1\baselineskip,
  font       = {\normalsize\bfseries},
]{subsubsection}

\RedeclareSectionCommand[
  beforeskip = -1\baselineskip,
  afterskip  = -0.5em,
  font       = {\normalsize\mdseries\scshape\addfontfeatures{Letters = {UppercaseSmallCaps}}},
]{paragraph}
%%%

%%% Typographic enhancements
\usepackage{microtype}
%%%

%%% Language-specific settings
\usepackage{polyglossia}
\setmainlanguage{ukrainian}
\setotherlanguages{english}
%%%

%%% Captions
\usepackage{caption}
\usepackage{subcaption}

%\DeclareCaptionLabelFormat{closing}{#2)}
%\captionsetup[subtable]{labelformat = closing}

%\captionsetup[subfigure]{labelformat = closing}

\captionsetup[table]{
  aboveskip = 0\baselineskip,
  belowskip = 0\baselineskip,
}

\captionsetup[figure]{
  aboveskip = 1\baselineskip,
  belowskip = 0\baselineskip,
}

\captionsetup[subfigure]{
  labelformat = simple,
  labelformat = brace,
  justification = RaggedRight,
  singlelinecheck = false,
}
%%%

%%% Hyphenated ragged typesetting
\usepackage{ragged2e}
%%%

%%% Table typesetting
\usepackage{booktabs}
\usepackage{longtable}

\usepackage{multirow}

\usepackage{array}
\newcolumntype{v}[1]{>{\RaggedRight\arraybackslash\hspace{0pt}}p{#1}}
\newcolumntype{b}[1]{>{\Centering\arraybackslash\hspace{0pt}}p{#1}}
\newcolumntype{n}[1]{>{\RaggedLeft\arraybackslash\hspace{0pt}}p{#1}}
%%%

%%% Drawing
\usepackage{tikz}
\usepackage{tikzscale}
\usetikzlibrary{datavisualization}
\usetikzlibrary{datavisualization.formats.functions}
\usetikzlibrary{positioning}
\usetikzlibrary{patterns}
\usetikzlibrary{intersections}
\usetikzlibrary{arrows.meta} % Stealth arrow tips
\usetikzlibrary{graphs}
\usetikzlibrary{graphdrawing}
\usegdlibrary{layered}
\usetikzlibrary{quotes}

\usepackage{pgfplots}
\usepgfplotslibrary{fillbetween}
%%%

%%% SI units typesetting
\usepackage{siunitx}
\sisetup{
  output-decimal-marker = {,},
  exponent-product      = {\cdot},
  inter-unit-product    = \ensuremath{{} \cdot {}},
  per-mode              = symbol,
}
%%%

% Code Highlighting
\usepackage{minted}
\setmintedinline{
  style = bw,
  breaklines,
}

\newminted[bashterm]{text}{%
  autogobble,%
  breaklines,%
  style=bw,%
}

\newminted[codegeneric]{text}{%
  autogobble,%
  style=bw,%
  breaklines,%
  fontsize=\small,%
}

\newmintinline{bash}{%
}

\newmintinline[minttext]{text}{%
  breaklines,%
  breakanywhere,%
}

%%% Framing code listings
\usepackage{tcolorbox}
\tcbuselibrary{breakable}
\tcbuselibrary{minted}
\tcbuselibrary{skins}

% Text file listing
\newtcblisting[
  auto counter,
  list inside,
  number within = section,
]{listingplaintext}[3][]{%
  minted language = text,
  minted style    = bw,
  minted options  = {
    autogobble,
    linenos,
    tabsize = 4,
    breaklines,
    breakanywhere,
    fontsize = \footnotesize,
  },
  empty,
  sharp corners,
  coltitle = black,
  borderline horizontal = {1pt}{0pt}{black},
  titlerule = {0.5pt},
  titlerule style = {
    black,
  },
  toptitle = 0.3em,
  bottomtitle = 0.3em,
  before skip      = \intextsep,
  after  skip      = \intextsep,
  title            = {Лістинг \thetcbcounter: #2},
  list entry       = {\protect\numberline{\thetcbcounter}#2},
  left = 0em,
  right = 0em,
  %
  listing only,
  breakable,
  %
  label = {#3},%
}

\newtcbinputlisting[
  use counter from = listingplaintext,
  list inside,
  number within = section
]{\inputplaintext}[4][]{%
  minted language = text,
  minted style    = bw,
  minted options  = {
    autogobble,
    linenos,
    tabsize = 4,
    breaklines,
    breakanywhere,
    fontsize = \footnotesize,
  },
  empty,
  sharp corners,
  coltitle = black,
  borderline horizontal = {1pt}{0pt}{black},
  titlerule = {0.5pt},
  titlerule style = {
    black,
  },
  toptitle = 0.3em,
  bottomtitle = 0.3em,
  before skip      = \intextsep,
  after  skip      = \intextsep,
  title            = {Лістинг \thetcbcounter: #3},
  list entry       = {\protect\numberline{\thetcbcounter}#3},
  left = 0em,
  right = 0em,
  %
  listing file={#2},
  listing only,
  breakable,
  %
  label = {#4}
}

\newtcblisting[
  use counter from = listingplaintext,
  list inside,
  number within = section,
]{listingpython}[3][]{%
  minted language = python,
  minted style    = bw,
  minted options  = {
    autogobble,
    linenos,
    tabsize = 4,
    breaklines,
    breakanywhere,
    fontsize = \footnotesize,
  },
  empty,
  sharp corners,
  coltitle = black,
  borderline horizontal = {1pt}{0pt}{black},
  titlerule = {0.5pt},
  titlerule style = {
    black,
  },
  toptitle = 0.3em,
  bottomtitle = 0.3em,
  before skip      = \intextsep,
  after  skip      = \intextsep,
  title            = {Лістинг \thetcbcounter: #2},
  list entry       = {\protect\numberline{\thetcbcounter}#2},
  left = 0em,
  right = 0em,
  %
  listing only,
  breakable,
  %
  label = {#3},
  %
  #1%
}

\newtcbinputlisting[
  use counter from = listingplaintext,
  list inside,
  number within = section
]{\inputpython}[4][]{%
  minted language = python,
  minted style    = bw,
  minted options  = {
    autogobble,
    linenos,
    tabsize = 4,
    breaklines,
    breakanywhere,
    fontsize = \footnotesize,
  },
  empty,
  sharp corners,
  coltitle = black,
  borderline horizontal = {1pt}{0pt}{black},
  titlerule = {0.5pt},
  titlerule style = {
    black,
  },
  toptitle = 0.3em,
  bottomtitle = 0.3em,
  before skip      = \intextsep,
  after  skip      = \intextsep,
  title            = {Лістинг \thetcbcounter: #3},
  list entry       = {\protect\numberline{\thetcbcounter}#3},
  left = 0em,
  right = 0em,
  %
  listing file={#2},
  listing only,
  breakable,
  %
  label = {#4}
}

% Linux command-line listing
\newtcblisting{linuxterm}%
{%
  % Syntax highlighing options
  listing only,%
  minted language = bash,%
  minted options={%
    autogobble,%
    linenos%
  },%
  % Presentation options
  empty,%
  %% Margins
  sharp corners,%
  toptitle = 0.0em,%
  bottomtitle = 0.0em,%
  left = 0em,%
  right = 0em,%
  before skip = \intextsep,%
  after skip = \intextsep,%
}

\newtcblisting{linuxtermout}%
{%
  % Syntax highlighing options
  listing only,%
  minted language = text,%
  minted options={%
    autogobble,%
    linenos%
  },%
  % Presentation options
  empty,%
  %% Margins
  sharp corners,%
  toptitle = 0.0em,%
  bottomtitle = 0.0em,%
  left = 0em,%
  right = 0em,%
  before skip = \intextsep,%
  after skip = \intextsep,%
}

% Dockerfile listings
\newtcblisting[
  use counter from = listingplaintext,
  list inside,
  number within = section,
]{listingdocker}[3][]{%
  minted language = dockerfile,
  minted style    = bw,
  minted options  = {
    autogobble,%
    linenos,
    tabsize = 4,
    breaklines,
    breakanywhere,
    fontsize = \footnotesize,
  },
  empty,
  sharp corners,
  coltitle = black,
  borderline horizontal = {1pt}{0pt}{black},
  titlerule = {0.5pt},
  titlerule style = {
    black,
  },
  toptitle = 0.3em,
  bottomtitle = 0.3em,
  before skip      = \intextsep,
  after  skip      = \intextsep,
  title            = {Лістинг \thetcbcounter: #2},
  list entry       = {\protect\numberline{\thetcbcounter}#2},
  left = 0em,
  right = 0em,
  %
  listing only,
  breakable,
  %
  label = {#3},%
}

% Docker Compose listings
\newtcblisting[
  use counter from = listingplaintext,
  list inside,
  number within = section,
]{listingdockercompose}[3][]{%
  minted language = yaml,
  minted style    = bw,
  minted options  = {
    autogobble,%
    linenos,
    tabsize = 4,
    breaklines,
    breakanywhere,
    fontsize = \footnotesize,
  },
  empty,
  sharp corners,
  coltitle = black,
  borderline horizontal = {1pt}{0pt}{black},
  titlerule = {0.5pt},
  titlerule style = {
    black,
  },
  toptitle = 0.3em,
  bottomtitle = 0.3em,
  before skip      = \intextsep,
  after  skip      = \intextsep,
  title            = {Лістинг \thetcbcounter: #2},
  list entry       = {\protect\numberline{\thetcbcounter}#2},
  left = 0em,
  right = 0em,
  %
  listing only,
  breakable,
  %
  label = {#3},%
}

% SWI Prolog listings
\newtcblisting[
  use counter from = listingplaintext,
  list inside,
  number within = section,
]{listingprolog}[3][]{%
  minted language = prolog,
  minted style    = bw,
  minted options  = {
    autogobble,%
    linenos,
    tabsize = 4,
    breaklines,
    breakanywhere,
    fontsize = \footnotesize,
  },
  empty,
  sharp corners,
  coltitle = black,
  borderline horizontal = {1pt}{0pt}{black},
  titlerule = {0.5pt},
  titlerule style = {
    black,
  },
  toptitle = 0.3em,
  bottomtitle = 0.3em,
  before skip      = \intextsep,
  after  skip      = \intextsep,
  title            = {Лістинг \thetcbcounter: #2},
  list entry       = {\protect\numberline{\thetcbcounter}#2},
  left = 0em,
  right = 0em,
  %
  listing only,
  breakable,
  %
  label = {#3},%
}

\newtcbinputlisting[
  use counter from = listingplaintext,
  list inside,
  number within = section
]{\inputprolog}[4][]{%
  minted language = prolog,
  minted style    = bw,
  minted options  = {
    autogobble,
    linenos,
    tabsize = 4,
    breaklines,
    breakanywhere,
    fontsize = \footnotesize,
  },
  empty,
  sharp corners,
  coltitle = black,
  borderline horizontal = {1pt}{0pt}{black},
  titlerule = {0.5pt},
  titlerule style = {
    black,
  },
  toptitle = 0.3em,
  bottomtitle = 0.3em,
  before skip      = \intextsep,
  after  skip      = \intextsep,
  title            = {Лістинг \thetcbcounter: #3},
  list entry       = {\protect\numberline{\thetcbcounter}#3},
  left = 0em,
  right = 0em,
  %
  listing file={#2},
  listing only,
  breakable,
  %
  label = {#4}
}

% JSON

\newtcbinputlisting[
  use counter from = listingplaintext,
  list inside,
  number within = section
]{\inputjson}[4][]{%
  minted language = json,
  minted style    = bw,
  minted options  = {
    autogobble,
    linenos,
    tabsize = 4,
    breaklines,
    breakanywhere,
    fontsize = \footnotesize,
  },
  empty,
  sharp corners,
  coltitle = black,
  borderline horizontal = {1pt}{0pt}{black},
  titlerule = {0.5pt},
  titlerule style = {
    black,
  },
  toptitle = 0.3em,
  bottomtitle = 0.3em,
  before skip      = \intextsep,
  after  skip      = \intextsep,
  title            = {Лістинг \thetcbcounter: #3},
  list entry       = {\protect\numberline{\thetcbcounter}#3},
  left = 0em,
  right = 0em,
  %
  listing file={#2},
  listing only,
  breakable,
  %
  label = {#4}
}


% Customize minted line numbers
\renewcommand{\theFancyVerbLine}{\ttfamily\scriptsize\arabic{FancyVerbLine}}

%%%

%%% Typeset menus and keys
\usepackage{menukeys}[
  os=win,
]
%%%

%%% Links and hyperreferences
\usepackage{hyperref}
\hypersetup{
  bookmarksnumbered = true,
  colorlinks      = false,
  linkbordercolor = red,
  urlbordercolor  = lightblue,
  pdfborderstyle  = {/S/U/W 1.5},
}
%%%

%%% Length adjustment

% Set baselineskip, default is 14.5 pt
\linespread{1.068966} % ~15.5 pt
\setlength{\emergencystretch}{1em}
\setlength{\parindent}{1.5em}
\newlength{\gridunitwidth}
\setlength{\gridunitwidth}{\textwidth / 12}
%%%

%%% Custom commands
\newcommand{\allcaps}[1]{%
  {%
    \addfontfeatures{%
      Letters = UppercaseSmallCaps,
      LetterSpace = 8,%
    }%
    #1%
  }%
}
\newcommand{\filename}[1]{\texttt{#1}}
\newcommand{\progname}[1]{\texttt{#1}}
\newcommand{\commandname}[1]{\texttt{#1}}
\newcommand{\modulename}[1]{\texttt{#1}}
\newcommand{\transeng}[1]{{англ.}~\textit{\textenglish{#1}}}
%%%

%%% Custom math commands
\newcommand{\longvar}[1]{\mathit{#1}}
\newcommand{\vect}[1]{\mathbfit{#1}}
\newcommand{\matr}[1]{\mathbfit{#1}}

\newcommand{\logequiv}{\mathrel{\Longleftrightarrow}} % Logically equivalent

\newcommand{\ssep}{\mid} % set builder separator

\DeclareMathOperator*{\minimize}{min} % minimize for linear programs
\DeclareMathOperator*{\rand}{rand} % rand()

\DeclarePairedDelimiter{\setpower}{\lvert}{\rvert} % set power
%%%

\begin{document}

\begin{titlepage}
    \begin{center}
      Міністерство освіти і~науки України\\
      Національний авіаційний університет\\
      Факультет кібербезпеки, комп'ютерної та~програмної інженерії\\
      Кафедра комп'ютеризованих систем управління

      \vspace{\fill}
        Лабораторна робота №~3\\
        з~дисципліни «Системи підтримки прийняття рішень»\\
        на~тему «Прийняття рішень в~умовах ризику. Критерій очікуваного значення»\\

      \vspace{\fill}

      \begin{flushright}
        Виконав:\\
        студент \allcaps{ФККПІ}\\
        групи \allcaps{СП}-425\\
        Клокун В.\,Д.\\
        Перевірила:\\
        Яковенко Л.\,В.
      \end{flushright}

      Київ 2020
    \end{center}
  \end{titlepage}

  \section{Мета роботи}
    Ознайомитись з~методом прийняття рішень в~умовах ризику, сформулювати отримане завдання прийняття рішення в~умовах ризику та~обрати оптимальну альтернативу.

  \section{Хід~роботи}
    \paragraph{Задача}
    Припустимо, у~вас~є~можливість вкласти гроші в~три~інвестиційні фонди відкритого типу: простий, спеціальний (що~забезпечує максимальний довгостроковий прибуток від~акцій дрібних компаній) та~глобальний. Прибуток від~інвестиції може змінитися в~залежності від~умов ринку. Існує 10\%-ва~ймовірність, що~ситуація на~ринку цінних паперів погіршиться, 50\%-ва~— що~ринок залишиться помірним і~40\%-ва~— ринок буде зростати. Наступна таблиця містить значення відсотків прибутку від~суми інвестицій.
    \begin{table}[!htbp]
      \begin{tabular}{
        v{3\gridunitwidth - 2\tabcolsep}
        n{3\gridunitwidth - 2\tabcolsep}
        n{3\gridunitwidth - 2\tabcolsep}
        n{3\gridunitwidth - 2\tabcolsep}
      }
        \toprule
        & \multicolumn{3}{c}{Прибуток від~інвестиції, \%} \\
        \cmidrule(lr){2-4}
        Альтернатива (фонди) & Погіршуючийся ринок & Помірний ринок & Зростаючий ринок\\
        \midrule
          Простий     &  +5 & +7 &  +8\\
          Спеціальний & −10 & +5 & +30\\
          Глобальний  &  +2 & +7 & +20\\
        \bottomrule
      \end{tabular}
    \end{table}

    Необхідно зобразити задачу у~вигляді дерева рішень та~визначити, який фонд слід обрати.

    \paragraph{Розв'язання}
    Побудуємо дерево рішень для сформульованої задачі~(рис.~\ref{fig:decision-tree}).
    \begin{figure}[!htbp]
      \begin{tikzpicture}
        [
          treenode/.style = {
            shape = rectangle,
            rounded corners,
            draw,
            align = center,
          },
          layered layout,
          root/.style     = {treenode},
          dummy/.style    = {circle, draw},
          grow                    = right,
          sibling distance        = 3em,
          level distance          = 10em,
          edge from parent/.style = {draw, -latex, near end},
          every node/.style       = {font=\footnotesize},
          sloped,
          edge from parent path = {
            (\tikzparentnode) |-   % Start from parent
            (\tikzchildnode)
          }
        ]
        \node [root] {1}
          child {
            node [dummy] {2}
            child {
              node [dummy] {5}
              child {
                node {+5}
                edge from parent node [above] {P = \num{0.1}}
              }
              edge from parent node [below] {Погіршуючийся}
            }
            child {
              node [dummy] {6}
              child {
                node {+7}
                edge from parent node [above] {P = \num{0.6}}
              }
              edge from parent node [above, pos = 0.7] {Помірний}
            }
            child {
              node [dummy] {7}
              child {
                node {+8}
                edge from parent node [above] {P = \num{0.4}}
              }
              edge from parent node [above] {Зростаючий}
            }
            edge from parent node [below] {Простий}
          }
          child {
            node [dummy] {3}
            child {
              node [dummy] {8}
              child {
                node {−10}
                edge from parent node [above] {P = \num{0.1}}
              }
              edge from parent node [below] {Погіршуючийся}
            }
            child {
              node [dummy] {9}
              child {
                node {+5}
                edge from parent node [above] {P = \num{0.6}}
              }
              edge from parent node [above, pos = 0.7] {Помірний}
            }
            child {
              node [dummy] {10}
              child {
                node {+30}
                edge from parent node [above] {P = \num{0.4}}
              }
              edge from parent node [above] {Зростаючий}
            }
            edge from parent node [below] {Спеціальний}
          }
          child {
            node [dummy] {4}
            child {
              node [dummy] {11}
              child {
                node {+2}
                edge from parent node [above] {P = \num{0.1}}
              }
              edge from parent node [below] {Погіршуючийся}
            }
            child {
              node [dummy] {12}
              child {
                node {+7}
                edge from parent node [above] {P = \num{0.6}}
              }
              edge from parent node [above, pos = 0.7] {Помірний}
            }
            child {
              node [dummy] {13}
              child {
                node {+20}
                edge from parent node [above] {P = \num{0.4}}
              }
              edge from parent node [above] {Зростаючий}
            }
            edge from parent node [above] {Глобальний}
          }
          ;
      \end{tikzpicture}
      \caption{Дерево рішень для сформульованої задачі}
      \label{fig:decision-tree}
    \end{figure}

    Розроблюємо програму, яка~розв'яже~задачу, оцінивши значення критерію очікуваного значення. Вона складатиметься з~модуля, який розв'язуватиме задачу~(лістинг~\ref{lst:main}). Запускаємо програму і~спостерігаємо результат~(рис.~\ref{fig:res}).

    \begin{figure}[!htbp]
      \centering
      \includegraphics[width = 12\gridunitwidth]{./assets/y04s02-decision-theory-lab-03-s02-report-p01.png}
      \caption{Результат роботи розробленої програми}
      \label{fig:res}
    \end{figure}

    Видно, що~розроблена реалізація розв'язала поставлену задачу і~визначила, який фонд варто обрати. З~результату зрозуміло, що~краще обрати альтернативу~2~— спеціальний фонд, бо~вона дає~найбільший очікуваний прибуток.

  \section{Висновок}
    Виконуючи дану лабораторну роботу, ми~ознайомились з~методом прийняття рішень в~умовах ризику, сформулювали отримане завдання прийняття рішення в~умовах ризику та~обрали оптимальну альтернативу.

  \appendix
  \section{Лістинг коду програмної реалізації}
    \inputpython{../s01-solution/solver.py}{Файл~\textenglish{\filename{solver.py}}}{lst:main}

\end{document}
