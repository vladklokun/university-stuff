\documentclass[
  a4paper,
  oneside,
  BCOR = 10mm,
  DIV = 12,
  12pt,
  headings = normal,
]{scrartcl}

%%% Length calculations
\usepackage{calc}
%%%

%%% Support for color
\usepackage{xcolor}
\definecolor{lightblue}{HTML}{03A9F4}
\definecolor{red}{HTML}{F44336}
%%%

%%% Including graphics
\usepackage{graphicx}
%%%

%%% Font selection
\usepackage{fontspec}

\setromanfont{STIX Two Text}[
  SmallCapsFeatures = {LetterSpace = 8},
]

\setsansfont{IBM Plex Sans}[
  Scale = MatchUppercase,
]

\setmonofont{IBM Plex Mono}[
  Scale = MatchUppercase,
]
%%%

%%% Math typesetting
\usepackage{amsmath}
\usepackage{mathtools}

\usepackage{unicode-math}
\setmathfont{STIX Two Math}

\usepackage{IEEEtrantools}
%%%

%%% List settings
\usepackage{enumitem}
\setlist[enumerate]{
  label*      = {\arabic*.},
  left        = \parindent,
  topsep      = 0\baselineskip,
  parsep      = 0\baselineskip,
  noitemsep, % override itemsep
}
% List settings for levels 2–4
\setlist[enumerate, 2, 3, 4]{
  label*      = {\arabic*.},
  left        = 0em,
  topsep      = 0\baselineskip,
  parsep      = 0\baselineskip,
  noitemsep, % override itemsep
}

\setlist[itemize]{
  label*      = {—},
  left        = \parindent,
  topsep      = 0\baselineskip,
  parsep      = 0\baselineskip,
  itemsep     = 1\baselineskip,
  noitemsep, % override itemsep
}

\setlist[description]{
  font        = {\rmfamily\upshape\bfseries},
  topsep      = 1\baselineskip,
  parsep      = 0\baselineskip,
  itemsep     = 0\baselineskip,
}

%%%

%%% Structural elements typesetting
\setkomafont{pagenumber}{\rmfamily\upshape}
\setkomafont{disposition}{\rmfamily\bfseries}

% Sectioning
\RedeclareSectionCommand[
  beforeskip = -1\baselineskip,
  afterskip  = 1\baselineskip,
  font       = {\normalsize\bfseries\scshape},
]{section}

\RedeclareSectionCommand[
  beforeskip = -1\baselineskip,
  afterskip  = 1\baselineskip,
  font       = {\normalsize\bfseries\itshape},
]{subsection}

\RedeclareSectionCommand[
  beforeskip = -1\baselineskip,
  afterskip  = 1\baselineskip,
  font       = {\normalsize\bfseries},
]{subsubsection}

\RedeclareSectionCommand[
  beforeskip = -1\baselineskip,
  afterskip  = -0.5em,
  font       = {\normalsize\mdseries\scshape\addfontfeatures{Letters = {UppercaseSmallCaps}}},
]{paragraph}
%%%

%%% Typographic enhancements
\usepackage{microtype}
%%%

%%% Language-specific settings
\usepackage{polyglossia}
\setmainlanguage{ukrainian}
\setotherlanguages{english}
%%%

%%% Captions
\usepackage{caption}
\usepackage{subcaption}

%\DeclareCaptionLabelFormat{closing}{#2)}
%\captionsetup[subtable]{labelformat = closing}

%\captionsetup[subfigure]{labelformat = closing}

\captionsetup[table]{
  aboveskip = 0\baselineskip,
  belowskip = 0\baselineskip,
}

\captionsetup[figure]{
  aboveskip = 1\baselineskip,
  belowskip = 0\baselineskip,
}

\captionsetup[subfigure]{
  labelformat = simple,
  labelformat = brace,
  justification = RaggedRight,
  singlelinecheck = false,
}
%%%

%%% Hyphenated ragged typesetting
\usepackage{ragged2e}
%%%

%%% Table typesetting
\usepackage{booktabs}
\usepackage{longtable}

\usepackage{multirow}

\usepackage{array}
\newcolumntype{v}[1]{>{\RaggedRight\arraybackslash\hspace{0pt}}p{#1}}
\newcolumntype{b}[1]{>{\Centering\arraybackslash\hspace{0pt}}p{#1}}
\newcolumntype{n}[1]{>{\RaggedLeft\arraybackslash\hspace{0pt}}p{#1}}
%%%

%%% Drawing
\usepackage{tikz}
\usepackage{tikzscale}
\usetikzlibrary{datavisualization}
\usetikzlibrary{datavisualization.formats.functions}
\usetikzlibrary{positioning}
\usetikzlibrary{patterns}
\usetikzlibrary{intersections}
\usetikzlibrary{arrows.meta} % Stealth arrow tips
\usetikzlibrary{graphs}
\usetikzlibrary{graphdrawing}
\usegdlibrary{layered}
\usetikzlibrary{quotes}

\usepackage{pgfplots}
\usepgfplotslibrary{fillbetween}
%%%

%%% SI units typesetting
\usepackage{siunitx}
\sisetup{
  output-decimal-marker = {,},
  exponent-product      = {\cdot},
  inter-unit-product    = \ensuremath{{} \cdot {}},
  per-mode              = symbol,
}
%%%

% Code Highlighting
\usepackage{minted}
\setmintedinline{
  style = bw,
  breaklines,
}

\newminted[bashterm]{text}{%
  autogobble,%
  breaklines,%
  style=bw,%
}

\newminted[codegeneric]{text}{%
  autogobble,%
  style=bw,%
  breaklines,%
  fontsize=\small,%
}

\newmintinline{bash}{%
}

\newmintinline[minttext]{text}{%
  breaklines,%
  breakanywhere,%
}

%%% Framing code listings
\usepackage{tcolorbox}
\tcbuselibrary{breakable}
\tcbuselibrary{minted}
\tcbuselibrary{skins}

% Text file listing
\newtcblisting[
  auto counter,
  list inside,
  number within = section,
]{listingplaintext}[3][]{%
  minted language = text,
  minted style    = bw,
  minted options  = {
    autogobble,
    linenos,
    tabsize = 4,
    breaklines,
    breakanywhere,
    fontsize = \footnotesize,
  },
  empty,
  sharp corners,
  coltitle = black,
  borderline horizontal = {1pt}{0pt}{black},
  titlerule = {0.5pt},
  titlerule style = {
    black,
  },
  toptitle = 0.3em,
  bottomtitle = 0.3em,
  before skip      = \intextsep,
  after  skip      = \intextsep,
  title            = {Лістинг \thetcbcounter: #2},
  list entry       = {\protect\numberline{\thetcbcounter}#2},
  left = 0em,
  right = 0em,
  %
  listing only,
  breakable,
  %
  label = {#3},%
}

\newtcbinputlisting[
  use counter from = listingplaintext,
  list inside,
  number within = section
]{\inputplaintext}[4][]{%
  minted language = text,
  minted style    = bw,
  minted options  = {
    autogobble,
    linenos,
    tabsize = 4,
    breaklines,
    breakanywhere,
    fontsize = \footnotesize,
  },
  empty,
  sharp corners,
  coltitle = black,
  borderline horizontal = {1pt}{0pt}{black},
  titlerule = {0.5pt},
  titlerule style = {
    black,
  },
  toptitle = 0.3em,
  bottomtitle = 0.3em,
  before skip      = \intextsep,
  after  skip      = \intextsep,
  title            = {Лістинг \thetcbcounter: #3},
  list entry       = {\protect\numberline{\thetcbcounter}#3},
  left = 0em,
  right = 0em,
  %
  listing file={#2},
  listing only,
  breakable,
  %
  label = {#4}
}

\newtcblisting[
  use counter from = listingplaintext,
  list inside,
  number within = section,
]{listingpython}[3][]{%
  minted language = python,
  minted style    = bw,
  minted options  = {
    autogobble,
    linenos,
    tabsize = 4,
    breaklines,
    breakanywhere,
    fontsize = \footnotesize,
  },
  empty,
  sharp corners,
  coltitle = black,
  borderline horizontal = {1pt}{0pt}{black},
  titlerule = {0.5pt},
  titlerule style = {
    black,
  },
  toptitle = 0.3em,
  bottomtitle = 0.3em,
  before skip      = \intextsep,
  after  skip      = \intextsep,
  title            = {Лістинг \thetcbcounter: #2},
  list entry       = {\protect\numberline{\thetcbcounter}#2},
  left = 0em,
  right = 0em,
  %
  listing only,
  breakable,
  %
  label = {#3},
  %
  #1%
}

\newtcbinputlisting[
  use counter from = listingplaintext,
  list inside,
  number within = section
]{\inputpython}[4][]{%
  minted language = python,
  minted style    = bw,
  minted options  = {
    autogobble,
    linenos,
    tabsize = 4,
    breaklines,
    breakanywhere,
    fontsize = \footnotesize,
  },
  empty,
  sharp corners,
  coltitle = black,
  borderline horizontal = {1pt}{0pt}{black},
  titlerule = {0.5pt},
  titlerule style = {
    black,
  },
  toptitle = 0.3em,
  bottomtitle = 0.3em,
  before skip      = \intextsep,
  after  skip      = \intextsep,
  title            = {Лістинг \thetcbcounter: #3},
  list entry       = {\protect\numberline{\thetcbcounter}#3},
  left = 0em,
  right = 0em,
  %
  listing file={#2},
  listing only,
  breakable,
  %
  label = {#4}
}

% Linux command-line listing
\newtcblisting{linuxterm}%
{%
  % Syntax highlighing options
  listing only,%
  minted language = bash,%
  minted options={%
    autogobble,%
    linenos%
  },%
  % Presentation options
  empty,%
  %% Margins
  sharp corners,%
  toptitle = 0.0em,%
  bottomtitle = 0.0em,%
  left = 0em,%
  right = 0em,%
  before skip = \intextsep,%
  after skip = \intextsep,%
}

\newtcblisting{linuxtermout}%
{%
  % Syntax highlighing options
  listing only,%
  minted language = text,%
  minted options={%
    autogobble,%
    linenos%
  },%
  % Presentation options
  empty,%
  %% Margins
  sharp corners,%
  toptitle = 0.0em,%
  bottomtitle = 0.0em,%
  left = 0em,%
  right = 0em,%
  before skip = \intextsep,%
  after skip = \intextsep,%
}

% Dockerfile listings
\newtcblisting[
  use counter from = listingplaintext,
  list inside,
  number within = section,
]{listingdocker}[3][]{%
  minted language = dockerfile,
  minted style    = bw,
  minted options  = {
    autogobble,%
    linenos,
    tabsize = 4,
    breaklines,
    breakanywhere,
    fontsize = \footnotesize,
  },
  empty,
  sharp corners,
  coltitle = black,
  borderline horizontal = {1pt}{0pt}{black},
  titlerule = {0.5pt},
  titlerule style = {
    black,
  },
  toptitle = 0.3em,
  bottomtitle = 0.3em,
  before skip      = \intextsep,
  after  skip      = \intextsep,
  title            = {Лістинг \thetcbcounter: #2},
  list entry       = {\protect\numberline{\thetcbcounter}#2},
  left = 0em,
  right = 0em,
  %
  listing only,
  breakable,
  %
  label = {#3},%
}

% Docker Compose listings
\newtcblisting[
  use counter from = listingplaintext,
  list inside,
  number within = section,
]{listingdockercompose}[3][]{%
  minted language = yaml,
  minted style    = bw,
  minted options  = {
    autogobble,%
    linenos,
    tabsize = 4,
    breaklines,
    breakanywhere,
    fontsize = \footnotesize,
  },
  empty,
  sharp corners,
  coltitle = black,
  borderline horizontal = {1pt}{0pt}{black},
  titlerule = {0.5pt},
  titlerule style = {
    black,
  },
  toptitle = 0.3em,
  bottomtitle = 0.3em,
  before skip      = \intextsep,
  after  skip      = \intextsep,
  title            = {Лістинг \thetcbcounter: #2},
  list entry       = {\protect\numberline{\thetcbcounter}#2},
  left = 0em,
  right = 0em,
  %
  listing only,
  breakable,
  %
  label = {#3},%
}

% SWI Prolog listings
\newtcblisting[
  use counter from = listingplaintext,
  list inside,
  number within = section,
]{listingprolog}[3][]{%
  minted language = prolog,
  minted style    = bw,
  minted options  = {
    autogobble,%
    linenos,
    tabsize = 4,
    breaklines,
    breakanywhere,
    fontsize = \footnotesize,
  },
  empty,
  sharp corners,
  coltitle = black,
  borderline horizontal = {1pt}{0pt}{black},
  titlerule = {0.5pt},
  titlerule style = {
    black,
  },
  toptitle = 0.3em,
  bottomtitle = 0.3em,
  before skip      = \intextsep,
  after  skip      = \intextsep,
  title            = {Лістинг \thetcbcounter: #2},
  list entry       = {\protect\numberline{\thetcbcounter}#2},
  left = 0em,
  right = 0em,
  %
  listing only,
  breakable,
  %
  label = {#3},%
}

\newtcbinputlisting[
  use counter from = listingplaintext,
  list inside,
  number within = section
]{\inputprolog}[4][]{%
  minted language = prolog,
  minted style    = bw,
  minted options  = {
    autogobble,
    linenos,
    tabsize = 4,
    breaklines,
    breakanywhere,
    fontsize = \footnotesize,
  },
  empty,
  sharp corners,
  coltitle = black,
  borderline horizontal = {1pt}{0pt}{black},
  titlerule = {0.5pt},
  titlerule style = {
    black,
  },
  toptitle = 0.3em,
  bottomtitle = 0.3em,
  before skip      = \intextsep,
  after  skip      = \intextsep,
  title            = {Лістинг \thetcbcounter: #3},
  list entry       = {\protect\numberline{\thetcbcounter}#3},
  left = 0em,
  right = 0em,
  %
  listing file={#2},
  listing only,
  breakable,
  %
  label = {#4}
}


% Customize minted line numbers
\renewcommand{\theFancyVerbLine}{\ttfamily\scriptsize\arabic{FancyVerbLine}}

%%%

%%% Typeset menus and keys
\usepackage{menukeys}[
  os=win,
]
%%%

%%% Links and hyperreferences
\usepackage{hyperref}
\hypersetup{
  bookmarksnumbered = true,
  colorlinks      = false,
  linkbordercolor = red,
  urlbordercolor  = lightblue,
  pdfborderstyle  = {/S/U/W 1.5},
}
%%%

%%% Length adjustment

% Set baselineskip, default is 14.5 pt
\linespread{1.068966} % ~15.5 pt
\setlength{\emergencystretch}{1em}
\setlength{\parindent}{1.5em}
\newlength{\gridunitwidth}
\setlength{\gridunitwidth}{\textwidth / 12}
%%%

%%% Custom commands
\newcommand{\allcaps}[1]{%
  {%
    \addfontfeatures{%
      Letters = UppercaseSmallCaps,
      LetterSpace = 8,%
    }%
    #1%
  }%
}
\newcommand{\filename}[1]{\texttt{#1}}
\newcommand{\progname}[1]{\texttt{#1}}
\newcommand{\commandname}[1]{\texttt{#1}}
\newcommand{\modulename}[1]{\texttt{#1}}
\newcommand{\transeng}[1]{{англ.}~\textit{\textenglish{#1}}}
%%%

%%% Custom math commands
\newcommand{\longvar}[1]{\mathit{#1}}
\newcommand{\vect}[1]{\mathbfit{#1}}
\newcommand{\matr}[1]{\mathbfit{#1}}

\newcommand{\logequiv}{\mathrel{\Longleftrightarrow}} % Logically equivalent

\newcommand{\ssep}{\mid} % set builder separator

\DeclareMathOperator*{\minimize}{min} % minimize for linear programs
\DeclareMathOperator*{\rand}{rand} % rand()

\DeclarePairedDelimiter{\setpower}{\lvert}{\rvert} % set power
%%%

\begin{document}

\begin{titlepage}
    \begin{center}
      Міністерство освіти і~науки України\\
      Національний авіаційний університет\\
      Факультет кібербезпеки, комп'ютерної та~програмної інженерії\\
      Кафедра комп'ютеризованих систем управління

      \vspace{\fill}
        Лабораторна робота №~1\\
        з~дисципліни «Системи підтримки прийняття рішень»\\
        на~тему «Введення в~теорію прийняття рішень»\\

      \vspace{\fill}

      \begin{flushright}
        Виконав:\\
        студент \allcaps{ФККПІ}\\
        групи \allcaps{СП}-425\\
        Клокун В.\,Д.\\
        Перевірила:\\
        Яковенко Л.\,В.
      \end{flushright}

      Київ 2020
    \end{center}
  \end{titlepage}

  \section{Мета роботи}
    Ознайомитись з~основним принципом теорії прийняття рішень, методами та~класами задач теорії прийняття рішень.

  \section{Хід~роботи}
    \subsection{Задача}
      Яку~модель мотоцикла запустити в~серію? Вихідні дані для~прийняття рішення наведені в~таблиці~\ref{tab:task01}. Розберіть чотири критерії прийняття рішення: песимістичний, оптимістичний, середнього прибутку, мінімальної вигоди.

      \begin{table}[!htbp]
        \caption{Дані про~мотоцикли, які~можна запустити в~серію}
        \label{tab:task01}
        \begin{tabular}{
            v{4\gridunitwidth - 2\tabcolsep}
            n{4\gridunitwidth - 2\tabcolsep}
            n{4\gridunitwidth - 2\tabcolsep}
        }
          \toprule
            Ціна на~бензин & Мотоцикл «Вітязь», прибуток & Мотоцикл «Комар», прибуток\\
          \midrule
            Низька~(20\%)  & 900 & 700\\
            Середня~(60\%) & 700 & 600\\
            Висока~(20\%)  & 100 & 400\\
          \bottomrule
        \end{tabular}
      \end{table}

      \paragraph{Песимістичний критерій}
        У~песимістичному випадку ціна на~бензин буде найвищою, тому варто випустити у~серію мотоцикл~«Комар», так~як~він~принесе більший прибуток при~низькій ціні на~бензин: $400 > 100$.

      \paragraph{Оптимістичний критерій}
        В~оптимістичному випадку ціна на~бензин буде найнижчою, тому варто випустити у~серію мотоцикл~«Вітязь», так~як~він~принесе більший прибуток при~високій ціні на~бензин: $900 > 700$.

      \paragraph{Критерій середнього прибутку}
        Щоб~прийняти рішення за~критерієм середнього прибутку, необхідно скласти рівняння середнього прибутку для~можливих альтернатив. Нехай~$P_1$~— середній прибуток для~мотоцикла~«Вітязь», а~$P_2$~— для~мотоцикла~«Комар». Тоді:
        \begin{IEEEeqnarray*}{rCl}
          P_1 &=& 900 \cdot \num{0.2} + 700 \cdot \num{0.6} + 100 \cdot \num{0.2} = 600,\\
          P_2 &=& 700 \cdot \num{0.2} + 600 \cdot \num{0.6} + 400 \cdot \num{0.2} = 580.
        \end{IEEEeqnarray*}
        Як~видно, середній прибуток від~випуску мотоцикла~«Вітязь» більший, тому за~критерієм середнього прибутку варто випустити у~серію саме його.

      \paragraph{Критерій мінімальної вигоди}
        Випускаючи мотоцикл~«Вітязь», у~найгіршому випадку виробництво отримає~$100$~од. Тоді мінімальна вигода~$900 - 100 = 800$. Випускаючи мотоцикл~«Комар», підприємство отримає не~менше~$400$, тоді мінімальна вигода:~$700 - 400 = 300$. Тоді варто запустити мотоцикл~«Вітязь».

    \subsection{Проаналізуйте твердження «максимум прибутку при~мінімумі витрат». Як~можна позбутися від~його суперечливості? Запропонуйте якомога більше способів}

      Це~твердження суперечливе, тому що~переслідує одразу дві~суперечливі цілі: «максимум прибутку» і~«мінімум витрат». Щоб~досягти кожну з~них~окремо, потрібні різні стратегії. Наприклад, мінімуму витрат можна досягти, якщо нічого не~робити і~не~отримувати прибуток. Водночас, максимуму прибутку можна досягти, якщо витратити значну кількість ресурсів. А~особа, що~приймає рішення, може не~мати необхідних ресурсів.

      Тому, щоб~позбавитись суперечності, можна:
      \begin{enumerate}
        \item Обрати основну мету, наприклад, максимізацію прибутку. І~слідувати стратегії її~досягнення.

        \item Уточнити будь-яку~з~умов. Наприклад, «Максимізувати прибуток так, щоб~витрати склали не~більше 10 000 умовних одиниць».

        \item Переформулювати твердження у~«досягти якнайбільшого прибутку, зменшивши витрати, наскільки це~можливо». Тоді можна застосувати багатокритеріальний аналіз і~знайти оптимальний розв'язок.
      \end{enumerate}

    \subsection{Чи~доцільно, на~Ваш~погляд, купити 1000~квитків лотереї з~метою розбагатіти?}

      Залежить від~умов лотереї. Якщо розглядати сучасні реалістичні лотереї, то~ні, не~доцільно, бо~математичне очікування виграшу від~1000~придбаних квитків буде значно меншим, порівняно з~їх~вартістю. Проте, якщо лотерея має~вигідні умови для~гравця, це~може бути доцільно.

      Наприклад, припустимо, що~є~лотерея, в~який головний приз~$J = 1000$. Тираж лотереї~$N = 1000$ квитків, кожен з~яких коштує~$P = \num{0,5}$~гривень. Очевидно, що~щоб~точно виграти в~лотерею, необхідно викупити всі~квитки. Це~буде коштувати~$LP = N \cdot P = 1000 \cdot \num{0,5} = 500$. Тоді зрозуміло, що~гравець витратить менше коштів, щоб~точно виграти в~лотерею, ніж~заробить, оскільки~$LP < J$, а~його прибуток складе:~$LP - J = 1000 - 500 = 500$. В~такому випадку доцільно придбати 1000~лотерейних квитків.

    \subsection{Чи~має~сенс твердження «мета роботи фірми~— максимізація прибутку»?}

      Так, це~твердження має~сенс, оскільки воно коректно відображає мету роботи багатьох підприємств~— отримання грошей. Однак, максимізація прибутку може бути не~єдиною метою: часто необхідно також зменшити витрати, створити інноваційний продукт або~вирішити задачу. Також метою деяких підприємств може бути надання суспільно необхідної послуги, на~кшталт підприємств, що~відповідають за~міський транспорт. Дійсно, їм~не~завадить максимізувати прибуток, але~все~ж~таки їх~основна мета~— це~дати можливість жителям міста пересуватись, інакше місто перестане нормально функціонувати і~почне втрачати гроші.

  \section{Висновок}
    Виконуючи дану лабораторну роботу, ми~ознайомились з~основним принципом теорії прийняття рішень, методами та~класами задач теорії прийняття рішень.

\end{document}
