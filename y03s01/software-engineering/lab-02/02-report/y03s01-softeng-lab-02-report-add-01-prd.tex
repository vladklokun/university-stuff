\documentclass[
	a4paper,
	oneside,
	BCOR = 10mm,
	DIV = 12,
	12pt,
	headings = normal,
]{scrartcl}

%%% Length calculations
\usepackage{calc}
%%%

%%% Support for color
\usepackage{xcolor}
\definecolor{lightblue}{HTML}{03A9F4}
\definecolor{red}{HTML}{F44336}
%%%

%%% Including graphics
\usepackage{graphicx}
%%%

%%% Font selection
\usepackage{fontspec}

\setromanfont{STIX Two Text}[
	SmallCapsFeatures = {LetterSpace = 8},
]

\setsansfont{IBM Plex Sans}[
	Scale = MatchUppercase,
]

\setmonofont{IBM Plex Mono}[
	Scale = MatchUppercase,
]
%%%

%%% Math typesetting
\usepackage{amsmath}

\usepackage{unicode-math}
\setmathfont{STIX Two Math}
%%%

%%% List settings
\usepackage{enumitem}
\setlist[enumerate]{
	label*      = {\arabic*.},
	leftmargin  = *,
	labelindent = \parindent,
	topsep      = 1\baselineskip,
	parsep      = 0\baselineskip,
	itemsep     = 1\baselineskip,
}

\setlist[itemize]{
	label*      = {—},
	leftmargin  = *,
	labelindent = \parindent,
	topsep      = 1\baselineskip,
	parsep      = 0\baselineskip,
	itemsep     = 1\baselineskip,
}

\setlist[description]{
	font        = {\rmfamily\upshape\bfseries},
	topsep      = 1\baselineskip,
	parsep      = 0\baselineskip,
	itemsep     = 0\baselineskip,
}

%%%

%%% Structural elements typesetting
\setkomafont{pagenumber}{\rmfamily}
\setkomafont{disposition}{\rmfamily\bfseries}

% Sectioning
\RedeclareSectionCommand[
	beforeskip = -1\baselineskip,
	afterskip  = 1\baselineskip,
	font       = {\normalsize\bfseries\scshape},
]{section}

\RedeclareSectionCommand[
	beforeskip = -1\baselineskip,
	afterskip  = 1\baselineskip,
	font       = {\normalsize\bfseries},
]{subsection}

\RedeclareSectionCommand[
	beforeskip = -1\baselineskip,
	afterskip  = 1\baselineskip,
	font       = {\normalsize\bfseries\itshape},
]{subsubsection}

\RedeclareSectionCommand[
	beforeskip = -1\baselineskip,
	afterskip  = -1em,
	font       = {\normalsize\mdseries\scshape\addfontfeatures{Letters = {UppercaseSmallCaps}}},
]{paragraph}
%%%

%%% Typographic enhancements
\usepackage{microtype}
%%%

%%% Language-specific settings
\usepackage{polyglossia}
\setmainlanguage{ukrainian}
%%%

%%% Captions
\usepackage{caption}
\usepackage{subcaption}

%\DeclareCaptionLabelFormat{closing}{#2)}
%\captionsetup[subtable]{labelformat = closing}

%\captionsetup[subfigure]{labelformat = closing}

\captionsetup[table]{
	aboveskip = 0\baselineskip,
	belowskip = 1\baselineskip,
}

\captionsetup[figure]{
	aboveskip = 1\baselineskip,
	belowskip = 0\baselineskip,
}

\captionsetup[subfigure]{
	labelformat = simple,
	labelformat = brace,
}
%%%

%%% Hyphenated ragged typesetting
\usepackage{ragged2e}
%%%

%%% Table typesetting
\usepackage{booktabs}
\usepackage{longtable}

\usepackage{multirow}

\usepackage{array}
\newcolumntype{v}[1]{>{\RaggedRight\arraybackslash\hspace{0pt}}p{#1}}
\newcolumntype{b}[1]{>{\Centering\arraybackslash\hspace{0pt}}p{#1}}
\newcolumntype{n}[1]{>{\RaggedLeft\arraybackslash\hspace{0pt}}p{#1}}
%%%

%%% Drawing
\usepackage{tikz}
\usepackage{tikzscale}
\usetikzlibrary{positioning}
\usetikzlibrary{arrows.meta} % Stealth arrow tips
%%%

%%% Links and hyperreferences
\usepackage{hyperref}
\hypersetup{
	bookmarksnumbered = true,
	colorlinks      = false,
	linkbordercolor = red,
	urlbordercolor  = lightblue,
	pdfborderstyle  = {/S/U/W 1.5},
}
%%%

%%% Length adjustments
% Set baselineskip, default is 14.5 pt
\linespread{1.068966} % ~15.5 pt
\setlength{\emergencystretch}{1em}
\setlength{\parindent}{1.5em}
\newlength{\gridunitwidth}
\setlength{\gridunitwidth}{\textwidth / 12}
%%%

%%% Custom commands
\newcommand{\allcaps}[1]{{\addfontfeatures{LetterSpace = 5, Kerning = Off}#1}}
%%%

\begin{document}

	% \tableofcontents

	\section{Загальні відомості}
		\subsection{Найменування системи}
			Повне найменування системи~— «Автоматизована інформаційна система аеропорту „Бориспіль“».

			Умовне найменування системи~— «\allcaps{АІС} аеропорту „Бориспіль“», \allcaps{АІС}, Система управління аеропортом, Система.

		\subsection{Номер договору}
			Договір~№135 від~2017.10.01.

		\subsection{Найменування Розробника та Замовника робіт}
			Розробник: Закрите акціонерне товариство «Програмні системи». Адреса: 03237, м.~Київ, пр-т~Вернадського, буд.~3. Тел.: \mbox{+38 044 922 33 55}, факс: \mbox{+38 044 922 33 44}. 

			Замовник: Державне підприємство «Міжнародний аеропорт „Бориспіль“». Адреса: Київська~обл., Бориспільский район, с.~Гора, вул.~Бориспіль, буд.~7. Тел.: \mbox{+38 044 281 78 78}, факс: \mbox{+38 044 281 71 22}. 

		\subsection{Перелік документів, на підставі яких проводяться роботи}
			Роботи зі~створення \allcaps{АІС} проводяться на~підставі Договору~№135 від~2017.10.01 на~поставку, впровадження та~супровід прикладного програмного забезпечення для~інформаційної системи аеропорту.

		\subsection{Планові терміни початку та завершення робіт}
			Планова дата початку робіт зі~створення Системи~— 2018.01.01, закінчення робіт~— 2019.01.01.

		\subsection{Відомості про джерела та порядок фінансування робіт}
			Фінансування робіт здійснюється за кошти Державного підприємства «Міжнародний аеропорт „Бориспіль“», порядок фінансування робіт визначається умовами Договору~№135 від~2017.10.01.

		\subsection{Порядок оформлення та пред'явлення замовнику результатів роботи}
			Роботи зі~створення Системи виконуються і~приймаються поетапно. Після закінчення кожного з~етапів робіт Розробник представляє Замовникові відповідну документацію і~Акт здачі-приймання робіт, підписаний Розробником. Після закінчення етапів «Пусконалагоджувальні роботи» і~«Дослідна експлуатація» Розробник додатково повідомляє Замовника про~готовність Системи та~її~частин до~випробувань.

	\section{Призначення і цілі створення системи}
		\subsection{Призначення системи}
			Автоматизована інформаційна система аеропорту «Бориспіль» призначена~для:
			\begin{enumerate}[noitemsep]
				\item Надання пасажирам інформації про~рейси та~їх~поточний стан.
				\item Покрокової навігації пасажирів у~процесі посадки на~бажаний рейс або~прибуття до~аеропорту. 
			\end{enumerate}

			\subsection{Цілі створення системи}
				Основними цілями створення та~впровадження системи~є:
				\begin{enumerate}[noitemsep]
					\item Автоматизація використання існуючих внутрішніх даних про~поточний стан рейсу як~джерела інформації для~інформаційних елементів аеропорту.
					\item Створення автоматичного механізму інформування пасажирів про~стан найближчих рейсів.
					\item Автоматизація інформування пасажирів про~необхідні дії для~посадки на~рейс або~прибуття.
				\end{enumerate}

	\section{Характеристика об'єктів автоматизації}
		Об'єктом автоматизації є~процеси обміну даними між~структурами аеропорту, а~також процеси надання пасажирам інформації про~рейси. Автоматизація обміну даними планується між такими структурами аеропорту:
		\begin{enumerate}[noitemsep]
			\item Підрозділ контролю польотів.
			\item Підрозділ логістики.
			\item Підрозділ інформації та навігації.
		\end{enumerate}

	\section{Вимоги до системи}
		\subsection{Вимоги до~системи в~цілому}
			\subsubsection{Вимоги до структури та функціонування системи}
				\paragraph{Структура системи}
					В~основу Автоматизованої інформаційної системи повинна бути покладена модульна архітектура, де~кожен модуль виконує окрему функцію~(табл.~\ref{tab:is-modules-summary-short}). Для~забезпечення стійкості до~відмов у~інформаційній системі повинен існувати модуль керування, який відстежує стан підключених модулів.
				
					\begin{table}[!htbp]
						\centering
						\caption{Коротка характеристика модулів Системи}
						\label{tab:is-modules-summary-short}
						\begin{tabular}{
							v{3\gridunitwidth - 2\tabcolsep}
							v{9\gridunitwidth - 2\tabcolsep}
						}
							\toprule
								Назва модуля & Призначення\\
							\midrule
								«Керування»  & Керує підключеними модулями, відстежує їх поточний стан\\
								«АТС»        & Обробляє дані контролю польотів\\
								«Логістика»  & Обробряє інформацію та оновлює стан логістичних структур аеропорту\\
								«Навігація»  & Надає інформацію клієнтам аеропорту\\
							\bottomrule
						\end{tabular}
					\end{table}
					Рекомендується запобігати централізації системи, оскільки в такому випадку зменшиться стійкість до відмов, тому робота усієї Системи не повинна залежати від дієздатності модуля «Керування».
				
				\paragraph{Способи та засоби зв'язку та інформаційного обміну}
					Обмін даними між підсистемами повинен здійснюватись через єдиний інформаційний простір і~за~допомогою використання стандартизованих протоколів і форматів обміну даними. Всі~програмні компоненти підсистем повинні функціонувати в~межах єдиного логічного простору, забезпеченого інтегрованими засобами серверів даних і~серверів додатків.
					
				\paragraph{Сумісність із суміжними системами}
					Система повинна забезпечувати інтеграцію і сумісність на інформаційному рівніз іншими системами. Інформаційна сумісність забезпечується на рівні експорту~— імпорту \allcaps{JSON}-документів.
					
				\paragraph{Режими функціонування системи}
					Стандартний режим роботи системи~— режим взаємодії. Тим не~менш, кожен модуль Системи повинен коректно функціонувати в автономному режимі, правильно оброблюючи наявні або відсутні дані. 
					
				\paragraph{Діагностика}
					Необхідно передбачити можливість зручного регулярного діагностування та~моніторингу Системи в~автоматичному і~ручному режимах. Автоматична діагностика і~моніторинг може проводитись модулем «Керування», а~ручна~— відповідальним персоналом.
					
				\paragraph{Перспективи розвитку системи}
					Система повинна бути побудована з~використанням стандартизованих і~ефективно супроводжуваних рішень; бути реалізована як~відкрита система, тобто допускати модифікацію та~нарощування функціональних і~апаратних можливостей, удосконалення інформаційного забезпечення.
				
			\subsubsection{Вимоги до персоналу системи}
				\paragraph{Кількість персоналу}
					Для зручної експлуатації системи необхідний штат персоналу, до~якого входять:
					\begin{enumerate}[noitemsep]
						\item Оператори модулів.
						\item Адміністратори забезпечення модулів.
						\item Адміністратор Системи.
					\end{enumerate}
					Оскільки система розширювана, то~для~визначення кількості персоналу варто користуватись такою методикою:
					\begin{enumerate}[noitemsep]
						\item На~кожні 3~інформаційні зв'язки модуля~— 1~оператор.
						\item На~кожні 10~серверів у~модулі~— 1~адміністратор забезпечення.
						\item За роботу Системи відповідає 1~адміністратор Системи.
					\end{enumerate}
					
				\paragraph{Кваліфікація персоналу}
					\emph{Оператори модулів} повинні мати базові навички роботи за~персональними комп'\-ю\-те\-ра\-ми під~управлінням операційної системи Fedora Workstation, пройти тренування з~використання модуля або~мати попередній досвід роботи зі~схожими інформаційними системами. \emph{Адміністратори забезпечення} повинні мати профільну технічну освіту або~сертифікат про~компетентність у~профільних сферах (Cisco, CompTIA тощо). \emph{Адміністратор Системи} повинен мати кваліфікацію «інженер» або сертифікати, які підтверджують високий рівень компетентності (Cisco CCNP, CCIE, CCAr тощо).

				\paragraph{Показники відповідності призначенню}
					Відповідність призначенню повинна зберігатись протягом усього терміну експлуатації Системи Замовником. Система повинна передбачати розробку та~підключення нових модулів для~структур аеропорту в~рамках передбачуваної архітектури, створення нових типів повідомлень, якими обмінюються модулі, тощо. Розробник не~несе відповідальність за~відповідність призначенню Системи при~виході за~рамки вищезазначених змін.
				
			\subsubsection{Вимоги до~надійності}
				\paragraph{Показники надійності}
					Цільовий графік роботи Системи передбачає повне співпадіння робочого часу Системи з~робочим часом аеропорту. Допускається повна втрата працездатності Системи не~довше, ніж на~1~хвилину. Допускається втрата працездатності одного з~модулів системи не~довше, ніж на~10~хвилин.

				\paragraph{Надійність технічних та~програмних засобів}
					Для~забезпечення стійкості до~відмов у~інформаційній системі повинен існувати модуль керування, який відстежує стан підключених модулів. Апаратне забезпечення, на~якому запущений модуль, повинно мати резерв~— хоча~б один допоміжний дублюючий сервер, якому буде передаватись управління у~разі відмови.

			\subsubsection{Вимоги до безпеки}
				Під час встановлення, налагодження та~експлуатації технічних засобів Системи повинні виконуватись заходи електробезпеки відповідно до~«Правил улаштування електроустановок» та~«Правил техніки безпеки при експлуатації електроустановок споживачів». Апаратне забезпечення повинно відповідати вимогам пожежної безпеки у~виробничих приміщеннях. Повинно бути забезпечено дотримання загальних вимог безпеки відповідно до~ГОСТ~12.2003-91.

			\subsubsection{Вимоги до захисту інформації від несанкціонованого доступу}

		\subsection{Вимоги до функцій, виконуваних системою}
			Автоматизована інформаційна система аеропорту «Бориспіль» повинна складатись з модулів~(табл.~\ref{tab:is-modules-summary-short}). Модулі, що ходять до складу Системи, повинні виконувати мінімальний набір необхідних завдань, бути зручними в заміні, створенні та інтеграції. Незважаючи на це, будь-який модуль, підключений до системи, повинен мати такий базовий функціонал:
			\begin{enumerate}[noitemsep]
				\item Статус.
				\item Звітність.
				\item Діагностичні повідомлення.
			\end{enumerate}

			\subsubsection{Модуль «Керування»}
				Основними функціями модуля «Керування» є:
				\begin{enumerate}[noitemsep]
					\item Відстеження стану модулів, підключених до системи.
					\item Зображення стану модулів на підключені пристрої виводу операторів. 
					\item Керування підключеними модулями.
				\end{enumerate}
						Процес відстеження стану підключених модулів полягає у~відправленні спеціальних запитів на стан підключеним до Системи модулям для перевірки їх дієздатності. Відправка таких повідомлень проводиться з~інтервалом 1~хвилина. Крім того, у разі виникнення несправностей або збоїв підключені модулі повинні відправляти відповідне діагностичне повідомлення, яке містить:

		\subsection{Вимоги до видів забезпечення}
		
	\section{Склад і зміст робіт зі створення системи}

	\section{Порядок контролю та приймання системи}

	\section{Вимоги до документування}

	\section{Порядок внесення змін}

\end{document}
