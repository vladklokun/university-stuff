\documentclass[
	a4paper,
	oneside,
	BCOR = 10mm,
	DIV = 12,
	12pt,
	headings = normal,
]{scrartcl}

%%% Length calculations
\usepackage{calc}
%%%

%%% Support for color
\usepackage{xcolor}
\definecolor{lightblue}{HTML}{03A9F4}
\definecolor{red}{HTML}{F44336}
%%%

%%% Including graphics
\usepackage{graphicx}
%%%

%%% Font selection
\usepackage{fontspec}

\setromanfont{STIX Two Text}[
	SmallCapsFeatures = {LetterSpace = 8},
]

\setsansfont{IBM Plex Sans}[
	Scale = MatchUppercase,
]

\setmonofont{IBM Plex Mono}[
	Scale = MatchUppercase,
]
%%%

%%% Math typesetting
\usepackage{amsmath}

\usepackage{unicode-math}
\setmathfont{STIX Two Math}
%%%

%%% List settings
\usepackage{enumitem}
\setlist[enumerate]{
	label*      = {\arabic*.},
	leftmargin  = *,
	labelindent = \parindent,
	topsep      = 1\baselineskip,
	parsep      = 0\baselineskip,
	itemsep     = 1\baselineskip,
}

\setlist[itemize]{
	label*      = {—},
	leftmargin  = *,
	labelindent = \parindent,
	topsep      = 1\baselineskip,
	parsep      = 0\baselineskip,
	itemsep     = 1\baselineskip,
}

\setlist[description]{
	font        = {\rmfamily\upshape\bfseries},
	topsep      = 1\baselineskip,
	parsep      = 0\baselineskip,
	itemsep     = 0\baselineskip,
}

%%%

%%% Structural elements typesetting
\setkomafont{pagenumber}{\rmfamily}
\setkomafont{disposition}{\rmfamily\bfseries}

% Sectioning
\RedeclareSectionCommand[
	beforeskip = -1\baselineskip,
	afterskip  = 1\baselineskip,
	font       = {\normalsize\bfseries\scshape},
]{section}

\RedeclareSectionCommand[
	beforeskip = -1\baselineskip,
	afterskip  = 1\baselineskip,
	font       = {\normalsize\bfseries\itshape},
]{subsection}

\RedeclareSectionCommand[
	beforeskip = -1\baselineskip,
	afterskip  = 1\baselineskip,
	font       = {\normalsize\bfseries},
]{subsubsection}

\RedeclareSectionCommand[
	beforeskip = -1\baselineskip,
	afterskip  = -0.5em,
	font       = {\normalsize\mdseries\scshape\addfontfeatures{Letters = {UppercaseSmallCaps}}},
]{paragraph}
%%%

%%% Typographic enhancements
\usepackage{microtype}
%%%

%%% Language-specific settings
\usepackage{polyglossia}
\setmainlanguage{ukrainian}
\setotherlanguages{english}
%%%

%%% Captions
\usepackage{caption}
\usepackage{subcaption}

%\DeclareCaptionLabelFormat{closing}{#2)}
%\captionsetup[subtable]{labelformat = closing}

%\captionsetup[subfigure]{labelformat = closing}

\captionsetup[table]{
	aboveskip = 0\baselineskip,
	belowskip = 0\baselineskip,
}

\captionsetup[figure]{
	aboveskip = 1\baselineskip,
	belowskip = 0\baselineskip,
}

\captionsetup[subfigure]{
	labelformat = simple,
	labelformat = brace,
}
%%%

%%% Hyphenated ragged typesetting
\usepackage{ragged2e}
%%%

%%% Table typesetting
\usepackage{booktabs}
\usepackage{longtable}

\usepackage{multirow}

\usepackage{array}
\newcolumntype{v}[1]{>{\RaggedRight\arraybackslash\hspace{0pt}}p{#1}}
\newcolumntype{b}[1]{>{\Centering\arraybackslash\hspace{0pt}}p{#1}}
\newcolumntype{n}[1]{>{\RaggedLeft\arraybackslash\hspace{0pt}}p{#1}}
%%%

%%% Drawing
\usepackage{tikz}
\usepackage{tikzscale}
\usetikzlibrary{positioning}
\usetikzlibrary{arrows.meta} % Stealth arrow tips
%%%

%%% Links and hyperreferences
\usepackage{hyperref}
\hypersetup{
	bookmarksnumbered = true,
	colorlinks      = false,
	linkbordercolor = red,
	urlbordercolor  = lightblue,
	pdfborderstyle  = {/S/U/W 1.5},
}
%%%

%%% Length adjustments
% Set baselineskip, default is 14.5 pt
\linespread{1.068966} % ~15.5 pt
\setlength{\emergencystretch}{1em}
\setlength{\parindent}{1.5em}
\newlength{\gridunitwidth}
\setlength{\gridunitwidth}{\textwidth / 12}
%%%

%%% Custom commands
\newcommand{\allcaps}[1]{{\addfontfeatures{LetterSpace = 8, Kerning = Off}#1}}
%%%

\begin{document}

	\tableofcontents
	\newpage

	\section{Загальні відомості}
		\subsection{Найменування системи}
			\paragraph{Повне найменування системи}
				«Автоматизована інформаційна система аеропорту „Бориспіль“».

			\paragraph{Умовне найменування системи}
				«\allcaps{АІС} аеропорту „Бориспіль“», \allcaps{АІС}, Система управління аеропортом, Система.

		\subsection{Номер договору}
			Договір~№135 від~2017.10.01.

		\subsection{Найменування Розробника та Замовника робіт}
			\paragraph{Розробник}
				Закрите акціонерне товариство «Програмні системи». Адреса: 03237, м.~Київ, пр-т~Вернадського, буд.~3. Тел.: \mbox{+38 044 922 33 55}, факс: \mbox{+38 044 922 33 44}. 

			\paragraph{Замовник}
				Державне підприємство «Міжнародний аеропорт „Бориспіль“». Адреса: Київська~обл., Бориспільский район, с.~Гора, вул.~Бориспіль, буд.~7. Тел.: \mbox{+38 044 281 78 78}, факс: \mbox{+38 044 281 71 22}. 

		\subsection{Перелік документів, на підставі яких проводяться роботи}
			Роботи зі~створення \allcaps{АІС} проводяться на~підставі Договору~№135 від~2017.10.01 на~поставку, впровадження та~супровід прикладного програмного забезпечення для~інформаційної системи аеропорту.

		\subsection{Планові терміни початку та завершення робіт}
			\paragraph{Планова дата початку робіт}
				2018.01.01
			\paragraph{Планова дата закінчення робіт}
				2019.01.01.

		\subsection{Відомості про джерела та порядок фінансування робіт}
			Фінансування робіт здійснюється за кошти Державного підприємства «Міжнародний аеропорт „Бориспіль“», порядок фінансування робіт визначається умовами Договору~№135 від~2017.10.01.

		\subsection{Порядок оформлення та пред'явлення замовнику результатів роботи}
			Роботи зі~створення Системи виконуються і~приймаються поетапно. Після закінчення кожного з~етапів робіт Розробник представляє Замовникові відповідну документацію і~Акт здачі-приймання робіт, підписаний Розробником. Після закінчення етапів «Пусконалагоджувальні роботи» і~«Дослідна експлуатація» Розробник додатково повідомляє Замовника про~готовність Системи та~її~частин до~випробувань.

	\section{Призначення і цілі створення системи}
		\subsection{Призначення системи}
			Автоматизована інформаційна система аеропорту «Бориспіль» призначена~для:
			\begin{enumerate}[noitemsep]
				\item Надання пасажирам інформації про~рейси та~їх~поточний стан.
				\item Покрокової навігації пасажирів у~процесі посадки на~бажаний рейс або~прибуття до~аеропорту. 
			\end{enumerate}

			\subsection{Цілі створення системи}
				Основними цілями створення та~впровадження системи~є:
				\begin{enumerate}[noitemsep]
					\item Автоматизація використання існуючих внутрішніх даних про~поточний стан рейсу як~джерела інформації для~інформаційних елементів аеропорту.
					\item Створення автоматичного механізму інформування пасажирів про~стан найближчих рейсів.
					\item Автоматизація інформування пасажирів про~необхідні дії для~посадки на~рейс або~прибуття.
				\end{enumerate}

	\section{Характеристика об'єктів автоматизації}
		Об'єктом автоматизації є~процеси обміну даними між~структурами аеропорту, а~також процеси надання пасажирам інформації про~рейси. Автоматизація обміну даними планується між такими структурами аеропорту:
		\begin{enumerate}[noitemsep]
			\item Підрозділ контролю польотів.
			\item Підрозділ логістики.
			\item Підрозділ інформації та навігації.
		\end{enumerate}

	\section{Вимоги до системи}
		\subsection{Вимоги до~системи в~цілому}
			\subsubsection{Вимоги до структури та функціонування системи}
				\paragraph{Структура системи}
					В~основу Автоматизованої інформаційної системи повинна бути покладена модульна архітектура, де~кожен модуль виконує окрему функцію~(табл.~\ref{tab:is-modules-summary-short}). Для~забезпечення стійкості до~відмов у~інформаційній системі повинен існувати модуль керування, який відстежує стан підключених модулів.

					\begin{table}[!htbp]
						\centering
						\caption{Коротка характеристика модулів Системи}
						\label{tab:is-modules-summary-short}
						\begin{tabular}{
							v{3\gridunitwidth - 2\tabcolsep}
							v{9\gridunitwidth - 2\tabcolsep}
						}
							\toprule
								Назва модуля     & Призначення\\
							\midrule
								«Керування»      & Керує підключеними модулями, відстежує їх поточний стан\\
								«Повітряний рух» & Обробляє дані контролю повітряним рухом\\
								«Логістика»      & Обробряє інформацію та оновлює стан логістичних структур аеропорту\\
								«Навігація»      & Надає інформацію клієнтам аеропорту\\
							\bottomrule
						\end{tabular}
					\end{table}
					Рекомендується запобігати централізації системи, оскільки в такому випадку зменшиться стійкість до відмов, тому робота усієї Системи не повинна залежати від дієздатності модуля «Керування».

				\paragraph{Способи та засоби зв'язку та інформаційного обміну}
					Обмін даними між підсистемами повинен здійснюватись через єдиний інформаційний простір і~за~допомогою використання стандартизованих протоколів і форматів обміну даними. Всі~програмні компоненти підсистем повинні функціонувати в~межах єдиного логічного простору, забезпеченого інтегрованими засобами серверів даних і~серверів додатків.

				\paragraph{Сумісність із суміжними системами}
					Система повинна забезпечувати інтеграцію і сумісність на інформаційному рівні з~іншими системами. Інформаційна сумісність забезпечується на рівні експорту~— імпорту \allcaps{JSON}-документів.

				\paragraph{Режими функціонування системи}
					Стандартний режим роботи системи~— режим взаємодії. Тим не~менш, кожен модуль Системи повинен коректно функціонувати в автономному режимі, правильно оброблюючи наявні або відсутні дані.

				\paragraph{Діагностика}
					Необхідно передбачити можливість зручного регулярного діагностування та~моніторингу Системи в~автоматичному і~ручному режимах. Автоматична діагностика і~моніторинг може проводитись модулем «Керування», а~ручна~— відповідальним персоналом.

				\paragraph{Перспективи розвитку системи}
					Система повинна бути побудована з~використанням стандартизованих і~ефективно супроводжуваних рішень; бути реалізована як~відкрита система, тобто допускати модифікацію та~нарощування функціональних і~апаратних можливостей, удосконалення інформаційного забезпечення.

			\subsubsection{Вимоги до персоналу системи}
				\paragraph{Кількість персоналу}
					Для зручної експлуатації системи необхідний штат персоналу, до~якого входять:
					\begin{enumerate}[noitemsep]
						\item Оператори модулів.
						\item Адміністратори забезпечення модулів.
						\item Адміністратор Системи.
					\end{enumerate}
					Оскільки система розширювана, то~для~визначення кількості персоналу варто користуватись такою методикою:
					\begin{enumerate}[noitemsep]
						\item На~кожні 3~інформаційні зв'язки модуля~— 1~оператор.
						\item На~кожні 10~серверів у~модулі~— 1~адміністратор забезпечення.
						\item За роботу Системи відповідає 1~адміністратор Системи.
					\end{enumerate}

				\paragraph{Кваліфікація персоналу}
					\emph{Оператори модулів} повинні мати базові навички роботи за~персональними комп'\-ю\-те\-ра\-ми під~управлінням операційної системи Fedora Workstation, пройти тренування з~використання модуля або~мати попередній досвід роботи зі~схожими інформаційними системами. \emph{Адміністратори забезпечення} повинні мати профільну технічну освіту або~сертифікат про~компетентність у~профільних сферах (Cisco, CompTIA тощо). \emph{Адміністратор Системи} повинен мати кваліфікацію «інженер» або сертифікати, які підтверджують високий рівень компетентності (\textenglish{Cisco \allcaps{CCNP}, \allcaps{CCIE}, \allcaps{CC}Ar} тощо).

				\paragraph{Показники відповідності призначенню}
					Відповідність призначенню повинна зберігатись протягом усього терміну експлуатації Системи Замовником. Система повинна передбачати розробку та~підключення нових модулів для~структур аеропорту в~рамках передбачуваної архітектури, створення нових типів повідомлень, якими обмінюються модулі, тощо. Розробник не~несе відповідальність за~відповідність призначенню Системи при~виході за~рамки вищезазначених змін.
				
			\subsubsection{Вимоги до~надійності}
				\paragraph{Показники надійності}
					Цільовий графік роботи Системи передбачає повне співпадіння робочого часу Системи з~робочим часом аеропорту. Допускається повна втрата працездатності Системи не~довше, ніж на~1~хвилину. Допускається втрата працездатності одного з~модулів системи не~довше, ніж на~10~хвилин.

				\paragraph{Надійність технічних та~програмних засобів}
					Для~забезпечення стійкості до~відмов у~інформаційній системі повинен існувати модуль керування, який відстежує стан підключених модулів. Апаратне забезпечення, на~якому запущений модуль, повинно мати резерв~— хоча~б один допоміжний дублюючий сервер, якому буде передаватись управління у~разі відмови.

			\subsubsection{Вимоги до безпеки}
				Під час встановлення, налагодження та~експлуатації технічних засобів Системи повинні виконуватись заходи електробезпеки відповідно до~«Правил улаштування електроустановок» та~«Правил техніки безпеки при експлуатації електроустановок споживачів». Апаратне забезпечення повинно відповідати вимогам пожежної безпеки у~виробничих приміщеннях. Повинно бути забезпечено дотримання загальних вимог безпеки відповідно до~ГОСТ~12.2003-91.

			\subsubsection{Вимоги до захисту інформації від несанкціонованого доступу}
				Захист інформації від несанкціонованого доступу повинен бути організований за допомогою списків контролю доступу акаунтів користувачів до ресурсів системи.

				Усі комунікації системи повинні бути зашифровані за схемою \textenglish{Authenticated Encryption with Associated Data (\allcaps{AEAD})}. Для шифрування комунікацій мають бути використані такі шифри:
				\begin{enumerate}[noitemsep]
					\item Симетричний шифр \textenglish{\allcaps{AES}-256}.
					\item Асиметричний шифр \textenglish{\allcaps{ECDHE RSA}}.
					\item Код аутентифікації \textenglish{\allcaps{SHA}-256}.
				\end{enumerate}

			\subsubsection{Вимоги до збереження інформації}
				Система повинна забезпечувати збереження інформації у разі непередбачуваних подій, зокрема:
				\begin{enumerate}[noitemsep]
					\item Відмова серверного обладнання.
					\item Відмова ліній зв'язку.
				\end{enumerate}
				Для збереження інформації у таких обставинах використовуються такі засоби:
				\begin{enumerate}[noitemsep]
					\item Щогодинне резервне копіювання до~зовнішньої системи з~тривалістю збереження копій 48~годин.
					\item Щотижневе резервне копіювання до~зовнішньої системи з~тривалістю збереження копій 14~днів.
					\item Щомісячне резервне копіювання до~зовнішньої системи з~тривалістю збереження копій 62~дні.
					\item Щорічне резервне копіювання до~зовнішньої системи з~тривалістю збереження копій 730~днів.
					\item Використання журнальованих файлових систем.
					\item Використання \textenglish{\allcaps{RAID}}-масивів у конфігураціях з дуплікацією даних або файлової системи~\textenglish{\allcaps{ZFS}}.
				\end{enumerate}

		\subsection{Вимоги до функцій, виконуваних системою}
			Автоматизована інформаційна система аеропорту «Бориспіль» повинна складатись з модулів~(табл.~\ref{tab:is-modules-summary-short}). Модулі, що входять до складу Системи, повинні виконувати мінімальний набір необхідних завдань, бути зручними в заміні, створенні та інтеграції. Незважаючи на це, будь-який модуль, підключений до системи, повинен мати такий базовий функціонал:
			\begin{enumerate}[noitemsep]
				\item Відображення статусу модуля.
				\item Виведення діагностичних повідомлень.
				\item Ведення звітності.
			\end{enumerate}

			Під функцією відображення статусу модуля розуміється відстеження його поточного стану та~можливість пересилки інформації про~нього іншим модулям у~вигляді повідомлення. Статус модуля повинен відображати лише базову інформацію про~стан модуля~(табл.~\ref{tab:is-mod-status-meaning}), більш детальну інформацію повинні надавати діагностичні повідомлення.
			\begin{table}[!htbp]
				\centering
				\caption{Перелік можливих статусів модуля та їх значення}
				\label{tab:is-mod-status-meaning}
				\begin{tabular}{
					v{3\gridunitwidth - 2\tabcolsep}
					v{9\gridunitwidth - 2\tabcolsep}
				}
					\toprule
						Статус & Значення\\
					\midrule
						\textenglish{Fully Operational} & Модуль працює як слід, усі системи в нормі\\
						\textenglish{Operating on Backup} & Модуль працює як слід, однак, на резервному забезпеченні. Необхідна діагностика\\
						\textenglish{Fatal Error} & Сталась критична помилка, і модуль не може продовжувати роботу. Необхідна негайна діагностика\\
						\textenglish{Offline} & Модуль не працює\\
					\bottomrule
				\end{tabular}
			\end{table}

			Під функцією виведення діагностичних повідомлень розуміється створення повідомлень у разі виникнення помилок у роботі модуля або його складових, при завершенні діагностики модуля або будь-яких інших процесах, які передбачають створення діагностичного повідомлення. Діагностичні повідомлення повинні бути у форматі~\allcaps{JSON}~(табл.~\ref{tab:is-mod-diag-msg-format}).

			\begin{table}[!htbp]
				\centering
				\caption{Формат діагностичного повідомлення}
				\label{tab:is-mod-diag-msg-format}
				\begin{tabular}{
					v{3\gridunitwidth - 2\tabcolsep}
					v{3\gridunitwidth - 2\tabcolsep}
					v{6\gridunitwidth - 2\tabcolsep}
				}
					\toprule
						Поле & Приклад & Опис даних\\
					\midrule
						Дата та~час & 2007-04-05T12:30-02:00 & Поточна дата та~час у~форматі \allcaps{ISO}~8601 із~зазначенням зсуву часового поясу\\
						Ідентифікатор модуля & \textenglish{mod-logistics} & Ідентифікатор модуля, який повідомив про помилку\\
						Тип події & \textenglish{Warning} & Тип події, що сталась \\
						Ідентифікатор події & \textenglish{W103} & Ідентифікатор події (повідомлення, помилки тощо), яка сталась\\
						Повний текст повідомлення & \textenglish{mod-logistics: Warning 103 (in time-sync): NTP Server not responding} & Повний текст діагностичного повідомлення із зазначенням необхідної діагностичної інформації: суті проблеми, місцем виникнення тощо\\
					\bottomrule
				\end{tabular}
			\end{table}

			Під функцією ведення звітності розуміється запис інформаційних та~діагностичних повідомлень у~спеціальний звітний файл. В~цей файл заноситься вся доступна інформація з~метою подальшої обробки зовнішніми засобами: спеціальними сценаріями, майбутніми модулями тощо.

			\subsubsection{Модуль «Керування»}
				Основними функціями модуля «Керування»~є:
				\begin{enumerate}[noitemsep]
					\item Відстеження стану модулів, підключених до системи.
					\item Зображення стану модулів на підключені пристрої виводу операторів. 
					\item Керування підключеними модулями.
				\end{enumerate}
				Процес відстеження стану підключених модулів полягає у~відправленні спеціальних запитів на~стан підключеним до~Системи модулям для~перевірки їх~дієздатності. Відправка таких повідомлень проводиться з~інтервалом 1~хвилина. 

				Функція зображення стану модулів на~підключені пристрої виводу операторів полягає у~графічному представленні станів підключених модулів у~форматі статусної веб-сторінки~(табл.~\ref{tab:is-status-page-format}).
				\begin{table}[!htbp]
					\centering
					\caption{Формат статусної сторінки}
					\label{tab:is-status-page-format}
					\begin{tabular}{
						v{8\gridunitwidth - 2\tabcolsep}
						n{4\gridunitwidth - 2\tabcolsep}
					}
						\toprule
							Модуль & Статус\\
						\midrule
							«Керування» & Fully Operational\\
							«Повітряний рух» & Operating on Backup\\
							«Логістика» & Fatal Error\\
							«Навігація» & \textenglish{Offline}\\
						\bottomrule
					\end{tabular}
				\end{table}

				Функція керування підключеними модулями полягає у~відправленні управляючих повідомлень цільовим модулями на~обробку та~виконання сервісних команд: виконання діагностики, перезавантаження, вимкнення тощо. Процес керування підключеними модулями виглядає так:
				\begin{enumerate}[noitemsep]
					\item Модуль «Керування» надсилає керуюче повідомлення у~форматі~\allcaps{JSON} цільовому модулю.
					\item Цільовий модуль приймає повідомлення.
					\item Цільовий модуль виконує отриману команду.
					\item Цільовий модуль надсилає модулю «Керування» повідомлення про~статус виконання інструкції у~форматі~\allcaps{JSON}.
					\item Модуль «Керування» отримує повідомлення і~діє відповідно до~своїх інструкцій. Наприклад, якщо цільовий модуль не~виконав відправлену команду успішно, модуль «Керування» може створити звіт про~цю~подію та~надіслати команду ще раз.
				\end{enumerate}

			\subsubsection{Модуль «Повітряний рух»}
				Модуль~«Повітряний рух» відповідає за~отримання даних про~положення повітряного транспорту від~зовнішніх стандартизованих систем контролю за~повітряним рухом (не~входять до~Системи) в~автоматичному режимі та~їх~підготовки для~обробки подальшими модулями. Отримані дані перетворюються у~формат~\allcaps{JSON} та~оброблюються таким чином, щоб мати такі дані:
				\begin{enumerate}[noitemsep]
					\item Дата та час із зазначенням зсуву часового поясу.
					\item Ідентифікатор повітряного транспорту.
					\item Поточне положення обраного повітряного транспорту.
					\item Приблизний час прибуття до аеропорту.
				\end{enumerate}
				
				Модуль~«Повітряний рух» повинен бути готовим надавати оброблену інформацію модулю~«Логістика» у~будь-який момент.

			\subsubsection{Модуль «Логістика»}
				Модуль «Логістика» призначений для~обробки інформації та~оновлення стану логістичних структур аеропорту. Функціями цього модуля~є:
				\begin{enumerate}[noitemsep]
					\item Отримання інформації про~рейси від~модуля~«Повітряний рух».
					\item Моніторинг та~оновлення даних про~пересування рейсів у~межах аеропорту.
					\item Передача даних модулю~«Навігація».
				\end{enumerate}

				Функція отримання інформації про~рейси від~модуля~«Повітряний рух» очевидна з~назви: модуль~«Логістика» отримує оновлення про~положення повітряного транспорту від~модуля~«Повітряний рух» по~мірі їх~надходження.

				Функція моніторингу та~оновлення даних про~пересування рейсів у~межах аеропорту полягає у~відстеженні пересування повітряного транспорту, що~виконує рейс. Кожна одиниця повітряного транспорту описується спеціальною структурою даних~(табл.~\ref{tab:is-mod-logistics-air-vehicle-description-struct}). Безпосередньо сам процес оновлення статусів виконується на~основі даних про~місцезнаходження повітряного транспорту у~межах аеропорту, а~також даних диспетчерів.

				\begin{table}[!htbp]
					\caption{Структура даних для опису статусу повітряного транспорту у логістичних процесах аеропорту}
					\label{tab:is-mod-logistics-air-vehicle-description-struct}
					\centering
					\begin{tabular}{
						v{3\gridunitwidth - 2\tabcolsep}
						v{9\gridunitwidth - 2\tabcolsep}
					}
						\toprule
							Поле & Опис\\
						\midrule
							Ідентифікатор & Ідентифікатор повітряного транспорту (\textenglish{call sign})\\
							Номер рейсу & Номер рейсу, з яким пов'язаний даний повітряний транспорт\\
							Поточний статус & Стан, в якому зараз перебуває повітряний транспорт\\
							Наступний статус & Наступний очікуваний стан, до якого повинен перейти повітряний транспорт\\
							Очікуваний час переходу & Очікуваний час, що знадобиться для переходу від поточного статусу до наступного\\
							Тип місцезнаходження & Тип місця, де перебуває повітряний транспорт (злітна смуга, аеродром, ангар, термінал тощо)\\
							Ідентифікатор місцезнаходження & Ідентифікатор споруди або місця, в якому знаходиться повітряний транспорт (для нумерованих місць). Наприклад, ангар~№1, злітна смуга~№3 тощо\\
							Номер виходу & Номер виходу, на~якому здійснюється посадка або~прибуття\\
						\bottomrule
					\end{tabular}
				\end{table}

				Функція передачі даних модулю~«Навігація» полягає у готовності надавати інформацію про стан даного повітряного транспорту модулю~«Навігація» по мірі надходження оновлень.

			\subsubsection{Модуль «Навігація»}
				Модуль~«Навігація» відповідає за надання навігаційної інформації пасажирам. Його функціями~є:
				\begin{enumerate}[noitemsep]
					\item Отримання та~обробка інформації про повітряний транспорт та~рейси, яку надає модуль~«Логістика».
					\item Зображення навігаційних вказівок на інформаційні дисплеї.
				\end{enumerate}

				Функція отримання та~обробки інформації про~повітряний транспорт полягає в~прийнятті повідомлень модуля~«Логістика» та~обробці отриманих оновлень. Під~обробкою оновлень розуміється виділення серед усіх отриманих оновлень тих, що~важливі для~пасажирів аеропорту: запланований час, місце призначення, номер рейсу, термінал, основні зміни статусу (прибув, вилетів, затримується, скасований, відкрита реєстрація, починається посадка тощо).

				Функція зображення навігаційних вказівок полягає у~використанні оброблених даних та~їх виведення на~інформаційні елементи, встановлені на~території аеропорту. Презентація цих елементів розрахована на~електронні дисплеї і~має бути виконано у~вигляді веб-сторінки, яка динамічно оновлюється по~мірі надходження відповідної інформації від~модуля~«Логістика». Така веб-сторінка повинна бути адаптивною, тобто адекватно підлаштовуватись під різні формати екранів, включаючи мобільні телефони, планшетні комп'ютери, настільні комп'ютери, великі інформаційні дисплеї тощо.

				Крім того, дані модуля навігації повинні бути відкритими для~отримання іншими модулями, наприклад, майбутніми модулями сайту. Це~дозволить виконувати одну функцію одним модулем.

				Інформаційне табло прильотів має містити таку інформацію: час, призначення, номер рейсу, термінал та~статус рейсу~(приклад у~табл.~\ref{tab:tableau-arrivals}). Записи групуються за~часом, рейси в~однаковий час з~однаковим місцем призначення мають бути зображені як~один запис, але із~зазначенням усіх номерів рейсів.

				\begin{table}[!htbp]
					\caption{Приклад формату інформаційно-навігаційного табло прильотів}
					\label{tab:tableau-arrivals}
					\centering
					\begin{tabular}{
							% *{6}{l}	
							v{2\gridunitwidth - 2\tabcolsep}
							v{3\gridunitwidth - 2\tabcolsep}
							v{2\gridunitwidth - 2\tabcolsep}
							v{2\gridunitwidth - 2\tabcolsep}
							v{3\gridunitwidth - 2\tabcolsep}
						}
						\toprule
							Час   & Призначення & Рейс    & Термінал & Статус\\
						\midrule
							22:25 & Бодрум      & 7W 7032 & D        & Прибув\\
							22:40 & Батумі      & YE 1216 & F        & Очікується о~22:23\\
						\bottomrule
					\end{tabular}
				\end{table}
				
				Інформаційне табло відльотів містить таку~ж інформацію, що~й~табло прильотів, однак, із~зазначенням номерів виходів для~посадки у~графі «Термінал»~(табл.~\ref{tab:tableau-departures}).

				\begin{table}[!htbp]
					\caption{Приклад формату інформаційно-навігаційного табло відльотів}
					\label{tab:tableau-departures}
					\centering
					\begin{tabular}{
							%*{6}{l}
							v{2\gridunitwidth - 2\tabcolsep}
							v{3\gridunitwidth - 2\tabcolsep}
							v{2\gridunitwidth - 2\tabcolsep}
							v{2\gridunitwidth - 2\tabcolsep}
							v{3\gridunitwidth - 2\tabcolsep}
						}
						\toprule
							Час   & Призначення & Рейс    & Термінал & Статус\\
						\midrule
							22:55 & Тель-Авів   & LY 2654 & D 2      & Посадка\\
							23:40 & Доха        & QR 298  & F 27–30  & Реєстрація\\
						\bottomrule
					\end{tabular}
				\end{table}

		\subsection{Вимоги до видів забезпечення}
			\subsubsection{Вимоги до інформаційного забезпечення}
				Для коректної роботи Системи у~запланованому вигляді необхідне підключення до~існуючих систем аеропорту «Бориспіль», а~саме системи керування повітряним рухом. Обмін інформацією повинен бути організований за~допомогою інформаційного мережевого каналу, яким передаються повідомлення у~форматі~\allcaps{JSON}. Конкретна схема повідомлень визначається в~залежності від~типу та~призначення повідомлення, проте схема усіх повідомлень повинна мати загальні поля.

				\begin{table}[!htb]
					\caption{Загальні поля усіх повідомлень}
					\label{tab:all-msg-json-schema}
					\centering
					\begin{tabular}{
						v{3\gridunitwidth - 2\tabcolsep}
						v{9\gridunitwidth - 2\tabcolsep}
					}
						\toprule
							Поле & Опис\\
						\midrule
							Час створення & Час створення повідомлення у~форматі \allcaps{ISO}~8601 із~зазначенням зсуву часового поясу\\
							Джерело & Ідентифікатор модуля або~системи, який створив повідомлення\\
							Призначення & Ідентифікатор модуля або~системи, якому призначене повідомлення\\
						\bottomrule
					\end{tabular}
				\end{table}

			\subsubsection{Вимоги до~лінгвістичного забезпечення}
				\paragraph{Мови програмування}
					Розробка даної Системи має вестись з~використанням мов програмування високого рівня і~мов розмітки, а~саме: \textenglish{Rust} для бек-енду та~основних складових системи, \textenglish{Python} для різноманітних утиліт, \textenglish{\allcaps{HTML}} та~\textenglish{\allcaps{CSS}} для розмітки веб-сторінок.

				\paragraph{Мови взаємодії Системи та~адміністраторів}
					Англійська є~основною мовою взаємодії між Системою та~адміністраторами. Функціонал Системи повинен передбачати підтримку створення та~завантаження перекладів елементів взаємодії на~різні мови.

				\paragraph{Мови взаємодії Системи та~користувачів}
					Українська є~основною мовою взаємодії між Системою та~користувачами. Функціонал Системи повинен передбачати підтримку створення та~завантаження перекладів елементів взаємодії на~різні мови. Обов'язково мають існувати переклади на~такі мови: англійська, російська, китайська.

			\subsubsection{Вимоги до програмного забезпечення}
				Автоматизована інформаційна система аеропорту «Бориспіль» є~прикладним програмним забезпеченням, яке повинне працювати під~управлінням операційної системи \textenglish{Linux Fedora Server} версій~27 і~вище.

			\subsubsection{Вимоги до технічного забезпечення}
				Система розрахована на роботу на серверному обладнанні, яке монтується у стандартні стійки форм-фактору 42U. У якості серверного обладнання рекомендуються сервери лінійки \textenglish{Dell~R830}. Кожна серверна стійка має бути обладнана системою охолодження та онлайн джерелами безперебійного живлення достатньої потужності, які надають чисту синусоідальну форму електричного сигналу. Мережеве з'єднання серверів має виконуватись за допомогою мідних або оптичних інтерфейсів. Мідні мережеві кабелі мають відповідати категорії~5e та~краще. Управління серверним устаткуванням виконується за допомогою консолей~\textenglish{\allcaps{KVM}}, оснащених монітором, клавіатурою та мишою. Для роботи модуля «Навігація» необхідні інформаційні мультимедійні елементи, а~саме монітори та~гучномовці. Ширина моніторів має бути не~меншою, ніж 50~см.

				Кожному оператору має бути надана робоча станція \textenglish{Dell OptiPlex 3060 Micro}, оснащена монітором, клавіатурою, мишою та навушниками.
		
	\section{Склад і зміст робіт зі створення системи}
		Роботи зі створення Системи поділяються на такі стадії та етапи:
		\begin{enumerate}[noitemsep]
			\item Формування вимог.
				\begin{enumerate}[noitemsep, topsep = 0\baselineskip]
					\item Обстеження об'єкта та обгрунтування необхідності створення Системи.
					\item Формування вимог користувача до Системи.
					\item Оформлення звіту про виконану роботу і заявки на розробку Системи.
				\end{enumerate}
			\item Розробка концепції Системи
				\begin{enumerate}[noitemsep, topsep = 0\baselineskip]
					\item Дослідження об'єкта.
					\item Проведення необхідних дослідницьких робіт.
					\item Розробка варіантів концепції Системи, що задовольняє вимоги користувача.
					\item Оформлення звіту про виконану роботу.
				\end{enumerate}
			\item Технічне завдання.
				\begin{enumerate}[noitemsep, topsep = 0\baselineskip]
					\item Розробка та затвердження технічного завдання на створення Системи.
				\end{enumerate}
			\item Ескізний проект.
				\begin{enumerate}[noitemsep, topsep = 0\baselineskip]
					\item Розробка попередніх проектних рішень по Системі та її складовим.
					\item Розробка документації на Систему та її складові.
				\end{enumerate}
			\item Технічний проект.
				\begin{enumerate}[noitemsep, topsep = 0\baselineskip]
					\item Розробка проектних рішень по Системі та її складовим.
					\item Розробка документації на Систему та її складові.
					\item Розробка та оформлення документації на поставку виробів для комплектації Системи.
					\item Розробка завдань на проектування у суміжних частинах проекту об'єкту автоматизації.
				\end{enumerate}
			\item Робоча документація.
				\begin{enumerate}[noitemsep, topsep = 0\baselineskip]
					\item Розробка робочої документації на Систему та її складові.
					\item Розробка чи адаптація програм.
				\end{enumerate}
			\item Введення в експлуатацію.
				\begin{enumerate}[noitemsep, topsep = 0\baselineskip]
					\item Підготовка об'єкту автоматизації до вводу Системи в експлуатацію.
					\item Підготовка персоналу.
					\item Комплектація Системи необхідними виробами.
					\item Будівельно-монтажні роботи.
					\item Пусконалагоджувальні роботи.
					\item Проведення попередніх випробувань.
					\item Проведення дослідної експлуатації.
					\item Проведення прийомних дослідів.
				\end{enumerate}
			\item Супровід Системи.
				\begin{enumerate}[noitemsep, topsep = 0\baselineskip]
					\item Виконання роботі відповідно до гарантійних зобов'язань.
					\item Післягарантійне обслуговування.
				\end{enumerate}
		\end{enumerate}

	\section{Вимоги до~робіт з~підготовки об'єкта автоматизації до~вводу Системи в~експлуатацію}
		Для вводу системи в~експлуатацію об'єкт автоматизації має бути адаптований для~надання інформації підрозділу контролю за~повітряним рухом модулю~«Повітряний рух». Цей етап передбачає:
		\begin{enumerate}[noitemsep]
			\item Організація збереження даних контролю за~повітряним рухом.
			\item Організація доступу до~даних контролю за~повітряним рухом.
			\item Надання Системі доступу до~даних контролю за~повітряним рухом.
			\item Налагодження та~тестування обробки отриманих даних.
		\end{enumerate}

	\section{Вимоги до документування}
		Документація до~Системи повинна бути представлена у~паперовому та~електронному вигляді на~жорстких дисках або~захищеному файловому сервері з~повною резервної копією. Вихідні тексти програм надаються виключно в~електронному вигляді. Документація повинна бути оформлена українською, англійською та~російською мовами.
		Текст документації повинен бути закодований у~форматі~\textenglish{\allcaps{UTF}}-8. Ілюстративний матеріал повинен бути виконаний у~векторному форматі, де~це~можливо і~доцільно. Використання растрового формату для ілюстрацій допускається виключно у~випадку неможливості або~недоцільності створення векторної графіки для даного формату зображення.

	\section{Порядок внесення змін}
		Дане технічне завдання може доповнюватись і змінюватись у процесі розробки та приймальних випробувань в установленому порядку за взаємною згодою Замовника і Розробника.

\end{document}
