\documentclass[
	a4paper,
	oneside,
	DIV = 12,
	12pt,
	headings = normal,
]{scrartcl}

%%% Length calculations
\usepackage{calc}
%%%

%%% Support for color
\usepackage{xcolor}
\definecolor{lightblue}{HTML}{03A9F4}
\definecolor{red}{HTML}{F44336}
%%%

%%% Including graphics
\usepackage{graphicx}
%%%

%%% Font selection
\usepackage{fontspec}

\setromanfont{STIX Two Text}[
	SmallCapsFeatures = {LetterSpace = 5},
]

\setsansfont{IBM Plex Sans}[
	Scale = MatchUppercase,
]

\setmonofont{IBM Plex Mono}[
	Scale = MatchUppercase,
]
%%%

%%% Math typesetting
\usepackage{amsmath}

\usepackage{unicode-math}
\setmathfont{STIX Two Math}
%%%

%%% List settings
\usepackage{enumitem}
\setlist[enumerate]{
	label*      = {\arabic*.},
	leftmargin  = *,
	labelindent = \parindent,
	topsep      = 1\baselineskip,
	parsep      = 0\baselineskip,
	itemsep     = 1\baselineskip,
}

\setlist[itemize]{
	label*      = {—},
	leftmargin  = *,
	labelindent = \parindent,
	topsep      = 1\baselineskip,
	parsep      = 0\baselineskip,
	itemsep     = 1\baselineskip,
}

\setlist[description]{
	font        = {\rmfamily\upshape\bfseries},
	topsep      = 1\baselineskip,
	parsep      = 0\baselineskip,
	itemsep     = 0\baselineskip,
}

%%%

%%% Structural elements typesetting
\setkomafont{pagenumber}{\rmfamily}
\setkomafont{disposition}{\rmfamily\bfseries}

% Sectioning
\RedeclareSectionCommand[
	beforeskip = -1\baselineskip,
	afterskip  = 1\baselineskip,
	font       = {\normalsize\bfseries\scshape},
]{section}

\RedeclareSectionCommand[
	beforeskip = -1\baselineskip,
	afterskip  = 1\baselineskip,
	font       = {\normalsize\bfseries},
]{subsection}

\RedeclareSectionCommand[
	beforeskip = -1\baselineskip,
	afterskip  = 1\baselineskip,
	font       = {\normalsize\bfseries},
]{subsubsection}
%%%

%%% Typographic enhancements
\usepackage{microtype}
%%%

%%% Language-specific settings
\usepackage{polyglossia}
\setmainlanguage{ukrainian}
%%%

%%% Captions
\usepackage{caption}
\usepackage{subcaption}

%\DeclareCaptionLabelFormat{closing}{#2)}
%\captionsetup[subtable]{labelformat = closing}

%\captionsetup[subfigure]{labelformat = closing}

\captionsetup[table]{
	aboveskip = 0\baselineskip,
	belowskip = 1\baselineskip,
}

\captionsetup[figure]{
	aboveskip = 1\baselineskip,
	belowskip = 0\baselineskip,
}

\captionsetup[subfigure]{
	labelformat = simple,
	labelformat = brace,
}
%%%

%%% Table typesetting
\usepackage{booktabs}
\usepackage{longtable}

\usepackage{multirow}

\usepackage{array}
\newcolumntype{v}[1]{>{\raggedright\arraybackslash\hspace{0pt}}p{#1}}
\newcolumntype{b}[1]{>{\centering\arraybackslash\hspace{0pt}}p{#1}}
\newcolumntype{n}[1]{>{\raggedleft\arraybackslash\hspace{0pt}}p{#1}}
%%%

%%% Links and hyperreferences
\usepackage{hyperref}
\hypersetup{
	bookmarksnumbered = true,
	colorlinks      = false,
	linkbordercolor = red,
	urlbordercolor  = lightblue,
	pdfborderstyle  = {/S/U/W 1.5},
}
%%%

%%% Length adjustments
\setlength{\emergencystretch}{1em}
\setlength{\parindent}{1.5em}
\newlength{\gridunitwidth}
\setlength{\gridunitwidth}{\textwidth / 12}
%%%

%%% Custom commands
\newcommand{\allcaps}[1]{{\addfontfeatures{LetterSpace = 5, Kerning = Off}#1}}
%%%

\begin{document}
	\begin{titlepage}
		\begin{center}
			Міністерство освіти і науки України\\
			Національний авіаційний університет\\
			Навчально-науковий інститут комп'ютерних інформаційних технологій\\
			Кафедра комп'ютеризованих систем управління

			\vspace{\fill}
				Лабораторна робота №1.1\\
				з дисципліни «Інженерія програмного забезпечення»\\
				на тему «Опис і аналіз інформаційної системи»\\
				Варіант №3

			\vspace{\fill}

			\begin{flushright}
				Виконав:\\
				студент ННІКІТ\\
				групи СП-325\\
				Клокун В.\,Д.\\
				Перевірила:\\
				Голего Н.\,М.
			\end{flushright}

			Київ 2018
		\end{center}
	\end{titlepage}

	\section{Мета}
		Вибрати тип інформаційної системи та~спроектувати із застосуванням структурного моделювання.

	\section{Завдання}
		Обрати тип інформаційної системи відповідно до~індивідуального завдання, провести збір інформації про компанії, що~розробляють та~експлуатують аналогічні системи; з'ясувати проблемні питання, що~виникають в~ході експлуатації, сформулювати цілі розробки; зробити опис інформаційної системи.
		
	\section{Хід роботи}
		Відповідно до~номеру варіанту завданням лабораторної роботи є~опис і~аналіз інформаційної системи аеропорту. Відповідно до поставленого завдання необхідний опис був створений і задокументований~(додаток~\ref{sec:is-analysis-and-description}).

	\section{Висновок}
		Проблемами існуючою інформаційної системи аеропорту~«Бориспіль» є~монолітність, невеликий рівень стійкості та~незручність підтримки і~розширення. Нова інформаційна система вирішує існуючі проблеми, пропонуючи модульну архітектуру, яка~розподіляє єдину систему на~компоненти, що~взаємодіють між собою. Такий підхід дозволяє вносити зміни у~існуючі компоненти або~додавати нові не~турбуючи систему як~цілісний об'єкт, що~значно спрощує підтримку такої системи. Крім того, виділення компонентів дозволяє підвищити стійкість до~відмов шляхом дуплікації та~використання резервних апаратних потужностей, а~також рівень інформаційної безпеки, оскільки компрометація одного модуля не~означатиме повну компрометацію системи.

		Якщо система не~буде введена в~експлуатацію, організація ризикує значно ускладнити підтримку існуючої системи у~процесі її оновлення та~доповнення; ставить під загрозу інформаційну безпеку існуючої інфраструктури та~клієнтської бази; нехтує підвищенням рівня зручності для~кінцевого користувача~— пасажира.

		Розробці запропонованої інформаційної системи може сприяти використання такої \allcaps~{CASE}-системи, як~\allcaps{IBM} RationalRose та/або~Umbrello \allcaps{UML} Modeller, що~дозволять побудувати моделі складових запропонованої інформаційної системи, а~отже й~точніше окреслити роботу, необхідну для~впровадження системи.

		Наведені вище переваги системи наочно ілюструють доцільність реалізації проекту, а~опис і~аналіз інформаційної системи~(додаток~\ref{sec:is-analysis-and-description}) пропонують дієві механізми та~концепти для~її реалізації.

	\appendix
	\newpage
	\section{Опис інформаційної системи}
	\label{sec:is-analysis-and-description}
		\subsection{Коротка інформація про~компанію «Міжнародний аеропорт „Бориспіль“»}
		Міжнародний аеропорт «Бориспіль» є~найбільшим і~найпотужнішим в~Україні. Він забезпечує понад~67\% авіаційних пасажирських перевезень України і~обслуговує понад~10~млн. пасажирів на~рік. Наразі має дві злітно-посадкові смуги довжиною 4~000~м і~3~500~м та~чотири пасажирських термінали (один експлуатується, а~інші три законсервовані). Міжнародний аеропорт «Бориспіль» є~базовим аеропортом авіакомпанії \allcaps{МАУ}. Це єдиний аеропорт України, з~якого виконуються трансконтинентальні рейси. Аеропорт є~членом Міжнародної асоціації повітряного транспорту~(\allcaps{IATA}), Міжнародної організації цивільної авіації~(\allcaps{ICAO}) та~Міжнародної ради аеропортів~(\allcaps{ACI} Europe). 

		Основні бізнес-процеси компанії~— організація пасажирських перельотів, прийом, відправлення та~обслуговування повітряного транспорту та~перевезень.
		
		Конкуренцію компанії складає «Міжнародний аеропорт „Київ“», який, однак, є~другим за~пасажиропотоком порівняно з~«Борисполем».

		\subsection{Адреси та телефони}
			\allcaps{ДП}~\allcaps{МА}~«Бориспіль», Київська обл., Бориспільський район, с.~Гора, вул. Бориспіль,~7. Телефон: +38~044~281~78~78, факс: +38~044~281~71~22.

		\subsection{Контактні особи}
			Інтереси Державного підприємства «Міжнародний аеропорт „Бориспіль“» представляють відповідальні співробітники~(табл.~\ref{tab:contacts}), з~якими можна зв'язатись з~12:00 до~20:00.

			\begin{table}[!htbp]
				\caption{Контактні особи Міжнародного аеропорту «Бориспіль»}
				\label{tab:contacts}
				\centering
				\begin{tabular}{
						v{3\gridunitwidth - 2\tabcolsep}
						v{3\gridunitwidth - 2\tabcolsep}
						n{3\gridunitwidth - 2\tabcolsep}
						n{3\gridunitwidth - 2\tabcolsep}
					}
					\toprule
						\allcaps{ПІБ} & Посада & Телефон & E-mail\\
					\midrule
						Вовченко\newline{}Тимур Олексійович & Генеральний директор & \mbox{+38 097 601 46 09} & \url{vto@kbp.aero}\\
						Покровський Станіслав Георгійович & Виконавчий директор & \mbox{+38 063 564 15 43} & \url{psg@kbp.aero}\\
						Рогова Руслана~Іванівна & Менеджер зі~зв'язків з~громадскістю & \mbox{+38 093 555 22 11} & \url{rri@kbp.aero}\\
					\bottomrule
				\end{tabular}
			\end{table}

		\subsection{Співробітники}
			На момент проведення аналізу штат компанії складає 668~працівників.

		\subsection{Цілі проекту}
			Основними цілями проекту розробки інформаційної системи аеропорту є:
			\begin{itemize}
				\item розробка та~впровадження автоматизованої системи для~підтримки облікових та~інформаційних процесів компанії.
				\item забезпечення тісного модульного зв'язку між~складовими компонентами аеропорту;
				\item підвищення ефективності роботи складових аеропорту;
			\end{itemize}

		\subsection{Бачення виконання проекту і границі проекту}
			В рамках проекту розгортання нової системи передбачається здійснити лише у~таких підрозділах:
			\begin{enumerate}
				\item Контролю польотів.
				\item Логістики.
				\item Пасажирської навігації.
			\end{enumerate}
			Суть майбутньої системи полягатиме в~збереженні та~автоматичному обміні інформацією про поточні рейси між підрозділом контролю польотів, логістики та~пасажирської навігації. Врахування інших процесів аеропорту~(управління кадрами, нарахування заробітної плати, технічне обслуговування літаків) не~передбачається, однак, може бути реалізоване в~межах модульної архітектури.

			\subsection{Існуючий рівень забезпечення}
				У~міжнародному аеропорті «Бориспіль» на~даний момент існує інформаційна інфраструктура та~\allcaps{IT}-відділ для її підтримки~(деталі у~табл.~{\ref{tab:current-automation-specs}}).

				\begin{table}[!htb]
					\caption{Основні характеристики існуючого рівня забезпечення}
					\label{tab:current-automation-specs}
					\centering
					\begin{tabular}{
							v{6\gridunitwidth - 2\tabcolsep}
							n{6\gridunitwidth - 2\tabcolsep}
						}
							\toprule
								Характеристика & Деталі\\
							\midrule
								\multicolumn{2}{c}{\textit{Людські ресурси}}\\
								Кількість працівників відділу~\allcaps{IT} & 50\\
								Найменша кількість працівників відділу~\allcaps{IT}, активна в~дану робочу~зміну & 20\\
								\multicolumn{2}{c}{\textit{Матеріальні ресурси}}\\
								Кількість робочих станцій, шт & 200\\
								Апаратне забезпечення робочих станцій & Dell OptiPlex 3060 Micro\\
								Операційна система робочих станцій & Fedora Workstation 28\\
								Кількість серверів, шт & 40\\
								Апаратне забезпечення серверів & Dell R830\\
								Операційна система серверів & Fedora Server 28\\
								Зв'язок з~віддаленими об'єктами & Доступ до~мережі Інтернет, $2 \times 1$~Гбіт/с\\
								Кількість робочих станцій, одночасно працюючих в~мережі & 100\\
								Кількість одночасно активних серверів & 20\\
								Кількість інформаційних LCD-дисплеїв & 100\\
							\bottomrule
					\end{tabular}
				\end{table}

			\subsection{Бачення інформаційної системи}
				Основною вимогою компанії «Міжнародний аеропорт „Бориспіль“» є~легкість розширення, крім того вимагається високий рівень інформаційної безпеки, легкість супроводу та~стійкість до~відмов. Враховуючи поставлені вимоги, для~реалізації системи оптимальним є~використання модульної архітектури. Модулі повинні виконувати мінімальний набір необхідних завдань, бути легкими в~заміні, створенні та~додаванні. Таким чином, ключовими функціональними вимогами до~програмної системи є:
				\begin{itemize}
					\item Модульна архітектура, зручний та~надійний міжмодульний зв'язок.
					\item Захист від~несанкціонованого доступу.
					\item Ізоляція інформаційної системи від~зовнішніх мереж проміжним шаром модулів зовнішнього зв'язку.
					\item Стійкість до відмов окремих модулів та~можливість заміни їх забезпечення без~втрати робочого часу.
					\item Моніторинг статусу модулів та~сповіщення про~відмови компонентів або~важливі зміни.
					\item Моніторинг переміщень повітряного транспорту на~основі даних керування повітряним рухом~(\allcaps{ATC}).
					\item Управління статусом рейсів на~основі даних керування повітряним рухом.
					\item Управління внутрішньоаеропортовою логістикою на~основі даних про статус рейсу.
					\item Управління матеріалами пасажирської навігації на~основі даних логістики. Відображення статусу прильотів та~відльотів у~системах пасажирської навігаціях: інформаційних дисплеях, онлайн-ресурсах тощо.
					\item Ведення окремих журналів роботи кожного модуля.
					\item Ведення окремих журналів інформації, обробленої модулем: внесені зміни, розклади рейсів, деталізація пересувань тощо.
					\item Надання даних журналів у~зручній формі для~побудови звітів.
				\end{itemize}

				Оскільки архітектура інформаційної системи передбачає її автономність, створення системи буде виконуватись з~нуля, тому взаємодія з~існуючими системами та~компонентами у~їх поточному вигляді не~підтримується. Однак передбачені функції системи дозволять адаптувати існуючі рішення без відчутних складнощів.

				Для забезпечення стійкості до відмов у інформаційній системі повинен існувати модуль статусу, який відстежує стан підключених модулів; апаратне забезпечення, на якому запущений модуль, повинно мати резерв~— хоча б один допоміжний дублюючий сервер, якому буде передаватись управління у разі відмови. Для реалізації необхідної системи необхідні такі модулі:
				\begin{enumerate}
					\item Модуль «Статус»~— відстежує стан підключених модулів, у~разі потреби перемикати управління на~резервний модуль, заносити.
					\item Модуль~«\allcaps{ATC}»~— відповідає за~обробку даних контролю польотів: маршрут та~поточне положення літаків, приблизний час прибуття, ідентифікатор смуги для~взльоту або~посадки тощо. Надає дані модулю~«Логістика».
					\item Модуль «Логістика»~— обробляє інформацію та~оновлює стан логістичних структур аеропорту: внутрішнє пересування повітряного транспорту, статус транспорту, стадія у~циклі знаходження в~аеропорті, приблизний час до~переходу до~наступної стадії тощо.
					\item Модуль~«Навігація»~— зчитує дані про~поточне положення та~статус певного повітряного транспорту (рейсу), надані модулем~«Логістика», та~відображає на~інформаційно-навігаційні засоби на~території аеропорту.
				\end{enumerate}

				Така інформаційна система дозволить показати пасажирам зручну та~швидку у~використанні навігацію по~прильотам і~відльотам для~виводу на~інформаційні дисплеї у~вестибюлях аеропорту~(табл.~\ref{tab:tableau-arrivals},~\ref{tab:tableau-departures}), працюватиме з~високим рівнем надійності та~істотно зменшить ризик людської помилки при~передачі даних між~підрозділами.

				\begin{table}[!htbp]
					\caption{Приклад інформаційно-навігаційного табло прильотів}
					\label{tab:tableau-arrivals}
					\centering
					\begin{tabular}{
							% *{6}{l}	
							v{2\gridunitwidth - 2\tabcolsep}
							v{3\gridunitwidth - 2\tabcolsep}
							v{2\gridunitwidth - 2\tabcolsep}
							v{2\gridunitwidth - 2\tabcolsep}
							v{3\gridunitwidth - 2\tabcolsep}
						}
						\toprule
							Час   & Призначення & Рейс    & Термінал & Статус\\
						\midrule
							22:25 & Бодрум      & 7W 7032 & D        & Прибув\\
							22:40 & Батумі      & YE 1216 & F        & Очікується о 22:23\\
						\bottomrule
					\end{tabular}
				\end{table}

				\begin{table}[!htbp]
					\caption{Приклад інформаційно-навігаційного табло відльотів}
					\label{tab:tableau-departures}
					\centering
					\begin{tabular}{
							%*{6}{l}
							v{2\gridunitwidth - 2\tabcolsep}
							v{3\gridunitwidth - 2\tabcolsep}
							v{2\gridunitwidth - 2\tabcolsep}
							v{2\gridunitwidth - 2\tabcolsep}
							v{3\gridunitwidth - 2\tabcolsep}
						}
						\toprule
							Час   & Призначення & Рейс    & Термінал & Статус\\
						\midrule
							22:55 & Тель-Авів   & LY 2654 & D 2      & Посадка\\
							23:40 & Доха        & QR 298  & F 27–30  & Реєстрація\\
						\bottomrule
					\end{tabular}
				\end{table}

			\subsection{Бачення обліку інформаційної системи}
				Облік в~інформаційній системі передбачає ведення журналів інформації, що~була оброблена модулями. Модуль~«Статус» зберігає дані про~функціонування підключених модулів у~текстовому форматі, відсортованому за~часом події. До~таких даних відносяться:
				\begin{enumerate}
					\item Результати регулярної (наприклад, щохвилинної) перевірки статусу цільового модуля.
					\item Усі повідомлення, відіслані цільовим модулем та~отримані модулем~«Статус».
				\end{enumerate}
				Модуль «\allcaps{ATC}» зберігає дані про~рух повітряного транспорту відповідно до~вимог регуляторних інстанцій (наприклад, \allcaps{ICAO}). Модуль~«Логістика» зберігає дані про~пересування та~зміни стану повітряного транспорту, що~прибув до~аеропорту та~рухається у~його межах, у~текстовому форматі, до якого входять:
				\begin{enumerate}
					\item Дата та час у форматі~\allcaps{ISO}~8601.
					\item Ідентифікатор повітряного транспорту.
					\item Ідентифікатор рейсу.
					\item Поточний стан.
					\item Очікуваний наступний стан.
					\item Приблизний час до переходу до очікуваного наступного стану.
				\end{enumerate}
				Модуль~«Навігація» зберігає текстові записи про~доступність рейсу для~пасажирів у~текстовому форматі, до~якого входять усі дані з~інформаційного табло~(табл.~\ref{tab:tableau-arrivals}, \ref{tab:tableau-departures}), однак, представлення часу розширюється до~повного представлення відповідно до~стандарту \allcaps{ISO}~8601.

\end{document}
