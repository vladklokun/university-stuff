\documentclass[
	a4paper,
	oneside,
	DIV = 12,
	12pt,
	headings = normal,
]{scrartcl}

%%% Length calculations
\usepackage{calc}
%%%

%%% Support for color
\usepackage{xcolor}
\definecolor{lightblue}{HTML}{03A9F4}
\definecolor{red}{HTML}{F44336}
%%%

%%% Including graphics
\usepackage{graphicx}
%%%

%%% Font selection
\usepackage{fontspec}

\setromanfont{STIX Two Text}[
	SmallCapsFeatures = {LetterSpace = 5},
]

\setsansfont{IBM Plex Sans}[
	Scale = MatchUppercase,
]

\setmonofont{IBM Plex Mono}[
	Scale = MatchUppercase,
]
%%%

%%% Math typesetting
\usepackage{amsmath}

\usepackage{unicode-math}
\setmathfont{STIX Two Math}
%%%

%%% List settings
\usepackage{enumitem}
\setlist[enumerate]{
	label*      = {\arabic*.},
	leftmargin  = *,
	labelindent = \parindent,
	topsep      = 1\baselineskip,
	parsep      = 0\baselineskip,
	itemsep     = 1\baselineskip,
}

\setlist[itemize]{
	label*      = {—},
	leftmargin  = *,
	labelindent = \parindent,
	topsep      = 1\baselineskip,
	parsep      = 0\baselineskip,
	itemsep     = 1\baselineskip,
}

\setlist[description]{
	font        = {\rmfamily\upshape\bfseries},
	topsep      = 1\baselineskip,
	parsep      = 0\baselineskip,
	itemsep     = 0\baselineskip,
}

%%%

%%% Structural elements typesetting
\setkomafont{pagenumber}{\rmfamily}
\setkomafont{disposition}{\rmfamily\bfseries}

% Sectioning
\RedeclareSectionCommand[
	beforeskip = -1\baselineskip,
	afterskip  = 1\baselineskip,
	font       = {\normalsize\bfseries\scshape},
]{section}

\RedeclareSectionCommand[
	beforeskip = -1\baselineskip,
	afterskip  = 1\baselineskip,
	font       = {\normalsize\bfseries},
]{subsection}

\RedeclareSectionCommand[
	beforeskip = -1\baselineskip,
	afterskip  = 1\baselineskip,
	font       = {\normalsize\bfseries},
]{subsubsection}
%%%

%%% Typographic enhancements
\usepackage{microtype}
%%%

%%% Language-specific settings
\usepackage{polyglossia}
\setmainlanguage{ukrainian}
%%%

%%% Captions
\usepackage{caption}
\usepackage{subcaption}

%\DeclareCaptionLabelFormat{closing}{#2)}
%\captionsetup[subtable]{labelformat = closing}

%\captionsetup[subfigure]{labelformat = closing}

\captionsetup[table]{
	aboveskip = 0\baselineskip,
	belowskip = 1\baselineskip,
}

\captionsetup[figure]{
	aboveskip = 1\baselineskip,
	belowskip = 0\baselineskip,
}

\captionsetup[subfigure]{
	labelformat = simple,
	labelformat = brace,
}
%%%

%%% Table typesetting
\usepackage{booktabs}
\usepackage{longtable}

\usepackage{multirow}

\usepackage{array}
\newcolumntype{v}[1]{>{\raggedright\arraybackslash\hspace{0pt}}p{#1}}
\newcolumntype{b}[1]{>{\centering\arraybackslash\hspace{0pt}}p{#1}}
\newcolumntype{n}[1]{>{\raggedleft\arraybackslash\hspace{0pt}}p{#1}}
%%%

%%% Drawing
\usepackage{tikz}
\usepackage{tikzscale}
\usetikzlibrary{positioning}
\usetikzlibrary{arrows.meta} % Stealth arrow tips
%%%

%%% Links and hyperreferences
\usepackage{hyperref}
\hypersetup{
	bookmarksnumbered = true,
	colorlinks      = false,
	linkbordercolor = red,
	urlbordercolor  = lightblue,
	pdfborderstyle  = {/S/U/W 1.5},
}
%%%

%%% Length adjustments
% Set baselineskip, default is 14.5 pt
\linespread{1.068966} % ~15.5 pt
\setlength{\emergencystretch}{1em}
\setlength{\parindent}{1.5em}
\newlength{\gridunitwidth}
\setlength{\gridunitwidth}{\textwidth / 12}
%%%

%%% Custom commands
\newcommand{\allcaps}[1]{{\addfontfeatures{LetterSpace = 5, Kerning = Off}#1}}
%%%

\begin{document}
	\begin{titlepage}
		\begin{center}
			Міністерство освіти і науки України\\
			Національний авіаційний університет\\
			Навчально-науковий інститут комп'ютерних інформаційних технологій\\
			Кафедра комп'ютеризованих систем управління

			\vspace{\fill}
				Лабораторна робота №1.1\\
				з дисципліни «Інженерія програмного забезпечення»\\
				на тему «Опис і аналіз інформаційної системи»\\
				Варіант №3

			\vspace{\fill}

			\begin{flushright}
				Виконав:\\
				студент ННІКІТ\\
				групи СП-325\\
				Клокун В.\,Д.\\
				Перевірила:\\
				Голего Н.\,М.
			\end{flushright}

			Київ 2018
		\end{center}
	\end{titlepage}

	\section{Мета}
		Вибрати тип інформаційної системи та~спроектувати із~застосуванням структурного моделювання.

	\section{Завдання}
		Обрати тип інформаційної системи відповідно до~індивідуального завдання, провести збір інформації про компанії, що~розробляють та~експлуатують аналогічні системи; з'ясувати проблемні питання, що~виникають в~ході експлуатації, сформулювати цілі розробки; зробити опис інформаційної системи.
		
	\section{Хід роботи}
		Відповідно до~номеру варіанту завданням лабораторної роботи є~опис і~аналіз інформаційної системи аеропорту. У~результаті виконання роботи згідно з~завданням був проведений аналіз і~розроблений опис необхідної інформаційної системи~(структурна схема на рис.~\ref{fig:01-is-structure-diagram}).

		\begin{figure}[!htbp]
			\centering
			\includegraphics{./assets/01-is-structure-diagram.tikz}
			\caption{Структура інформаційної системи}
			\label{fig:01-is-structure-diagram}
		\end{figure}

%%%
		Як видно зі~структурної схеми інформаційної системи, в~її~основу покладена модульна архітектура. Для~забезпечення стійкості до~відмов у~інформаційній системі повинен існувати модуль керування, який відстежує стан підключених модулів; апаратне забезпечення, на~якому запущений модуль, повинно мати резерв~— хоча~б один допоміжний дублюючий сервер, якому буде передаватись управління у~разі відмови. Для~реалізації необхідної системи необхідні такі модулі:
		\begin{enumerate}
			\item Модуль «Керування»~— керує підключеними модулями, відстежує їх~поточний стан, щоб у~разі потреби перемикнути управління на~резервне забезпечення модуля. За~роботою модуля наглядає оператор, оскільки проблеми в~його роботі можуть призвести до~серйозних наслідків.
			\item Модуль~«\allcaps{ATC}»~— відповідає за~обробку даних контролю польотів: маршрут та~поточне положення літаків, приблизний час прибуття, ідентифікатор смуги для~взльоту або~посадки тощо. Надає дані модулю~«Логістика». За~роботою модуля наглядає оператор, оскільки надання невірних даних може призвести до~перебоїв у~роботі аеропорту.
			\item Модуль «Логістика»~— обробляє інформацію та~оновлює стан логістичних структур аеропорту: внутрішнє пересування повітряного транспорту, статус транспорту, стадія у~циклі знаходження в~аеропорті, приблизний час до~переходу до~наступної стадії тощо. За~роботою модуля наглядає оператор, який аналізує та~перевіряє правильність ходу процесів, що~виконуються в~аеропорті.
			\item Модуль~«Навігація»~— зчитує дані про~поточне положення та~статус певного повітряного транспорту (рейсу), надані модулем~«Логістика», та~відображає на~інформаційно-навігаційні засоби на~території аеропорту.
		\end{enumerate}

	\section{Висновок}
		Проблемами існуючої інформаційної системи аеропорту~«Бориспіль» є~монолітність, невеликий рівень стійкості та~незручність підтримки і~розширення. Нова інформаційна система вирішує існуючі проблеми, пропонуючи модульну архітектуру, яка~розподіляє єдину систему на~компоненти, що~взаємодіють між собою. Такий підхід дозволяє вносити зміни у~існуючі компоненти або~додавати нові не~турбуючи систему як~цілісний об'єкт, що~значно спрощує підтримку такої системи. Крім того, виділення компонентів дозволяє підвищити стійкість до~відмов шляхом дуплікації та~використання резервних апаратних потужностей, а~також рівень інформаційної безпеки, оскільки компрометація одного модуля не~означатиме повну компрометацію системи.

		Якщо система не~буде введена в~експлуатацію, організація ризикує значно ускладнити підтримку існуючої системи у~процесі її оновлення та~доповнення; ставить під загрозу інформаційну безпеку існуючої інфраструктури та~клієнтської бази; нехтує підвищенням рівня зручності для~кінцевого користувача~— пасажира.

		Розробці запропонованої інформаційної системи може сприяти використання такої \allcaps~{CASE}-системи, як~\allcaps{IBM} RationalRose та/або~Umbrello \allcaps{UML} Modeller, що~дозволять побудувати моделі складових запропонованої інформаційної системи, а~отже й~точніше окреслити роботу, необхідну для~впровадження системи.

		Наведені вище переваги системи наочно ілюструють доцільність реалізації проекту, а~опис і~аналіз інформаційної системи пропонують дієві механізми та~концепти для~її реалізації.

\end{document}
