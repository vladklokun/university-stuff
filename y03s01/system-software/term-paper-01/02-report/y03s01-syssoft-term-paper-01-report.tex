\documentclass[
	a4paper,
	oneside,
	% BCOR = 10mm,
	DIV = 12,
	fontsize = 13pt,
	headings = normal,
	numbers = endperiod,
]{scrartcl}

%%% Length calculations
\usepackage{calc}
%%%

%%% Support for color
\usepackage{xcolor}
\definecolor{lightblue}{HTML}{03A9F4}
\definecolor{red}{HTML}{F44336}
%%%

%%% Including graphics
\usepackage{graphicx}
%%%

%%% Font selection
\usepackage{fontspec}

\setromanfont{STIX Two Text}[
	SmallCapsFeatures = {LetterSpace = 8},
]

\setsansfont{IBM Plex Sans}[
	Scale = MatchUppercase,
]

\setmonofont{IBM Plex Mono}[
	Scale = MatchUppercase,
]
%%%

%%% Math typesetting
\usepackage{amsmath}

\usepackage{amsthm}
\newtheoremstyle{mythm}%
	{1\baselineskip} % Space above
	{1\baselineskip} % Space below
	{\itshape} % Body font
	{} % Indent amount
	{\upshape\bfseries} % Theorem head font
	{.} % Punctuation after theorem head
	{0.5em} % Space after theorem head
	{} % Theorem head spec
\theoremstyle{mythm}
\newtheorem{mythm}{Теорема}
\newtheorem{mydef}{Визначення}

\usepackage{unicode-math}
\setmathfont{STIX Two Math}
%%%

%%% List settings
\usepackage{enumitem}
\setlist[enumerate]{
	label*      = {\arabic*.},
	leftmargin  = *,
	labelindent = \parindent,
	topsep      = 1\baselineskip,
	parsep      = 0\baselineskip,
	itemsep     = 1\baselineskip,
}

\setlist[itemize]{
	label*      = {—},
	leftmargin  = *,
	% labelindent = \parindent,
	align       = left,
	topsep      = 1\baselineskip,
	parsep      = 0\baselineskip,
	itemsep     = 0\baselineskip,
}

\setlist[description]{
	font        = {\rmfamily\upshape\bfseries},
	topsep      = 1\baselineskip,
	parsep      = 0\baselineskip,
	itemsep     = 0\baselineskip,
}

\newlist{termpaperinfo}{enumerate}{3}
\setlist[termpaperinfo]{
	label*      = {\arabic*.},
	leftmargin  = *,
	align       = left,
	topsep      = 1\baselineskip,
	parsep      = 0\baselineskip,
	itemsep     = 1\baselineskip,
}

%%%

%%% Structural elements typesetting
\setkomafont{pagenumber}{\rmfamily}
\setkomafont{disposition}{\rmfamily\bfseries}

% Sectioning
\RedeclareSectionCommand[
	beforeskip = -1\baselineskip,
	afterskip  = 1\baselineskip,
	font       = {\normalsize\bfseries},
]{section}

% \renewcommand*{\sectionformat}{\thesection\autodot\enskip}

\renewcommand{\sectionlinesformat}[4]{%
	\centering{}#3#4%
}

\RedeclareSectionCommand[
	beforeskip = -1\baselineskip,
	afterskip  = 1\baselineskip,
	font       = {\normalsize\bfseries},
]{subsection}

\RedeclareSectionCommand[
	beforeskip = -1\baselineskip,
	afterskip  = 1\baselineskip,
	font       = {\normalsize\bfseries},
]{subsubsection}

\RedeclareSectionCommand[
	beforeskip = -1\baselineskip,
	afterskip  = -0.5em,
	font       = {\normalsize\mdseries\scshape\addfontfeatures{Letters = {UppercaseSmallCaps}}},
]{paragraph}
%%%

%%% Typographic enhancements
\usepackage{microtype}
%%%

%%% Language-specific settings
\usepackage{polyglossia}
\setmainlanguage{ukrainian}
\setotherlanguages{english}
%%%

%%% Captions
\usepackage{caption}
\usepackage{subcaption}

%\DeclareCaptionLabelFormat{closing}{#2)}
%\captionsetup[subtable]{labelformat = closing}

%\captionsetup[subfigure]{labelformat = closing}

\captionsetup[table]{
	aboveskip = 0\baselineskip,
	belowskip = 0\baselineskip,
}

\captionsetup[figure]{
	aboveskip = 1\baselineskip,
	belowskip = 0\baselineskip,
}

\captionsetup[subfigure]{
	labelformat = simple,
	labelformat = brace,
}
%%%

%%% Hyphenated ragged typesetting
\usepackage{ragged2e}
%%%

%%% Table typesetting
\usepackage{booktabs}
\usepackage{longtable}

\usepackage{multirow}

\usepackage{array}
\newcolumntype{v}[1]{>{\RaggedRight\arraybackslash\hspace{0pt}}p{#1}}
\newcolumntype{b}[1]{>{\Centering\arraybackslash\hspace{0pt}}p{#1}}
\newcolumntype{n}[1]{>{\RaggedLeft\arraybackslash\hspace{0pt}}p{#1}}
%%%

%%% Drawing
\usepackage{tikz}
\usepackage{tikzscale}
\usetikzlibrary{positioning}
\usetikzlibrary{arrows.meta} % Stealth arrow tips
%%%

%%% SI units typesetting
\usepackage{siunitx}
\sisetup{
	output-decimal-marker = {,},
	exponent-product      = {\cdot},
	inter-unit-product    = \ensuremath{{} \cdot {}},
	per-mode              = symbol,
}
%%%

%%% Framing code listings
\usepackage{tcolorbox}
\tcbuselibrary{breakable}
\tcbuselibrary{minted}
\tcbuselibrary{skins}

\newtcbinputlisting[auto counter, list inside, number within = section]{\inputpython}[4][]{%
	minted language = python,
	minted style    = bw,
	minted options  = {
		linenos,
		tabsize = 4,
		breaklines,
		breakbytokenanywhere,
		fontsize = \footnotesize,
	},
	%
	% empty,
	sharp corners,
	colframe         = black,
	colback          = black!0,
	leftrule         = 0em,
	rightrule        = 0em,
	toprule          = 0pt, % orig = 0pt
	bottomrule       = 0pt, % orig = 0pt
	titlerule        = 0.5pt,
	colbacktitle     = black!0,
	coltitle         = black,
	toptitle         = 0.3em,
	bottomtitle      = 0.1em,
	borderline north = {1pt}{0pt}{black},
	borderline south = {1pt}{0pt}{black},
	before skip      = \intextsep,
	after  skip      = \intextsep,
	title            = {Лістинг \thetcbcounter: #3},
	list entry       = {\protect\numberline{\thetcbcounter}#3},
	left = 0em,
	right = 0em,
	%
	listing file={#2},
	listing only,
	breakable,
	%
	label = {#4},
	%
	#1
}

% Customize minted line numbers
\renewcommand{\theFancyVerbLine}{\ttfamily\scriptsize\arabic{FancyVerbLine}}

%%%

%%% Bibliography
\usepackage[
	style    = gost-numeric,
	language = auto,
	autolang = other,
	sorting  = none,
]{biblatex}
\addbibresource{y03s01-syssoft-term-paper-01-bibliography.bib}
%%%

%%% Links and hyperreferences
\usepackage{hyperref}
\hypersetup{
	bookmarksnumbered = true,
	colorlinks      = false,
	linkbordercolor = red,
	urlbordercolor  = lightblue,
	pdfborderstyle  = {/S/U/W 1.5},
}
%%%

%%% Length adjustments
% Set baselineskip, default is 14.5 pt
\linespread{1.068966} % ~15.5 pt
\setlength{\emergencystretch}{1em}
\setlength{\parindent}{1.5em}
\newlength{\gridunitwidth}
\setlength{\gridunitwidth}{\textwidth / 12}
%%%

%%% Custom commands
\newcommand{\blankspace}[1]{\underline{\hspace{#1}}}
\newcommand{\allcaps}[1]{{\addfontfeatures{LetterSpace = 8, Kerning = Off}#1}}
\newcommand{\filename}[1]{\texttt{#1}}
\newcommand{\progname}[1]{\texttt{#1}}
%%%

\begin{document}

\begin{titlepage}
		\begin{center}
			Міністерство освіти і науки України\\
			Національний авіаційний університет\\
			Навчально-науковий інститут комп'ютерних інформаційних технологій\\
			Кафедра комп'ютеризованих систем управління

			\vspace{\fill}
				Курсова робота\\
				з дисципліни «Системне програмне забезпечення»\\

				\vspace*{3\baselineskip}

				Пояснювальна записка\\
				Тема: реалізація наївного баєсового класифікатора на~мові~програмування~\textenglish{Python}

			\vspace{\fill}

			\begin{flushright}
				Виконав:\\
				студент групи СП-325\\
				Клокун В.\,Д.\\
			\end{flushright}

			Київ — 2018
		\end{center}
	\end{titlepage}

	\section*{Завдання на~виконання курсової~роботи\\студента групи~СП-325 Клокуна Владислава~Денисовича}
	% {\centering{}студента групи~СП-325 Клокуна Владислава Денисовича\par}
	\begin{termpaperinfo}
		\item Тема курсової роботи: реалізація наївного баєсового класифікатора на~мові~програмування~\textenglish{Python} для класифікації спостережень, що~містять неперервні дані.
		\item Термін виконання курсової роботи:\\ з~«\blankspace{1cm}» \blankspace{4cm}~2018~р. по~«\blankspace{1cm}»~\blankspace{4cm}~2018~р.
		\item Вхідні дані до роботи: набір даних для класифікації.
		\item Етапи виконання курсової роботи:
			\begin{itemize}
				\item Огляд теоретичних відомостей про наївний баєсов класифікатор.
				\item Реалізація та тестування наївного баєсового класифікатора.
			\end{itemize}
		\item Перелік обов'язкових додатків і графічного матеріалу:
			\begin{itemize}
				\item FIXME.
			\end{itemize}
	\end{termpaperinfo}

	{%
		\newlength{\blanklinematch}
		\setlength{\blanklinematch}{1cm + \widthof{» } + 5cm}
		\noindent%
		\begin{tabular}{
				@{}ll
			}
			Завдання отримав: & «\blankspace{1cm}» \blankspace{5cm}~2018~р.\\
			Підпис студента:  & \phantom{«}\blankspace{\blanklinematch}~(Клокун В.\,Д.)\\
			% Підпис студента:  & \blankspace{\widthof{«} + 1cm + \widthof{»} + 5cm}~(Клокун В.\,Д.)
		\end{tabular}
		\par
	}

	\newpage
	\tableofcontents

	\newpage
	\section{Теоретична частина}
		\subsection{Короткі теоретичні відомості}
			Припустимо, що~в~ході деякого експерименту проводились спостереження, під час проведення яких збирались неперервні (недискретні) дані про~результат події. Також були визначені категорії (або класи), до~яких ці~дані можуть належати. Поставлена задача класифікувати дані спостережень. \emph{Класифікація}~— це~задача визначення, до~якої з~категорій належить певне спостереження~\cite{wiki-stat-classification}. \emph{Класифікатор}~— це~алгоритм, який виконує класифікацію~\cite{wiki-stat-classification}.

			\emph{Наївний баєсів класифікатор}~— це ймовірнісний класифікатор, який використовує теорему Баєса для класифікації спостережень. Такі класифікатори отримують на вхід спостереження, оцінюють його і~роблять припущення про~клас, до~якого воно~належить. Вхідні дані, тобто спостереження, представляються у~вигляді вектора відомих значень випадкових змінних, які називаються \emph{ознаками}. Результатом роботи класифікатора є~певне значення цільової змінної або змінних, які зазвичай називаються класовими, і позначають клас, до~якого належить спостереження.

			Принцип класифікації полягає в~обчисленні умовних імовірностей~(визначення~\ref{def:conditional-probability}) того, що~вхідні дані належать до~кожного з~класів~(події, які~нас цікавлять), за~умови, що~ознаки мають певні значення~(події, які ми спостерігаємо). Після обчислення кожної з~умовних імовірностей знаходиться найбільша та~робиться висновок про~належність спостереження до~кожного з~класів. Саме тому наївний баєсів класифікатор називають~ймовірнісним.

			\begin{mydef}[Умовна ймовірність]
				\label{def:conditional-probability}
				Нехай~$A$ і~$B$~— події. Позначимо ймовірність настання кожної з~них незалежно одна від~одної як~$P(A)$ і~$P(B)$ відповідно. Тоді \emph{умовною імовірністю}~$P(A \mid B)$ називається ймовірність настання події~$A$ за~умови, що~подія~$B$ настала.
			\end{mydef}
			
			Наприклад, нехай подія~$A$~— дане спостереження належить до~певного класу, подія~$B$~— ознаки спостереження мають певні значення. Тоді щоб знайти ймовірність, що дане спостереження з~певним значенням ознак належить до~певного класу, необхідно обчислити умовну ймовірність~$P(A \mid B)$. Для~обчислення цієї імовірності необхідно використати теорему~Баєса~(теорема~\ref{thm:theorem-bayes}).

			\begin{mythm}[Баєса]
				\label{thm:theorem-bayes}
				Нехай~$P(A \mid B)$~— умовна ймовірність настання події~$A$ за~умови, що~подія~$B$ настала; $P(B \mid A)$~— умовна ймовірність настання події~$B$ за~умови, що~подія~$A$ настала і~$P(B)$~— імовірність настання події~$B$, причому~$P(B) \neq 0$. Тоді умовна ймовірність~$P(A \mid B)$ обчислюється так:
				\[
					P(A \mid B) = \frac{P(B \mid A) \, P(A)}{P(B)}.
				\]
			\end{mythm}

			Оскільки при обчисленні умовних імовірностей використовується теорема Баєса, такий імовірнісний класифікатор називається баєсовим.

		\subsection{Класифікація неперервних даних за~допомогою наївного баєсового класифікатора}

	\newpage
	\section{Практична частина}

	\newpage
	\addsec{Висновки}

	\newpage
	\printbibliography

\end{document}
