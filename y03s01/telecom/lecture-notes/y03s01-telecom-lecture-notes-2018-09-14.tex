\documentclass[
	a4paper,
	oneside,
	BCOR = 10mm,
	DIV = 12,
	12pt,
	headings = normal,
]{scrartcl}

%%% Length calculations
\usepackage{calc}
%%%

%%% Support for color
\usepackage{xcolor}
\definecolor{lightblue}{HTML}{03A9F4}
\definecolor{red}{HTML}{F44336}
%%%

%%% Including graphics
\usepackage{graphicx}
%%%

%%% Font selection
\usepackage{fontspec}

\setromanfont{STIX Two Text}[
	SmallCapsFeatures = {LetterSpace = 8},
]

\setsansfont{IBM Plex Sans}[
	Scale = MatchUppercase,
]

\setmonofont{IBM Plex Mono}[
	Scale = MatchUppercase,
]
%%%

%%% Math typesetting
\usepackage{amsmath}

\usepackage{unicode-math}
\setmathfont{STIX Two Math}
%%%

%%% List settings
\usepackage{enumitem}
\setlist[enumerate]{
	label*      = {\arabic*.},
	leftmargin  = *,
	labelindent = \parindent,
	topsep      = 1\baselineskip,
	parsep      = 0\baselineskip,
	itemsep     = 1\baselineskip,
}

\setlist[itemize]{
	label*      = {—},
	leftmargin  = *,
	labelindent = \parindent,
	topsep      = 1\baselineskip,
	parsep      = 0\baselineskip,
	itemsep     = 1\baselineskip,
}

\setlist[description]{
	font        = {\rmfamily\upshape\bfseries},
	topsep      = 1\baselineskip,
	parsep      = 0\baselineskip,
	itemsep     = 0\baselineskip,
}

%%%

%%% Structural elements typesetting
\setkomafont{pagenumber}{\rmfamily}
\setkomafont{disposition}{\rmfamily\bfseries}

% Sectioning
\RedeclareSectionCommand[
	beforeskip = -1\baselineskip,
	afterskip  = 1\baselineskip,
	font       = {\normalsize\bfseries\scshape},
]{section}

\RedeclareSectionCommand[
	beforeskip = -1\baselineskip,
	afterskip  = 1\baselineskip,
	font       = {\normalsize\bfseries},
]{subsection}

\RedeclareSectionCommand[
	beforeskip = -1\baselineskip,
	afterskip  = 1\baselineskip,
	font       = {\normalsize\bfseries\itshape},
]{subsubsection}

\RedeclareSectionCommand[
	beforeskip = -1\baselineskip,
	afterskip  = -1em,
	font       = {\normalsize\mdseries\scshape\addfontfeatures{Letters = {UppercaseSmallCaps}}},
]{paragraph}
%%%

%%% Typographic enhancements
\usepackage{microtype}
%%%

%%% Language-specific settings
\usepackage{polyglossia}
\setmainlanguage{ukrainian}
%%%

%%% Captions
\usepackage{caption}
\usepackage{subcaption}

%\DeclareCaptionLabelFormat{closing}{#2)}
%\captionsetup[subtable]{labelformat = closing}

%\captionsetup[subfigure]{labelformat = closing}

\captionsetup[table]{
	aboveskip = 0\baselineskip,
	belowskip = 1\baselineskip,
}

\captionsetup[figure]{
	aboveskip = 1\baselineskip,
	belowskip = 0\baselineskip,
}

\captionsetup[subfigure]{
	labelformat = simple,
	labelformat = brace,
}
%%%

%%% Hyphenated ragged typesetting
\usepackage{ragged2e}
%%%

%%% Table typesetting
\usepackage{booktabs}
\usepackage{longtable}

\usepackage{multirow}

\usepackage{array}
\newcolumntype{v}[1]{>{\RaggedRight\arraybackslash\hspace{0pt}}p{#1}}
\newcolumntype{b}[1]{>{\Centering\arraybackslash\hspace{0pt}}p{#1}}
\newcolumntype{n}[1]{>{\RaggedLeft\arraybackslash\hspace{0pt}}p{#1}}
%%%

%%% Drawing
\usepackage{tikz}
\usepackage{tikzscale}
\usetikzlibrary{
	positioning,
	arrows.meta, % Stealth arrow tips
	decorations.pathreplacing, % % grouping brace
	fit, % fit
}

\tikzset{>=Stealth}

\tikzset{mynode/.style = {
		font = \strut,
		draw,
		rectangle,
		fill = white,
		minimum height = 4\baselineskip,
		minimum width  = 2\gridunitwidth,
		text width = 3\gridunitwidth,
		align = center,
	},
}

\tikzset{caption/.style = {
		font = \scriptsize,
	},
}
%%%

%%% Date and time typesetting
\usepackage[
	ukrainian,
	useregional,
	showdow,
]{datetime2}
%%%

%%% Links and hyperreferences
\usepackage{hyperref}
\hypersetup{
	bookmarksnumbered = true,
	colorlinks      = false,
	linkbordercolor = red,
	urlbordercolor  = lightblue,
	pdfborderstyle  = {/S/U/W 1.5},
}
%%%

%%% Length adjustments
% Set baselineskip, default is 14.5 pt
\linespread{1.068966} % ~15.5 pt
\setlength{\emergencystretch}{1em}
\setlength{\parindent}{1.5em}
\newlength{\gridunitwidth}
\setlength{\gridunitwidth}{\textwidth / 12}
\newlength{\gridunitheight}
\setlength{\gridunitheight}{\textheight / 12}
%%%

%%% Custom commands
\newcommand{\allcaps}[1]{{\addfontfeatures{LetterSpace = 8, Kerning = Off}#1}}

\newcommand{\mynote}[1]{(\textit{Прим.}~#1)}
%%%

\begin{document}

\begin{titlepage}
		\vspace*{4\gridunitheight}
		% \begin{center}
			\noindent\textsc{Конспект}\\[\baselineskip]
			{\setlength{\tabcolsep}{0em}
			\begin{tabular}{
					v{3\gridunitwidth - 2\tabcolsep}
					v{6\gridunitwidth - 2\tabcolsep}
			}
				з дисципліни: & «Технології комунікацій»\\
				викладач:     & Пушкін Ю.\,О.\\
			\end{tabular}
			}
		% \end{center}
		\vspace*{\fill}
	\end{titlepage}

	\tableofcontents
	\newpage

	\section{\DTMDate{2018-09-14}}
		\subsection{Информация, сообщение, сигнал}
			Понятие «информация» имеет много различных аспектов, и~в связи с~этим существует несколько различных подходов к~её~определению. \emph{Информация}~— совокупность сведений о~каком-то событии, явлении, предмете, являющихся объектом хранения, передачи и~преобразования. Для выполнения указанных действий используют условные символы~— буквы, жесты, математические знаки, позволяющие выразить информацию в~необходимой форме. Совокупность знаков, которые используют для~хранения, передачи и~обработки, называют \emph{сообщением}. 
			
			В~различных технических системах информация представляется в~двоичной форме. Соответственно, сообщением может служить последовательность конечного числа двоичных символов.

			Различают дискретные и~непрерывные сообщения. Дискретные сообщения формируются в~результате последовательной выдачи источником сообщения отдельных знаков.

			Множество различных знаков называют \emph{алфавитом источника сообщений}, а~их~количество~— объёмом алфавита. Непрерывные сообщения не~разделены на~элементы, они описываются функциями времени, принимающими непрерывное множество значений. Пример~— телевизионное изображение.

			Передача сообщений на~расстояние осуществляется с~помощью какого-либо материального носителя или~физического процесса (волны, ток, колебания и~т.\,д.). Физический процесс, посредством которого сообщение передаётся на~расстояние, называется \emph{сигналом}. В~современных системах управления чаще всего используются электрические сигналы.

			Процесс изменения параметров носителя принято называть модуляцией. Различают такие виды сигналов~(рис.~2.1 \mynote{Лектор сказал искать рисунки в~тексте лаб}):
			\begin{enumerate}[noitemsep]
				\item Непрерывные по~уровню и~по~времени~(2.1а).
				\item Непрерывные по~уровню и~дискретные по~времени~(2.1б).
				\item Дискретные по~уровню и~непрерывные по~времени~(2.1в).
				\item Дискретные по~уровню и~по~времени~(2.1г).
			\end{enumerate}

			Сигналы первого вида, называемые непрерывными, задаются на~конечном или~бесконечном временном интервале и~могут принимать любые значение в~некотором диапазоне. 

			Сигналы второго вида задаются в~определённый дискретный момент времени и~могут принимать любые значения из~некоторого диапазона. Их можно получить из~непрерывных сигналов путём взятия отсчёта в~определённый момент времени. Это преобразование называется дискретизацией по~времени.

			Сигналы третьего вида, называемые квантоваными по~времени, задаются на~некотором временном интервале и~характеризуются тем, что принимают вполне только определённые дискретные значения. Их~можно получить из~непрерывных сигналов, применяя к~ним операцию квантования по~уровню.

			Сигналы четвёртого вида также называются дискретными, задаются в~определённые дискретные моменты и~принимают определённые значения. Их~можно получить из~непрерывных сигналов, осуществляя операцию дискретизации по~времени и~квантования по~уровню.

			\subsection{Обобщённая структурная схема системы связи}
				Системой связи называют совокупность технических средств, предназначенных для~передачи информации от~передатчика сообщений и~получателя информации.

				Структурная схема простейшей системы связи показана на~рис.~2.2. Источником сообщения~(1) может быть человек или~различного рода устройство, он~осуществляет выбор сообщений из~ансамбля сообщений. Если сообщение на~выходе источника имеет не~электрическую природу, то~для его передачи в~системе связи оно преобразуется в~первичный электрический сигнал.

				Первичные сигналы являются низкочастотными. Для~передачи на~большие расстояния используются специальные электромагнитные колебания высокой частоты, называемые переносчиками, которые могут эффективно распространяться по~линиям связи. 

				В передающем устройстве~(2) первичный сигнал превращается во~вторичный (высокочастотный) сигнал~$S(t)$. В~качестве переносчика могут использоваться электромагнитные колебания, имеющие гармоничную или~импульсную форму.

				Для того, чтобы заложить в~переносчики информацию, применяют операцию модуляции, которая заключается в~изменении одного или~нескольких параметров переносчика по~закону передаваемого сообщения. Например, в~гармоническом переносчике можно изменять амплитуду, частоту или~фазу колебаний. При этом возможны три вида модуляции: амплитудная, частотная или~фазовая.

				Устройство, осуществляющее изменение одного или~нескольких параметров переносчика, называется модулятором.

				Линия связи~(3)~— это среда, используемая для~передачи сигнала. Они могут быть проводными и~беспроводными. В~реальной системе связи сигнал передаётся при наличии помех, под которыми понимают любые случайные воздействия, накладывающиеся на~сигнал и~затрудняющие его приём. Поэтому сигнал~$S(t)$, в~общем случае отличается от~$S(t)$, который был на~выходе передающего устройства. 

				Совокупность технических средств передачи информации, включающая среду распространения, и~обеспечивающая передачу сигнала от~некоторой точки~$A$ до~точки~$B$~(рис 2.3) называют \emph{каналом}. Если сигнал, поступающий на~вход канала и~снимаемый на~его выходе, является дискретным, то~канал также называется дискретным.

				Любая телекоммуникационная система характеризуется рядом показателей, характеристиками канала. Наиболее существенными из~них с~точки зрения передачи информации такие:
				\begin{enumerate}
					\item Достоверность передачи информации~— степень соответствия принятых сообщений переданным. Она зависит от~параметров самой системы, степени её~технического совершенства и~условий работы.
					\item Помехоустойчивость~— способность системы противостоять вредному воздействию помех на~передачу сообщений. Количественно помехоустойчивость телекоммуникационных систем можно характеризовать вероятностью ошибок~$P_{\text{ош}}$ при~заданном отношении мощностей сигнала и~помехи в~полосе частот, занимаемой сигналом, или~требуемым отношением, при~котором обеспечивается заданная.
					\item Скорость передачи информации. \emph{Технической скоростью}~$V_{\text{т}}$ называется число элементарных сигналов~(символов), передаваемых по~каналу в~единицу времени. Она зависит от~свойств линии связи и~быстродействия аппаратуры. Единицей измерения технической скорости служит бод~(baud)~— скорость, при которой за~одну секунду передаётся один символ. \emph{Информационная скорость} (скорость передачи информации)~— среднее количество информации относительно заданного сообщения, которое передаётся по~каналу за~единицу времени. Для~практического применения телекоммуникационных систем важно выяснить, до~какого предела и~каким путём можно увеличить скорость передачи информации по~каналу.
					\item Предельные возможности канала по~передачи сообщений характеризуются пропускной способностью~— максимальная скорость передачи информации по~данному каналу, которую можно достигнуть при~самых совершенных способах передачи и~приёма. Пропускная способность передачи канала измеряется числом двоичных единиц информации в~секунду. Пропускная способность канала является характеристикой его самого и~не~зависит от~сигнала. 
				\end{enumerate}
				
				С~целью наилучшего соответствия характеристики сигнала и~канала связи обычно применяется канальное кодирование. Устройство, осуществляющее заданную операцию, называется \emph{кодером канала}. При~реализации современных систем телекоммуникации предпочтение отдают цифровым методам обработки и~передачи сигнала. Цифровые системы имеют ряд существенных преимуществ: представление сообщений в~цифровой форме обеспечивают более высокую помехоустойчивость, возможность более полного использование пропускной способности канала, стабильность параметров передачи и~гибкость при~построении телекоммуникационных систем.

\end{document}
