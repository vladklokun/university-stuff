\documentclass[
	a4paper,
	oneside,
	BCOR = 10mm,
	DIV = 12,
	12pt,
	headings = normal,
]{scrartcl}

%%% Length calculations
\usepackage{calc}
%%%

%%% Support for color
\usepackage{xcolor}
\definecolor{lightblue}{HTML}{03A9F4}
\definecolor{red}{HTML}{F44336}
%%%

%%% Including graphics
\usepackage{graphicx}
%%%

%%% Font selection
\usepackage{fontspec}

\setromanfont{STIX Two Text}[
	SmallCapsFeatures = {LetterSpace = 8},
]

\setsansfont{IBM Plex Sans}[
	Scale = MatchUppercase,
]

\setmonofont{IBM Plex Mono}[
	Scale = MatchUppercase,
]
%%%

%%% Math typesetting
\usepackage{amsmath}

\usepackage{unicode-math}
\setmathfont{STIX Two Math}
%%%

%%% List settings
\usepackage{enumitem}
\setlist[enumerate]{
	label*      = {\arabic*.},
	leftmargin  = *,
	labelindent = \parindent,
	topsep      = 1\baselineskip,
	parsep      = 0\baselineskip,
	itemsep     = 0\baselineskip,
}

% List type for enumerations with long items
\newlist{enumlong}{enumerate}{3}
\setlist[enumlong]{
	label*      = {\arabic*.},
	leftmargin  = *,
	labelindent = \parindent,
	topsep      = 1\baselineskip,
	parsep      = 0\baselineskip,
	itemsep     = 1\baselineskip,
}

\setlist[itemize]{
	label*      = {—},
	leftmargin  = *,
	labelindent = \parindent,
	topsep      = 1\baselineskip,
	parsep      = 0\baselineskip,
	itemsep     = 1\baselineskip,
}

\setlist[description]{
	font        = {\rmfamily\upshape\bfseries},
	topsep      = 1\baselineskip,
	parsep      = 0\baselineskip,
	itemsep     = 0\baselineskip,
}

%%%

%%% Structural elements typesetting
\setkomafont{pagenumber}{\rmfamily}
\setkomafont{disposition}{\rmfamily\bfseries}

% Sectioning
\RedeclareSectionCommand[
	beforeskip = -1\baselineskip,
	afterskip  = 1\baselineskip,
	font       = {\normalsize\bfseries\scshape},
]{section}

\RedeclareSectionCommand[
	beforeskip = -1\baselineskip,
	afterskip  = 1\baselineskip,
	font       = {\normalsize\bfseries\itshape},
]{subsection}

\RedeclareSectionCommand[
	beforeskip = -1\baselineskip,
	afterskip  = 1\baselineskip,
	font       = {\normalsize\bfseries},
]{subsubsection}

\RedeclareSectionCommand[
	beforeskip = -1\baselineskip,
	afterskip  = -1em,
	font       = {\normalsize\mdseries\scshape\addfontfeatures{Letters = {UppercaseSmallCaps}}},
]{paragraph}
%%%

%%% Typographic enhancements
\usepackage{microtype}
%%%

%%% Language-specific settings
\usepackage{polyglossia}
\setmainlanguage{russian}
\setotherlanguages{english}
%%%

%%% Captions
\usepackage{caption}
\usepackage{subcaption}

%\DeclareCaptionLabelFormat{closing}{#2)}
%\captionsetup[subtable]{labelformat = closing}

%\captionsetup[subfigure]{labelformat = closing}

\captionsetup[table]{
	aboveskip = 0\baselineskip,
	belowskip = 1\baselineskip,
}

\captionsetup[figure]{
	aboveskip = 1\baselineskip,
	belowskip = 0\baselineskip,
}

\captionsetup[subfigure]{
	labelformat = simple,
	labelformat = brace,
}
%%%

%%% Hyphenated ragged typesetting
\usepackage{ragged2e}
%%%

%%% Table typesetting
\usepackage{booktabs}
\usepackage{longtable}

\usepackage{multirow}

\usepackage{array}
\newcolumntype{v}[1]{>{\RaggedRight\arraybackslash\hspace{0pt}}p{#1}}
\newcolumntype{b}[1]{>{\Centering\arraybackslash\hspace{0pt}}p{#1}}
\newcolumntype{n}[1]{>{\RaggedLeft\arraybackslash\hspace{0pt}}p{#1}}
%%%

%%% Drawing
\usepackage{tikz}
\usepackage{tikzscale}
\usetikzlibrary{
	positioning,
	arrows.meta, % Stealth arrow tips
	decorations.pathreplacing, % % grouping brace
	fit, % fit
}

\tikzset{>=Stealth}

\tikzset{mynode/.style = {
		font = \strut,
		draw,
		rectangle,
		fill = white,
		minimum height = 4\baselineskip,
		minimum width  = 2\gridunitwidth,
		text width = 3\gridunitwidth,
		align = center,
	},
}

\tikzset{caption/.style = {
		font = \scriptsize,
	},
}
%%%

%%% Date and time typesetting
\usepackage[
	ukrainian,
	useregional,
	showdow,
]{datetime2}
%%%

%%% Links and hyperreferences
\usepackage{hyperref}
\hypersetup{
	bookmarksnumbered = true,
	colorlinks      = false,
	linkbordercolor = red,
	urlbordercolor  = lightblue,
	pdfborderstyle  = {/S/U/W 1.5},
}
%%%

%%% Length adjustments
% Set baselineskip, default is 14.5 pt
\linespread{1.068966} % ~15.5 pt
\setlength{\emergencystretch}{1em}
\setlength{\parindent}{1.5em}
\newlength{\gridunitwidth}
\setlength{\gridunitwidth}{\textwidth / 12}
\newlength{\gridunitheight}
\setlength{\gridunitheight}{\textheight / 12}
%%%

%%% Custom commands
\newcommand{\allcaps}[1]{{\addfontfeatures{LetterSpace = 8, Kerning = Off}#1}}

\newcommand{\mynote}[1]{(\textit{Прим.}~#1)}
%%%

\begin{document}

\begin{titlepage}
		\vspace*{4\gridunitheight}
		% \begin{center}
			\noindent\textsc{Конспект}\\[\baselineskip]
			{\setlength{\tabcolsep}{0em}
			\begin{tabular}{
					v{3\gridunitwidth - 2\tabcolsep}
					v{6\gridunitwidth - 2\tabcolsep}
			}
				з дисципліни: & «Телекомунікаційні~технології комп'ютерних~мереж»\\
				викладач:     & Пушкін Юрій Олександрович\\
			\end{tabular}
			}
		% \end{center}
		\vspace*{\fill}
	\end{titlepage}

	\tableofcontents
	\newpage

	\section{\DTMDate{2018-09-14}}
		\subsection{Информация, сообщение, сигнал}
			Понятие «информация» имеет много различных аспектов, и~в связи с~этим существует несколько различных подходов к~её~определению. \emph{Информация}~— совокупность сведений о~каком-то событии, явлении, предмете, являющихся объектом хранения, передачи и~преобразования. Для выполнения указанных действий используют условные символы~— буквы, жесты, математические знаки, позволяющие выразить информацию в~необходимой форме. Совокупность знаков, которые используют для~хранения, передачи и~обработки, называют \emph{сообщением}. 
			
			В~различных технических системах информация представляется в~двоичной форме. Соответственно, сообщением может служить последовательность конечного числа двоичных символов.

			Различают дискретные и~непрерывные сообщения. Дискретные сообщения формируются в~результате последовательной выдачи источником сообщения отдельных знаков.

			Множество различных знаков называют \emph{алфавитом источника сообщений}, а~их~количество~— объёмом алфавита. Непрерывные сообщения не~разделены на~элементы, они описываются функциями времени, принимающими непрерывное множество значений. Пример~— телевизионное изображение.

			Передача сообщений на~расстояние осуществляется с~помощью какого-либо материального носителя или~физического процесса (волны, ток, колебания и~т.\,д.). Физический процесс, посредством которого сообщение передаётся на~расстояние, называется \emph{сигналом}. В~современных системах управления чаще всего используются электрические сигналы.

			Процесс изменения параметров носителя принято называть модуляцией. Различают такие виды сигналов~(рис.~2.1 \mynote{Лектор сказал искать рисунки в~тексте лаб.}):
			\begin{enumerate}
				\item Непрерывные по~уровню и~по~времени~(2.1а).
				\item Непрерывные по~уровню и~дискретные по~времени~(2.1б).
				\item Дискретные по~уровню и~непрерывные по~времени~(2.1в).
				\item Дискретные по~уровню и~по~времени~(2.1г).
			\end{enumerate}

			Сигналы первого вида, называемые непрерывными, задаются на~конечном или~бесконечном временном интервале и~могут принимать любые значения в~некотором диапазоне. 

			Сигналы второго вида задаются в~определённый дискретный момент времени и~могут принимать любые значения из~некоторого диапазона. Их можно получить из~непрерывных сигналов путём взятия отсчёта в~определённый момент времени. Это преобразование называется дискретизацией по~времени.

			Сигналы третьего вида, называемые квантоваными по~времени, задаются на~некотором временном интервале и~характеризуются тем, что принимают вполне только определённые дискретные значения. Их~можно получить из~непрерывных сигналов, применяя к~ним операцию квантования по~уровню.

			Сигналы четвёртого вида также называются дискретными, задаются в~определённые дискретные моменты и~принимают определённые значения. Их~можно получить из~непрерывных сигналов, осуществляя операцию дискретизации по~времени и~квантования по~уровню.

		\subsection{Обобщённая структурная схема системы связи}
			Системой связи называют совокупность технических средств, предназначенных для~передачи информации от~передатчика сообщений и~получателя информации.

			Структурная схема простейшей системы связи показана на~рис.~2.2. Источником сообщения~(1) может быть человек или~различного рода устройство, он~осуществляет выбор сообщений из~ансамбля сообщений. Если сообщение на~выходе источника имеет не~электрическую природу, то~для его передачи в~системе связи оно преобразуется в~первичный электрический сигнал.

			Первичные сигналы являются низкочастотными. Для~передачи на~большие расстояния используются специальные электромагнитные колебания высокой частоты, называемые переносчиками, которые могут эффективно распространяться по~линиям связи. 

			В передающем устройстве~(2) первичный сигнал превращается во~вторичный (высокочастотный) сигнал~$S(t)$. В~качестве переносчика могут использоваться электромагнитные колебания, имеющие гармоничную или~импульсную форму.

			Для того, чтобы заложить в~переносчики информацию, применяют операцию модуляции, которая заключается в~изменении одного или~нескольких параметров переносчика по~закону передаваемого сообщения. Например, в~гармоническом переносчике можно изменять амплитуду, частоту или~фазу колебаний. При этом возможны три вида модуляции: амплитудная, частотная или~фазовая.

			Устройство, осуществляющее изменение одного или~нескольких параметров переносчика, называется модулятором.

			Линия связи~(3)~— это среда, используемая для~передачи сигнала. Они могут быть проводными и~беспроводными. В~реальной системе связи сигнал передаётся при наличии помех, под которыми понимают любые случайные воздействия, накладывающиеся на~сигнал и~затрудняющие его приём. Поэтому сигнал~$S(t)$, в~общем случае отличается от~$S(t)$, который был на~выходе передающего устройства. 

			Совокупность технических средств передачи информации, включающая среду распространения, и~обеспечивающая передачу сигнала от~некоторой точки~$A$ до~точки~$B$~(рис 2.3) называют \emph{каналом}. Если сигнал, поступающий на~вход канала и~снимаемый на~его выходе, является дискретным, то~канал также называется дискретным.

			Любая телекоммуникационная система характеризуется рядом показателей, характеристиками канала. Наиболее существенными из~них с~точки зрения передачи информации такие:
			\begin{enumlong}
				\item Достоверность передачи информации~— степень соответствия принятых сообщений переданным. Она зависит от~параметров самой системы, степени её~технического совершенства и~условий работы.
				\item Помехоустойчивость~— способность системы противостоять вредному воздействию помех на~передачу сообщений. Количественно помехоустойчивость телекоммуникационных систем можно характеризовать вероятностью ошибок~$P_{\text{ош}}$ при~заданном отношении мощностей сигнала и~помехи в~полосе частот, занимаемой сигналом, или~требуемым отношением, при~котором обеспечивается заданная.
				\item Скорость передачи информации. \emph{Технической скоростью}~$V_{\text{т}}$ называется число элементарных сигналов~(символов), передаваемых по~каналу в~единицу времени. Она зависит от~свойств линии связи и~быстродействия аппаратуры. Единицей измерения технической скорости служит бод~(baud)~— скорость, при которой за~одну секунду передаётся один символ. \emph{Информационная скорость} (скорость передачи информации)~— среднее количество информации относительно заданного сообщения, которое передаётся по~каналу за~единицу времени. Для~практического применения телекоммуникационных систем важно выяснить, до~какого предела и~каким путём можно увеличить скорость передачи информации по~каналу.
				\item Предельные возможности канала по~передачи сообщений характеризуются пропускной способностью~— максимальная скорость передачи информации по~данному каналу, которую можно достигнуть при~самых совершенных способах передачи и~приёма. Пропускная способность передачи канала измеряется числом двоичных единиц информации в~секунду. Пропускная способность канала является характеристикой его самого и~не~зависит от~сигнала. 
			\end{enumlong}

			С~целью наилучшего соответствия характеристики сигнала и~канала связи обычно применяется канальное кодирование. Устройство, осуществляющее заданную операцию, называется \emph{кодером канала}. При~реализации современных систем телекоммуникации предпочтение отдают цифровым методам обработки и~передачи сигнала. Цифровые системы имеют ряд существенных преимуществ: представление сообщений в~цифровой форме обеспечивают более высокую помехоустойчивость, возможность более полного использование пропускной способности канала, стабильность параметров передачи и~гибкость при~построении телекоммуникационных систем.

	\section{\DTMDate{2018-09-20}}
		\subsection{Телекоммуникационные каналы}
			В~настоящее время введено понятие телекоммуникационного канала (канала связи) как совокупности технических устройств и~линий связи, необходимых для~передачи сигналов между пунктами связи. Линию связи, представляющую собой физическую среду, в~которой распространяется сигнал, называют \emph{физическим каналом}. Классификация физических каналов приведена на~рис.~4.1.

			\subsubsection{Кабели на~витой паре}
				Витая пара представляет собой два изолированных медных провода, скрученных вместе. Скрутка проводов позволяет уменьшить индуктивность проводов, приводящую к~ограничению технической скорости. Кроме того, скрутка способствует уменьшению электрических помех, наводимых соседними парами, а~также внешними источниками. Кабели на~витой паре характеризуются исключительной простотой. К~недостаткам такого кабеля относятся:
			\begin{enumerate}
				\item Низкий уровень защищённости от~электрических помех.
				\item Сравнительно большой уровень собственных излучений, способствующих затуханию сигнала.
				\item Возможности прослушивания передаваемого сигнала.
			\end{enumerate}
			Для уменьшения данных недостатков на практике используется экранированный кабель~(\allcaps{STP}). Провода в~кабеле имеют определённый цвет изоляции.

			Телекоммуникационные кабели могут содержать от~2000 до~3000~витых пар, число пар в~кабелях внутренней проводки~— примерно~200. Входное полное сопротивление~(импеданс) витой пары можно смоделировать двумя последовательно включёнными сопротивлениями, параллельно одному из~которых включена ёмкость~(рис.~4.2). На~рис.~4.2а представлена эквивалентная схема замещения полного входного сопротивления телефонной линии для~диапазона звуковых частот. На~рис~4.2б показана высокочастотная эквивалентная схема для~частот выше звукового диапазона. Затухание сигнала в~кабеле на~витой паре сильно зависит от~частоты передаваемого сигнала. Затухание возрастает пропорционально частоте, но~в~определённых частотных полосах затухание остаётся практически постоянным~(рис.~4.3).

			Характеристики кабелей на~витой паре регламентируются в~стандарте \allcaps{EIA/TIA}~568. Этот стандарт включает 5~категорий неэкранированных кабелей на~витой паре:
			\begin{description}
				\item[Категория~1] Традиционный телефонный кабель, по~которому можно передавать только речь. 
				\item[Категория 2] Кабель, предназначенный для~передачи данных со~скоростью 4~Мбит/с. Состоит из~4~витых пар.
				\item[Категория~3] Кабель, предназначенный для~передачи данных со~скоростью 10~Мбит/с.
				\item[Категория~4] Кабель, предназначенный для~передачи данных со~скоростью 16~Мбит/с.
				\item[Категория~5] Кабель, предназначенный для~передачи данных со~скоростью 100~Мбит/с.
			\end{description}
			Максимальная длина сегмента кабеля~\allcaps{UTP}~— 100~м.

			Экранированный кабель на~витой паре применяется в~локальных вычислительных сетях на~основе \textenglish{Token Ring}, компьютеры соединяются по~кольцу. Электрические параметры кабеля тип~1 примерно соответствуют параметрам кабеля категории~5. 

			\subsubsection{Коаксиальные кабели}
				Этот кабель состоит из~центрального проводника, изолированного твёрдым нейлоном или~полиэтиленом~(рис.~4.4). Изолирующий слой затем покрывается проволочной оплёткой или~фольгой. Существуют кабели, в~которых присутствуют и~оплётка, и~фольга, которые выполняют функции экрана, защищающего центральный проводник от~внешних помех. Поверх этого изолятор покрывают поливинилхлоридом. Стоимость коаксиального кабеля в~несколько раз выше кабеля на~витой паре, и~монтировать его гораздо сложнее. Наличие экрана существенно увеличивает помехозащищённость и~снижает собственное излучение. Несанкционированное подключение к~такому кабелю сложнее, чем к~кабелю на~витой паре. Пропускная способность кабеля в~режиме модуляции высокочастотного сигнала может достигать 500~Мбит/с. Допустимая длина сегмента~— несколько километров. Входное полное сопротивление коаксиального кабеля обычно равно 50~Ом. Кабель для~телевизионных фидеров имеет сопротивление 75~Ом.

				В~локальных вычислительных сетях, реализованных на~основе технологии \textenglish{Ethernet}, получили распространение два типа коаксиальных кабелей, которые получили название «тонкий» и~«толстый» \textenglish{Ethernet}.

			\subsubsection{Волоконно-оптические кабели}
				Волоконно-оптические кабели предназначены на~расстояния оптических сигналов. Основным элементом волоконно-оптических кабелей являются световоды~— тонкие нити из~высокопрочных материалов. Оптическим сигналом служит модулированное оптическое излучение лазера или~светодиода.

				Для~описания процесса распространения оптических волн пользуются волновым и~лучевым методом. Первый метод основан на~решении уравнений и~позволяет получить точное решение электродинамической задачи. На~практике широкое распространение получили лучевые методы~(методы геометрической оптики).

				В~геометрической оптике световые волны изображаются световыми лучами, которые в~однородной среде распространяются прямолинейно. При~попадании на~границу раздела двух сред с~разными значения показателей преломления, световой луч изменяет своё направление. И~в~общем случае появляются преломленный и~отражённый лучи. Углы, которые образуют падающий, отражённый и~преломлённый лучи с~нормалью границы раздела сред, восстановленный в~точке падения, называют соответственно углами падения, отражения и~преломления.

				Процесс распространения световых в~оптически более плотной среде, окружённый менее плотной, показан на~рис.~4.5. Траектория, показанная сплошной линией, соответствует световому лучу, который падает на~границу раздела сред, отражается от~неё и~возвращается в~область более плотной среды, где~распространяется зигзагообразно. Такие лучи называются направляемыми, и~их~траектории полностью расположены внутри среды распространения.

				Траектория, показанная штриховой линией, соответствует лучу, который падает на~границу раздела сред под~определённым углом, и~испытывает не~только отражение, но~и~преломление проникая в~менее плотную среду. Такие лучи называются лучами излучений. 

				Оптические волокна, у~которых показатель преломления на~границе раздела сердцевины и~оболочки изменяется скачком, называются ступенчатыми. Волокна, у~которых показатель преломление изменяется плавно, называются градиентными.

				Важнейшим параметром оптических волокон являются потери, приводящие к~ослаблению сигнала. Ещё одним важным параметром является полоса частот. Она определяет объём информации, который можно передавать по~оптическому кабелю. Ограничение частоты, применительно к~цифровым системам передачи, обусловлено тем, что импульс на~приёме приходит размытым, искажённым в~следствии различия скоростей распространения отдельных его составляющих в~световоде. Данное явление носит название дисперсии. Сравнивая дисперсионные характеристики различных световодов, можно отметить, что лучшими данными обладают одномодовые световоды. Хорошие показатели также у~градиентных световодов. Явление дисперсии также приводит к~ограничению пропускной способности и~дальности передачи. 

			\subsubsection{Радиоканалы}
				Полоса частот, используемых в~радиосвязи составляет от~3~Гц до~3000~ГГц, что соответствует длинам волны от~$10^{8}$~м до~$10^{-6}$~м. Классификация частот и~соответствующих им~длин волн приведена в~табл.~4.2.

				По~способу распространения радиоволн различают каналы с~открытым и~закрытым распространением. В~каналах с~закрытым распространением электромагнитная энергия распространяется по~направляющим линиям, для~них характерны малый уровень помех и~постоянство параметров сигнала, что позволяет передавать информацию с~высокой достоверностью и~скоростью.

				В~диапазонах от~крайне низких до~низких частот на~небольших расстояниях поле в~месте приёма создаётся за~счёт дифракционного огибания выпуклой поверхности Земли. На~больших расстояниях радиоволны распространяются в~своеобразном сферическом волноводе, внутренняя стенка которого образуется поверхностью Земли, а~внешняя~— ионосферой.
				
				В~распространении волн высоких частот принимает участие даже ионосфера, однако, если волны длиннее 1~км, они отражаются от~нижнего её~слоя практически зеркально, то~дециметровые волны проникают в~неё практически полностью, что~приводит к~эффекту многолучёвости. Дециметровые волны применяются для~глобальной связи и~радиовещания. С~их помощью можно передать информацию сравнительно большого объёма в~пределах земного шара со~сравнительно небольшой мощностью передатчика.

				Гектаметровые волны днём распространяются как земные, а~ночью~— как ионосферные. Дальность распространения не~превышает 500~км (над~морем~— 1000~км). Волны частотой от~30~ГГц и~выше распространяются в~пределах прямой видимости.

				Новую эру в~освоении высокочастотной области радиодиапазона в~радиосвязи открыл запуск искусственных спутников Земли. Линия спутниковой связи состоит из~оконечных земных станций и~одного или~нескольких спутников-ретрансляторов, вращающихся вокруг Земли. 

				В~системах, в~которых используются открытые каналы, передача оптических сигналов передаётся непосредственно через атмосферу. Могут использоваться при~объединении различных локальных вычислительных сетей, системах безопасности и~т.\,д.

				Выпускаемая в~настоящее время аппаратура позволяет обеспечить оптических сигналов на~расстояние до~3~км. Однако в~зависимости от~погодных условий это расстояние может уменьшаться. Оптическая система, устанавливаемая на~каждом конце линии связи, состоят из~двух станций, содержащих излучающий лазерный блок, принимающий оптический блок и~дополнительное электронное оборудование.

				Излучение лазера модулируется передаваемым цифровым потоком управляющего интерфейса. После прохождения излучения через атмосферу линзовая оптическая система на~приёмнике противоположной станции фокусирует его на~фотоприёмник. Электронные системы станций усиливают принятый сигнал и~осуществляют синхронизацию исходного цифрового потока.

				Оптическая линия связи хорошо защищена от~несанкционированного доступа к~данным, поскольку сигнал невидим и~хорошо сфокусирован, попытка доступа к~данным невозможна без~нарушения связи.

				Для надёжной работы во~всепогодных условиях оптический приёмопередающий блок снабжают системой антиобледенения, исключающей образование конденсата, влаги и~ледяной корки. 

				В~качестве примера в~таблице~4.3 приведены характеристики оптической атмосферной системы связи. 

		\section{\DTMDate{2018-09-28}}
			\subsection{Спектральный анализ сигналов на~линиях связи}
				Известно, что любой периодический процесс можно представить в~виде суммы синусоидальных колебаний различных частот и~различных амплитуд. Каждая составляющая синусоида называется также гармоникой, а~набор всех гармоник называют спектральным разложением исходного сигнала. 

				Техника нахождения спектра любого исходного сигнала хорошо известна: для~некоторых сигналов, которые хорошо описываются аналитически, спектр легко вычисляются из~формул~Фурье. Для сигналов произвольной формы, встречающихся на~практике, спектр можно найти с~помощью специальных приборов~— спектральных анализаторов,~— которые отображают спектр сигнала на~экране.

				Искажение передающим каналом синусоиды какой-либо частоты приводит к~искажениям передаваемого сигнала. При~передаче импульсных сигналов, характерных для~компьютерных сетей, искажаются высокочастотные и~низкочастотные гармоники. В~результате фронты импульсов теряют свою прямоугольную форму~(рис.~3.5), в~следствии этого на~приёмном конце линии сигналы могут плохо распознаваться.

				Линия связи искажает передаваемые сигналы из-за того, что её~физические параметры отличаются от~идеальных. Так, например, медные провода всегда представляют собой некоторую распределённую по~длине комбинацию активного сопротивления и~ёмкостной и~индуктивной нагрузки~(рис.~3.6).

				В~результате для~синусоид различных частот линия будет обладать различным полным сопротивлением, а~значит и~передаваться они будут по-разному.

				Волоконно-оптический кабель также имеет отклонения, мешающие идеальному распространению света. Кроме искажений сигналов, вносимых внутренними физическими параметрами линии, существуют и~внешние помехи, которые вносят свой вклад в~искажение формы сигналов на~выходе линии.

			\subsection{Амплитудно-частотная характеристика, полоса пропускания и~затухание}
				Степень искажения синусоидальных сигналов линиями связи оценивается с~помощью таких характеристик как амплитудно-частотная характеристика, полоса пропускания и~затухание на~определённой частоте.

				Амплитудно-частотная характеристика~(рис.~3.7) показывает, как затухает амплитуда синусоиды на~выходе линии связи по~сравнению с~амплитудой на~её входе для~всех возможных частот передаваемого сигнала. Часто вместо амплитуды в~этой характеристике используют мощность сигнала. Знание амплитудно-частотной характеристики реальной линии позволяет определить форму выходного сигнала практически любого входного сигнала.

				На~практике вместо амплитудно-частотной характеристики применяются и~другие упрощённые характеристики~— полоса пропускания и~затухание. Полоса пропускания~— это непрерывный диапазон частот, для~которого отношение амплитуды выходного сигнала ко~входному превышает некоторый заранее заданный предел~(обычно это~$0{,}5$). То есть полоса пропускания определяет диапазон частот синусоидального сигнала, при~котором этот сигнал передаётся по~линии связи без~значительных искажений.

				Затухание определяется как относительное уменьшение амплитуды или~мощности сигнала при~передачи по~линии сигнала определённой частоты. Затухание обычно измеряется в~децибелах и~вычисляется так:
				\[
					A = 10 \log_{10} \left( \frac{P_{\text{вых}}}{P_{\text{вх}}} \right).
				\]

				Пропускная способность линии характеризует максимально возможную скорость передачи данных по~линии связи. Пропускная способность измеряется в~битах в~секунду. Пропускная способность зависит не~только от~характеристик линии, но~и~от~спектра передаваемых сигналов. Если значимые гармоники попадают в~полосу пропускания линии, то~такой сигнал будет хорошо передаваться по~линии связи, и~приёмник сможет правильно определить информацию.

				Выбор способа представления дискретной информации в~виде сигналов, подаваемых в~линии связи, называется физическим или~линейным кодированием. От~выбранного способа кодирования зависит спектр сигналов и~пропускная способность линии. Таким образом, для~одного способа кодирования линии может обладать одной пропускной способностью, а~для~другого~— другой. 

				Большинство способов кодирования используют изменение какого-либо параметра периодического сигнала (частоты, амплитуды, фазы синусоиды) или~же знак потенциала последовательности импульсов.

				Периодический сигнал, параметры которого изменяются, называется несущим сигналом или~несущей частотой, если в~качестве сигнала используется синусоида. Если сигнал изменяется так, что можно различить только два его состояния, то~любое его изменение будет соответствовать наименьшей единице информации~— биту. Если же сигнал может иметь более двух различных состояний, любое его изменение будет нести несколько бит информации.

				Количество изменений информационного параметра несущего периодического сигнала в~секунду измеряется в~бодах. Период времени между соседними изменениями информационного сигнала называется тактом работы передатчика.

				На~пропускную способность линии оказывает влияние не~только физическое, но~и~логическое кодирование. Логическое кодирование выполняется до~физического кодирования и~подразумевает замену бит исходной информации новой последовательностью бит, несущей ту же информацию, но~обладающей кроме этого дополнительными свойствами.

				При~логическом кодировании чаще всего исходная последовательность бит заменяется более длинной последовательностью, поэтому пропускная способность канала по~отношению к~полезной информации при~этом уменьшается.

			\subsection{Помехоустойчивость и~достоверность}
				Помехоустойчивость линии определяет её~способность уменьшать уровень помех, создаваемых во~внешней среде. Помехоустойчивость линии зависит от~типа используемой физической среды, а~также от~экранирующих, подавляющих помех, средств самой линии.

				Перекрёстные наводки на~ближнем конце (\textenglish{\allcaps{NEXT}}) определяют помехоустойчивость кабеля внутренним источникам помех, когда электромагнитное поле сигнала, передаваемое выходом передатчика по~одной паре проводников, наводит на~другую пару проводников сигнал помехи. Если ко~второй паре будет подключен приёмник, он~может принять помеху. Чем меньше значение \textenglish{\allcaps{NEXT}}, тем лучше кабель. Показатель \textenglish{\allcaps{NEXT}} обычно используется применительно к~кабелю с~несколькими витыми парами.

		\section{\DTMDate{2018-10-04}}
			\subsection{Методы передачи данных}
				При~передаче дискретных данных по~каналам связи применяются 2~основных типа кодировки: на~основе синусоидального несущего сигнала и~на~основе последовательности прямоугольных импульсов. Первый способ также называется модуляцией или~аналоговой модуляцией, подчёркивая тот факт, что~кодировка осуществляется за~счёт изменения параметров аналогового сигнала. Второй способ называется цифровой кодировкой. Эти способы отличаются шириной спектра результирующего сигнала и~сложностью аппаратуры для~их~реализации.

				В~настоящее время данные чаще всего имеют аналоговую форму, а~передаются по~каналам связи в~дискретном виде. Процесс представления аналоговой информации в~дискретной форме называется дискретной модуляцией. Термины «модуляция» и~«кодирование» часто используют как~синонимы. 

				\subsubsection{Аналоговая модуляция}
					Аналоговая модуляция применяется для~передачи дискретных данных по~каналам с~узкой полосой частот. Типичная амплитудно-частотная характеристика канала тональной частоты представлена на~рис.~5.1. Этот канал передаёт частоты в~диапазоне от~300 до~3400~Гц. Таким образом, его полоса пропускания равна~3100~Гц. Строгое ограничение полосы пропускания тонального канала связано с~использованием аппаратуры уплотнения и~коммутации канала в~телефонных сетях.

					Устройство, которое выполняет функции модуляции несущей синусоиды передающей стороны и~демодуляции на~приёмной стороне, носит название «модем» (модулятор-демодулятор).

				\subsubsection{Методы аналоговой модуляции}
					Аналоговая модуляция является таким способом физического кодирования, при~котором информация кодируется изменением амплитуды, частоты или~фазы синусоидального сигнала несущей частоты. Основные способы аналоговой модуляции показаны на~рис.~5.2. На~диаграмме~(рис.~5.2а) показана последовательность бит исходной информации, представленная потенциалами высокого уровня, что~соответствует логической единице, и~потенциалам нулевого уровня для~логического нуля. Такой способ кодирования называется потенциальным кодом.
					
					При~амплитудной модуляции~(рис.~5.2б) для~логической единицы выбирается один уровень амплитуды, а~для~логического нуля~— другой. При~частотной модуляции~(рис.~5.2в) значение нуля и~единицы передаются синусоидами с~различной частотой. При~фазовой модуляции~(рис.~5.2г), значениям нуля и~единицы соответствуют сигналы одинаковой частоты, но~с~различной фазой.

				\subsubsection{Спектр модулированного сигнала}
					Спектр результирующего модулированного сигнала зависит от~типа модуляции и~скорости модуляции (то~есть желаемой скорости передачи). Рассмотрим спектр сигнала при~потенциальной кодировке. Пусть логическая единица кодируется положительным потенциалом, а~логический ноль~— отрицательным той~же величины. Для~упрощения предположим, что~передаётся информация, состоящая из~бесконечной последовательности чередующихся нулей и~единиц, как это показано на~рис.~5.2а. Для~потенциальной кодировки спектр получается из~формулы~Фурье для~периодической функции. Если дискретные данные передаются с~битовой скоростью~$M$~бит/с, то~спектр состоит из~постоянной составляющей нулевой частоты и~бесконечного ряда гармоник с~частотами~$f_0, 3f_0, 5f_0, 7f_0, \dots$,~где~$f_0 = M / 2$. Амплитуды этих гармоник убывают достаточно медленно с~коэффициентом~$1/3, 1/5, 1/7, \dots$ от~амплитуды гармоники~$f_0$~(рис.~5.3). В~результате следует, что~спектр потенциального кода требует для~качественной передачи широкую полосу пропускания.

					При амплитудной модуляции спектр состоит из~синусоиды несущей частоты~$f_c$ и~двух боковых гармоник~($f_c + f_m$, $f_c - f_m$), где~$f_m$~— частота изменения информационного параметра синусоиды, которая совпадает со~скоростью передачи данных при~использование двух уровней амплитуды~(рис.~5.3б). 

					При фазовой и~частотной модуляции спектр сигнала получается более сложным, так как боковых гармоник здесь образуется более двух, но~они симметрично расположены относительно основной несущей частоты, а~их~амплитуды быстро убывают.

				\subsubsection{Цифровое кодирование}
					При~цифровом кодировании дискретной информации применяются потенциальные и~импульсные коды. В~потенциальных кодах для~представления логических единиц и~нулей используется только значение потенциала сигнала, а~его перепады, формирующие законченные импульсы во~времени, не~принимаются. 

					Импульсные коды позволяют представить двоичные данные либо импульсами определённой полярности, либо частью импульса~— перепадом потенциала определённого направления. 

				\subsubsection{Требования к~методам цифрового кодирования}
					При~использовании прямоугольных импульсов необходимо выбрать такой способ кодировки, который достигал~бы несколько целей:
					\begin{enumerate}
						\item Имел при~одной и~той~же битовой скорости наименьшую ширину спектра.
						\item Обеспечивал синхронизацию между передатчиком и~приёмником.
						\item Обладал способностью распознавать ошибки.
						\item Обладал низкой стоимостью реализации.
					\end{enumerate}

					Более узкий спектр сигнала позволяет на~одной и~той~же линии связи добиваться более высокой скорости передачи данных. Кроме того, часто к~спектру сигнала предъявляются требования отсутствия постоянной составляющей, то~есть наличие постоянного тока между передатчиком и~приёмником. В~частности, применяются различные трансформаторные схемы гальванической разрядки, которые препятствуют прохождению тока.

					Синхронизация передатчика и~приёмника нужна для~того, чтобы приёмник точно знал, в~какой момент времени необходимо считывать информацию с~линии связи. На~небольших расстояниях хорошо работает схема, основанная на~отдельной тактирующей линии связи~(рис.~5.4), так что информация снимается с~линии связи только в~момент прихода такого импульса.
					
					В~сетях использование этой схемы вызывает трудности из-за неоднородности проводников в~кабелях. На~больших расстояниях неравномерность скорости распространения сигнала может привести к~тому, что~тактовый импульс придёт несколько позже или~раньше данных. Другой причиной, по~которой в~сетях отказываются от~использования тактирующих импульсов является экономия дорогостоящих проводников.

					Поэтому в~сетях применяются так называемые самосинхронизирующиеся коды, сигналы которых несут для приёмника указание о~том, в~какой момент времени надо осуществлять распознавание очередного бита. Любой резкий перепад сигнала (так называемый «фронт») может служить хорошим указанием для~синхронизации приёмника с~передатчиком. 

					При~использовании синусоид в~качестве несущего сигнала результирующий код обладает свойством самосинхронизации, так как изменения амплитуды несущей частоты даёт возможность приёмнику определить момент появления входного кода. 

					Требования, предъявляемые к~методам кодирования, являются взаимно противоречивыми, поэтому каждый из~них обладает своими преимуществами и~своими недостатками.

				\subsubsection{Потенциальный код без~возвращения к~нулю}
					На~рис.~5.5а показан упомянутый ранее метод потенциального кодирования, называемый также «кодирование без~возвращения к~нулю» (\textenglish{No Return to Zero, \allcaps{NRZ}}). Название отражает то~обстоятельство, что~при~передаче последовательности единиц сигнал не~возвращается к~нулю в~течение такта, как в~других методах кодирования. Метод \textenglish{\allcaps{NRZ}} прост в~реализации, обладает хорошей распознаваемостью ошибок, но не~обладает свойством самосинхронизации. При передачи длинной последовательности единиц или~нулей сигнал на~линии не~изменяется, поэтому приёмник лишён возможности определять по~входному сигналу моменты времени, когда нужно в~очередной раз считывать данные.

					Другим серьёзным недостатком метода \textenglish{\allcaps{NRZ}} является наличие низкочастотной составляющей, которая приближается к~нулю при~передаче длинных последовательностей единиц и~нулей. В~результате в~чистом виде код~\textenglish{\allcaps{NRZ}} в~сетях не~используется.

				\subsubsection{Метод биполярного кодирования с~интерактивной инверсией}
					Одной из~модификаций метода~\textenglish{\allcaps{NRZ}} является метод биполярного кодирования с~альтернативной инверсией~(\textenglish{\allcaps{AMI}}). В~этом методе~(рис.~5.5б) используется 3~уровня потенциалов: отрицательный, нулевой и~положительный. Для~кодирования логического нуля используется нулевой потенциал, а~логическая единица кодируется либо положительным, либо отрицательным потенциалом, но при~этом потенциал каждой новой единицы противоположен предыдущей по~знаку. 

					Код~\textenglish{\allcaps{AMI}} представляет также некоторые возможности по~распознаванию ошибочных сигналов: нарушение строгого чередования полярности сигналов говорит о~ложности импульса или~исчезновения с~линии корректного импульса. 

				\subsubsection{Биполярный импульсный код}
					Кроме потенциальных кодов в~сетях используют и~импульсные коды, когда данные представлены полным импульсов или~же его частью (фронтом). Простым случаем такого биполярного кода является биполярный импульсный код, в~котором единица представлена импульсом одной полярности, а~ноль~— другой~(рис.~5.5в). Каждый импульс длится половину такта. Такой код обладает отличными самосинхронизирующими свойствами, но~может присутствовать постоянная составляющая.

				\subsubsection{Манчестерский код}
					В~локальных сетях до~недавнего времени был самым распространённым методом кодирования~(рис.~5.5г). Он применялся в~технологиях \textenglish{Ethernet} и~\textenglish{Token Ring}. В~Манчестерском коде для~кодирования единиц и~нулей используется перепад потенциала (фронт импульса). При~таком кодировании каждый такт делится на~две части: информация кодируется перепадом потенциалов для~единицы~— низкого уровня к~высокому, ноль~— обратным перепадом от~высокого к~низкому. Так как сигнал изменяется по~крайней мере один раз за~такт, передача одного бита данных, то~Манчестерский код обладает хорошими самосинхронизирующими свойствами. Полоса пропускания Манчестерского кода хуже, чем у~биполярного импульсного. Манчестерский код имеет ещё одно преимущество перед биполярным кодированием: в~последнем для~передачи сигнала используются 3~уровня сигнала, а в~Манчестерском~— 2.

\end{document}
