\documentclass[
	a4paper,
	oneside,
	BCOR = 10mm,
	DIV = 12,
	12pt,
	headings = normal,
]{scrartcl}

%%% Length calculations
\usepackage{calc}
%%%

%%% Support for color
\usepackage{xcolor}
\definecolor{lightblue}{HTML}{03A9F4}
\definecolor{red}{HTML}{F44336}
%%%

%%% Including graphics
\usepackage{graphicx}
%%%

%%% Font selection
\usepackage{fontspec}

\setromanfont{STIX Two Text}[
	SmallCapsFeatures = {LetterSpace = 8},
]

\setsansfont{IBM Plex Sans}[
	Scale = MatchUppercase,
]

\setmonofont{IBM Plex Mono}[
	Scale = MatchUppercase,
]
%%%

%%% Math typesetting
\usepackage{amsmath}

\usepackage{unicode-math}
\setmathfont{STIX Two Math}
%%%

%%% List settings
\usepackage{enumitem}
\setlist[enumerate]{
	label*      = {\arabic*.},
	leftmargin  = *,
	labelindent = \parindent,
	topsep      = 1\baselineskip,
	parsep      = 0\baselineskip,
	itemsep     = 1\baselineskip,
}

\setlist[itemize]{
	label*      = {—},
	leftmargin  = *,
	labelindent = \parindent,
	topsep      = 1\baselineskip,
	parsep      = 0\baselineskip,
	itemsep     = 1\baselineskip,
}

\setlist[description]{
	font        = {\rmfamily\upshape\bfseries},
	topsep      = 1\baselineskip,
	parsep      = 0\baselineskip,
	itemsep     = 0\baselineskip,
}

%%%

%%% Structural elements typesetting
\setkomafont{pagenumber}{\rmfamily}
\setkomafont{disposition}{\rmfamily\bfseries}

% Sectioning
\RedeclareSectionCommand[
	beforeskip = -1\baselineskip,
	afterskip  = 1\baselineskip,
	font       = {\normalsize\bfseries\scshape},
]{section}

\RedeclareSectionCommand[
	beforeskip = -1\baselineskip,
	afterskip  = 1\baselineskip,
	font       = {\normalsize\bfseries\itshape},
]{subsection}

\RedeclareSectionCommand[
	beforeskip = -1\baselineskip,
	afterskip  = 1\baselineskip,
	font       = {\normalsize\bfseries},
]{subsubsection}

\RedeclareSectionCommand[
	beforeskip = -1\baselineskip,
	afterskip  = -0.5em,
	font       = {\normalsize\mdseries\scshape\addfontfeatures{Letters = {UppercaseSmallCaps}}},
]{paragraph}
%%%

%%% Typographic enhancements
\usepackage{microtype}
%%%

%%% Language-specific settings
\usepackage{polyglossia}
\setmainlanguage{ukrainian}
\setotherlanguages{english}
%%%

%%% Captions
\usepackage{caption}
\usepackage{subcaption}

%\DeclareCaptionLabelFormat{closing}{#2)}
%\captionsetup[subtable]{labelformat = closing}

%\captionsetup[subfigure]{labelformat = closing}

\captionsetup[table]{
	aboveskip = 0\baselineskip,
	belowskip = 0\baselineskip,
}

\captionsetup[figure]{
	aboveskip = 1\baselineskip,
	belowskip = 0\baselineskip,
}

\captionsetup[subfigure]{
	labelformat = simple,
	labelformat = brace,
}
%%%

%%% Hyphenated ragged typesetting
\usepackage{ragged2e}
%%%

%%% Table typesetting
\usepackage{booktabs}
\usepackage{longtable}

\usepackage{multirow}

\usepackage{array}
\newcolumntype{v}[1]{>{\RaggedRight\arraybackslash\hspace{0pt}}p{#1}}
\newcolumntype{b}[1]{>{\Centering\arraybackslash\hspace{0pt}}p{#1}}
\newcolumntype{n}[1]{>{\RaggedLeft\arraybackslash\hspace{0pt}}p{#1}}
%%%

%%% Drawing
\usepackage{tikz}
\usepackage{tikzscale}
\usetikzlibrary{positioning}
\usetikzlibrary{arrows.meta} % Stealth arrow tips
%%%

%%% SI units typesetting
\usepackage{siunitx}
\sisetup{
	output-decimal-marker = {,},
	exponent-product      = {\cdot},
	inter-unit-product    = \ensuremath{{} \cdot {}},
	per-mode              = symbol,
}
%%%

%%% Links and hyperreferences
\usepackage{hyperref}
\hypersetup{
	bookmarksnumbered = true,
	colorlinks      = false,
	linkbordercolor = red,
	urlbordercolor  = lightblue,
	pdfborderstyle  = {/S/U/W 1.5},
}
%%%

%%% Length adjustments
% Set baselineskip, default is 14.5 pt
\linespread{1.068966} % ~15.5 pt
\setlength{\emergencystretch}{1em}
\setlength{\parindent}{1.5em}
\newlength{\gridunitwidth}
\setlength{\gridunitwidth}{\textwidth / 12}
%%%

%%% Custom commands
\newcommand{\allcaps}[1]{{\addfontfeatures{LetterSpace = 8, Kerning = Off}#1}}
%%%

\begin{document}

\begin{titlepage}
		\begin{center}
			Міністерство освіти і науки України\\
			Національний авіаційний університет\\
			Навчально-науковий інститут комп'ютерних інформаційних технологій\\
			Кафедра комп'ютеризованих систем управління

			\vspace{\fill}
				Лабораторна робота №1\\
				з дисципліни «Телекомунікаційні~технології комп'ютерних~мереж»\\
				на тему «Канали передачі даних»\\

			\vspace{\fill}

			\begin{flushright}
				Виконав:\\
				студент \allcaps{ННІКІТ}\\
				групи СП-325\\
				Клокун В.\,Д.\\
				Перевірив:\\
				Пушкін Ю.\,О.
			\end{flushright}

			Київ 2018
		\end{center}
	\end{titlepage}

	\section{Мета роботи}
		Ознайомлення з видами і характеристиками різноманітних каналів передачі даних.

	\section{Контрольні запитання}
		\subsection{Класифікація кабелів в залежності від області застосування (кручена пара)}
			В~залежності від~основної області застосування і~відповідно конструкції, кабельні вироби для~структурованих комп'ютерних мереж на~основі кручених пар поділяються на~4 основні види:
			\begin{enumerate}[noitemsep]
				\item Горизонтальний кабель.
				\item Магістральний кабель.
				\item Кабель для шнурів.
				\item Провід для перемичок.
			\end{enumerate}

		\subsection{Які бувають види скруток кручених пар?}
			За~видами скрутки провідників горизонтального кабеля розрізняють парну і~четвірочну скрутки.

		\subsection{Для чого застосовується екранування?}
			Екранування застосовують для~підвищення перехідного загасання, зниження рівня електромагнітної інтерференції і~для~підвищення перешкодозахищеності.

		\subsection{Які бувають конструкції горизонтальних кабелів?}
			Бувають такі конструкції горизонтальних кабелів типу «кручена пара»:
			\begin{enumerate}[noitemsep]
				\item \allcaps{UTP}~— неекранована кручена пара.
				\item \allcaps{STP}~— екранована кручена пара, екран індивідуальний.
				\item \allcaps{S/UTP}~— екранована кручена пара, екран загальний.
				\item \allcaps{S/STP}~— екранована кручена пара, екран індивідуальний та~загальний.
			\end{enumerate}

		\subsection{Які бувають типи оптоволоконних кабелів?}
			За призначенням волоконно-оптичні кабелі можна розділити на:
			\begin{enumerate}
				\item Монтажні (сполучні). Використовуються для механічної комутації і~підключення апаратури.
				\item Об'єктові. Використовуються для~високошвидкісних з'єднань усередині будівель. Як~правило, у~них використовується покриття, що~погано поширює горіння, виділяє малу кількість диму і~не~містить галогенів (\textenglish{\allcaps{LSF/0H}~— low smoke and fume, zero halogen}).
				\item Міські, зонові. З'єднують будинки, райони, міста, області. Зазвичай мережі, побудовані з~їх~використанням, мають довжину від~1 до~100~км.
				\item Магістральні. Призначені для~передачі інформаційних потоків на~великі відстані, для~цього використовуються кабелі з~дуже якісними оптичними волокнами.
			\end{enumerate}
			За місцем прокладки:
			\begin{enumerate}[noitemsep]
				\item По~підземних комунікаціях телефонних та~інших служб.
				\item Призначені для~прокладки в~грунті. Посилена броня, захист від~гризунів.
				\item Підвісні (на~стовпах освітлення, трубостійках, контактних опорах залізниць, опорах ліній електропередач тощо).
				\item Підводні.
			\end{enumerate}

		\subsection{Конструкція оптичного волокна}
			Типова конструкція кабельного сердечника модульного типу складається з:
			\begin{enumerate}[noitemsep]
				\item Оптичного волокна в буфері або службової жили з м'якого мідного дроту.
				\item Гідрофобного заповнювача.
				\item Оболонки оптичного модуля.
				\item Скріплюючого елементу.
				\item Центрального силового елементу.
				\item Проміжної оболонки кабелю.
				\item Силового елементу.
				\item Захисної оболонки.
			\end{enumerate}

		\subsection{Що таке зварювання оптичного волокна?}
			Зварювання оптичного волокна~— це~один з~основних способів створення не\-роз'\-єм\-них з'єд\-нань оптичних волокон. Суть зварювання оптичного волокна полягає у~нагріві кінців волокон електричної дугою. Перевагами такого способу є~надійність, довговічність і~крихітний рівень загасання.

		\subsection{Чим відрізняються одномодові і багатомодові кабелі?}
			В~одномодовому кабелі використовується центральний провідник малого діаметру, що~близько до~довжини хвилі світла,~— від~\SI{5}{\micro\metre} до~\SI{10}{\micro\metre}. При цьому практично всі промені світла розповсюджуються вздовж оптичної осі світлодіода, не~відбиваючись від~зовнішнього провідника. Такий кабель має дуже низький коефіцієнт затухання.
			
			У~багатомодових кабелях використовуються більш широкі центральні провідники~— зазвичай від~\SI{50}{\micro\metre} до~\SI{100}{\micro\metre}, в~яких одночасно розповсюджуються декілька світлових променів під~різними кутами.

			Таким чином, багатомодові кабелі використовують в~основному для~передачі даних на~швидкостях не~більше~10~Гбіт/с на~невеликі відстані~(300–\SI{2000}{\metre}), а~одномодові~— для передачі даних на~надвисоких швидкостях на~відстані до~декількох сотень кілометрів.

		\subsection{Що таке мода?}
			Поняття «мода» описує режим розповсюдження світлових променів у~серцевині кабеля. Мода~— це~кут відбивання променя світла, його шлях.

		\subsection{Які ви знаєте параметри оптичних кабелів?}
			Оптичні кабелі характеризуються такими параметрами:
			\begin{enumerate}[noitemsep]
				\item Центральний силовий елемент.
				\item Кількість оптичних волокон в модулі~(шт.).
				\item Діаметр модуля~(\si{\milli\meter}).
				\item Максимальний зовнішній діаметр кабеля~(\si{\milli\meter}).
				\item Мінімальний радіус кривизни~($\diameter$).
				\item Допустима розтягуюча сила~(\si{\newton}).
				\item Температура експлуатації~(\si{\degreeCelsius}).
				\item Допустима роздавлююча сила~(\si{\newton\per\centi\meter}).
			\end{enumerate}
			В свою чергу оптичні волокна характеризуються такими параметрами:
			\begin{enumerate}
				\item Затухання. Потужність сигналу зменшується через поглинання світла матеріалом волокна та~домішками, розсіювання світла через неоднорідність густини волокна, а~також через кабельні спотворення, обумовлені деформацією кабеля під час монтажу. Вимірюється в~\si{\decibel\per\kilo\meter}.
				\item Хроматична дисперсія. Сигнал спотворюється через те, що~хвилі різної довжини розповсюджуються вздовж волокна з~різною швидкістю. Оскільки прямокутний імпульс має спектр ненульової ширини, хвилі, що~його складають, досягають виходу волокна з~різними затримками через хроматичну дисперсію, і~фронт імпульсу стає «розмитим». Хроматична дисперсія оцінюється відношенням різниці часу розповсюдження двох хвиль в~пікосекундах у~волокні певної довжини, зазвичай~\SI{1}{\kilo\metre} до~різниці довжин хвиль в~наносекундах, і~тому вимірюється в~\si{\pico\second\per\nano\metre\kilo\metre}.
				\item Поляризаційна модова дисперсія. Світлова мода має дві взаємно перпендикулярні поляризаційні складові. У~хвилі з~ідеальним поперечним перерізом, тобто колом, ці складові розповсюджуються з~однаковою швидкістю. Оскільки реальні хвилі завжди мають певну «овальність», то~й~швидкості складових відрізняються, що~призводить до~поляризаційної дисперсії. Цей вид дисперсії зростає пропорційно кореню квадратному від довжини кабеля, тому вимірюється в~\si{\nano\second\per\sqrt{\kilo\meter}}.
			\end{enumerate}

		\subsection{Які бувають типи коаксіальних кабелів?}
			За~техніко-експлуатаційними характеристиками розрізняють широко- та~вузькосмугові коаксіальні кабелі. Широкосмугові кабелі використовуються для~аналогового широкосмугового передавання. Вузькосмугові кабелі застосовують для~цифрового передавання.

			Також відрізняють «товстий» і~«тонкий» коаксіальні кабелі. «Товстий» коаксіальний кабель був розроблений для мереж \textenglish{Ethernet 10Base-5}. Цей кабель має зовнішній діаметр приблизно~\SI{12}{\milli\meter}, діаметр внутрішнього провідника~\SI{2.17}{\milli\meter}, хвильовий опір~\SI{50}{\ohm}, і~затухання на~частоті~\SI{10}{\mega\hertz} не~гірше~\SI{18}{\deci\bel\per\kilo\meter}, однак, погано гнеться, що~робить його незручним для монтажу.

			«Тонкий» коаксіальний кабель призначений для мереж \textenglish{Ethernet 10Base-2}. Його зовнішній діаметр складає близько~\SI{50}{\milli\meter}, діаметр внутрішнього провідника~— \SI{0.89}{\milli\meter} та~хвильовий опір~\SI{50}{\ohm}. Його механічні та~електричні характеристики гірші за~характеристики «товстого» кабеля.

		\subsection{Як можна зовні відрізнити товстий і тонкий коаксіал?}
			За~діаметром. Крім цього, «товстий» коаксіальний кабель зазвичай вироблявся яскраво-жовтого кольору, а~тонкий~— чорного або~сірого.

	\section{Висновок}
		Виконуючи дану лабораторну роботу ми~ознайомились з~видами і~характеристиками різноманітних каналів передачі даних.

\end{document}
