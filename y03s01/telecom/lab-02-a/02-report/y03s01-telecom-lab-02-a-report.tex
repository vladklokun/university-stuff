\documentclass[
	a4paper,
	oneside,
	BCOR = 10mm,
	DIV = 12,
	12pt,
	headings = normal,
]{scrartcl}

%%% Length calculations
\usepackage{calc}
%%%

%%% Support for color
\usepackage{xcolor}
\definecolor{lightblue}{HTML}{03A9F4}
\definecolor{red}{HTML}{F44336}
%%%

%%% Including graphics
\usepackage{graphicx}
%%%

%%% Font selection
\usepackage{fontspec}

\setromanfont{STIX Two Text}[
	SmallCapsFeatures = {LetterSpace = 8},
]

\setsansfont{IBM Plex Sans}[
	Scale = MatchUppercase,
]

\setmonofont{IBM Plex Mono}[
	Scale = MatchUppercase,
]
%%%

%%% Math typesetting
\usepackage{amsmath}

\usepackage{unicode-math}
\setmathfont{STIX Two Math}

\usepackage{ieeetrantools}
%%%

%%% List settings
\usepackage{enumitem}
\setlist[enumerate]{
	label*      = {\arabic*.},
	leftmargin  = *,
	labelindent = \parindent,
	topsep      = 1\baselineskip,
	parsep      = 0\baselineskip,
	itemsep     = 1\baselineskip,
}

\setlist[itemize]{
	label*      = {—},
	leftmargin  = *,
	labelindent = \parindent,
	topsep      = 1\baselineskip,
	parsep      = 0\baselineskip,
	itemsep     = 1\baselineskip,
}

\setlist[description]{
	font        = {\rmfamily\upshape\bfseries},
	topsep      = 1\baselineskip,
	parsep      = 0\baselineskip,
	itemsep     = 0\baselineskip,
}

%%%

%%% Structural elements typesetting
\setkomafont{pagenumber}{\rmfamily}
\setkomafont{disposition}{\rmfamily\bfseries}

% Sectioning
\RedeclareSectionCommand[
	beforeskip = -1\baselineskip,
	afterskip  = 1\baselineskip,
	font       = {\normalsize\bfseries\scshape},
]{section}

\RedeclareSectionCommand[
	beforeskip = -1\baselineskip,
	afterskip  = 1\baselineskip,
	font       = {\normalsize\bfseries\itshape},
]{subsection}

\RedeclareSectionCommand[
	beforeskip = -1\baselineskip,
	afterskip  = 1\baselineskip,
	font       = {\normalsize\bfseries},
]{subsubsection}

\RedeclareSectionCommand[
	beforeskip = -1\baselineskip,
	afterskip  = -0.5em,
	font       = {\normalsize\mdseries\scshape\addfontfeatures{Letters = {UppercaseSmallCaps}}},
]{paragraph}
%%%

%%% Typographic enhancements
\usepackage{microtype}
%%%

%%% Language-specific settings
\usepackage{polyglossia}
\setmainlanguage{ukrainian}
\setotherlanguages{english}
%%%

%%% Captions
\usepackage{caption}
\usepackage{subcaption}

%\DeclareCaptionLabelFormat{closing}{#2)}
%\captionsetup[subtable]{labelformat = closing}

%\captionsetup[subfigure]{labelformat = closing}

\captionsetup[table]{
	aboveskip = 0\baselineskip,
	belowskip = 0\baselineskip,
}

\captionsetup[figure]{
	aboveskip = 1\baselineskip,
	belowskip = 0\baselineskip,
}

\captionsetup[subfigure]{
	labelformat = simple,
	labelformat = brace,
}
%%%

%%% Hyphenated ragged typesetting
\usepackage{ragged2e}
%%%

%%% Table typesetting
\usepackage{booktabs}
\usepackage{longtable}

\usepackage{multirow}

\usepackage{array}
\newcolumntype{v}[1]{>{\RaggedRight\arraybackslash\hspace{0pt}}p{#1}}
\newcolumntype{b}[1]{>{\Centering\arraybackslash\hspace{0pt}}p{#1}}
\newcolumntype{n}[1]{>{\RaggedLeft\arraybackslash\hspace{0pt}}p{#1}}
%%%

%%% Drawing
\usepackage{tikz}
\usepackage{tikzscale}
\usetikzlibrary{positioning}
\usetikzlibrary{arrows.meta} % Stealth arrow tips
%%%

%%% SI units typesetting
\usepackage{siunitx}
\sisetup{
	output-decimal-marker = {,},
	exponent-product      = {\cdot},
	inter-unit-product    = \ensuremath{{} \cdot {}},
	per-mode              = symbol,
}
%%%

%%% Links and hyperreferences
\usepackage{hyperref}
\hypersetup{
	bookmarksnumbered = true,
	colorlinks      = false,
	linkbordercolor = red,
	urlbordercolor  = lightblue,
	pdfborderstyle  = {/S/U/W 1.5},
}
%%%

%%% Length adjustments
% Set baselineskip, default is 14.5 pt
\linespread{1.068966} % ~15.5 pt
\setlength{\emergencystretch}{1em}
\setlength{\parindent}{1.5em}
\newlength{\gridunitwidth}
\setlength{\gridunitwidth}{\textwidth / 12}
%%%

%%% Custom commands
\newcommand{\allcaps}[1]{{\addfontfeatures{LetterSpace = 8, Kerning = Off}#1}}
%%%

\begin{document}

\begin{titlepage}
		\begin{center}
			Міністерство освіти і науки України\\
			Національний авіаційний університет\\
			Навчально-науковий інститут комп'ютерних інформаційних технологій\\
			Кафедра комп'ютеризованих систем управління

			\vspace{\fill}
				Лабораторна робота №\\
				з дисципліни «Телекомунікаційні~технології комп'ютерних~мереж»\\
				на тему «Формування коду Хемінга»\\
				Варіант №6

			\vspace{\fill}

			\begin{flushright}
				Виконав:\\
				студент \allcaps{ННІКІТ}\\
				групи СП-325\\
				Клокун В.\,Д.\\
				Перевірив:\\
				Пушкін Ю.\,О.
			\end{flushright}

			Київ 2018
		\end{center}
	\end{titlepage}

	\section{Мета роботи}
		Ознайомитись з методиками формування простого і посиленого кодів Хемінга. Здобути практичні навички побудови кодів.
		
	\section{Хід роботи}
		Відповідно до варіанта для виконання роботи дано число~$N = 164_{10}$.

		\subsection{Формування простого коду Хемінга}
			Нехай слово~$A$ закодоване простим кодом Хемінга. Для формування слова~$A$ перетворюємо задане число~$N$ в~двійкову систему числення:
			\[
				N = 164_{10} = 10100100_{2}.
			\]
				Як бачимо, кількість біт передаваної інформації~$m = 8$. Простий код Хемінга розрахований на коригування 1~помилки в~даних, тому кількість контрольних розрядів~$k$ має задовольняти нерівність:
			\[
				k \geqslant \log_{2} (k + m + 1) \quad \Longrightarrow \quad k \geqslant \log_{2} (k + 9).
			\]
			Найменшим числом, яке задовольняє нерівність, є~$k_{\text{min}} = 4$, що і буде кількістю контрольних розрядів. Таким чином, довжина слова~$|A|$ (загальна кількість розрядів):
			\[
				|A| = m + k_{\text{min}} = 8 + 4 = 12.
			\]

			Код Хемінга передбачає, що контрольні розряди розташовуються на позиціях слова~$a_i$, де $i = 2^N, N \in \{0, 1, 2, \dots\}$. Тому запишемо слово~$A$, залишаючи контрольні розряди ($i = 1, 2, 4, 8$) пустими:
			\[
				\begin{tikzpicture}
					\node (a01) at (0,0)        { };
					\node[right of = a01] (a02) { };
					\node[right of = a02] (a03) {1};
					\node[right of = a03] (a04) { };
					\node[right of = a04] (a05) {0};
					\node[right of = a05] (a06) {1};
					\node[right of = a06] (a07) {0};
					\node[right of = a07] (a08) { };
					\node[right of = a08] (a09) {0};
					\node[right of = a09] (a10) {1};
					\node[right of = a10] (a11) {0};
					\node[right of = a11] (a12) {0};
					\node[left of = a01] (A)   {$A = $};

					\node[above of = a01] (a01_t) {$a_{1}$};
					\node[above of = a02] (a02_t) {$a_{2}$};
					\node[above of = a03] (a03_t) {$a_{3}$};
					\node[above of = a04] (a04_t) {$a_{4}$};
					\node[above of = a05] (a05_t) {$a_{5}$};
					\node[above of = a06] (a06_t) {$a_{6}$};
					\node[above of = a07] (a07_t) {$a_{7}$};
					\node[above of = a08] (a08_t) {$a_{8}$};
					\node[above of = a09] (a09_t) {$a_{9}$};
					\node[above of = a10] (a10_t) {$a_{10}$};
					\node[above of = a11] (a11_t) {$a_{11}$};
					\node[above of = a12] (a12_t) {$a_{12}$};
				\end{tikzpicture}
			\]
			Обчислимо контрольні розряди:
			\begin{IEEEeqnarray*}{rCl}
				a_1 &=& a_1 \oplus a_3 \oplus a_5 \oplus a_7 \oplus a_9 \oplus a_{11} = 0 \oplus 1 \oplus 0 \oplus 0 \oplus 0 \oplus 0 = 1,\\
				a_2 &=& a_2 \oplus a_3 \oplus a_6 \oplus a_7 \oplus a_{10} \oplus a_{11} = 0 \oplus 1 \oplus 1 \oplus 0 \oplus 1 \oplus 0 = 1,\\
				a_4 &=& a_4 \oplus a_5 \oplus a_6 \oplus a_7 \oplus a_{12} = 0 \oplus 0 \oplus 1 \oplus 0 \oplus 0 = 1,\\
				a_9 &=& a_9 \oplus a_{10} \oplus a_{11} \oplus a_{12} = 0 \oplus 1 \oplus 0 \oplus 0 = 1.\\
			\end{IEEEeqnarray*}
			Впишемо знайдені контрольні розряди:
			\[
				\begin{tikzpicture}
					\node (a01) at (0,0)        {1};
					\node[right of = a01] (a02) {1};
					\node[right of = a02] (a03) {1};
					\node[right of = a03] (a04) {1};
					\node[right of = a04] (a05) {0};
					\node[right of = a05] (a06) {1};
					\node[right of = a06] (a07) {0};
					\node[right of = a07] (a08) {1};
					\node[right of = a08] (a09) {0};
					\node[right of = a09] (a10) {1};
					\node[right of = a10] (a11) {0};
					\node[right of = a11] (a12) {0};
					\node[left of = a01] (A)   {$A = $};

					\node[above of = a01] (a01_t) {$a_{1}$};
					\node[above of = a02] (a02_t) {$a_{2}$};
					\node[above of = a03] (a03_t) {$a_{3}$};
					\node[above of = a04] (a04_t) {$a_{4}$};
					\node[above of = a05] (a05_t) {$a_{5}$};
					\node[above of = a06] (a06_t) {$a_{6}$};
					\node[above of = a07] (a07_t) {$a_{7}$};
					\node[above of = a08] (a08_t) {$a_{8}$};
					\node[above of = a09] (a09_t) {$a_{9}$};
					\node[above of = a10] (a10_t) {$a_{10}$};
					\node[above of = a11] (a11_t) {$a_{11}$};
					\node[above of = a12] (a12_t) {$a_{12}$};
				\end{tikzpicture}
			\]
			Закодувавши задане число кодом Хемінга, отримали слово~$A = 111101001100$.

	\section{Висновок}


\end{document}
