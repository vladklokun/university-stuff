\documentclass[
	a4paper,
	oneside,
	BCOR = 10mm,
	DIV = 12,
	12pt,
	headings = normal,
]{scrartcl}

%%% Length calculations
\usepackage{calc}
%%%

%%% Support for color
\usepackage{xcolor}
\definecolor{lightblue}{HTML}{03A9F4}
\definecolor{red}{HTML}{F44336}
%%%

%%% Including graphics
\usepackage{graphicx}
%%%

%%% Font selection
\usepackage{fontspec}

\setromanfont{STIX Two Text}[
	SmallCapsFeatures = {LetterSpace = 8},
]

\setsansfont{IBM Plex Sans}[
	Scale = MatchUppercase,
]

\setmonofont{IBM Plex Mono}[
	Scale = MatchUppercase,
]
%%%

%%% Math typesetting
\usepackage{amsmath}

\usepackage{unicode-math}
\setmathfont{STIX Two Math}
%%%

%%% List settings
\usepackage{enumitem}
\setlist[enumerate]{
	label*      = {\arabic*.},
	leftmargin  = *,
	labelindent = \parindent,
	topsep      = 1\baselineskip,
	parsep      = 0\baselineskip,
	itemsep     = 1\baselineskip,
}

\setlist[itemize]{
	label*      = {—},
	leftmargin  = *,
	labelindent = \parindent,
	topsep      = 1\baselineskip,
	parsep      = 0\baselineskip,
	itemsep     = 1\baselineskip,
}

\setlist[description]{
	font        = {\rmfamily\upshape\bfseries},
	topsep      = 1\baselineskip,
	parsep      = 0\baselineskip,
	itemsep     = 0\baselineskip,
}

%%%

%%% Structural elements typesetting
\setkomafont{pagenumber}{\rmfamily}
\setkomafont{disposition}{\rmfamily\bfseries}

% Sectioning
\RedeclareSectionCommand[
	beforeskip = -1\baselineskip,
	afterskip  = 1\baselineskip,
	font       = {\normalsize\bfseries\scshape},
]{section}

\RedeclareSectionCommand[
	beforeskip = -1\baselineskip,
	afterskip  = 1\baselineskip,
	font       = {\normalsize\bfseries\itshape},
]{subsection}

\RedeclareSectionCommand[
	beforeskip = -1\baselineskip,
	afterskip  = 1\baselineskip,
	font       = {\normalsize\bfseries},
]{subsubsection}

\RedeclareSectionCommand[
	beforeskip = -1\baselineskip,
	afterskip  = -0.5em,
	font       = {\normalsize\mdseries\scshape\addfontfeatures{Letters = {UppercaseSmallCaps}}},
]{paragraph}

%%%

%%% Typographic enhancements
\usepackage{microtype}
%%%

%%% Language-specific settings
\usepackage{polyglossia}
\setmainlanguage{ukrainian}
\setotherlanguages{english}
%%%

%%% Captions
\usepackage{caption}
\usepackage{subcaption}

%\DeclareCaptionLabelFormat{closing}{#2)}
%\captionsetup[subtable]{labelformat = closing}

%\captionsetup[subfigure]{labelformat = closing}

\captionsetup[table]{
	aboveskip = 0\baselineskip,
	belowskip = 0\baselineskip,
}

\captionsetup[figure]{
	aboveskip = 1\baselineskip,
	belowskip = 0\baselineskip,
}

\captionsetup[subfigure]{
	labelformat = simple,
	labelformat = brace,
}
%%%

%%% Hyphenated ragged typesetting
\usepackage{ragged2e}
%%%

%%% Table typesetting
\usepackage{booktabs}
\usepackage{longtable}

\usepackage{multirow}

\usepackage{array}
\newcolumntype{v}[1]{>{\RaggedRight\arraybackslash\hspace{0pt}}p{#1}}
\newcolumntype{b}[1]{>{\Centering\arraybackslash\hspace{0pt}}p{#1}}
\newcolumntype{n}[1]{>{\RaggedLeft\arraybackslash\hspace{0pt}}p{#1}}
%%%

%%% Drawing
\usepackage{tikz}
\usepackage{tikzscale}
\usetikzlibrary{positioning}
\usetikzlibrary{arrows.meta} % Stealth arrow tips
%%%

%%% SI units typesetting
\usepackage{siunitx}
\sisetup{
	output-decimal-marker = {,},
	exponent-product      = {\cdot},
	inter-unit-product    = \ensuremath{{} \cdot {}},
	per-mode              = symbol,
}
%%%

%%% Date and time typesetting
\usepackage[
	ukrainian,
	useregional,
	showdow,
]{datetime2}
%%%

%%% Links and hyperreferences
\usepackage{hyperref}
\hypersetup{
	bookmarksnumbered = true,
	colorlinks      = false,
	linkbordercolor = red,
	urlbordercolor  = lightblue,
	pdfborderstyle  = {/S/U/W 1.5},
}
%%%

%%% Length adjustments
% Set baselineskip, default is 14.5 pt
\linespread{1.068966} % ~15.5 pt
\setlength{\emergencystretch}{1em}
\setlength{\parindent}{1.5em}
\newlength{\gridunitwidth}
\setlength{\gridunitwidth}{\textwidth / 12}
\newlength{\gridunitheight}
\setlength{\gridunitheight}{\textheight / 12}
%%%

%%% Custom commands
\newcommand{\allcaps}[1]{{\addfontfeatures{LetterSpace = 8, Kerning = Off}#1}}
\newcommand{\allsmallcaps}[1]{{\addfontfeatures{Letters = UppercaseSmallCaps, LetterSpace = 8, Kerning = Off}#1}}
%%%

\begin{document}
	\begin{titlepage}
		\vspace*{4\gridunitheight}
		% \begin{center}
			\noindent\textsc{Конспект}\\[\baselineskip]
			{\setlength{\tabcolsep}{0em}
			\begin{tabular}{
					v{3\gridunitwidth - 2\tabcolsep}
					v{6\gridunitwidth - 2\tabcolsep}
			}
				з дисципліни: & «Технології мультимедіа»\\
				викладач:     & Апенько Наталія Вікторівна\\
			\end{tabular}
			}
		% \end{center}
		\vspace*{\fill}
	\end{titlepage}

	\tableofcontents
	\newpage

	\section{\DTMDate{2018-09-04}}
		\subsection{Основні поняття мультимедіа}
			Мультимедіа~— це комплекс апаратних і~програмних засобів, що~дозволяють працювати з~різними форматами даних у~рамках єдиного інформаційного середовища. Мультимедіа~— це~новий тип зберігання інформації в~єдиній цифровій формі. Є~3~основні принципи мультимедіа:
			\begin{enumerate}[noitemsep]
				\item Представлення інформації за~допомогою безлічі комбінацій.
				\item Наявність кількох сюжетних ліній в~змісті продукту.
				\item Художній дизайн інтерфейсів і~засобів навігації.
			\end{enumerate}

			Особливістю технології є~такі можливості мультимедіа, які~активно використовуються в~поданій інформації:
			\begin{itemize}[noitemsep]
				\item можливість зберігання великого обсягу самої різної інформації на~одному носії;
				\item можливість збільшення деталізації на~екрані зображення або~його найцікавіших фрагментів при~збереженні якості зображення (це~особливо важливо для~презентації творів мистецтва і~унікальних історичних документів);
				\item можливість порівняння зображені і~обробки його різноманітними програмними засобами з~наукового-дослідними або~пізнавальними цілями;
				\item можливість виділення в~текстовому або~іншому візуальному матеріалі;
				\item можливість здійснення безперервного музичного або~будь-якого іншого аудіосупроводу відповідного статичному чи~динамічному візуальному ряду;
				\item можливість використання відеофрагментів, відеозаписів і~так далі, функції стоп-кадру, покадрового гортання відеозапису;
				\item можливість включення баз даних, методик обробки образів, анімації;
				\item можливість підключення до глобальної мережі;
				\item можливість роботи з~різними додатками;
				\item можливість створення власних галерей з~представленої в~продукті інформації;
				\item можливість запам'ятовування пройденої інформації і~створення закладок;
				\item можливість автоматичного перегляду всього змісту продукту, включення до~складу продукту ігрових компонентів з~інформаційним середовищем;
				\item можливість вільної навігації і~виходу в~основне меню з~будь-якої точки продукту.
			\end{itemize}

		\subsection{Класифікація мультимедійних додатків}
			Мультимедіа можна класифікувати з~різних точок зору, а~саме:
			\begin{enumerate}[noitemsep]
				\item На~основі підтримки взаємодії.
				\item На~основі використання різних мультимедійних, телекомунікаційних технологій.
			\end{enumerate}
			Області застосування мультимедійних додатків:
			\begin{itemize}[noitemsep]
				\item навчання з~використанням комп'ютерних технологій;
				\item відеоенциклопедії;
				\item інтерактивні путівники;
				\item тренажери;
				\item інформаційна і~рекламна служба;
				\item розважальні сфери;
				\item презентаційна сфера;
				\item творча сфера;
				\item військові технології.
			\end{itemize}

\end{document}
